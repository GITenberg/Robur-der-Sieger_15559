\documentclass[oneside,12pt]{book}
\usepackage[german]{babel}%required
\renewcommand{\thefootnote}{\fnsymbol{footnote}}

% comment (delete) the following four lines for a non-fraktur version
\usepackage{yfonts}%required for fraktur version
\newenvironment{antiqua}{\normalfont}{}
\protect\renewcommand{\thepage}{\textfrak{\arabic{page}}}
\newcommand{\s}{s:}

% un-comment the following four lines for a non-fraktur version
%	\newcommand{\s}{s}
%	\newcommand{\frakfamily}{\null}
%	\newenvironment{antiqua}{}{}
%	\sloppy

\begin{document}
\thispagestyle{empty}

\begin{verbatim}


Project Gutenberg's Robur der Sieger, by Jules Verne

This eBook is for the use of anyone anywhere at no cost and with
almost no restrictions whatsoever.  You may copy it, give it away or
re-use it under the terms of the Project Gutenberg License included
with this eBook or online at www.gutenberg.net


Title: Robur der Sieger

Author: Jules Verne

Release Date: April 4, 2005 [EBook #15559]

Language: German

Character set encoding: TeX

*** START OF THIS PROJECT GUTENBERG EBOOK ROBUR DER SIEGER ***




Produced by K.F. Creiner and the Online Distributed Proofreading Team.





\end{verbatim}

\newpage
\frakfamily

\thispagestyle{empty}
\begin{center}
Juliu{\s} Verne'{\s} Reiseromane. Band 51.

\bigskip\bigskip\bigskip\bigskip\bigskip\bigskip
\Huge Robur der Sieger

\bigskip\bigskip\bigskip\bigskip\bigskip\bigskip
\large

von \\
\bigskip\bigskip\bigskip\bigskip
\Huge Juliu{\s} Verne \\
\bigskip\bigskip\bigskip\bigskip\bigskip
\large
Bibliographische Anstalt Adolph Schumann. \\
Leipzig
\end{center}
\vfill
\normalsize
\begin{antiqua}
[Hinweis: Andere \"Ubersetzungen des Romans
{\glqq}Robur-le-Conqu\'erant{\grqq} sind unter dem Titel {\glqq}Robur
der Eroberer{\grqq} erschienen.]
\end{antiqua}


\newpage\begin{center}\label{kap01}
{\large \begin{antiqua}I.\end{antiqua}\\
Worin die gelehrte Welt sich ebenso wenig Rath wei{\ss}, wie die
ungelehrte.\\\bigskip}
\end{center}



Paff! ... Paff!

Zwei Pistolensch\"usse knallten zu gleicher Zeit. Eine Kuh, welche
eben in der Entfernung von f\"unfzig Schritten vor\"uber trabte,
bekam eine Kugel in'{\s} R\"uckgrat ... und sie ging die Sache doch
gar nicht{\s} an.

Von den beiden Gegnern war keiner getroffen worden.

Wer waren jene beide Herren? Niemand wei{\ss} e{\s}, und gerade hier
w\"are ja Gelegenheit gewesen, ihre Namen der Nachwelt zu
\"uberliefern. E{\s} l\"a{\ss}t sich \"uber sie nicht{\s} weiter
sagen, al{\s} da{\ss} der \"altere ein Engl\"ander, der j\"ungere
Duellant ein Amerikaner war. Desto leichter l\"a{\ss}t sich die
Oertlichkeit bestimmen, an der jener unschuldige Wiederk\"auer eben
sein letzte{\s} Gra{\s}b\"undelchen abgeweidet hatte; diese ist
n\"amlich am rechten Ufer de{\s} Niagara und unweit der
H\"angebr\"ucke zu suchen, welche drei Meilen unterhalb der
ber\"uhmten F\"alle da{\s} canadische Ufer mit dem amerikanischen
verbindet.

Der Engl\"ander schritt jetzt auf den Amerikaner zu.

{\glqq}Ich bleibe nicht{\s}destoweniger dabei, da{\ss} e{\s} die
Melodie von \begin{antiqua}Rule Britannia\end{antiqua} war, sagte er.

-- Nein, der \begin{antiqua}Yankee Doodle\end{antiqua}!{\grqq}
versetzte der Andere.

Der Streit schien auf'{\s} Neue entbrennen zu sollen, al{\s} sich
einer der Zeugen -- ohne Zweifel im Interesse de{\s} weidenden
Vieh{\s} -- mit den Worten einmischte:

{\glqq}Nehmen wir an, e{\s} w\"are der \begin{antiqua}Rule
Doodle\end{antiqua} und der \begin{antiqua}Yankee
Britannia\end{antiqua} gewesen und begeben wir un{\s} nun zum
Fr\"uhst\"uck.{\grqq}

Diese{\s} Compromi{\ss} zwischen den beiden Nationalges\"angen
Amerika{\s} und Gro{\ss}britannien{\s} wurde zur allgemeinen
Befriedigung angenommen. L\"ang{\s} de{\s} linken Niagara-Ufer{\s}
zur\"uckwandelnd, beeilten sich Amerikaner und Engl\"ander, an der
einladenden Tafel de{\s} H\^otel{\s} auf Goat I{\s}land -- einem
neutralen Gebiete zwischen den beiden F\"allen -- Platz zu nehmen.
W\"ahrend ihrer Besch\"aftigung mit gekochten Eiern und dem
lande{\s}\"ublichen Schinken mit kaltem Roastbeef, einem
Zwischengericht von im Munde fast brennenden Pickle{\s} und mit
Hochfluthen von Thee, welche die weltbekannten Wasserf\"alle
eifers\"uchtig machen k\"onnten, wollen wir sie nicht weiter
st\"oren, zumal kaum anzunehmen ist, da{\ss} von ihnen im Laufe
dieser Erz\"ahlung noch ferner die Rede sein wird.

Wer hatte nun Recht -- der Engl\"ander oder der Amerikaner? E{\s}
w\"are schwer gewesen, diese Frage zu entscheiden. Jedenfall{\s}
liefert jene{\s} Duell den Bewei{\s} f\"ur die leidenschaftliche
Erregung der Geister nicht allein in der Neuen, sondern auch in der
Alten Welt, und zwar \"uber ein Ereigni{\ss} oder eine
unerkl\"arliche Erscheinung, welche seit etwa einem Monate alle
K\"opfe verwirrte.

\begin{antiqua}... Os sublime dedit coelumque tueri\end{antiqua}\\
hat Ovid einst zu Ehren der Menschheit gesungen. In der That hatte
man seit dem Erscheinen de{\s} ersten Menschen auf der Erdkugel noch
niemal{\s} den Himmel so vielfach betrachtet.

Gerade in der vorhergegangenen Nacht hatte n\"amlich eine Trompete
au{\s} der Luft ihre metallenen T\"one herabgeschmettert \"uber
denjenigen Theil von Canada, der sich zwischen dem Ontario- und dem
Erie-See au{\s}dehnt. Die Einen hatten darau{\s} den
\begin{antiqua}Yankee Doodle\end{antiqua}, die Anderen da{\s}
\begin{antiqua}Rule Britannia\end{antiqua} zu h\"oren vermeint,
darau{\s} entstand auch obiger angels\"achsische Zweikampf, der mit
dem Fr\"uhst\"uck auf Goat I{\s}land endigte. Vielleicht war e{\s}
weder der eine, noch der andere Nationalgesang gewesen; nur dar\"uber
herrschte bei Niemand ein Zweifel, da{\ss} die betreffenden T\"one
die Eigenth\"umlichkeit gehabt hatten, al{\s} schienen sie vom Himmel
zur Erde hernieder zu steigen.

Sollte man etwa gar an eine Himmel{\s}posaune denken, die ein Engel
oder ein Erzengel geblasen h\"atte? ... Waren e{\s} nicht vielmehr
lustige Luftschiffer gewesen, die sich de{\s} sonoren Instrumente{\s}
bedienten, von dem die Reclame so au{\s}gebreiteten Gebrauch macht?
Nein, von einem Ballon, von Luftschiffern konnte nicht die Rede sein.
In hohen Himmel{\s}regionen vollzog sich ein au{\ss}ergew\"ohnliche{\s}
Ereigni{\ss}, dessen Natur und Ursprung kein Mensch zu entr\"athseln
vermochte. Heute zeigte sich da{\s}selbe \"uber Amerika,
vierundzwanzig Stunden sp\"ater \"uber Europa, acht Tage sp\"ater in
Asien \"uber dem Himmlischen Reiche. Wenn die Trompete, welche da{\s}
Vor\"uberziehen jener Erscheinung ank\"undigte, nicht die de{\s}
J\"ungsten Gerichte{\s} war, welche, ja, welche war e{\s} dann?

In allen Landen der Erde, in K\"onigreichen wie in Republiken,
entstand de{\s}halb eine gewisse Unruhe, welche gestillt werden
mu{\ss}te. Vernimmt Einer in seinem Hause eigenth\"umliche und
unerkl\"arliche Ger\"ausche, w\"urde er nicht schnellsten{\s} die
Ursache derselben zu ermitteln suchen, und wenn da{\s} vergeblich
w\"are, w\"urde er nicht sein Hau{\s} verlassen, um ein andere{\s} zu
bewohnen? Ganz sicherlich! Hier war da{\s} Hau{\s} freilich die
Erdkugel, und e{\s} gab doch kein Mittel, diese zu verlassen und etwa
mit dem Monde, Mar{\s}, Venu{\s}, Jupiter oder einem anderen Planeten
de{\s} Sonnensystem{\s} zu vertauschen.

E{\s} galt demnach unbedingt, aufzukl\"aren, wa{\s} im unendlichen
leeren Raume, doch innerhalb der Erdatmosph\"are, vorging. Ohne Luft
ist ja ein Ger\"ausch unm\"oglich, und da man hier ein solche{\s}
vernahm -- immer jene fast sagenhafte Trompete -- mu{\ss}te die
Erscheinung auch in der Lufth\"ulle stattfinden, deren Dichtigkeit
sich nach oben zu immer mehr vermindert und die sich \"uber unserem
Sph\"aroid nur wenige Meilen hoch verbreitet.

Nat\"urlich bem\"achtigten sich die Tage{\s}bl\"atter der
vorliegenden Frage, behandelten sie unter allen Gesicht{\s}punkten,
beleuchteten oder verdunkelten dieselbe, berichteten falsche oder
wahre Thatsachen, erregten oder beruhigten ihre Leser im Interesse
der H\"ohe ihrer Auflage -- und wiegelten endlich die schon halb
verwirrten Massen nicht wenig auf. Welch' Wunder! Die Politik hatte
den Laufpa{\ss} erhalten und die Gesch\"afte gingen de{\s}halb doch
nicht schlecht. Aber um wa{\s} handelte e{\s} sich \"uberhaupt?

Man befragte alle gro{\ss}en Observatorien der ganzen Welt. Wenn
diese keine Antwort gaben, wozu n\"utzten dann solche Observatorien
eigentlich? Wenn die Astronomen, welche selbst in der Entfernung von
hunderttausend Millionen Meilen noch einen Lichtpunkt zu zwei und
drei Sternen aufzul\"osen verm\"ogen, nicht im Stande waren, den
Ursprung einer ko{\s}mischen Erscheinung zu ergr\"unden, die nur
wenige Kilometer \"uber ihnen auftrat, wozu hatte man Astronomen?

Man konnte auch in der That kaum sch\"atzung{\s}weise angeben, wie
viel Teleskope, Brillen, Fernr\"ohre, Lorgnetten, Binocle{\s} und
Monocle{\s} w\"ahrend der sch\"onen Sommernacht nach dem Himmel
gerichtet waren, noch wie viele Augen sich vor die Oculare und
Instrumente von jeder Art und Vergr\"o{\ss}erung hefteten. Vielleicht
mehrere Hunderttausend, und da{\s} ist nur gering angeschlagen.
Zehnmal mehr, al{\s} man am Firmament mit unbewaffnetem Auge
sichtbare Sterne z\"ahlt. Nein, noch keiner, auf allen Punkten der
Erdkugel gleichzeitig beobachteten Sonnenfinsterni{\ss} hatte man
solche Ehre angethan!

Die Observatorien antworteten, aber unzul\"anglich. Jede{\s} gab
seine Meinung ab, die stet{\s} von der aller anderen abwich, so
da{\ss} sich darau{\s} w\"ahrend der letzten Wochen de{\s} April und
der ersten de{\s} Mai ein wirklicher B\"urgerkrieg unter der
Gelehrtenwelt entwickelte.

Da{\s} Observatorium von Pari{\s} erwie{\s} sich sehr
zur\"uckhaltend. Keine seiner Abtheilungen sprach sich entschieden
au{\s}. In der Abtheilung f\"ur mathematische Astronomie hatte man
e{\s} f\"ur unter seiner W\"urde gehalten, Beobachtungen anzustellen;
in der f\"ur die Meridianmessung hatte man nicht{\s} entdeckt; in der
f\"ur physikalische Beobachtungen hatte man nicht{\s} wahrgenommen;
in der f\"ur Geod\"asie nicht{\s} bemerkt; in der f\"ur Meteorologie
war Niemand etwa{\s} aufgefallen; in der f\"ur die Berechnungen hatte
man nicht{\s} gesehen. Da{\s} war wenigsten{\s} ein offene{\s}
Gest\"andni{\ss}. Dieselbe Offenherzigkeit bekundete da{\s}
Observatorium von Montsouci{\s}, wie die magnetische Station im Park
Saint-Maur. Dieselbe Achtung vor der Wahrheit bewie{\s} da{\s}
L\"angenbureau. Nun ja, Frankreich hei{\ss}t ja da{\s} Land, wo man
{\glqq}frank{\grqq}, d.~h. offen spricht.

Die Provinz war etwa{\s} entschiedener in ihrer Aeu{\ss}erung. Etwa
in der Nacht zwischen dem 6. und 7. Mai hatte sich ein Lichtschein
elektrischen Ursprunge{\s} gezeigt, der 20 Secunden nicht
\"uberdauerte. Am Pic-Du-Midi war derselbe zwischen 9 und 10 Uhr
Abend{\s} beobachtet worden; im meteorologischen Observatorium de{\s}
Puy-de-D\^ome hatte man ihn zwischen 1 und 2 Uhr Morgen{\s} bemerkt;
auf dem Mont Ventoux in der Provence zwischen 2 und 3 Uhr; in Nizza
zwischen 3 und 4 Uhr; auf den Semnoz-Alpen endlich zwischen Annecy,
le Bourget und dem Genfer See im Augenblicke, al{\s} der
Tage{\s}schimmer sich eben bi{\s} zum Zenith erhob.

Offenbar konnte man diese Beobachtungen unm\"oglich in Bausch und
Bogen verwerfen. E{\s} unterlag keinem Zweifel, da{\ss} der
Lichtschein an verschiedenen Punkten, und zwar im Verlauf einiger
Stunden, wahrgenommen worden war. Derselbe ging also entweder von
mehreren Herden au{\s}, die sich durch die Erdatmosph\"are
hinbewegten, oder, wenn er nur einem einzigen solchen angeh\"orte, so
mu{\ss}te dieser sich mit einer Schnelligkeit fortbewegen, welche
nahezu 200 Kilometer in der Stunde erreichte.

Hatte man denn aber im Laufe de{\s} Tage{\s} niemal{\s} etwa{\s}
Besondere{\s} in der Luft bemerkt?

Nein, niemal{\s}.

Erklang nicht wenigsten{\s} jene Trompete einmal durch die
Luftschichten?

Nein, zwischen Aufgang und Untergang der Sonne hatte man nicht den
leisesten Ton geh\"ort.

Im vereinigten K\"onigreich Gro{\ss}britannien wu{\ss}te man nicht
mehr au{\s}, noch ein. Die Observatorien gelangten zu keinerlei
Uebereinstimmung. Greenwich konnte sich nicht mit Oxford
verst\"andigen, obwohl Beide die Behauptung aufstellten, {\glqq}an
der ganzen Sache sei nicht{\s}{\grqq}.

{\glqq}Eine Gesicht{\s}t\"auschung! meinte da{\s} Eine.

-- Eine Geh\"or{\s}t\"auschung!{\grqq} erwiderte da{\s} Andere.

Dar\"uber lagen sie im Streit; auf eine T\"auschung lief e{\s} jedoch
allemal hinau{\s}. Die Verhandlungen zwischen den Sternwarten zu
Berlin und der zu Wien drohten zu internationalen Verwicklungen zu
f\"uhren. Ru{\ss}land bewie{\s} ihnen in der Person de{\s}
Vorsteher{\s} seiner Sternwarte zu Pulkowa, da{\ss} sie Beide Recht
h\"atten, da{\s} h\"ange nur von den Gesicht{\s}punkten ab, auf die
sie sich bez\"uglich Bestimmung der Natur jener Erscheinung stellten,
die in der Theorie unm\"oglich schien und in der Praxi{\s} m\"oglich
war.

In der Schweiz, auf der Sternwarte zu S\"anti{\s}, im Canton
Appenzell, auf dem Rigi, im G\"abri{\s}, in den
Beobachtung{\s}stationen de{\s} St.~Gotthard, St.~Bernhard, de{\s}
Julier, de{\s} Simplon, in denen von Z\"urich und de{\s} Sonnblick in
den Hohen Tauern, beflei{\ss}igte man sich einer ganz besonderen
Zur\"uckhaltung gegen\"uber einer Thatsache, die bi{\s}her Niemand zu
bekr\"aftigen vermocht hatte -- wa{\s} gewi{\ss} recht vern\"unftig
zu nennen ist.

In Italien dagegen, auf den meteorologischen Stationen de{\s}
Vesuv{\s} und de{\s} Aetna, welch' letztere sich in der alten Casa
Inghlese befindet, wie auf dem Monte Cavo, z\"ogerten die Beobachter
nicht im geringsten, die Wirklichkeit jener Erscheinung anzuerkennen,
und da{\s} auf Grund de{\s} Umstande{\s}, da{\ss} sie dieselbe einmal
am Tage in Form eine{\s} kleinen Dampfw\"olkchen{\s} und einmal in
der Nacht in Gestalt einer Sternschnuppe hatten wahrnehmen k\"onnen.
Ueber die eigentliche Natur derselben wu{\ss}ten sie freilich
ebenfall{\s} nicht{\s}.

In der That begann diese{\s} Geheimni{\s} allm\"ahlich die Vertreter
der Wissenschaft zu erm\"uden, erregte dagegen und erschreckte desto
mehr die Einf\"altigen und Unwissenden, welche, Dank einem hochweisen
Naturgesetze, von jeher in dieser Welt die ungeheure Mehrzahl
gebildet haben, noch bilden und in aller Zukunft bilden werden. Die
Astronomen und Meteorologen hatten also schon darauf verzichtet, sich
mit der Sache zu besch\"aftigen, al{\s} in der Nacht vom 26. zum 27.
auf der Sternwarte zu Cantokeino in Finnland, in Norwegen, in der
Nacht vom 28. zum 29. auf der de{\s} I{\s}fjord und auf Spitzbergen,
die Norweger auf einer und die Schweden auf der anderen Seite in der
Anschauung \"ubereingestimmt hatten, da{\ss} inmitten einer Art
Nordlichtscheine{\s} etwa{\s} wie ein gewaltiger Vogel oder ein
Luftungeheuer sichtbar gewesen sei. War e{\s} auch nicht gelungen,
dessen Structur genauer zu bestimmen, so unterlag e{\s} doch keinem
Zweifel, da{\ss} derselbe kleine K\"orper au{\s}geworfen habe, welche
gleich Bomben mit einem Knalle zersprangen.

In Europa neigte man wohl dazu, die Beobachtungen der Stationen von
Finnmarken und Spitzbergen nicht anzuzweifeln. Ganz besonder{\s}
merkw\"urdig erschien freilich, da{\ss} die Schweden und die Norweger
doch einmal \"uber einen Punkt einig zu sein schienen.

Man lachte und spottete \"uber die angebliche Entdeckung auf allen
Sternwarten S\"udamerika{\s}, in Brasilien und Peru, ebenso wie in La
Plata, auf denen von Australien, in Sidney, Adelaide, wie in
Melbourne, und da{\s} australische Lachen ist bekanntlich sehr
ansteckend.

Nur ein einziger Vorsteher einer meteorologischen Station verhielt
sich zustimmend bei dieser Frage, trotz der Sp\"otteleien, welche
seine Erkl\"arung derselben hervorrufen mochte. Da{\s} war ein
Chinese, der Director der Sternwarte zu Zi-Ka-Wey, die sich inmitten
einer au{\s}gedehnten Ebene, mindesten{\s} zehn Lieue{\s} vom Meere,
erhebt und welche bei ungemeiner Klarheit der Luft ein grenzenlo{\s}
weiter Horizont umschlie{\ss}t.

{\glqq}E{\s} k\"onnte ja sein, sagte er, da{\ss} der Gegenstand, um
den e{\s} sich handelt, ein besonder{\s} construirter Apparat, eine
fliegende Maschine w\"are.{\grqq}

Welcher Scherz!

Waren die vielfachen Widerspr\"uche nun schon in der Alten Welt sehr
lebhaft, so begreift man leicht, wie sie sich in jenem Theile der
Neuen Welt gestalten mu{\ss}ten, von dem die Vereinigten Staaten
da{\s} weitau{\s} gr\"o{\ss}te Gebiet einnehmen.

Ein Yankee liebt bekanntlich keine Umwege -- er w\"ahlt gew\"ohnlich
den, der am schnellsten zum Ziele f\"uhrt. So z\"ogerten auch die
amerikanischen Bunde{\s}staaten nicht im mindesten, ihre Ansichten
gegenseitig au{\s}zusprechen. Wenn sie sich dabei nicht gleich die
Objective ihrer Fernrohre an den Kopf warfen, so kam da{\s} nur
daher, da{\ss} sie dieselben jetzt, wo sie gerade am meisten
gebraucht wurden, erst h\"atten wieder ersetzen m\"ussen.

In dieser so viel Staub aufwirbelnden Frage standen die Sternwarten
von Washington im District Columbia und die von Cambridge im Staate
Duna denen de{\s} Darmouth-Colleg{\s} in Connecticut und von
Ann-Arbor in Michigan feindlich gegen\"uber. Ihr Streit betraf
\"ubrigen{\s} nicht die Natur de{\s} beobachteten K\"orper{\s},
sondern die genaue Zeit der Beobachtung, denn Alle behaupteten, ihn
in derselben Nacht, zu derselben Stunde, zur gleichen Minute und
Secunde wahrgenommen zu haben, obwohl die Flugbahn de{\s}
geheimni{\ss}vollen Wanderer{\s} der L\"ufte nur in m\"a{\ss}iger
H\"ohe \"uber dem Horizont liegen sollte. Von Connecticut bi{\s}
Michigan, von Duna nach Columbia ist aber die Entfernung eine so
gro{\ss}e, da{\ss} eine doppelte Beobachtung zu ein und demselben
Zeitpunkt al{\s} unm\"oglich angesehen werden konnte.

Dudley in Albany, Staat New-York, und West-Point, die
Milit\"arakademie, gaben allen ihren Collegen Unrecht in einer
Zuschrift, welche die gerade Aufsteigung und die Declination de{\s}
bewu{\ss}ten K\"orper{\s} bestimmte.

Sp\"ater stellte sich jedoch herau{\s}, da{\ss} diese Beobachter
einem Irrthume unterlegen waren und da{\ss} der betreffende K\"orper
nur eine Feuerkugel gewesen war, welche durch die mittleren
Luftschichten hinblitzte. Um diese Feuerkugel handelte e{\s} sich
aber offenbar nicht. Wie k\"onnte auch eine solche Feuerkugel eine
Trompete geblasen haben?

Wa{\s} nun die erw\"ahnte Trompete anging, versuchte man vergeblich
deren schmetternden Ton al{\s} eine einfache Geh\"or{\s}t\"auschung
hinzustellen. Jedenfall{\s} hatten sich bei dieser Gelegenheit die
Ohren der Leute ebenso wenig get\"auscht, wie deren Augen.
Unz\"ahlige Beobachter hatten vielmehr entschieden etwa{\s} gesehen
und gleichzeitig geh\"ort. In der sehr dunklen Nacht -- vom 12. zum
13. Mai -- war e{\s} den Beobachtern de{\s} Yale-Colleg{\s} an der
Hochschule von Sheffield sogar gelungen, einige Tacte eine{\s}
musikalischen Satze{\s} in \begin{antiqua}A-dur\end{antiqua} und im
Viervierteltacte in Noten zu fixiren, welche vollkommen mit einem
Theile der Melodie de{\s} bekannten \begin{antiqua}Chant du
d\'epart\end{antiqua} -- eine{\s} Soldatenliede{\s} beim Au{\s}zug
zum Kampfe -- \"ubereinstimmten.

{\glqq}Sehr sch\"on! riefen dazu die Witzbolde, da h\"atten wir ja
ein franz\"osische{\s} Orchester, da{\s} seine Weisen mitten in der
Luft ert\"onen l\"a{\ss}t!{\grqq}

Scherzen hei{\ss}t aber nicht antworten. Diese Bemerkung machte auch
da{\s} von der Atlantic Iron Work{\s} Company gegr\"undete
Observatorium zu Boston, dessen Anschauungen in Fragen der Astronomie
und Meteorologie f\"ur die gelehrte Welt allm\"ahlich schon die
Bedeutung von Gesetzen gewannen.

Ferner gab auch noch da{\s}, Dank der Freigebigkeit de{\s} Mr.
Kilgoor im Jahre 1870 auf dem Berge Lookout entstandene Observatorium
von Cincinnati eine Erkl\"arung ab, jene{\s} Institut, da{\s} sich
durch seine mikrometrischen Messungen der Doppelsterne so
vortheilhaft bekannt gemacht hat. Sein Director sprach sich in vollem
guten Glauben dahin au{\s}, da{\ss} den weitverbreiteten Ger\"uchten
unzweifelhaft etwa{\s} zu Grunde liege, da{\ss} sich zu nahe
aneinanderliegenden Zeiten an sehr verschiedenen Stellen in der
Atmosph\"are ein in Bewegung befindlicher K\"orper zeige, da{\ss}
\"uber dessen Natur, Gr\"o{\ss}enverh\"altnisse, Geschwindigkeit und
Flugbahn aber kein Urtheil m\"oglich sei.

Da erhielt ein Journal von allergr\"o{\ss}ter Verbreitung, der
New-York Herald, von einem Abonnenten folgende anonyme Mittheilung:
{\glqq}Noch d\"urfte der Wettkampf unvergessen sein, der vor einigen
Jahren herrschte zwischen den beiden Erben der Begum von Ragginahra,
dem franz\"osischen Arzt Sarrasin in seiner Stadt Franceville und dem
deutschen Ingenieur Herrn Schulze in seiner Stadt Stahlstadt, welche
Beide im s\"udlichen Theile von Oregon, Vereinigte Staaten, angelegt
waren.

{\glqq}Man kann auch nicht vergessen haben, da{\ss} Herr Schulze in
der Absicht, Franceville zu zerst\"oren, ein ungeheure{\s}
Gescho{\ss}, schon mehr eine Maschine, auf letztere Stadt
schleuderte, welche dieselbe mit einem Schlage vernichten sollte.

{\glqq}Noch weniger kann der Vergessenheit verfallen sein, da{\ss}
diese{\s} Gescho{\ss}, dessen Anfang{\s}\-ge\-schwin\-dig\-keit
beim Verlassen der M\"undung der Monstrekanone falsch berechnet war,
mit einer sechzehnmal gr\"o{\ss}eren Geschwindigkeit, al{\s}
gew\"ohnliche Geschosse -- n\"amlich f\"unfundsiebzig bi{\s} acht\/zig
geographische Meilen in der Stunde -- hinweg getragen wurde, da{\ss}
e{\s} auf die Erde nicht niedergefallen ist und nach seinem Uebergang
in den Zustand etwa einer Feuerkugel noch jetzt um unseren Planeten
kreist und in alle Ewigkeit kreisen mu{\ss}.

{\glqq}Warum sollte diese{\s} Riesengescho{\ss}, dessen Vorhandensein
nicht anzuzweifeln ist, nicht der in Frage stehende K\"orper
sein?{\grqq}

Da{\s} war ja recht scharfsinnig von dem Abonnenten de{\s} New-York
Herald ... aber die Trompete~...? In dem Projectil de{\s} Herrn
Schulze hatte sich bestimmt keine Trompete befunden.

Alle bi{\s}herigen Erkl\"arungen erkl\"arten also nicht{\s}, alle
Beobachter beobachteten einfach falsch.

E{\s} blieb sonach nur noch die von dem Director von Zi-Ka-Wey
aufgestellte Hypothese. Aber, mein Gott, der Mann war ja Chinese!

Man darf nicht etwa glauben, da{\ss} sich der Bev\"olkerung der Alten
und der Neuen Welt endlich ein gewisser Ueberdru{\ss} bem\"achtigt
h\"atte. Im Gegentheil, die Er\"orterungen dauerten in gleicher
Lebhaftigkeit fort, ohne da{\ss} irgendwo eine Uebereinstimmung
erzielt wurde. Gleichwohl trat einmal eine Art Pause ein. E{\s}
vergingen n\"amlich einige Tage, ohne da{\ss} etwa{\s} von dem
fraglichen Gegenstande, von der Feuerkugel oder wa{\s} e{\s} sonst
war, gemeldet wurde und ohne da{\ss} sich der bekannte Trompetenton
au{\s} der Luft h\"oren lie{\ss}. War jener K\"orper also irgendwo
auf die Erde niedergefallen, vielleicht an einem Punkte, der sein
Wiederauf\/finden besonder{\s} erschwerte -- etwa gar in'{\s} Meer? Lag
er jetzt in der unendlichen Tiefe de{\s} Atlantischen, de{\s}
Pacifischen oder de{\s} Indischen Ocean{\s}? Wer h\"atte da{\s} sagen
k\"onnen?

Da vollzog sich aber zwischen dem 2. und dem 9. Juni eine neue Reihe
von Thatsachen, deren Erkl\"arung durch die Annahme eine{\s} rein
ko{\s}mischen Ph\"anomen{\s} schlechterding{\s} unm\"oglich war.

Im Laufe jener acht Tage fand man n\"amlich auf den entlegensten
Punkten eine Fahne gerade an den schwerst zug\"anglichen Stellen von
Kirchen u.~s.~w. befestigt; so wurden die Hamburger \"uberrascht durch
eine solche an der Spitze de{\s} Thurme{\s} von St.~Michael, die
T\"urken auf dem h\"ochsten Minaret der heiligen Sophien-Moschee, die
Einwohner von Rouen an der Spitze de{\s} metallenen Pfeile{\s} ihrer
Kathedrale, die Stra{\ss}burger am obersten Punkte de{\s}
M\"unster{\s}, die Amerikaner auf dem Kopfe ihrer Bilds\"aule der
Freiheit am Eingange de{\s} Hafen{\s} und am Gipfel de{\s}
Washington-Denkmal{\s} in Boston, die Chinesen an der Spitze de{\s}
Tempel{\s} der f\"unfhundert Geister in Canton, die Hindu{\s} am
sechzehnten Stockwerk der Pyramide de{\s} Tempel{\s} zu Tanjur, die
R\"omer am Kreuze de{\s} St.~Peter{\s}-Dome{\s}, die Engl\"ander am
Kreuz der St.~Paul{\s}-Kirche in London, die Egypter an der obersten
Spitze der Pyramide von Gizeh, die Wiener an dem Reich{\s}adler auf
der Spitze de{\s} St.~Stephan{\s}thurme{\s}, die Pariser am
Blitzableiter de{\s} dreihundert Meter hohen eisernen Thurme{\s} der
Au{\s}stellung von 1889 und noch andere mehr.

Diese Fahne aber zeigte ein schwarze{\s} Flaggentuch, da{\s} in der
Mitte eine goldene Sonne und ring{\s}um verstreut einzelne Sterne
enthielt.



\newpage\begin{center}\label{kap02}
{\large \begin{antiqua}II.\end{antiqua}\\\medskip
In welchem die Mitglieder de{\s} Weldon-Institut{\s} mit einander
streiten, ohne zu einer Uebereinstimmung zu gelangen.\\\bigskip}
\end{center}



{\glqq}Und der Erste, der da{\s} Gegentheil behauptet~...

-- Oho, da{\s} wird man behaupten, wenn ein Grund daf\"ur vorliegt!

-- Und auch trotz Ihrer Drohungen!~...

-- Achten Sie auf Ihre Worte, Bat Fyn!

-- Und Sie auf die Ihrigen, Onkel Prudent!

-- Ich bleibe dabei, da{\ss} sich die Schraube nur am Hintertheil
befinden darf!

-- Wir auch! Wir auch! erschallten f\"unfzig Stimmen wie au{\s} einer
Kehle.

-- Sie mu{\ss} am Vordertheil sein! rief Phil Evan{\s}.

-- Am Vordertheil! br\"ullten f\"unfzig andere Stimmen eben so stark,
wie jene fr\"uheren.

-- Wir werden nie zu ein und derselben Ansicht kommen!

-- Niemal{\s}! ... Niemal{\s}!

-- Nun, warum streiten wir dann \"uberhaupt noch?

-- Da{\s} ist kein Streit ... e{\s} ist nur eine Er\"orterung!{\grqq}

Da{\s} h\"atte freilich kein Mensch geglaubt, der die scharfe
Entgegnung, die Vorw\"urfe und da{\s} Geschrei h\"orte, welche den
Sitzung{\s}saal seit einer guten Viertelstunde erf\"ullten.

Gedachter Saal war n\"amlich der gr\"o{\ss}te de{\s}
Weldon-Institute{\s} ... und jene{\s} vor allen ber\"uhmten Club{\s}
in der Walnut Street zu Philadelphia, Pennsylvanien, Vereinigte
Staaten von Nordamerika.

In genannter Stadt war e{\s} erst am Vortage bei Gelegenheit der Wahl
eine{\s} Ga{\s}laternenanz\"under{\s} zu \"offentlichen Kundgebungen,
ger\"auschvollen Versammlungen und zu reichlich au{\s}getheilten
Schl\"agen gekommen. Daher r\"uhrte eine noch nicht bes\"anftigte
Reizbarkeit und stammte wohl auch jene au{\ss}ergew\"ohnliche
Erregung, welche die Mitglieder de{\s} Weldon-Institut{\s} eben
zeigten. Und hierbei handelte e{\s} sich nur um eine einfache
Vereinigung von {\glqq}Ballonisten{\grqq}, welche \"uber die noch
heutigen Tage{\s} brennende Frage der Lenkbarkeit der Ballon{\s}
verhandelten.

Der Vorgang aber spielte sich in einer Stadt der Vereinigten Staaten
ab, welche an schneller Entwickelung selbst New-York, Chicago,
Cincinnati und San Franci{\s}co \"uberholt hat -- einer Stadt, welche
weder ein Hafenplatz, noch der Mittelpunkt von Petroleum- oder
Steinkohlenbergwerken, auch kein Brennpunkt der Industrie, so wenig
wie der Kreuzung{\s}punkt eine{\s} vielstrahligen Bahnnetze{\s} ist
-- in einer Stadt, die an Gr\"o{\ss}e schon Manchester, Edinburgh,
Liverpool, Wien, Peter{\s}burg und Dublin \"ubertrifft -- einer
Stadt, die einen Park besitzt, in dem die sieben Park{\s} der
Hauptstadt von England zusammen Platz finden -- einer Stadt endlich,
welche jetzt nahezu 1,200.000 Einwohner z\"ahlt und sich nach London,
Pari{\s}, New-York und Berlin al{\s} die f\"unfte Stadt der Welt
betrachtet.

Philadelphia ist fast eine Stadt au{\s} Marmor mit seinen vielen
monumentalen Geb\"auden und \"offentlichen Anstalten, welche ihre{\s}
Gleichen nirgend{\s} finden. Da{\s} bedeutendste aller Colleg{\s} der
Neuen Welt ist da{\s} Colleg Girard, und da{\s} hat seinen Sitz in
Philadelphia. Die gr\"o{\ss}te Eisenbr\"ucke der Erde ist die, welche
den Schuylkill \"uberspannt, und diese befindet sich in Philadelphia.
Der sch\"onste Tempel der Freimaurerei ist der Maurertempel in
Philadelphia; endlich besteht der gr\"o{\ss}te Club von Freunden und
Bef\"orderern der Luftschifffahrt ebenfall{\s} in Philadelphia, und
wer Gelegenheit gehabt h\"atte, diesen am Abend de{\s} 12. Juni zu
besuchen, der w\"urde sich dabei au{\s}gezeichnet unterhalten haben.

In erw\"ahntem gro{\ss}en Saale bewegten, dr\"angten sich,
gestikulirten, sprachen, verhandelten und stritten -- Alle den Hut
auf dem Kopfe -- wohl hundert Ballonisten unter dem hohen Vorsitz
eine{\s} Pr\"asidenten, dem ein Schriftf\"uhrer und ein Schatzmeister
zur Seite standen. E{\s} waren da{\s} keine Ingenieur{\s} von Fach;
nein, einfache Liebhaber alle{\s} Dessen, wa{\s} mit der Aerostatik
in Beziehung stand, aber begeisterte Liebhaber, und vor Allem Feinde
Derjenigen, welche den Aerostaten Apparate, {\glqq}schwerer al{\s}
die Luft{\grqq}, fliegende Maschinen, Luftschiffe u. dgl.
entgegenzustellen beabsichtigen. Da{\ss} diese wackeren Leute
nimmermehr die Lenkbarkeit de{\s} Ballon{\s} erfinden w\"urden, war
gewi{\ss} mehr al{\s} wahrscheinlich. Auf jeden Fall hatte ihr
Vorsitzender Noth genug, um sie selbst geh\"orig zu lenken und zu
leiten.

Dieser in Philadelphia sattsam bekannte Pr\"asident war der Onkel
Prudent -- Prudent seinem Familiennamen nach. Wa{\s} die weitere
Bezeichnung {\glqq}Onkel{\grqq} betrifft, so braucht man sich in
Amerika \"uber diese nicht zu wundern, wo Jeder zum Onkel werden
kann, ohne einen Neffen oder eine Nichte zu haben. Man sagt dort
ebenso Onkel, wie anderw\"art{\s} Vater von Leuten, welche auf eine
Vaterschaft nicht den geringsten Anspruch haben.

Onkel Prudent war eine gewichtige Pers\"onlichkeit und trotz
seine{\s} Namen{\s} oft genannt gerade wegen seiner K\"uhnheit,
daneben sehr reich, wa{\s} selbst in den Vereinigten Staaten nicht
von Nachtheil sein soll. Wie h\"atte er da{\s} auch nicht sein
sollen, da er einen gro{\ss}en Theil der Niagarafall-Actien sein
eigen nannte? Jener Zeit hatte sich n\"amlich in Buffalo eine
Gesellschaft von Ingenieuren zur Au{\s}beutung der ber\"uhmten
F\"alle gegr\"undet. Die 7500 Cubikmeter, welche der Niagara jede
Secunde hinabw\"alzt, k\"onnen 7 Millionen Dampfpferdekr\"afte
erzeugen. Diese ungeheure, in einem Umkreise von 500 Kilometer nach
allen Fabriken und Werkst\"atten vertheilte Kraftmenge lieferte eine
j\"ahrliche Ersparni{\ss} von 1200 Millionen Mark, von dem ein Theil
in die Cassen der Gesellschaft -- speciell in die Taschen de{\s}
Onkel Prudent -- zur\"uckflo{\ss}. Uebrigen{\s} war er Junggeselle,
lebte h\"ochst einfach und hatte al{\s} h\"au{\s}lichen
pers\"onlichen Beistand niemand Anderen, al{\s} seinen Diener
Frycollin, der eigentlich am allerwenigsten verdiente, im Dienste
eine{\s} so k\"uhnen, unternehmenden Herrn zu stehen. Aber e{\s}
giebt einmal Regelwidrigkeiten.

Da{\ss} der Onkel Prudent Freunde hatte, da er so reich war, versteht
sich ja von selbst; aber er hatte auch Feinde, weil er Vorsitzender
jene{\s} Club{\s} war -- unter Allen alle die, welche selbst nach
diesem Amte strebten; und al{\s} der hitzigsten Einer ist hier der
Schriftf\"uhrer de{\s} Weldon-Institute{\s} zu erw\"ahnen.

E{\s} war da{\s} der ebenfall{\s} sehr reiche Phil Evan{\s}, der
Director der Walton Watch Company, einer gewaltigen Uhrenfabrik,
welche tagt\"aglich 500 St\"uck Zeitmesser erzeugt und Producte
liefert, die sich den besten der Schweiz an die Seite stellen
k\"onnen. Phil Evan{\s} h\"atte also f\"ur einen der gl\"ucklichen
Menschen der Welt selbst in den Vereinigten Staaten gelten k\"onnen,
wenn man von jener Stellung de{\s} Onkel Prudent absah. Wie
letzterer, war auch er 45 Jahre alt, von scheinbar
unersch\"utterlicher Gesundheit, wie jener von unzweifelhafter
K\"uhnheit, und sorgte er sich wenig darum, die gewissen Vorz\"uge
de{\s} Junggesellenstande{\s} gegen die oft zweifelhaften Vortheile
der Ehe zu vertauschen. Wahrlich, da{\s} waren zwei M\"anner, wie
geschaffen, einander zu verstehen, die sich doch nicht verstanden,
und Beide, wa{\s} wohl zu bemerken ist, von ungemein stark
entwickeltem Charakter, der Eine, Onkel Prudent, hitzig, der Andere,
Phil Evan{\s}, ei{\s}kalt bi{\s} zum Ueberma{\ss}e.

Und woher kam e{\s}, da{\ss} Phil Evan{\s} nicht zum Vorsitzenden
de{\s} Club{\s} ernannt worden war? Die Stimmenzahl f\"ur Onkel
Prudent und f\"ur ihn war die genau gleiche gewesen. Wohl zwanzig Mal
wurde die Abstimmung wiederholt, aber auch zwanzig Mal ergab sich
eine Majorit\"at weder f\"ur den Einen, noch f\"ur den Anderen.
Da{\s} war eine peinliche Lage, welche wahrscheinlich die
Leben{\s}zeit der beiden Candidaten h\"atte \"uberdauern k\"onnen.

Da schlug ein Mitglied de{\s} Club{\s} ein Mittel vor, die
Stimmengleichheit aufzuheben. E{\s} war Jem Cip, der Schatzmeister
de{\s} Weldon-Institute{\s}. Jem Cip war eingefleischter
Vegetarianer, mit anderen Worten, au{\s}schlie{\ss}licher
Gem\"useesser, einer der Leute, die jede Fleischnahrung, wie alle
gegohrenen Getr\"anke verwarfen -- halb Brahmanen und halb
Muselm\"anner -- der Rival eine{\s} Nievmann, Pitmann, Ward und
Davie, welche der Secte dieser unschuldigen Thoren einen gewissen
Namen gemacht haben.

Bei vorliegender Gelegenheit wurde Jem Cip von einem anderen Mitglied
de{\s} Club{\s} unterst\"utzt, von William T. Forbe{\s}, dem Director
einer gro{\ss}en Anstalt, in der Glucose durch Behandlung von Lumpen
mit Schwefels\"aure hergestellt wurde -- ein Verfahren, nach dem man
also Zucker au{\s} alter W\"asche zu erzeugen vermag. E{\s} war ein
gut situirter Mann, dieser William T. Forbe{\s}, und Vater von zwei
reizenden, bejahrteren T\"ochtern, der Mi{\ss} Dorothee, genannt
Doll, und der Mi{\ss} Martha, genannt Mat, die in der besten
Gesellschaft von Philadelphia den Ton angaben.

Der von William T. Forbe{\s} nebst einigen Anderen unterst\"utzte
Vorschlag Jem Cip'{\s} ging nun dahin, den Vorsitzenden de{\s}
Club{\s} durch den Mittelpunkt zu bestimmen.

Wahrlich, dieser Wahlmodu{\s} k\"onnte in allen F\"allen angewendet
werden, wo e{\s} sich darum handelt, den W\"urdigsten zu erw\"ahlen,
und sehr viele, h\"ochst vern\"unftige Amerikaner dachten auch schon
daran, denselben bei der Ernennung de{\s} Pr\"asidenten der
Vereinigten Staaten zur Anwendung zu bringen.

Auf zwei tadello{\s} wei{\ss}e Tafeln wurde hierzu je eine schwarze
Linie gezogen. Die L\"ange beider war mathematisch genau die gleiche,
denn man hatte dieselbe mit ebenso viel Sorgfalt abgemessen, al{\s}
handelte e{\s} sich dabei um die Grundlinien de{\s} ersten
Dreieck{\s} einer Triangulation{\s}arbeit. Hierauf wurden beide
Tafeln am n\"amlichen Tage inmitten de{\s} Sitzung{\s}saale{\s} der
Gesellschaft aufgestellt; die beiden Wettbewerber versahen sich Jeder
mit einer sehr feinspitzigen Nadel und gingen wieder gleichzeitig auf
die, Jedem durch da{\s} Loo{\s} zugefallene Tafel zu. Derjenige der
beiden Rivalen aber, welcher seine Nadel am n\"achsten dem
Mittelpunkte der Linie einstechen w\"urde, sollte damit zum
Vorsitzenden de{\s} Weldon-Institute{\s} gew\"ahlt sein.

E{\s} versteht sich von selbst, da{\ss} hierbei jede{\s} Hilf{\s}mittel,
jede{\s} Umhertappen verboten und nur die Sicherheit de{\s} Blick{\s}
entscheidend war. E{\s} galt, nach volk{\s}th\"umlichem Au{\s}druck,
den Zirkel im Auge zu haben.

Onkel Prudent stach seine Nadel ein und zu gleicher Zeit Phil
Evan{\s}. Darauf wurde nachgemessen, welcher der beiden Konkurrenten
sich dem Mittelpunkte am meisten gen\"ahert hatte.

Welche{\s} Wunder! Die beiden M\"anner hatten so vortreffliche{\s}
Augenma{\ss} entwickelt, da{\ss} die Messungen keinen
sch\"atzen{\s}werthen Unterschied ergaben. War von ihnen auch nicht
genau der mathematische Mittelpunkt getroffen worden, so erwie{\s}
sich der Raum zwischen diesem und den beiden Nadeln kaum merkbar und
schien bei beiden obendrein noch gleich gro{\ss} zu sein.

Die Versammlung befand sich nun in neuer Verlegenheit.

Zum Gl\"uck bestand eine{\s} der Mitglieder, Truk Milnor, darauf, die
Messungen mit Hilfe eine{\s} mit Perreaux' mikrometischer Maschine
getheilten Lineal{\s} noch einmal vorzunehmen, welche die
M\"oglichkeit gew\"ahrt noch ein F\"unfzehnhundertstel eine{\s}
Millimeter{\s} abzulesen. Auf dem Lineal waren in der That
f\"unfzehnhundert Abtheilungen auf einem solchen kleinen Raum
mittelst Diamant eingeritzt, und bei Abmessung der Entfernung der
Stiche von den betreffenden Mittelpunkten erhielt man folgende{\s}
Resultat:

Onkel Prudent hatte sich dem Mittelpunkt auf weniger al{\s} sech{\s}
f\"unfzehnhundertstel Millimeter gen\"ahert, Phil Evan{\s} auf nahezu
neun f\"unfzehnhundertstel.

Daher kam e{\s}, da{\ss} Phil Evan{\s} nur Schriftf\"uhrer de{\s}
Weldon-Institute{\s} wurde, w\"ahrend Onkel Prudent die W\"urde
de{\s} Pr\"asidenten de{\s}selben erhielt.

Einer Entfernung von drei f\"unfzehnhundertstel, mehr hatte e{\s}
nicht bedurft, um Phil Evan{\s} mit Ha{\ss} gegen Onkel Prudent zu
erf\"ullen, mit einem Ha{\ss}, der, wenn er ihn auch in sich
verschlo{\ss}, doch nicht minder grimmig war.

Jener Zeit, und zwar seit dem letzten Viertel diese{\s} neunzehnten
Jahrhundert{\s}, hatte die Frage der lenkbaren Ballon{\s} immerhin
schon einige Fortschritte zu verzeichnen, die mit Triebschraube
au{\s}ger\"usteten Gondeln, welche Henry Giffard 1852 an seinem
verl\"angerten Ballon anbrachte, ferner Dupuy de L\^ome, 1872, die
Gebr\"uder Tissandier 1883 und die Capit\"ane Kreb{\s} und Renard im
Jahre 1884 hatten mindesten{\s} einige Ergebnisse erzielt, denen man
Rechnung tragen mu{\ss}te.

Doch wenn diese Apparate in einem schwereren Medium al{\s} sie
selbst, unter dem Drucke einer Schraube man\"ovrirend, eine schr\"age
Richtung gegen den Wind einhielten, sogar gegen einen widrigen
Luftzug aufkamen, um nach ihrem Au{\s}gang{\s}punkt zur\"uckzukehren,
also wirklich gelenkt worden waren, so konnte da{\s} doch nur unter
ganz besonder{\s} g\"unstigen Umst\"anden erreicht werden. In
gro{\ss}en, geschlossenen au{\s}gedehnten Hallen allerding{\s}! In
recht ruhiger Atmosph\"are -- da{\s} ging auch noch recht gut. Bei
einem leichten Winde von f\"unf bi{\s} sech{\s} Meter in der Secunde
war e{\s} vielleicht eben noch zu erzwingen -- Alle{\s} in Allem
hatte man eigentlich praktisch verwendbare Resultate aber noch nicht
erzielt. Gegen einen Windm\"uhlenwind von acht Metern in der Secunde
w\"urden jene Apparate nahezu station\"ar geblieben sein; vor einer
frischen Brise von zehn Metern in der Secunde hatten sie in Gefahr
geschwebt, zerrissen zu werden; und bei einer jener Cyclonen, welche
hundert Meter in der Secunde \"uberschreiten, w\"urde man von ihnen
kein St\"uckchen wieder gefunden haben.

Selbst nach den scheinbar gl\"anzend gelungenen Versuchen der
Capit\"ane Kreb{\s} und Renard d\"urfte al{\s} bewiesen angesehen
werden, da{\ss} die Aerostaten, wenn sie an Bewegung{\s}f\"ahigkeit
auch ein wenig gewonnen hatten, mit dieser doch gerade nur gegen eine
schwache Brise aufzukommen vermochten. E{\s} war also nach wie vor
al{\s} unm\"oglich zu betrachten, diese Art der Fortbewegung durch
die Luft praktisch zu verwenden.

W\"ahrend man sich aber so eifrig mit dem Problem der Lenkbarkeit der
Aerostaten, da{\s} hei{\ss}t mit den Mitteln besch\"aftigte, diesen
eine eigene Geschwindigkeit zu verleihen, hatte die Frage der Motoren
unzweifelhaft weit schnellere Fortschritte gemacht. An Stelle der
Dampfmaschinen und der Verwendung der blo{\ss}en Mu{\s}kelkraft waren
allm\"ahlich die elektrischen Motore getreten. Die Batterien mit
doppeltchromsaurem Natron, deren Elemente auf hohe Spannung
angeordnet waren, wie sie die Gebr\"uder Tissandier ben\"utzten,
erzielten eine Schnelligkeit von etwa vier Metern in der Secunde. Die
zw\"olf Pferdekraft entwickelnden dynamo-elektrischen Maschinen der
Capit\"ane Kreb{\s} und Renard gestatteten, eine Geschwindigkeit von
im Mittel sech{\s} Meter in der Secunde zu erreichen.

Bei ihren Versuchen waren Mechaniker und Elektriker bestrebt gewesen,
sich dem frommen Wunsche zu n\"ahern, eine {\glqq}Dampfpferdekraft in
einem Taschenuhrgeh\"ause{\grqq} zu erzeugen. Die Effecte der
S\"aule, deren Zusammensetzung die Capit\"ane Kreb{\s} und Renard
geheim gehalten hatten, wurden ebenfall{\s} bald \"ubertroffen, und
nach ihnen fanden die Aeronauten Gelegenheit, Motore zu verwenden,
deren Leichtigkeit im gleichen Verh\"altni{\ss} mit ihrer
Kraftwirkung wuch{\s}.

Die Anh\"anger der M\"oglichkeit einer Lenkbarkeit der Ballon{\s}
hatten also gewi{\ss} Ursache, ihren Muth aufrecht zu erhalten, und
doch, wie viele klare K\"opfe haben e{\s} verworfen, an die
Ben\"utzung solcher zu glauben. In der That, wenn der Aerostat einen
Angriff{\s}punkt der ihm innewohnenden Kraft in der Luft findet, so
ist er doch mit seiner gro{\ss}en Masse in diese eingetaucht. Und wie
k\"onnte derselbe, da er wieder den Str\"omungen der Atmosph\"are
eine so breite Angriff{\s}fl\"ache bietet, jemal{\s}, und wenn sein
Triebwerk noch so m\"achtig w\"are, direct gegen einen widrigen Wind
aufkommen?

Diese Frage lag noch immer vor, man hoffte dieselbe jedoch durch
Anwendung sehr gro{\ss}er Apparate zu l\"osen.

E{\s} ergab sich \"ubrigen{\s}, da{\ss} bei diesem Wettstreite der
Erfinder in der Herstellung eine{\s} sehr kr\"aftigen und dennoch
leichten Motor{\s} die Amerikaner sich dem gew\"unschten Ziele am
meisten gen\"ahert hatten. Ein auf der Anwendung einer neuen S\"aule
beruhender dynamo-elektrischer Apparat, dessen Construction
vorl\"aufig noch Geheimni{\ss} blieb, war seinem Erfinder, einem
bi{\s}her unbekannten Chemiker in Boston, abgekauft worden. Mit
gr\"o{\ss}ter Sorgfalt durchgef\"uhrte Berechnungen und mit
\"au{\ss}erster Genauigkeit entworfene Diagramme ergaben, da{\ss}
dieser Apparat, wenn er auf eine Schraube von angepa{\ss}ter
Gr\"o{\ss}e wirkte, eine Fortbewegung von acht\/zehn bi{\s} zwanzig
Metern in der Secunde gew\"ahrleisten mu{\ss}te.

Wahrlich, da{\s} w\"are gro{\ss}artig gewesen!

{\glqq}Und da{\s} Ding ist nicht theuer!{\grqq} hatte Onkel Prudent
hinzu gesetzt, al{\s} er dem Erfinder gegen regelrecht
au{\s}gef\"ullte Quittung da{\s} letzte P\"ackchen von hunderttausend
Papierdollar{\s} einh\"andigte, mit dem man ihm seine Erfindung
bezahlte.

Unverz\"uglich ging da{\s} Weldon-Institut an'{\s} Werk. Handelt
e{\s} sich um ein Versuch{\s}unternehmen, da{\s} irgend welchen
praktischen Nutzen verspricht, so wird da{\s} Geld in amerikanischen
Taschen stet{\s} leicht locker. Die n\"othigen Mittel str\"omten
zusammen, so da{\ss} selbst die Gr\"undung einer Actiengesellschaft
umgangen werden konnte. Dreihunderttausend Dollar{\s} (also 600.000
fl. = 1$\frac{1}{5}$ Millionen Mark) f\"ullten gleich nach dem ersten Aufruf
die Cassen de{\s} Club{\s}. Die Arbeiten begannen unter Leitung
de{\s} hervorragendsten Luftschiffer{\s} der Vereinigten Staaten,
Harry W. Tinder'{\s}, der sich unter tausend Anderen vorz\"uglich
durch drei k\"uhne Fahrten ber\"uhmt gemacht hat: die eine, bei der
er sich bi{\s} 1200 Meter erhob, d.~h. h\"oher aufstieg, al{\s}
Gay-Lussac, Coxwell, Sivel, Croc\'e-Spinelli, Tissandier, Glaisher;
die zweite, w\"ahrend der er ganz Amerika von New-York bi{\s} San
Franci{\s}co \"uberflog und um mehrere hundert Lieue{\s} die
l\"angste Reise Nadar'{\s}, Godard'{\s} und vieler Anderen hinter
sich lie{\ss}, ohne John Wise zu rechnen, der von St.~Loui{\s} bi{\s}
nach der Grafschaft Jefferson elfhundertf\"unfzig Meilen zur\"uckgelegt
hatte; die dritte endlich, welche mit einem furchtbaren Sturze au{\s}
der H\"ohe von f\"unfzehnhundert Fu{\ss} endigte, bei dem er sich
doch nur den rechten Daumen verstauchte, w\"ahrend der minder vom
Gl\"ucke beg\"unstigte Pil\^atre de Rozier bei einem Sturze von nur
siebenhundert Fu{\ss} augenblicklich den Tod fand.

Zur Zeit, mit der diese Erz\"ahlung beginnt, konnte man schon
beurtheilen, da{\ss} da{\s} Weldon-Institut die Angelegenheit
kr\"aftig gef\"ordert hatte. In den Turner-Werften zu Philadelphia
erhob sich schon ein ungeheurer Aerostat, dessen Haltbarkeit durch
F\"ullung mit stark comprimirter Luft gepr\"uft werden sollte. Vor
Allem w\"urde dieser den Namen eine{\s} Monstre-Ballon{\s} verdienen.

Wie viel fa{\ss}te der G\'eant Nadar'{\s}? Sech{\s}tausend
Cubikmeter. Wie viel der Ballon John Wise'{\s}? Zwanzigtausend
Cubikmeter. Welchen Fassung{\s}raum hatte der Ballon Giffard auf der
Au{\s}stellung von 1878? F\"unfundzwanzigtausend Cubikmeter bei
acht\/zehn Meter Halbmesser. Vergleicht man diese drei Aerostaten mit
dem de{\s} Weldon-Institute{\s}, dessen Volumen vierzigtausend
Cubikmeter betrug, so begreift man leicht, da{\ss} Onkel Prudent und
seine Clubgenossen einigerma{\ss}en Recht hatten, sich vor Stolz
aufzubl\"ahen.

Dieser Ballon, der nicht dazu bestimmt war, die h\"ochsten Schichten
der Atmosph\"are zu erreichen, nannte sich nicht
{\glqq}Excelsior{\grqq}, eine Bezeichnung, welche sonst bei den
Amerikanern sehr beliebt ist, nein, er war einfach \begin{antiqua}Go
a head\end{antiqua}, d.~h. {\glqq}Vorw\"art{\s}{\grqq} getauft, und
e{\s} er\"ubrigte also nur noch, da{\ss} er seinen Namen
rechtfertigte, indem er der Leitung seine{\s} Capit\"an{\s}
allenthalben entsprach.

Jener Zeit war die dynamo-elektrische Maschine nach dem vom
Weldon-Institute angekauften Patente fast vollendet und man durfte
darauf rechnen, da{\ss} der \begin{antiqua}Go a head\end{antiqua} 
seinen Flug durch da{\s} Luftmeer begonnen haben werde.

Immerhin waren bekanntlich alle mechanischen Schwierigkeiten noch
nicht \"uberwunden.

Sehr viele Sitzungen waren zu diesem Zwecke abgehalten worden, nicht
etwa die Form der Schraube oder deren Gr\"o{\ss}enverh\"altnisse
fest\/zustellen, sondern um die Frage zu entscheiden, ob dieselbe am
Hintertheil de{\s} gro{\ss}en Apparate{\s} angebracht werden sollte,
wie die Gebr\"uder Tissandier wollten, oder am Vordertheile, wie
e{\s} die Capit\"ane Kreb{\s} und Renard schon gethan hatten. E{\s}
bedarf kaum der Erw\"ahnung, da{\ss} die Vertreter dieser beiden
Ansichten bei den bez\"uglichen Verhandlungen dar\"uber fast
handgemein wurden. Die Gruppe der {\glqq}Vorderm\"anner{\grqq} glich
an Zahl genau der der {\glqq}Hinterm\"anner{\grqq}. Onkel Prudent,
dessen Stimme bei sonstiger Stimmengleichheit die entscheidende
gewesen w\"are, Onkel Prudent, der unzweifelhaft au{\s} der Schule
de{\s} Professor{\s} Buridan hervorgegangen war, vermied e{\s}
kl\"uglich, sich zu \"au{\ss}ern.

Bei der Unm\"oglichkeit, ein Einverst\"andni{\ss} herbeizuf\"uhren,
war e{\s} nat\"urlich auch unm\"oglich, die Schraube an Ort und
Stelle zu setzen. Da{\s} konnte demnach lange dauern, wenn sich nicht
etwa die Regierung in'{\s} Mittel legte. In den Vereinigten Staaten
liebt e{\s} die Regierung aber bekanntlich nicht, sich in
Privatangelegenheiten einzumischen oder um da{\s} zu k\"ummern,
wa{\s} sie nicht direct angeht. Damit hat sie gewi{\ss} ganz Recht.

So war die Sachlage, und die Sitzung vom 13. Juni schien gar nicht
endigen oder vielmehr nur in einen ungeheuren Tumult au{\s}laufen zu
wollen -- der wie gew\"ohnlich mit Injurien begann, sich mit
Faustschl\"agen fortsetzte, dann zu Stockschl\"agen \"uberging und
mit dem Knallen der Revolver abschlo{\ss} -- al{\s} ein Zwischenfall
um acht Uhr siebenunddrei{\ss}ig Minuten diesen beliebten Verlauf
st\"orte.

Kalt und gemessen, wie ein Polizist inmitten der st\"urmischen Wogen
einer Volk{\s}versammlung, hatte sich der Th\"ursteher de{\s}
Weldon-Institut{\s} gen\"ahert und dem Vorsitzenden eine Karte
eingeh\"andigt. Er erwartete eben noch die Befehle, welche der Onkel
Prudent ihm zu ertheilen haben k\"onnte.

Onkel Prudent lie{\ss} die Dampftrompete ert\"onen, die ihm al{\s}
Pr\"asidentenglocke diente, denn hier h\"atte, um durchzudringen,
nicht einmal die gro{\ss}e Glocke de{\s} Kreml{\s} hingereicht.
Nicht{\s}destoweniger nahm der L\"arm nur noch zu. Da
{\glqq}entbl\"o{\ss}te der Pr\"asident den Kopf{\grqq} und Dank
diesem allerletzten Hilf{\s}mittel entstand wenigsten{\s} eine
leidliche Ruhe.

{\glqq}Eine Mittheilung an den Club! rief Onkel Prudent, nachdem er
sich eine Prise au{\s} der ungeheuren Dose, die ihn niemal{\s}
verlie{\ss}, zugelangt.

-- Reden Sie! Reden Sie! antworteten neunundneunzig Stimmen, die
hier\"uber zuf\"allig einer Meinung waren.

-- Ein Fremdling, geehrte Collegen, w\"unscht in unseren
Sitzung{\s}saal Eintritt zu erhalten.

-- Nimmermehr! widersetzten sich alle Stimmen.

-- Er w\"unscht un{\s}, fuhr Onkel Prudent fort, allem Anscheine nach
den Bewei{\s} zu liefern, da{\ss} e{\s} der greulichste Wahnwitz sei,
an die Lenkbarkeit von Ballon{\s} zu glauben.{\grqq}

Allgemeine{\s} Murren beantwortete diese Erkl\"arung.

{\glqq}Herein, herein mit ihm!

-- Wie nennt sich denn diese merkw\"urdige Pers\"onlichkeit? fragte
der Schriftf\"uhrer Phil Evan{\s}.

-- Robur, antwortete Onkel Prudent.

-- Robur! ... Robur! ... Robur!{\grqq} heulte die ganze Versammlung.

Und wenn bei Nennung diese{\s} eigenth\"umlichen Namen{\s} der
Tr\"ager de{\s}selben so schnell Zulassung fand, geschah e{\s}
eigentlich nur, weil da{\s} ganze Weldon-Institut sich Hoffnung
machte, auf den Mann den Ueberschu{\ss} seiner Erbitterung
abzusch\"utteln.

\enlargethispage{\baselineskip}

Der Sturm hatte sich also einen Augenblick gelegt -- wenigsten{\s}
scheinbar. Wie k\"onnte \"ubrigen{\s} ein Sturm so schnell
vor\"ubergehen bei einem Volk, welche{\s} jeden Monat zwei bi{\s}
drei solcher nach Europa unter der Form von Wirbelwinden entsendet?



\newpage\begin{center}\label{kap03}
{\large \begin{antiqua}III.\end{antiqua}\\\medskip
In dem eine neue Pers\"onlichkeit nicht besonder{\s} vorgestellt
zu werden braucht, da sie da{\s} selbst besorgt.\\\bigskip}
\end{center}



{\glqq}B\"urger der Vereinigten Staaten, ich hei{\ss}e
Robur\footnote[1]{\frakfamily Zu Deutsch: Die Kraft.} und bin diese{\s}
Namen{\s} w\"urdig. Trotz meiner vierzig Jahre sehe ich au{\s} wie
drei{\ss}ig, habe eine eiserne Constitution, eine unersch\"utterliche
Gesundheit, hervorragende Mu{\s}kelkraft und einen Magen, der selbst
in der Welt der Strau{\ss}e al{\s} vorz\"uglich gelten
w\"urde.{\grqq}

Die Versammlung lauschte. Jede{\s} Ger\"ausch hatte vorl\"aufig
aufgeh\"ort, al{\s} man diese unerwartete Vorrede \begin{antiqua}pro
facie sua\end{antiqua} vernahm. War e{\s} ein Narr oder ein
Sp\"otter, diese Pers\"onlichkeit? Wie dem auch sein mochte, er
machte Eindruck und wu{\ss}te sich diesen zu erzwingen. Jetzt ging
kein Lufthauch durch diese Menge, in der doch kurz vorher ein Orkan
w\"uthete. Die Windstille nach der hohen See.

Ueberdie{\s} schien Robur wirklich der Mann zu sein, f\"ur den er
sich au{\s}gab. Von mittlerer Gr\"o{\ss}e mit geometrischer Gestalt,
ein regelm\"a{\ss}ige{\s} Trapez bildend, deren gr\"o{\ss}te
Parallelseite von der Schulterbreite au{\s}gef\"ullt wurde; auf
dieser Linie sa{\ss} wieder auf einem kr\"aftigen Halse ein
gewaltiger sph\"aroidaler Kopf. Welchem Dickkopfe mochte derselbe zu
vergleichen sein? Dem eine{\s} Stiere{\s}, aber eine{\s} Stiere{\s}
mit hochintelligentem Gesicht. Darin funkelten ein paar Augen, welche
der geringste Widerspruch sicherlich in volle Gluth versetzte, und
\"uber letzteren waren die Augenbrauenmu{\s}keln -- ein Zeichen
entwickelter Energie -- fortw\"ahrend zusammengezogen. Die Haare
de{\s} Manne{\s} waren kurz, etwa{\s} krau{\s} und von metallischem
Glanze, al{\s} tr\"uge er ein Toupet von eisernem Stroh; seine breite
Brust hob und senkte sich mit Bewegungen gleich einem
Schmiedeblasebalg. Arme und H\"ande, Beine und F\"u{\ss}e erwiesen
sich de{\s} Rumpfe{\s} v\"ollig w\"urdig.

Schnurr- und Backenbart sah man bei ihm nicht, nur einen starken
Seemann{\s}-Kinnbart nach amerikanischer Mode, der die Anhaftepunkte
der Kinnlade frei lie{\ss}, deren Kaumu{\s}keln eine furchtbare Kraft
entwickeln mu{\ss}ten. Man hat berechnet -- wa{\s} berechnet man denn
nicht? -- da{\ss} der Druck der Kinnlade de{\s} Krokodil{\s} unter
gew\"ohnlichen Umst\"anden dem von vierhundert Atmosph\"aren gleich
kommt, w\"ahrend der eine{\s} Jagdhunde{\s} von mittlerer Gr\"o{\ss}e
hundert erreichen soll. Darau{\s} hat man auch folgende merkw\"urdige
Formel abgeleitet: wenn ein Kilogramm Hund acht Kilogramm
Mu{\s}kelkraft entwickelt, so entwickelt ein Kilogramm Krokodil deren
zw\"olf. Nun, ein Kilogramm de{\s} genannten Robur h\"atte deren
gewi{\ss} zehn entwickelt. Er hielt also zwischen Hund und Krokodil
in dieser Beziehung die Mitte.

Au{\s} welchem Lande diese{\s} merkw\"urdige Menschenkind stammte,
h\"atte man nur schwer errathen k\"onnen. Jedenfall{\s} dr\"uckte
sich der Mann ganz gel\"aufig englisch au{\s} und ohne jenen
schleppenden Tonfall, der den Yankee von Neu-England unterscheidet.

Er fuhr folgenderma{\ss}en fort:

{\glqq}Nun lassen Sie mich auch von meinen anderen Eigenschaften
sprechen, ehrenwerthe B\"urger. Sie sehen vor sich einen Ingenieur,
dessen geistige Natur seiner k\"orperlichen nicht nachsteht. Ich
f\"urchte mich vor Nicht{\s} und vor Niemand; besitze eine
Willen{\s}kraft, die noch nie vor einem Anderen gewichen ist. Hab'
ich mir einmal ein Ziel gesetzt, so w\"urde ganz Amerika, ja die
ganze Welt sich vergeblich verb\"unden, mich von Erreichung
de{\s}selben abzuhalten. Hab' ich einen Gedanken, so erwarte ich,
da{\ss} Andere ihn theilen, und vertrage keinen Widerspruch. Ich
betone diese Einzelnheiten, ehrenwerthe B\"urger, weil Sie mich
gr\"undlich kennen lernen m\"ussen. Sie finden vielleicht, da{\ss}
ich zu viel von mir selbst spreche? Thut nicht{\s}! Jetzt aber
\"uberlegen Sie sich Alle{\s}, ehe Sie mich unterbrechen, denn ich
bin hierhergekommen, Ihnen Dinge zu sagen, welche Ihnen vielleicht
nicht recht gefallen d\"urften.{\grqq}

Ein Grollen wie da{\s} der Brandung lief l\"ang{\s} der ersten
B\"anke de{\s} Saale{\s} hin, ein Zeichen, da{\ss} da{\s} Meer bald
wieder hoch aufwogen werde.

{\glqq}Reden Sie, ehrenwerther Fremdling,{\grqq} begn\"ugte sich
Onkel Prudent, der M\"uhe hatte, seine Ruhe zu bewahren, auf diese
Ansprache zu antworten.

Und Robur sprach wie vorher, ohne sich irgendwie um Beifall oder
Mi{\ss}fallen seiner Zuh\"orer zu k\"ummern.

{\glqq}Ja wohl, ich wei{\ss} Alle{\s}! Nach einem Jahrhundert
andauernder Experimente, die zu Nicht{\s} gef\"uhrt, nach Versuchen,
die ergebni{\ss}lo{\s} verliefen, giebt e{\s} noch immer verkehrt
beanlagte Geister, welche hartn\"ackig an die Lenkbarkeit von
Ballon{\s} glauben. Sie erdenken irgend einen Motor, einen
elektrischen oder einen anderen, der an ihre anspruch{\s}vollen,
d\"unnen H\"ullen angebracht wurde, welche letztere den
atmosph\"arischen Str\"omungen so breite Angriff{\s}fl\"achen
darbieten. Sie bildeten sich ein, Beherrscher eine{\s} Aerostaten
werden zu k\"onnen, wie man etwa ein Schiff auf der Oberfl\"ache
de{\s} Meere{\s} beherrscht. Weil einige Erfinder bei ganz oder doch
fast ganz stiller Witterung den Erfolg gehabt haben, entweder schief
durch den Wind oder einer ganz leichten Brise entgegen zu fahren,
de{\s}halb sollte die Lenkbarkeit von Apparaten, welche leichter
sind, al{\s} die Luft, zu praktischen Erfolgen f\"uhren? O gehen Sie!
Sie sind hier an hundert M\"anner, die an die Verwirklichung ihrer
Tr\"aume glauben und viele Tausende von Dollar{\s} nicht in'{\s}
Wasser, aber in die Luft werfen. Ich sage Ihnen, da{\s} hei{\ss}t
gegen eine Unm\"oglichkeit k\"ampfen!{\grqq}

Wunderbar, die Mitglieder de{\s} Weldon-Institut{\s} sagten
gegen\"uber dieser Behauptung jetzt kein Wort, al{\s} w\"aren sie
eben so taub wie langm\"uthig geworden, oder hielten sie nur an sich,
um zu sehen, wie weit dieser k\"uhne Widersacher zu gehen wagen
w\"urde?

Robur fuhr fort:

{\glqq}Nehmen wir einen Ballon. Um ein Kilogramm an Gewicht zu
verlieren, mu{\ss} derselbe ein Cubikmeter Ga{\s} aufnehmen. Ein
Ballon, der den Anspruch macht, mit Hilfe seine{\s} Mechani{\s}mu{\s}
dem Winde zu widerstehen, wenn der Druck einer steifen Brise auf
da{\s} Gro{\ss}segel eine{\s} Schiffe{\s} der Kraft von 400 Pferden
gleichkommt, wenn man bei dem Ungl\"uck{\s}falle mit der Taybr\"ucke
gesehen hat, da{\ss} ein Orkan einen Druck von 444 Kilogramm auf den
Quadratmeter au{\s}zu\"uben im Stande ist! Ein Ballon, wo die Natur
doch niemal{\s} ein fliegende{\s} Gesch\"opf nach diesem System
geschaffen hat, ob da{\s}selbe nun mit Fl\"ugeln, wie die V\"ogel,
oder mit Membranen, wie gewisse Fische und S\"augethiere,
au{\s}ger\"ustet wurden~...

-- S\"augethiere? rief eine{\s} der Mitglieder de{\s} Club{\s}.

-- Gewi{\ss}, die Fledermau{\s}, welche ja auch fliegt, wenn ich
nicht irre. Sollte der Herr, welcher mich unterbrach, wirklich nicht
wissen, da{\ss} die Fledermau{\s} ein S\"augethier ist, oder hat er
jemal{\s} eine Omelette au{\s} Fledermau{\s}eiern bereiten
sehen?{\grqq}

Darauf hielt der Heimgeschickte seine Unterbrechungen ferner f\"ur
sich, Robur dagegen fuhr mit demselben Eifer fort:

{\glqq}W\"are damit aber gesagt, da{\ss} der Mensch darauf verzichten
m\"usse, da{\s} Luftmeer zu beherrschen und durch Nutzbarmachung
diese{\s} wunderbaren Bef\"orderung{\s}mittel{\s} die Zust\"ande der
alternden Welt umzuwandeln? Gewi{\ss} nicht! So wie er der Herr der
Meere geworden durch da{\s} Schiff mit Ruder, Segel, Rad oder
Schraube, so wird er auch zum Herrn der Luft werden durch Apparate,
welche schwerer sind al{\s} diese, denn unbedingt m\"ussen jene
schwerer sein, um m\"achtiger sein zu k\"onnen.{\grqq}

Jetzt war in der Versammlung aber kein Halten mehr. Welche Breitseite
von Zurufen donnerte au{\s} jedem Munde, die alle auf Robur zielten,
wie eben so viele Gewehrl\"aufe oder Kanonenrohre! Sollten sie nicht
antworten auf solch' offenbare, in'{\s} Lager der Ballonisten
geschleuderte Krieg{\s}erkl\"arung? Wurde hiermit nicht der Kampf
zwischen dem {\glqq}leichter{\grqq} und {\glqq}schwerer al{\s} die
Luft{\grqq} au{\s}gesprochener Ma{\ss}en wieder aufgenommen?

Robur verzog keine Miene. Die Arme \"uber der Brust gekreuzt wartete
er e{\s} regung{\s}lo{\s} ab, bi{\s} wieder Ruhe eingetreten war.

Onkel Prudent befahl durch eine Handbewegung, da{\s} Feuer
einzustellen.

{\glqq}Ja, fuhr Robur fort, die Zukunft geh\"ort den Flugmaschinen.
Die Luft bietet den hinreichenden, soliden St\"utzpunkt. Man verleihe
einer S\"aule diese{\s} Medium{\s} eine aufsteigende Bewegung von 45
Meter in der Secunde, und ein Mensch w\"urde sich schon oberhalb
derselben erhalten, wenn die Sohlen seiner Schuhe nur ein Achtel
Quadratmeter Oberfl\"ache boten. W\"urde die Geschwindigkeit der
Lufts\"aule auf 90 Meter gesteigert, so k\"onnte er mit blo{\ss}en
F\"u{\ss}en darauf gehen. Treibt man nun durch die Fl\"ugel einer
archimedischen Schraube eine Luftmasse mit derselben Schnelligkeit
fort, so erzielt man da{\s}selbe Resultat.{\grqq}

Wa{\s} Robur hier sagte, hatten vor ihm alle Anh\"anger der
sogenannten Aviation au{\s}gesprochen, deren Arbeiten langsam, aber
sicher zur L\"osung de{\s} vorliegenden Problem{\s} zu f\"uhren
versprechen.

Die Ehre, diese einfachen Gedanken verbreitet zu haben, kommt Ponton
d'Ann\'ecourt, La Landelle, Nadar, Luzi, Louvrie, Liai{\s},
B\'el\'egnic, Moreau, den beiden Richard, Babinet, Jobert, Du Temple,
Salive{\s}, Penaud, De Villeneuve, Gauchol und Tatin, Michel Loup,
Edison, Planavergue und noch einer Menge anderer M\"anner zu.
Mehrmal{\s} aufgegeben und wieder aufgenommen, mu{\ss}te denselben
doch eine{\s} Tage{\s} der Sieg zu Theil werden. Und hatten von
dieser Seite die Feinde der Aviation, welche behaupteten, da{\ss} der
Vogel nur durch Erw\"armung der Luft, mit der er sich aufbl\"aht,
fliege, auf Antwort warten m\"ussen? Hatten die Erstgenannten nicht
vielmehr nachgewiesen, da{\ss} ein 5 Kilogramm wiegender Adler sich
h\"atte mit 50 Cubikmeter jene{\s} erw\"armten Fluidum{\s} anf\"ullen
m\"ussen, um sich dadurch allein frei schwebend zu erhalten?

Ganz da{\s}selbe wie{\s} auch hier Robur mit unerbittlicher Logik
nach, aber inmitten eine{\s} Heidenl\"arme{\s}, der sich von allen
Seiten erhob. Zum Schlu{\ss} warf er den Ballonisten noch folgende
Worte in'{\s} Gesicht:

{\glqq}Mit Ihren Aerostaten k\"onnen Sie nicht{\s} au{\s}richten,
werden Sie zu nicht{\s} kommen und niemal{\s} etwa{\s} wagen
d\"urfen. Der k\"uhnste Ihrer Aeronauten, John Wise, mu{\ss}te,
obwohl er schon eine Luftreise von 1200 Meilen \"uber da{\s} Festland
Amerika{\s} zur\"uckgelegt hatte, doch auf die Absicht, \"uber den
atlantischen Ocean zu fahren, verzichten. Und seit jener Zeit sind
Sie um keinen Schritt, um keinen einzigen auf diesem Wege
vorw\"art{\s} gekommen.

-- Mein Herr, begann da der Vorsitzende, der sich vergeblich
bem\"uhte, ruhig zu bleiben, Sie vergessen offenbar, wa{\s} unser
unsterblicher Franklin au{\s}gesprochen hat, al{\s} die erste
Mongolfi\`ere aufstieg, also zur Zeit der Geburt de{\s} Ballon{\s}.
{\glqq}Jetzt ist da{\s} nur ein Kind, aber e{\s} wird wachsen!{\grqq}
lautete seine Prophezeiung, und e{\s} ist gewachsen!

-- Nein, Herr Pr\"asident, nein, e{\s} ist nicht gewachsen ... e{\s}
ist nur gr\"o{\ss}er und dicker geworden, und da{\s} ist nicht da{\s}
N\"amliche.{\grqq} \footnote[1]{\frakfamily Wegen de{\s} Doppelsinne{\s}
de{\s} franz\"osischen {\glqq}\begin{antiqua}grandir\end{antiqua}{\grqq},
welche{\s} sowohl k\"orperlich wachsen, al{\s} auch an Bedeutung und
Ansehen zunehmen au{\s}dr\"uckt, nicht ganz wiederzugebende{\s}
Wortspiel.\hfill D.~Ueb.}

Da{\s} war ein directer Angriff gegen die Pl\"ane de{\s}
Weldon-Institut{\s}, welche{\s} die Herstellung eine{\s}
Monstre-Ballon{\s} beschlossen, unterst\"utzt und betrieben hatte.
Sofort kreuzten sich denn auch ziemlich bedrohliche Au{\s}rufe in dem
ger\"aumigen Saale, wie:

{\glqq}Nieder mit dem Eindringling!

-- Werft ihn von der Trib\"une herunter!

-- Um ihm zu beweisen, da{\ss} er schwerer ist al{\s} die Luft!{\grqq}

Und Aehnliche{\s} mehr.

Man begn\"ugte sich indessen noch mit Worten, ohne zu Th\"atlichkeiten
\"uberzugehen. Robur konnte also noch einmal seine Stimme erheben und
laut hinau{\s}rufen:

{\glqq}Fortschritte, B\"urger Ballonisten, sind nicht mit dem
Aerostaten, sondern nur mit fliegenden Maschinen zu erwarten. Der
Vogel fliegt auch, und der ist kein Ballon, sondern ein
Mechani{\s}mu{\s}!~...

-- Ja er fliegt wohl, schrie der vor Zorn keuchende Bat~T. Fyn, aber
er fliegt gegen alle Regeln der Mechanik.

-- Ach so!{\grqq} erwiderte Robur, die Achseln zuckend.

Dann fuhr er fort:

{\glqq}Seit man den Flug der gr\"o{\ss}eren und kleineren fliegenden
Thiere genau beobachtet hat, ist folgender sehr einfache Gedanke in
den Vordergrund getreten: E{\s} gilt auch hier die Natur nachzuahmen,
denn diese t\"auscht sich niemal{\s}. Zwischen dem Albatro{\s}, der
kaum zehn Fl\"ugelschl\"age in der Minute macht, und dem Pelikan, der
siebenzig macht~...

-- Einundsiebenzig! rief eine schnarrende Stimme.

-- Und der Biene, bei der man hundert\/zweiundneunzig in der Secunde
z\"ahlte~...

-- Hundertdreiundneunzig! rief ein Anderer au{\s} Scherz.

-- Und der Stubenfliege, welche dreihundertunddrei{\ss}ig fertig
bringt~...

-- Dreihundertdrei{\ss}igundeinhalb!

-- Und dem Mo{\s}quito, der Millionen macht~...

-- Nein ... Milliarden!{\grqq}

Robur lie{\ss} sich durch alle diese Einreden nicht au{\ss}er Fassung
bringen.

{\glqq}Zwischen diesen verschiedenen Zahlen ... nahm er wieder da{\s}
Wort.

-- Ist ein gro{\ss}er Unterschied! lie{\ss} sich eine Stimme h\"oren.

... Wird man die richtige w\"ahlen m\"ussen, um eine praktische
L\"osung der Aufgabe zu finden. Schon an dem Tage, wo De Lucy
nachweisen konnte, da{\ss} der Hirschk\"afer, jene{\s} Insect,
welche{\s} nur zwei Gramm wiegt, ein Gewicht von vierhundert Gramm,
d.~h. zweihundert Mal so viel wie sein eigene{\s} Gewicht, aufzuheben
vermochte, war eigentlich da{\s} Problem der Aviation gel\"ost.
Au{\ss}erdem wurde nachgewiesen, da{\ss} die Fl\"achenau{\s}dehnung
der Fl\"ugel in gleichem Verh\"altni{\ss} abnimmt, wie die
Gr\"o{\ss}e und da{\s} Gewicht de{\s} Thiere{\s} zunehmen. Seitdem
hat man schon mehr al{\s} sechzig verschiedene Apparate erdacht oder
auch au{\s}gef\"uhrt~...

-- Die noch niemal{\s} haben fliegen k\"onnen! rief der
Schriftf\"uhrer Phil Evan{\s}.

-- Welche geflogen sind oder noch fliegen werden, antwortete Robur,
ohne sich irre machen zu lassen. Ob man sie nun Streophoren,
Helicopteren, Orthoptheren nennt, oder ihrem Namen nach dem
lateinischen \begin{antiqua}navis\end{antiqua} die Silbe
{\glqq}\begin{antiqua}nef\end{antiqua}{\grqq} anh\"angt, meinetwegen
auch nach dem Worte \begin{antiqua}avis\end{antiqua} die Silbe
{\glqq}\begin{antiqua}efs\end{antiqua}{\grqq} -- jedenfall{\s}
kommt man zu dem Apparate, dessen endliche Herstellung den Menschen
zum Herren de{\s} Luftmeere{\s} machen mu{\ss}.

-- Aha, die Schraube! warf Phil Evan{\s} ein. Der Vogel hat aber
keine Schraube ... so weit man da{\s} wei{\ss}!

-- Zugegeben, erwiderte Robur, wie Penaud gezeigt hat, arbeitet
eigentlich der Vogel selbst al{\s} solche und ist seinem Fluge nach
Helicoptere, darum ist auch die Schraube der Motor der Zukunft~...

\begin{quote}
... {\glqq}Vor solchem Uebel,\\
Heilige Helice\footnote[1]{\frakfamily Der Name Helice in der Bedeutung
Schraube gebraucht. \hfill D.Ueb.}, beh\"ute un{\s}!{\grqq}~...
\end{quote}

tr\"allerte einer der Zuh\"orer, der zuf\"allig diese{\s} Motiv
au{\s} H\'erold'{\s} Zampa im Kopfe behalten hatte.

Alle wiederholten den Refrain im Chor und mit Intonationen, bei denen
sich der Componist sicher im Grabe herumdrehte.

Dann, al{\s} die letzten T\"one in einem entsetzlichen Durcheinander
verhallten, glaubte Onkel Prudent unter Ben\"utzung eine{\s}
augenblicklichen Stillschweigen{\s} sagen zu m\"ussen:

{\glqq}B\"urger Fremdling, bi{\s} hierher haben wir Sie reden lassen,
ohne Sie zu unterbrechen~...{\grqq}

E{\s} scheint demnach, al{\s} ob der Vorsitzende de{\s}
Weldon-Institut{\s} die fr\"uheren Einw\"urfe, die Zwischenrufe,
da{\s} tolle Durcheinander nicht f\"ur Unterbrechungen, sondern nur
f\"ur einfachen Meinung{\s}au{\s}tausch hielt.

{\glqq}Jedenfall{\s}, fuhr er fort, mu{\ss} ich Sie daran erinnern,
da{\ss} die Theorie der Aviation schon im Vorau{\s} durch die meisten
amerikanischen und fremden Ingenieure verurtheilt und v\"ollig
verworfen worden ist. Ein System, auf dessen Debetseite der Tod
Sarasin Volant'{\s} in Constantinopel, der de{\s} M\"onche{\s} Voador
in Lissabon, der Letuo'{\s} im Jahre 1852 und der Groof'{\s} 1864
steht, ohne die Opfer zu z\"ahlen, die ich augenblicklich vergessen
habe, und w\"are e{\s} nur der mythologische Icaru{\s}~...

-- Diese{\s} System, nahm Robur den Satz auf, ist nicht
verdammen{\s}werther, al{\s} da{\s}, dessen Opferliste die Namen
eine{\s} Pil\^atre de Rozier in Calai{\s}, der Madame Blanchard in
Pari{\s}, eine{\s} Donaldson und Grimwood, welche in den Michigan-See
fielen, eine{\s} Swel, Croc\'e-Spinelli, Eloy und so vieler Anderer
enth\"alt, welche gewi{\ss} nicht so leicht der Vergessenheit
anheimfallen.{\grqq}

Da{\s} hie{\ss} {\glqq}mit einem Hieb parirt{\grqq}, wie man in der
Fechtkunst sagen w\"urde.

{\glqq}Mit Ihren Ballon{\s}, fuhr Robur fort, werden Sie
\"ubrigen{\s}, dieselben m\"ogen noch so vervollkommnet sein,
niemal{\s} eine praktisch werthvolle Schnelligkeit erzielen, zehn
Jahre brauchen, um eine Reise um die Erde zu vollenden -- wa{\s} eine
Maschine in etwa acht Tagen abmachen d\"urfte.{\grqq}

Neue w\"uthende Proteste und Verneinungen, welche drei ganze Minuten
anhielten, bevor dann Phil Evan{\s} da{\s} Wort ergreifen konnte.

{\glqq}Mein Herr Aviator, Sie, der Sie un{\s} so viel von der
Herrlichkeit der Aviation vorreden, sind Sie denn jemal{\s} in dieser
Weise geflogen?

-- Ja, gewi{\ss}!

-- Und Sie h\"atten also den Kampf mit der Luft siegreich bestanden?

-- Vielleicht, mein Herr.

-- Hurrah, Robur, der Sieger! rief eine Stimme spottend.

-- Nun ja, Robur, der Sieger -- ich nehme diesen Namen an und werde
ihn f\"uhren, denn ich habe da{\s} Recht dazu.

-- Wir erlauben un{\s} inde{\ss} daran zu zweifeln! rief Jem Cip.

-- Meine Herren, erkl\"arte Robur, dessen Augenbrauen sich runzelten,
wenn ich eine ernsthafte Sache ernsthaft behandle, duld' ich e{\s}
nicht, da{\ss} mir Jemand eine Unzuverl\"assigkeit meiner Worte
vorwirft, und ich w\"urde gern den Namen de{\s} Herrn kennen lernen,
der mich in dieser Weise unterbrach.

-- Ich hei{\ss}e Jem Cip ... und bin Vegetarianer.

-- B\"urger Jem Cip, antwortete Robur, ich wei{\ss}, da{\ss} die
Pflanzenesser gew\"ohnlich l\"angere Eingeweide haben, al{\s} andere
Menschen -- mindesten{\s} um einen Fu{\ss} l\"anger. Da{\s} ist schon
viel ... Nun verleiten Sie mich nicht, die Ihrigen noch mehr zu
verl\"angern, indem ich bei den Ohren anfange~...

-- Durch die Th\"ur!

-- Hinau{\s} auf die Stra{\ss}e!

-- Man viertheile ihn!

-- Lynchen, lyncht den Kerl!

-- Verdrehen wir ihn zu einer Schraube!~...{\grqq}

Die Wuth der Ballonisten hatte ihren Gipfel erreicht. Schon sprangen
sie von den St\"uhlen auf und umdr\"angten die Trib\"une. Robur
verschwand unter einer Unmasse von Armen, welche sich, wie von einem
Sturme getrieben, auf- und abbewegten. Vergeben{\s} lie{\ss} die
Dampftrompete ihren heulenden Ton durch die Versammlung brausen. An
jenem Abende konnte Philadelphia wohl glauben, eine Feuer{\s}brunst
verzehre eine{\s} seiner Quartiere, und da{\s} ganze Wasser de{\s}
Schuylkill-Strome{\s} werde zum L\"oschen de{\s}selben nicht
hinreichen.

Pl\"otzlich entstand in der l\"armenden Masse eine Bewegung nach
r\"uck\-w\"art{\s}. Robur hatte eben die H\"ande wieder au{\s} den
Taschen gezogen und streckte sie gegen die vorderste Reihe der
w\"uthenden Gegner au{\s}.

Seine beiden H\"ande zeigten jetzt zwei sogenannte amerikanische
F\"auste, welche gleichzeitig Revolver bilden und die schon ein Druck
de{\s} Daumen{\s} ihre \"uberall verst\"andliche Sprache reden lassen
-- zwei kleine Taschen-Mitrailleusen.

Dann rief er, da{\s} Zur\"uckgehen der Angreifer und die
vor\"ubergehende Stille, welche dabei eintrat, schnell ben\"utzend:

{\glqq}Entschieden war e{\s} nicht Amerigo Vespucci, der die Neue
Welt entdeckt hat, sondern Sebastian Cabot. Sie sind keine
Amerikaner, B\"urger Ballonisten! Sie sind nur Cabo...{\grqq}

In diesem Augenblicke krachten auch schon vier oder f\"unf Sch\"usse
in die Luft, welche Niemand verwundeten. Inmitten de{\s}
Pulverdampfe{\s} verschwand der Ingenieur, und al{\s} jener sich
zerstreute, entdeckte man von ihm keine Spur mehr. Robur der Sieger
war davongeflogen, al{\s} ob irgend ein Aviation{\s}-Apparat ihn in
die L\"ufte entf\"uhrt h\"atte.



\newpage\begin{center}\label{kap04}
{\large \begin{antiqua}IV.\end{antiqua}\\\medskip
In dem der Verfasser infolge einer Bemerkung de{\s} Diener{\s}
Frycollin den Mond wieder zu Ehren zu bringen versucht.\\\bigskip}
\end{center}



Sicherlich schon mehr al{\s} einmal hatten die Mitglieder de{\s}
Weldon-Institut{\s}, wenn sie nach st\"urmischen Verhandlungen au{\s}
den Sitzungen kamen, Walnut-Street und die Nachbarstra{\ss}en noch
streitend und l\"armend durchzogen. Wiederholt waren von den
Bewohnern diese{\s} Stadttheile{\s} Klagen eingegangen \"uber die
ger\"auschvollen Au{\s}l\"aufer solcher Verhandlungen, welche bi{\s}
in ihre Wohnungen eindrangen, und mehr al{\s} einmal hatten
Polizisten einschreiten m\"ussen, um wenigsten{\s}
Verkehr{\s}st\"orungen zu beseitigen, da doch die meisten Leute sehr
wenig oder gar kein Interesse an solchen, die Luftschifffahrt
betreffenden Fragen nehmen. Doch vor diesem heutigen Abend hatte der
Tumult noch nie so gro{\ss}e Verh\"altnisse angenommen, niemal{\s}
w\"aren jene Klagen mehr begr\"undet und niemal{\s} die Einmischung
der Policemen nothwendiger gewesen.

Immerhin konnte man den Mitgliedern de{\s} Weldon-Institut{\s}
mildernde Umb\"ande zubilligen, da sie sich eine{\s} Ueberfalle{\s}
in den eigenen vier Pf\"ahlen, wie sie eben erlitten, gewi{\ss} nicht
versehen hatten. Den \"ubereifrigen Verfechtern de{\s} Grundgesetze{\s}
{\glqq}leichter, al{\s} die Luft{\grqq} hatte ein nicht minder
energischer Vertreter de{\s} {\glqq}schwerer, al{\s} die Luft{\grqq}
h\"ochst unangenehme Dinge in'{\s} Gesicht gesagt; und al{\s} ihm
daf\"ur die Behandlung zu Theil werden sollte, die er verdiente, war
der Mann spurlo{\s} verschwunden.

Da{\s} schrie nach Rache! Um derartige Beleidigungen ungestraft zu
lassen, h\"atten sie nicht amerikanische{\s} Blut in ihren Adern
haben m\"ussen. Die Nachkommen Amerigo'{\s} al{\s} solche eine{\s}
Cabot zu behandeln! War da{\s} nicht eine Beschimpfung, die um so
unverzeihlicher schien, weil sie eigentlich richtig, wenigsten{\s}
historisch berechtigt war?

Die Mitglieder de{\s} Club{\s} st\"urzen sich also truppweise erst in
die Walnut-Street, hierauf in die Nachbarstra{\ss}en und dann in
da{\s} ganze Quartier, wo alle Bewohner aufgescheucht werden.

Sie zwingen dieselben, eine Durchsuchung ihrer H\"auser vornehmen zu
lassen, um sich sp\"ater wegen de{\s} gewaltth\"atigen Angriff{\s} in
da{\s} Privatleben ihrer Mitb\"urger zu entschuldigen, wa{\s} gerade
bei den V\"olkern von angels\"achsischem Stamme sonst ganz
besonder{\s} respectirt wird. Vergebliche{\s} Aufgebot von
Bel\"astigungen und Nachforschungen. Robur wurde nirgend{\s}
gefunden; er hatte nicht die leiseste Spur hinterlassen. Und wenn er
mit dem \begin{antiqua}Go a head\end{antiqua}, dem Ballon de{\s}
Weldon-Institut{\s}, davongefahren w\"are, h\"atte er nicht mehr
unauf\/findlich gewesen sein k\"onnen. Nach einst\"undigen
Hau{\s}suchungen mu{\ss}ten sie darauf verzichten, und die Collegen
trennten sich, aber nicht ohne die eidliche Zusicherung, ihre
Nachforschungen \"uber da{\s} ganze Gebiet Nord- und
S\"udamerika{\s}, da{\s} die Neue Welt bildet, au{\s}zudehnen.

Gegen elf Uhr war die Ruhe in dem Quartier nahezu wieder hergestellt.
Philadelphia konnte sich wieder in sanften Schlummer versenken, wozu
die St\"adte, welche weniger Industrie haben, da{\s}
beneiden{\s}werthe Privilegium besitzen. Die verschiedenen Mitglieder
de{\s} Club{\s} dachten jetzt an nicht{\s} Andere{\s}, al{\s} an die
Heimkehr an den eigenen h\"au{\s}lichen Herd. Um nur einige der
hervorragendsten zu nennen, so begab sich William T. Forbe{\s}
eiligst nach dem Tische, auf dem Mi{\ss} Doll und Mi{\ss} Mat ihm den
Abendthee zubereitet und mit der selbst\/zubereiteten Glucose
vers\"u{\ss}t hatten; Truk Milnor schlug den Weg nach seiner Fabrik
ein, deren Ventilator die ganze Nacht hindurch in einer der
entfernteren Vorst\"adte sauste. Der Schatzmeister Jem Cip, dem
\"offentlich nachgesagt worden war, einen um einen Fu{\ss} l\"angeren
Darmcanal zu haben, al{\s} der Mensch ihn sonst mit sich
herumtr\"agt, begab sich nach seinem E{\ss}zimmer, wo ihn ein
vegetabilische{\s} Abendbrot erwartete.

Zwei der bedeutendsten Ballonisten -- aber nur zwei -- schienen nicht
daran zu denken, ihr Heim sogleich aufzusuchen. Sie hatten die
Gelegenheit wahrgenommen, in hitzigster Weise weiter zu plaudern.
E{\s} waren da{\s} die beiden Unvers\"ohnlichen, Onkel Prudent und
Phil Evan{\s}, der Vorsitzende und der Schriftf\"uhrer de{\s}
Weldon-Institut{\s}.

An der Th\"ur de{\s} Clubhause{\s} erwartete Frycollin, der Diener
de{\s} Onkel Prudent, wie gew\"ohnlich seinen Herrn.

Er folgte diesem auf Schritt und Tritt nach, ohne sich um den
Gegenstand de{\s} Gespr\"ach{\s} zu k\"ummern, der die beiden
Collegen schon in die Hitze gebracht hatte.

Wir gebrauchten auch nur euphemistisch da{\s} Zeitwort
{\glqq}plaudern{\grqq} f\"ur die Th\"atigkeit, welcher der
Vorsitzende und der Schriftf\"uhrer de{\s} Club{\s} sich mit gleichem
Eifer hingaben. In der That stritten und zankten sie sich mit einer
Energie, deren Ursprung in ihrer alten Rivalit\"at zu suchen war.

{\glqq}Nein, und dreimal nein! wiederholte Phil Evan{\s}, h\"atte ich
die Ehre gehabt, dem Weldon-Institut bei der heutigen Sitzung zu
pr\"asidiren, e{\s} w\"are niemal{\s} zu einem solchen Scandal
gekommen!

-- Und wa{\s} w\"urden Sie gethan haben, wenn Sie diese Ehre gehabt
h\"atten? fragte Onkel Prudent.

-- Ich h\"atte jenem \"offentlichen Beleidiger da{\s} Wort
abgeschnitten, noch ehe er den Mund \"offnete.

-- Mir scheint, um Jemand da{\s} Wort abzuschneiden, m\"usse man ihm
ein solche{\s} wenigsten{\s} erst au{\s}sprechen lassen.

-- Nicht in Amerika, mein Herr, nicht in Amerika!{\grqq}

Und w\"ahrend sie sich so mehr bittere al{\s} angenehme Reden{\s}arten
in'{\s} Gesicht warfen, schlenderten die beiden M\"anner mehrere
Stra{\ss}en dahin, die sie immer weiter von ihren Wohnungen
entfernten; sie durchschritten Quartiere, deren Lage sie sp\"ater zu
gro{\ss}en Umwegen zwingen mu{\ss}te.

Frycollin folgte noch immer nach, f\"uhlte sich aber doch etwa{\s}
beunruhigt, seinen Herrn sich nach so menschenleeren Oertlichkeiten
hin verirren zu sehen. Er liebte diese Gegenden nicht, vorz\"uglich
nicht so kurz vor Mitternacht. Dazu herrschte tiefe Dunkelheit, denn
der zunehmende Mond war eben nur dabei, {\glqq}seine
achtundzwanzigt\"agige Rundreise{\grqq} zu beginnen.

Frycollin sah sich scheu nach recht{\s} und link{\s} um, ob sie nicht
von verd\"achtigen Schatten belauscht w\"urden, und wirklich, er
glaubte f\"unf oder sech{\s} gro{\ss}e Teufel zu erkennen, die sie
nicht au{\s} den Augen zu verlieren schienen.

Instinctiv n\"aherte sich Frycollin seinem Herrn, um Alle{\s} in der
Welt h\"atte er jedoch nicht gewagt, ihn inmitten eine{\s}
Gespr\"ach{\s} zu unterbrechen, von dem er zuweilen einzelne Brocken
aufschnappte.

Der Zufall f\"ugte e{\s}, da{\ss} der Vorsitzende und der
Schriftf\"uhrer de{\s} Weldon-Institut{\s} sich, ohne darauf zu
achten, bi{\s} nach dem Fairmont-Park verirrten. Hier \"uberschritten
sie, in lebhaftem Wortwechsel begriffen, den Schuylkill-Strom auf der
ber\"uhmten Eisenbr\"ucke; sie begegneten nur sehr wenig Leuten und
befanden sich endlich mitten in jenen weiten Terrain{\s}, die sich
auf der einen Seite al{\s} ungeheure Wiesen au{\s}dehnen, auf der
anderen von herrlichem Baumbestand beschattet sind und in ihrer
Gesammtheit eine vielleicht in der ganzen Welt einzig dastehende
Anlage bilden.

Hier nahm der Schreck de{\s} Diener{\s} Frycollin pl\"otzlich noch
mehr zu, und da{\s} mit um so gr\"o{\ss}erer Berechtigung, da f\"unf
bi{\s} sech{\s} jener Schatten ihnen auch \"uber die Strombr\"ucke
nachgefolgt waren. Die Pupille seiner Augen hatte sich dabei so
erweitert, da{\ss} sie bi{\s} an den Rand der Iri{\s} reichte. Und
gleichzeitig schrumpfte sein ganzer K\"orper zusammen und zog sich
zur\"uck, al{\s} bes\"a{\ss}e er jene eigenth\"umliche
Zusammenziehbarkeit, welche den Mollu{\s}ken und auch gewissen
Wirbelthieren eigen ist.

Der Diener Frycollin war n\"amlich ein vollst\"andiger Hasenfu{\ss}.

Ein richtiger Neger und S\"udcaroliner, mit vierschr\"otigem Kopfe
auf einem mageren Rumpfe. Er z\"ahlte jetzt gerade 21 Jahre, war also
nicht einmal mehr zur Zeit seiner Geburt Sclave gewesen, taugte
de{\s}halb aber nicht viel mehr, al{\s} ein solcher. Ein
Grimassenschneider, Leckermaul und Faulpelz, aber vor Allem ein
Prahlhan{\s} sondergleichen, stand er seit drei Jahren bei Onkel
Prudent im Dienste. Hundert Mal war er schon nahe daran gewesen, vor
die Th\"ure gesetzt zu werden, doch hatte man ihn behalten -- um
nicht au{\s} dem Regen in die Traufe zu kommen. Und doch lief er hier
bei einem Herrn, der jeden Augenblick zu den tollk\"uhnsten
Unternehmungen bereit war, so oft Gefahr, in Lagen zu kommen, in
denen sein Hasenherz auf die h\"artesten Proben gestellt werden
mu{\ss}te. Daf\"ur fand er auch gewisse Entschuldigungen. Niemand
machte ihm besondere Vorw\"urfe wegen seiner Leckerhaftigkeit und
noch weniger wegen seiner Tr\"agheit. Ach, armer Frycollin, h\"attest
Du in der Zukunft lesen k\"onnen!

Warum war Frycollin auch nicht in Boston im Dienste einer gewissen
Familie Sneffel geblieben, die im Begriffe, eine Reise nach der
Schweiz anzutreten, darauf verzichtet hatte, weil daselbst
Schneelawinen vorkamen? War f\"ur Frycollin nicht diese{\s} Hau{\s}
da{\s} geeignete, aber nicht da{\s} de{\s} Onkel Prudent, wo da{\s}
k\"uhne Wagen in Permanenz erkl\"art war?

Nun, er befand sich einmal hier und sein Herr hatte sich mit der Zeit
an seine Fehler gew\"ohnt, \"ubrigen{\s} besa{\ss} er doch e~i~n~e
gute Eigenschaft. Obwohl Neger von Abstammung, sprach er doch nicht,
wie diese gew\"ohnlich -- und da{\s} hat einigen Werth, denn
nicht{\s} ist so widerlich, al{\s} der abscheuliche Jargon, in dem
die Anwendung de{\s} besitzanzeigenden F\"urworte{\s} und de{\s}
Infinitiv{\s} bi{\s} zum Mi{\ss}brauch getrieben wird.

E{\s} steht also fest, da{\ss} der Diener Frycollin ein feiger
Prahlhan{\s} war, und zwar nannte man ihn einen {\glqq}Prahlhan{\s}
gleich dem Monde{\grqq}.

E{\s} erscheint \"ubrigen{\s} nur gerecht, gegen diesen f\"ur die
blonde Ph\"obe beleidigenden Vergleich Einspruch zu erheben; warum
sollte man die sanfte Selene, die keusche Schwester de{\s}
strahlenden Apollo, der Prahlerei zeihen, da{\s} Gestirn, welche{\s},
so lange die Welt steht, stet{\s} der Erde gerade in'{\s} Gesicht
geblickt hat, ohne ihr jemal{\s} den R\"ucken zuzuwenden?

Doch wie dem auch sei, zu dieser Stunde -- e{\s} war jetzt bald
Mitternacht -- begann die {\glqq}blasse, verd\"achtige Scheibe{\grqq}
schon im Westen hinter den hohen Baumkronen de{\s} Park{\s} zu
verschwinden. Ihre durch da{\s} Gezweige hereindringenden Strahlen
erhellten nur noch da und dort den Erdboden, so da{\ss} e{\s} unter
den B\"aumen noch etwa{\s} finsterer war.

Da{\s} gestattete Frycollin, einen forschenden Blick umherschweifen
zu lassen.

{\glqq}Brr, machte er, die Schurken sind wahrlich noch da! Offenbar
kommen sie n\"aher heran.{\grqq}

Da hielt e{\s} ihn nicht mehr und er schritt auf seinen Herrn zu.

{\glqq}Master Onkel!{\grqq} redete er ihn an.

So nannte er ihn gew\"ohnlich und so wollte der Vorsitzende de{\s}
Weldon-Institut{\s} auch genannt sein.

Eben jetzt war der Streit der beiden Rivalen auf da{\s} Hitzigste
entbrannt; und da sie einander spazieren f\"uhrten, wurde Frycollin
sehr grob angewiesen, diesen Spaziergang mit\/zumachen, wie e{\s}
seine Pflicht und Schuldigkeit sei.

Und w\"ahrend die Beiden ohne Unterbrechung weiterstritten, gerieth
Onkel Prudent immer weiter hinau{\s} nach den ver\"odeten
Gra{\s}gr\"unden de{\s} Fairmont-Parke{\s} und entfernte sich immer
mehr vom Schuylkill und der Br\"ucke, die sie zur R\"uckkehr nach der
Stadt unbedingt \"uberschreiten mu{\ss}ten.

Alle Drei befanden sich jetzt inmitten einer Gruppe hoher B\"aume, in
deren Gipfeln noch da{\s} letzte Licht de{\s} Monde{\s} spielte. An
den Saum derselben schlo{\ss} sich eine gr\"o{\ss}ere Lichtung an,
ein weiter, ovaler Wiesenplan, wie geschaffen f\"ur Wettrennen. Hier
h\"atte nicht die kleinste Unebenheit de{\s} Boden{\s} den Galopp
eine{\s} Pferde{\s} gest\"ort und kein Busch oder Baum die Blicke der
Zuschauer bei der Verfolgung der mehrere englische Meilen langen
Bahnlinie gehindert.

Und doch, w\"aren Onkel Prudent und Phil Evan{\s} in ihre
Streitigkeiten nicht gar so sehr vertieft gewesen, h\"atten sie sich
nur einigerma{\ss}en aufmerksam umgesehen, so h\"atte ihnen nicht
entgehen k\"onnen, da{\ss} der weite freie Platz heute einen ganz
anderen Anblick darbot. War da{\s} ein Zauberspuk, der hier seit
gestern entstanden war? Wahrlich, man h\"atte da{\s} Ganze mit seinen
vielen Windm\"uhlen f\"ur ein Zauberwerk erkl\"aren k\"onnen, wenn
man die M\"uhlenfl\"ugel sah, die, jetzt unbeweglich, im Halbdunkel
Grimassen zu machen schienen.

Doch weder der Pr\"asident, noch der Schriftf\"uhrer de{\s}
Weldon-Institut{\s} bemerkte diese auf\/f\"allige Ver\"anderung der
Ansicht de{\s} Fairmont-Park{\s}; Frycollin sah sie ebenso wenig.
E{\s} schien ihm, al{\s} ob die unheimlichen Gestalten sich
n\"aherten und zusammenduckten, al{\s} r\"usteten sie sich zu einem
r\"auberischen Ueberfalle. Er zitterte au{\s} Angst an allen Gliedern
und war doch gleichzeitig wie gel\"ahmt, so hatte ihn die Furcht vor
den n\"achsten Minuten ergriffen.

Obwohl ihm die Kniee f\"ormlich schlotterten, gewann er doch noch die
Kraft, einmal zu rufen:

{\glqq}Master Onkel! ... Master Onkel!

-- Nun, wa{\s} giebt e{\s} denn?{\grqq} antwortete Onkel Prudent.

Vielleicht w\"aren er und Phil Evan{\s} nicht b\"ose dar\"uber
gewesen, ihren Zorn dadurch abzuk\"uhlen, da{\ss} sie dem
ungl\"ucklichen Diener eine t\"uchtige Tracht Pr\"ugel ertheilten;
dazu fanden sie aber ebenso wenig Zeit, wie letzterer, ihnen eine
weitere Antwort zu geben.

Unter den B\"aumen gellte pl\"otzlich ein lauter Pfiff. Gleichzeitig
flammte inmitten der Lichtung ein heller elektrischer Stern auf.

Da{\s} war zweifel{\s}ohne ein Signal und im vorliegenden Falle die
Mahnung, da{\ss} der Augenblick zu irgend einer Gewaltth\"atigkeit
gekommen sei.

Schneller, al{\s} man e{\s} au{\s}denken kann, st\"urzten sich schon
sech{\s} M\"anner durch da{\s} Unterholz, zwei auf Onkel Prudent,
zwei auf Phil Evan{\s} und zwei auf den Diener Frycollin. Die beiden
Letzten ganz \"uberfl\"ussiger Weise, denn der Neger w\"are ganz
unf\"ahig gewesen, sich zu wehren.

Obgleich \"uberrascht durch diesen Ueberfall, wollten der Vorsitzende
und der Schriftf\"uhrer de{\s} Weldon-Institut{\s} doch versuchen,
Widerstand zu leisten, hatten dazu aber weder Zeit, noch Kraft.
Binnen wenigen Secunden waren sie schon stumm gemacht durch einen
Knebel im Munde, blind durch eine Binde \"uber die Augen, und wurden,
\"uberw\"altigt und gefesselt, schnell durch die Waldlichtung hin
fortgeschleppt. Wa{\s} konnten sie ander{\s} annehmen, al{\s} da{\ss}
sie einer Rotte jener gewissenlosen Herumlungerer in die H\"ande
gefallen seien, welche Jeden au{\s}rauben, den sie noch zu sp\"ater
Stunde im Walde antrafen? Und doch t\"auschten sie sich. Man
durchsuchte nicht einmal ihre Taschen, obwohl Onkel Prudent stet{\s},
seiner Gewohnheit nach und also auch heute, mehrere Tausend
Dollar{\s} Papiergeld bei sich f\"uhrte.

Kurz, eine Minute nach diesem Ueberfalle f\"uhlten Onkel Prudent,
Phil Evan{\s} und der Diener Frycollin, da{\ss} sie, ohne da{\ss} ein
Wort zwischen den Angreifern gewechselt worden w\"are, nicht auf den
Rasen der Waldbl\"o{\ss}e, sondern auf eine Art Fu{\ss}boden
niedergelegt wurden, der unter ihrem Gewichte knarrte. Hier lehnte
man sie dann Einen an den Anderen. Darauf h\"orte man da{\s} Klirren
eine{\s} Riegel{\s} in seiner Klappe und die{\s} belehrte die drei
M\"anner, da{\ss} sie gefangen seien.

Nachher entstand ein seltsame{\s}, anhaltende{\s} Ger\"ausch, wie ein
Schnarren, ein frrr, dessen rrr sich ohne Ende fortsetzten, ohne
da{\ss} in der so ruhigen Nacht etwa{\s} Andere{\s} h\"orbar geworden
w\"are.

\begin{center}
\makebox[15em]{\hrulefill}\bigskip
\end{center}

Welche Unruhe herrschte am folgenden Tage in Philadelphia. Schon in
den Morgenstunden erfuhr die ganze Stadt, wa{\s} sich in der letzten
Sitzung de{\s} Weldon-Institut{\s} zugetragen: Die Erscheinung
jene{\s} r\"athselhaften Fremdling{\s}, eine{\s} Ingenieur{\s},
Namen{\s} Robur -- Robur der Sieger! -- die Streitigkeiten, welche er
offenbar absichtlich unter den Ballonisten erregt, und endlich sein
unerkl\"arliche{\s} Verschwinden.

E{\s} machte aber doch einen noch ganz anderen Eindruck, al{\s} man
sp\"ater davon h\"orte, da{\ss} auch der Vorsitzende und der
Schriftf\"uhrer de{\s} Club{\s} in der Nacht vom 12. zum 13. Juni
verschwunden seien.

Welche Nachsuchungen wurden da nicht in der Stadt und deren
Umgebungen angestellt! Vergeblich -- alle vergeblich. Die Zeitungen
von Philadelphia, nach ihnen die Journale von Pennsylvanien und
endlich die von ganz Amerika bem\"achtigten sich eifrig diese{\s}
Vorfall{\s} und erkl\"arten ihn auf hunderterlei Weise, von denen
keine die richtige war. Durch Annoncen und Maueranschl\"age wurden
betr\"achtliche Preise au{\s}gesetzt -- nicht allein f\"ur Den, der
die ehrenwerthen Verschwundenen wieder finden w\"urde, sondern auch
f\"ur Jeden, der nur auf eine F\"ahrte hinweisen k\"onnte, auf der
man ihren Spuren folgen konnte. Nicht{\s} hatte Erfolg. Und h\"atte
sich die Erde aufgethan gehabt, um sie zu verschlingen, so konnten
der Vorsitzende und der Schriftf\"uhrer de{\s} Weldon-Institut{\s}
nicht vollst\"andiger von der Oberfl\"ache der Erdkugel verschwunden
sein.

Die Regierung{\s}bl\"atter traten bei dieser Gelegenheit mit dem
Verlangen hervor, da{\s} Personal der Polizei in betr\"achtlichem
Ma{\ss}stabe zu vermehren, weil \"ahnliche Attentate gegen die besten
B\"urger der Vereinigten Staaten sich wiederholen k\"onnten -- und
sie hatten damit Recht.

Freilich verlangten die Bl\"atter der Opposition, da{\ss} da{\s}
Personal voll\-st\"andig, und zwar al{\s} unn\"utz verabschiedet werde,
da derartige Raubanf\"alle sich doch wiederholen k\"onnten, ohne
da{\ss} e{\s} m\"oglich w\"urde, die Urheber derselben zu entdecken
-- und vielleicht hatten sie damit nicht Unrecht.

Alle{\s} in Allem, die Polizei blieb, wa{\s} sie war und immer sein
wird in der besten der Welten, die nicht vollkommen ist und e{\s}
niemal{\s} werden wird.



\newpage\begin{center}\label{kap05}
{\large \begin{antiqua}V.\end{antiqua}\\
In dem die Einstellung der Feindseligkeiten zwischen dem Vorsitzenden
und dem Schriftf\"uhrer de{\s} Weldon-Institut{\s} beschlossen
wird.\\\bigskip}
\end{center}



Eine Binde \"uber den Augen zu tragen, einen Knebel im Munde, einen
Strick um die Handgelenke und einen solchen um die Kn\"ochel zu
haben, d.~h. also jeder M\"oglichkeit zu sehen, zu sprechen und sich
zu bewegen, beraubt zu sein, da{\s} war f\"ur den Onkel Prudent keine
Lage, in der er sich h\"atte wohl f\"uhlen k\"onnen, und ebenso wenig
f\"ur Phil Evan{\s} und den Diener Frycollin. Obendrein nicht einmal
zu wissen, wer die Urheber dieser Entf\"uhrung waren, nicht zu
wissen, wo man sich befand und welche{\s} Loo{\s} man zu erwarten
habe -- da{\s} mu{\ss}te gewi{\ss} auch da{\s} allergeduldigste Lamm
in Wuth bringen, und bekanntlich geh\"orten die Mitglieder de{\s}
Weldon-Institut{\s}, wa{\s} ihre Geduld betraf, nicht im geringsten
zur Familie der L\"ammer. Ber\"ucksichtigt man die nat\"urliche
Heftigkeit seine{\s} Charakter{\s}, so kann man sich leicht
vorstellen, in welcher Gem\"uth{\s}verfassung Onkel Prudent sich
jetzt befinden mochte.

Jedenfall{\s} mu{\ss}ten Phil Evan{\s} und er langsam einsehen,
da{\ss} e{\s} f\"ur sie Schwierigkeiten haben werde, am n\"achsten
Abend ihre Pl\"atze im Bureau de{\s} Club{\s} einzunehmen.

Frycollin war e{\s} mit den verbundenen Augen und dem geschlossenen
Munde \"uberhaupt unm\"oglich, irgend etwa{\s} zu denken; er war
schon mehr todt, al{\s} lebendig.

W\"ahrend einer Stunde trat in der Lage der Gefangenen keine
Aenderung ein. Kein Mensch lie{\ss} sich erblicken, sie zu besuchen
oder ihnen wenigsten{\s} die Freiheit der Bewegung und der Sprache
wieder zu geben, nach der sie doch so sehr verlangten. Jetzt sahen
sie sich auf erstickte Seufzer, auf ein dumpfe{\s},
{\glqq}Ach!{\grqq} angewiesen, da{\s} sich kaum durch ihre Knebel
pre{\ss}te, und beschr\"ankt auf schwache Bewegungen, wie sie etwa
ein seinem nat\"urlichen Element entrissener absterbender Karpfen
au{\s}f\"uhrt, und man begreift leicht, welchen stummen Zorn, welch'
verhaltene oder vielmehr eingeschn\"urte Wuth da{\s} in ihnen
erzeugen mu{\ss}te. Nach wiederholten vergeblichen
Befreiung{\s}versuchen verhielten sie sich eine Zeit lang ganz still.
Da ihnen der Gesicht{\s}sinn augenblicklich abging, bem\"uhten sie
sich, vielleicht durch den Geh\"orsinn einige Aufkl\"arung \"uber
diesen beunruhigenden Zustand der Dinge zu erlangen. Vergeblich aber
strengten sie sich an, ein andere{\s} Ger\"ausch zu h\"oren, al{\s}
da{\s} ununterbrochene und unerkl\"arliche {\glqq}frrr{\grqq}, da{\s}
hier den ganzen Umkrei{\s} zu beherrschen schien.

Inzwischen gelang e{\s} Phil Evan{\s}, der hier mit mehr Ruhe au{\s}
Werk ging, den Strick locker zu machen, der um seine Handgelenke lag.
Dann l\"oste sich allm\"ahlig der fesselnde Knoten, er schmiegte die
Finger dicht aneinander, und endlich erlangten seine H\"ande wieder
die gewohnte Bewegung{\s}freiheit.

Durch kr\"aftige{\s} Reiben stellte er den in ihnen halb
unterbrochenen Blutumlauf wieder her, und in der n\"achsten Minute
schon hatte Phil Evan{\s} die Binde abgerissen, die ihm die Augen
bedeckte, den Knebel au{\s} seinem Munde gel\"ost und alle St\"ucke
mit der feinen Klinge seine{\s} {\glqq}Bowie-Messer{\s}{\grqq}
zerschnitten. Ein Amerikaner, der nicht stet{\s} sein Bowie-Messer in
der Tasche h\"atte, w\"are eben kein Amerikaner mehr.

Wenn Phil Evan{\s} aber hierau{\s} die M\"oglichkeit, sich zu bewegen
und zu sprechen, wieder erlangte, so war da{\s} doch eben Alle{\s}.
Seine Augen fanden keine Gelegenheit zu n\"utzlicher Th\"atigkeit --
wenigsten{\s} jetzt nicht, denn in der Zelle, die sie einschlo{\ss},
herrschte vollst\"andige Finsterni{\ss}. Ein ganz schwacher
Lichtschein drang nur durch eine Art Schie{\ss}scharte herein, die in
sech{\s} bi{\s} sieben Fu{\ss} H\"ohe in der Wand angebracht war.

E{\s} versteht sich von selbst, da{\ss} Phil Evan{\s} keinen
Augenblick z\"ogerte, auch seinen Rivalen zu befreien. Einige Z\"uge
mit dem Bowie-Messer gen\"ugten zur Durchschneidung der Stricke,
welche dessen F\"u{\ss}e und H\"ande fesselten. In heller Wuth
ri{\ss} sich Onkel Prudent, al{\s} er sich kaum auf den F\"u{\ss}en
aufrichten konnte, die Binde herunter und den Knebel herau{\s} und
stammelte mit erstickter Stimme:

{\glqq}Ich danke Ihnen!

-- Nein! ... Hier ist nicht{\s} zu danken, antwortete der Andere.

-- Phil Evan{\s}?

-- Onkel Prudent?

-- Hier giebt e{\s} keinen Vorsitzenden und keinen Schriftf\"uhrer
de{\s} Weldon-Institut{\s}, ich denke, auch keine Gegner mehr.

-- Sie haben Recht, best\"atigte Phil Evan{\s}. Hier sind wir nur
zwei M\"anner, die sich zu r\"achen haben an einem Dritten, dessen
Gewaltstreich die strengste Wiedervergeltung herau{\s}fordert.

-- Und dieser Dritte~...

-- Ist jener Robur!~...

-- Ja, jener Robur!{\grqq}

Hier fand sich also einmal ein Punkt, bez\"uglich dessen die beiden
Ex-Concurrenten v\"ollig \"ubereinstimmten, ein Streit \"uber diesen
Gegenstand schien demnach ganz au{\s}geschlossen.

{\glqq}Und Ihr Diener? bemerkte da Phil Evan{\s} mit einem Fingerzeig
auf Frycollin, der wie ein Seehund schnaufte, wir m\"ussen auch ihn
befreien.

-- Noch nicht, erwiderte Onkel Prudent, er w\"urde un{\s} mit seinen
Klageliedern den Kopf warm machen, und wir haben jetzt Andere{\s} zu
thun, al{\s} auf sein Jammern zu achten.

-- Und wa{\s} denn, Onkel Prudent?

-- Un{\s} zu retten, wenn e{\s} m\"oglich ist.

-- Und selbst wenn e{\s} unm\"oglich ist!{\grqq}

Ein Zweifel daran, da{\ss} diese Entf\"uhrung jenem Fremdling, dem
Robur, zuzuschreiben sei, konnte dem Pr\"asidenten und seinem
Collegen gar nicht in den Sinn kommen. In der That h\"atten ja
einfache, ehrsame R\"auber sie unzweifelhaft ihrer Uhren, Edelsteine,
Brieftaschen und Portemonnaie{\s} entledigt und sie dann mit einem
Schnitt durch den Hal{\s} in den Schuylkill-Strom geworfen, statt sie
einschlie{\ss}en in ... Ja, in wa{\s}? -- Da{\s} war eine ernste
Frage, welche die schleunigste L\"osung verdiente, ehe sie mit
einiger Au{\s}sicht auf Erfolg an irgend welche Vorbereitungen zu
ihrer Flucht denken konnten.

{\glqq}Phil Evan{\s}, nahm Onkel Prudent wieder da{\s} Wort, wir
h\"atten wahrlich besser daran gethan, wenn wir beim Weggehen au{\s}
der Sitzung, statt Lieben{\s}w\"urdigkeiten, auf welche wir hier
nicht zur\"uckkommen wollen, au{\s}zutauschen, lieber etwa{\s}
weniger zerstreut gewesen w\"aren. Verlie{\ss}en wir die Stra{\ss}en
von Philadelphia nicht, so w\"are da{\s} Alle{\s} nicht geschehen.
Offenbar hatte jener Robur schon eine Ahnung davon, wa{\s} sein
Auftreten im Club bewirken w\"urde, er muthma{\ss}te die
Wuthau{\s}br\"uche, welche seine Herau{\s}forderungen entfesseln
mu{\ss}ten, und hatte vor der Th\"ure sicherlich einige seiner
Banditen, ihm im schlimmsten Falle beizuspringen. Al{\s} wir dann die
Walnut-Stra{\ss}e verlie{\ss}en, sp\"urten un{\s} seine Schergen auf,
folgten unseren Spuren und al{\s} sie sahen, da{\ss} wir un{\s}
unkluger Weise in die Alleen de{\s} Fairmont-Parke{\s} verirrten, da
hatten sie ja leichte{\s} Spiel.

-- Einverstanden, antwortete Phil Evan{\s}. Ja, wir haben sehr
Unrecht gethan, nicht unmittelbar unsere Wohnungen aufzusuchen.

-- Man hat immer Unrecht, nicht Recht zu haben,{\grqq} versetzte
Onkel Prudent.

Da ert\"onte ein langgezogener Seufzer au{\s} dem finsteren Winkel
der Zelle.

{\glqq}Wa{\s} war da{\s}? fragte Phil Evan{\s}.

-- O nicht{\s} ... Frycollin tr\"aumt nur.{\grqq}

Und Onkel Prudent fuhr ungest\"ort fort:

{\glqq}Zwischen dem Zeitpunkte, wo wir wenige Schritte vom Anfang der
Lichtung ergriffen wurden, und dem, wo man un{\s} in diesen Winkel
warf, sind kaum zwei Minuten verflossen. E{\s} liegt also auf der
Hand, da{\ss} jene Leute un{\s} nicht \"uber den Fairmont-Park
hinau{\s} verschleppt haben.

-- Denn wenn da{\s} geschehen w\"are, h\"atten wir doch von der
Fortschaffung etwa{\s} versp\"uren m\"ussen.

-- Einverstanden, erkl\"arte Onkel Prudent. E{\s} unterliegt also
keinem Zweifel, da{\ss} wir in einer Abtheilung irgend eine{\s}
Wagen{\s} eingesperrt sind, vielleicht in einem jener langen
Prairie-Reisewagen oder in dem Gef\"ahrte von Seilt\"anzern.

-- Ohne Zweifel. Bef\"anden wir un{\s} auf einem auf dem
Schuylkill-Strom vert\"auten Schiffe, so m\"u{\ss}te sich da{\s}
durch ein leichte{\s} Schwanken von Bord zu Bord, veranla{\ss}t durch
die Str\"omung, zu erkennen geben.

-- Einverstanden, stet{\s}, stet{\s}, wiederholte Onkel Prudent, und
ich meine, e{\s} ist, wo wir un{\s} noch in der Parklichtung
befinden, jetzt oder nie der geeignete Moment zur Flucht, um sp\"ater
jenen Robur wieder aufzusp\"uren~...

-- Und ihn diesen Angriff auf die Freiheit zweier B\"urger der
Vereinigten Staaten von Amerika theuer bezahlen zu lassen!

-- Theuer ... sehr theuer!

-- Doch, wer ist dieser Mann? ... Woher kommt er? ... Ist e{\s} ein
Engl\"ander, ein Deutscher, ein Franzose~...

-- Jedenfall{\s} ein elender Wicht, da{\s} gen\"ugt, antwortete Onkel
Prudent. Und nun an'{\s} Werk!{\grqq}

Mit au{\s}gestreckten H\"anden und gespreizten Fingern tasteten Beide
an der Wand de{\s} kleinen Raume{\s} umher, um einen Ri{\ss} oder
eine Spalte zu entdecken. Vergeblich. E{\s} fand sich hier ebenso
wenig davon, wie an der Th\"ur. Diese erwie{\s} sich fast hermetisch
geschlossen, und e{\s} w\"are unm\"oglich gewesen, da{\s} Schlo{\ss}
derselben zu sprengen. Man mu{\ss}te also ein Loch herzustellen
suchen, um durch da{\s}selbe zu entkommen. Dabei trat nun die Frage
hervor, ob die Bowie-Messer die Wand anzugreifen im Stande seien, ob
ihre Klingen sich nicht verbiegen oder bei dem Vorhaben gar
zerbrechen w\"urden.

{\glqq}Doch woher stammt jene{\s} Zittern, da{\s} gar nicht
aufh\"ort? fragte Phil Evan{\s}, der sich \"uber da{\s} immer
fortdauernde frrr nicht beruhigen konnte.

-- E{\s} ist ohne Zweifel der Wind, meinte Onkel Prudent.

-- Der Wind? ... Bi{\s} Mitternacht schien mir, al{\s} ob die Luft
ganz ruhig gewesen w\"are~...

-- Gewi{\ss}, Phil Evan{\s}, doch, wenn e{\s} der Wind nicht sein
soll, wa{\s} halten Sie denn f\"ur die Ursache?{\grqq}

Phil Evan{\s} versuchte, nachdem er die beste Klinge seine{\s}
Messer{\s} aufgeklappt, in die Wand nahe der Th\"ur einzuschneiden.
Vielleicht gen\"ugte e{\s} hier, eine Oeffnung zu machen, um diese
von au{\ss}en zu \"offnen. wenn sie nur durch einen Riegel versperrt
oder der Schl\"ussel im Schlosse stecken geblieben war.

Wenige Minuten Arbeit reichten hin, die Klinge de{\s}
Bowie-Messer{\s} zu verderben, die Spitze de{\s}selben abzubrechen
und e{\s} in eine tausendz\"ahnige S\"age zu verwandeln.

{\glqq}E{\s} greift wohl nicht, Phil Evan{\s}?

-- Nein.

-- Sollten wir un{\s} in einer Zelle au{\s} Stahl befinden?

-- Da{\s} nicht, Onkel Prudent; diese W\"ande geben angeschlagen
keinen metallischen Ton.

-- Also vielleicht au{\s} Eisenholz?

-- Nein, weder au{\s} Eisen, noch au{\s} Holz.

-- Au{\s} wa{\s} best\"ande sie denn dann?

-- Da{\s} ist unm\"oglich zu entscheiden; unbedingt aber ist e{\s}
eine Substanz, welche der Stahl nicht angreift.{\grqq}

Onkel Prudent loderte in hellem Zorn auf, er fluchte, stampfte den
widerhallenden Boden mit den F\"u{\ss}en und seine H\"ande suchten
einen eingebildeten Robur zu erw\"urgen.

{\glqq}Ruhig, Onkel Prudent, ermahnte ihn Phil Evan{\s}, ruhig.
Versuchen Sie einmal Ihr Gl\"uck.{\grqq}

Onkel Prudent versuchte e{\s}, da{\s} Bowie-Messer konnte aber nicht
in eine Wand einschneiden, die selbst dessen beste Klingen nicht zu
ritzen vermochten, al{\s} ob diese au{\s} Krystall w\"are.

Eine Flucht erschien also ganz unau{\s}f\"uhrbar, denn ohne Oeffnung
der Th\"ur war an eine solche doch gar nicht zu denken.

E{\s} galt demnach, f\"ur jetzt darauf zu verzichten, wa{\s} dem
Yankee-Temperament nicht eben leicht zu werden pflegt, und Alle{\s}
vom Zufall zu erwarten, wa{\s} hervorragenden praktischen Geistern
allemal zuwider ist. Nat\"urlich geschah da{\s} nicht ohne
Verw\"unschungen, furchtbare Drohungen und an die Adresse Robur'{\s}
gerichtete schwere pers\"onliche Beleidigungen, w\"ahrend er doch gar
nicht der Mann dazu schien, sich de{\s}halb ein graue{\s} Haar
wachsen zu lassen, wenn ander{\s} er sich im Privatleben ebenso
zeigte, wie bei seinem Auftreten im Weldon-Institute.

Inzwischen gab Frycollin einige unzweifelhafte Zeichen seiner
unbehaglichen Lage von sich. Ob er nun krampfhafte{\s} Kr\"ummen im
Magen empfand oder die Einschn\"urung ihm einen Krampf der Glieder
zugezogen hatte, jedenfall{\s} begann er j\"ammerlich zu lamentiren.

Onkel Prudent glaubte seinen Qualen ein Ende machen zu m\"ussen,
indem er die Stricke, welche den Neger fesselten, durchschnitt.

Fast h\"atte er Ursache gehabt, diese Regung von Mitleid zu bedauern.
Sofort begann Jener n\"amlich eine endlose Litanei, in der Au{\s}br\"uche
de{\s} Entsetzen{\s} und -- Klagen \"uber Hunger die Hauptrolle
spielten. Frycollin litt ebenso sehr im Kopfe, wie im Magen, so,
e{\s} w\"are schwierig gewesen, zu entscheiden, welchem dieser beiden
Organe am meisten Schuld an dem Jammern de{\s} Neger{\s} beizumessen war.

{\glqq}Frycollin!{\grqq} rief Onkel Prudent.

{\glqq}Master Onkel! Master Onkel!{\grqq} antwortete der Neger mit
kl\"aglichem Geschrei.

{\glqq}E{\s} ist m\"oglich, da{\ss} wir verdammt sind, in diesem
Gef\"angnisse Hunger{\s} zu sterben. Wir sind aber entschlossen,
Alle{\s}, wa{\s} irgend verzehrbar erscheint, zu verbuchen, um unser
Leben zu verl\"angern.

-- Und mich aufzuzehren? jammerte Frycollin.

-- Wie man e{\s} unter solchen Umst\"anden mit einem Neger stet{\s}
macht! Sorge also, Frycollin, da{\ss} Du Dich un{\s} nicht zu sehr
bemerkbar machst~...

-- Oder Du wirst fri--cas--sirt!{\grqq} setzte Phil Evan{\s} hinzu.

Frycollin bekam wirklich Angst, da{\ss} sein Leichnam in Anspruch
genommen werden k\"onnte, da{\s} Leben zweier M\"anner zu
verl\"angern, da{\s} jedenfall{\s} werthvoller war, al{\s} da{\s}
seinige. Er begn\"ugte sich also, nur noch im Stillen zu seufzen.

Inzwischen verstrich die Zeit und alle Versuche, die Th\"ur oder die
Wand gewaltsam zu \"offnen, waren erfolglo{\s} geblieben. Worau{\s}
diese Wand bestand, lie{\ss} sich unm\"oglich feststellen. Metall war
da{\s} nicht, Holz war e{\s} nicht und Stein war e{\s} auch nicht.
Der Fu{\ss}boden der Zelle schien \"ubrigen{\s} au{\s} demselben
Material hergestellt zu sein. Stie{\ss} man mit dem Fu{\ss}e auf
denselben, so gab da{\s} einen ganz seltsamen Ton, den unter die
bekannten Ger\"ausche zu classificiren, Onkel Prudent gewi{\ss} viele
M\"uhe gemacht h\"atte.

Dabei bemerkte man noch, da{\ss} der Fu{\ss}boden entschieden hohl
klang, so, al{\s} ob er nicht direct auf dem Boden der Lichtung
ruhte; ja, er schien bei dem unerkl\"arlichen frrr selbst leise zu
erzittern. Alle{\s} da{\s} war nicht gerade beruhigender Natur.

{\glqq}Onkel Prudent, begann Phil Evan{\s}.

-- Phil Evan{\s}? antwortete der Gefragte.

-- Meinen Sie, da{\ss} unsere Zelle ihre Lage ver\"andert hat?

-- Keine{\s}weg{\s}.

-- Und doch, al{\s} wir kaum eingesperrt waren, konnte ich deutlich
den frischen Geruch de{\s} Grase{\s} und den harzigen Duft der
B\"aume de{\s} Parke{\s} wahrnehmen. Jetzt kann ich Luft einfangen,
so viel ich will, e{\s} erscheint mir, al{\s} ob davon nicht{\s} zu
riechen w\"are.

-- Da{\s} ist freilich wahr.

-- Doch, wie soll man da{\s} erkl\"aren?

-- Erkl\"aren wir e{\s} auf ganz beliebige Weise, Phil Evan{\s}, nur
nicht durch die Hypothese, da{\ss} unser Gef\"angni{\ss} eine
Ort{\s}ver\"anderung erlitten habe. Ich wiederhole, da{\ss} wir e{\s}
unbedingt h\"atten f\"uhlen m\"ussen, wenn wir un{\s} auf einem in
Gang befindlichen Wagen oder auf einem dahingleitenden Schiffe
bef\"anden.{\grqq}

Frycollin lie{\ss} einen langgedehnten Seufzer h\"oren, den man
h\"atte f\"ur seinen letzten halten k\"onnen, wenn ihm nicht mehrere
andere nachgefolgt w\"aren.

{\glqq}Ich gab mich der Hoffnung hin, da{\ss} Robur un{\s} bald
veranlagen werde, vor ihn zu treten, fuhr Phil Evan{\s} fort.

-- Ich nicht minder, rief Onkel Prudent, aber ich werde ihm in'{\s}
Gesicht sagen~...

-- Wa{\s}?

-- Da{\ss} er erst wie ein Unversch\"amter in unsere Verhandlungen
eingegriffen hat, um schlie{\ss}lich gleich einem Schurken zu
handeln!{\grqq}

Eben jetzt bemerkte Phil Evan{\s}, da{\ss} der Tag zu grauen begann.
Ein noch schwacher Lichtschein drang durch die enge, am oberen Theile
der der Th\"ur gegen\"uberliegenden Wand angebrachte Schie{\ss}scharte
herein. E{\s} mochte also gegen vier Uhr Morgen{\s} sein, denn im
Juni und unter dieser Breite f\"arbt sich der Horizont von
Philadelphia zu dieser Stunde mit den ersten Morgenstrahlen.

Und doch, al{\s} Onkel Prudent seine Repetiruhr schlagen lie{\ss},
ein Meisterwerk, da{\s} au{\s} der Anstalt eine{\s} Collegen
hervorgegangen war -- meldete diese, da{\ss} e{\s} erst
dreivierteldrei Uhr war, obwohl die Uhr inzwischen bestimmt nicht
gestanden hatte.

{\glqq}Seltsam! sagte Phil Evan{\s}. Um dreivierteldrei Uhr sollte
e{\s} noch Nacht sein.

-- Meine Uhr m\"u{\ss}te dann bedeutend nachgeblieben sein, bemerkte
Onkel Prudent.

-- Eine Uhr der Waldon Watch Compagnie!{\grqq} rief Phil Evan{\s}
beleidigt.

Auf jeden Fall begann e{\s} jetzt Tag zu werden. Nach und nach hob
sich die Schie{\ss}scharte wei{\ss} von der tiefen Dunkelheit der
Zelle ab. Und doch, wenn da{\s} Morgengrauen fr\"uhzeitiger auftrat,
al{\s} e{\s} entsprechend dem 40. Breitengrade, unter dem
Philadelphia liegt, zu erwarten war, so erschien e{\s} doch nicht mit
der bekannten Schnelligkeit, wie in den niedrigen Breiten.

Onkel Prudent machte diese neue Beobachtung und erw\"ahnte der fast
unerkl\"arlichen Erscheinung.

{\glqq}Wir konnten vielleicht bi{\s} nach der Schie{\ss}scharte
hinaufklimmen{\grqq}, bemerkte Phil Evan{\s}, um von da au{\s}
Rundschau zu halten, wo wir \"uberhaupt sind.

-- Da{\s} k\"onnen wir,{\grqq} stimmte Onkel Prudent zu.

Dann wandte er sich an Frycollin:

{\glqq}Nun munter, Fry, auf die F\"u{\ss}e!{\grqq}

Der Neger erhob sich.

{\glqq}Lehne Dich mit dem R\"ucken gegen diese Wand, fuhr Onkel
Prudent fort, und Sie, Phil Evan{\s}, steigen gef\"alligst auf die
Schultern diese{\s} Burschen, w\"ahrend ich Sie von r\"uckw\"art{\s}
halte.

-- Recht gern,{\grqq} antwortete Phil Evan{\s}.

Einen Augenblick sp\"ater kletterte er schon auf Frycollin'{\s}
Schultern, so da{\ss} er zu der Schie{\ss}scharte hinau{\s}sehen
konnte.

Dieselbe war verschlossen, aber nicht durch ein Linsengla{\s}, wie
die Lichtpforte eine{\s} Schiffe{\s}, sondern durch eine
gew\"ohnliche Planscheibe. Obwohl sie nicht sehr stark war,
verhinderte sie doch den freien Au{\s}blick Phil Evan{\s}', dessen
Gesicht{\s}krei{\s} dadurch ziemlich beschr\"ankt wurde.

{\glqq}So zerbrechen Sie doch die Scheibe, sagte Onkel Prudent,
vielleicht k\"onnen Sie dann besser sehen.{\grqq}

Phil Evan{\s} f\"uhrte einen heftigen Schlag mit dem Heft seine{\s}
Bowie-Messer{\s} gegen die Scheibe, welche einen fast silbernen Ton
gab, aber nicht zerbrach.

Ein zweiter, noch kr\"aftigerer Schlag hatte nur da{\s}selbe Resultat.

{\glqq}Sch\"on, rief Phil Evan{\s}, unzerbrechliche{\s} Gla{\s}!{\grqq}

Wirklich mu{\ss}te diese Scheibe au{\s} dem nach der Methode de{\s}
Erfinder{\s} Siemen{\s} geh\"arteten Glase bestehen, da sie trotz der
wiederholten Schl\"age ganz blieb.

Uebrigen{\s} war e{\s} jetzt drau{\ss}en hell genug, um ziemlich weit
sehen zu k\"onnen, wenigsten{\s} innerhalb de{\s}
Gesicht{\s}felde{\s}, da{\s} die Einfassung der Schie{\ss}scharte
frei lie{\ss}.

{\glqq}Wa{\s} sehen Sie? fragte Onkel Prudent.

-- Nicht{\s}.

-- Wie? Keinen Wald?

-- Nein.

-- Nicht einmal die Gipfel der B\"aume?

-- Auch diese nicht.

-- Wir befinden un{\s} also nicht mehr inmitten der Lichtung?

-- Weder in der Lichtung, noch \"uberhaupt im Park.

-- Erkennen Sie denn auch nicht die D\"acher der H\"auser, die
Spitzen der Denkm\"aler? sagte Onkel Prudent, dessen Entt\"auschung
schon in einem Grade zunahm, da{\ss} sie nahe an Wuth grenzte.

-- Weder D\"acher, noch Spitzen.

-- Wa{\s}? Auch nicht einen Mast mit Flagge, nicht einen einzigen
Kirchthurm, nicht einmal einen Fabrik{\s}schornstein?

-- Nicht{\s} -- nicht{\s} al{\s} die leere Luft.{\grqq}

Eben jetzt \"offnete sich die Th\"ur der Zelle, in der ein Mann
sichtbar wurde. Da{\s} war Robur.

{\glqq}Ehrenwerthe Ballonisten, sagte eine ernste Stimme. Sie sind
nun frei, nach Belieben zu gehen, wohin Sie wollen~...

-- Frei! rief Onkel Prudent.

-- Ja ... da{\s} hei{\ss}t innerhalb der Grenzen de{\s}
{\glqq}Albatro{\s}{\grqq}!

Onkel Prudent und Phil Evan{\s} st\"urzten au{\s} der Zelle.

Und wa{\s} sahen sie da?

Zw\"olf- bi{\s} dreizehnhundert Meter unter ihnen die Oberfl\"ache
eine{\s} Lande{\s}, da{\s} sie vergeblich zu erkennen sich
bem\"uhten.



\newpage\begin{center}\label{kap06}
{\large \begin{antiqua}VI.\end{antiqua}\\
Welche{\s} Ingenieure, Mechaniker und andere Gelehrte vielleicht am
besten \"uberschlagen.\\\bigskip}
\end{center}



{\glqq}Wann wird der Mensch einmal aufh\"oren, in der Tiefe
umherzukriechen, um im Azur und im Frieden de{\s} Himmel{\s} zu
leben?{\grqq}

Die Antwort auf diese Frage Camille Flammarion'{\s} ist ziemlich
leicht. Da{\s} wird dann geschehen, wenn die Fortschritte der
Mechanik da{\s} Problem der Aviation, d.~h. der Nachahmung de{\s}
Vogelfluge{\s}, zu l\"osen gestatten, und vor Ablauf weniger Jahre --
da{\s} sah man ja vorau{\s} -- mu{\ss}te eine praktische Verwerthung
der Elektricit\"at zur L\"osung diese{\s} R\"athsel{\s} f\"uhren.

Schon im Jahre 1773, also ziemlich lange, bevor die Br\"uder
Montgolfier ihre erste Montgolfi\`ere und der Physiker Charle{\s}
seinen ersten Wasserstoffballon construirten, hatten einzelne
abenteuerliche K\"opfe davon getr\"aumt, mittelst mechanischer
Apparate die Luft so zu sagen zu erobern. Die ersten Erfinder hatten
also keine{\s}weg{\s} an Apparate gedacht, welche leichter al{\s} die
Luft waren, wa{\s} schon der Standpunkt der physikalischen
Wissenschaft ihrer Zeit nicht erlaubte. Sie gingen darauf au{\s}, die
Fortbewegung durch die Luft durch specifisch schwerere Apparate,
durch Flugmaschinen, welche die Bewegung de{\s} Vogel{\s} nachahmten,
zu erm\"oglichen.

Ganz da{\s}selbe hatte schon jener Thor, der Icaru{\s}, der Sohn
de{\s} Daedalu{\s}, gethan, dessen mit Wach{\s} angeheftete Fl\"ugel
ihm bei der Ann\"aherung an die Sonne abfielen.

Doch ohne bi{\s} auf mythologische Zeiten zur\"uckzugehen, ohne
eine{\s} Archyta{\s} von Tarent zu erw\"ahnen, begegnet man schon in
den Arbeiten eine{\s} Dante von Perousa, eine{\s} Leonard de Vinci
und Guidotti der Idee von Maschinen, welche bestimmt waren, sich in
der Atmosph\"are zu bewegen. Zweieinhalb Jahrhunderte sp\"ater traten
weit zahlreichere Erfinder auf. Im Jahre 1742 construirt sich der
Marqui{\s} de Baqueville ein System von Fl\"ugeln, versucht
da{\s}selbe \"uber der Seine und bricht beim Herabfallen den Arm.
1768 entwirft Paucton den Plan zu einem Apparat mit zwei Schrauben
zum Heben und zur Fortbewegung. 1781 baut Meerwein, der Architekt
de{\s} F\"ursten von Baden, eine Maschine zur Nachahmung de{\s}
Vogelflug{\s} und verwirft den Gedanken der Lenkbarkeit von
Ballon{\s}, welche eben erfunden worden waren. 1784 lassen Launoy und
Bienvenu eine Helicoptere aufsteigen, die von Federn bewegt wurde.
1808 Fliegversuch de{\s} Oesterreicher{\s} Jakob Degen. 1810
erscheint ein Schriftchen von Deniau au{\s} Nante{\s}, in dem der
Grundsatz de{\s} {\glqq}schwerer, al{\s} die Luft{\grqq} beleuchtet
ist. In den Zeitraum von 1811--40 fallen die Untersuchungn und
Versuche von Berblinger, Vigual, Sarti, Dubochet und Cagniard de
Latour. 1842 tritt der Engl\"ander Henson mit seinem System der
geneigten Ebene und durch Dampf bewegter Schraube auf; 1845 Cossu{\s}
mit seinem Apparat mit Steigschrauben; 1847 meldet sich Camille Vert
mit seiner Helicoptere au{\s} Federb\"ugeln; 1852 Letur mit seinem
lenkbaren Fallschirm, dessen praktische Pr\"ufung ihm da{\s} Leben
kostete. In demselben Jahre tritt Michel Loup hervor mit seinem
Vorschlage, bei dem die gleitende Bewegung in Verbindung mit vier
sich drehenden Fl\"ugeln gebracht ist. 1853 B\'el\'eguic mit seinem
durch Zugschrauben bewegten Aeroplan; Vaussin-Chardanne{\s} mit
seinem lenkbaren Drachen; George{\s} Cauley mit seinen
Fliegmaschinen, die ein Ga{\s}motor treiben sollte. Zwischen 1854 und
1863 sind zu nennen: Josef Bline, der Patente auf verschiedene
Systeme der Luftschifffahrt besitzt, Br\'eant, Carlingfort, Le
Bri{\s}, Du Temple, Bright, dessen Steigschrauben sich in verkehrter
Richtung bewegen; Smythie{\s}, Panafieu, Cro{\s}nier u.~A.~m. Endlich
wird im Jahre 1863 auf Betreiben Nadar'{\s} in Pari{\s} eine
Gesellschaft {\glqq}Schwerer, al{\s} die Luft{\grqq} gegr\"undet;
hier f\"uhren die Erfinder ihre Maschinen vor, von denen schon
verschiedene patentirt wurden, wie die von Ponton d'Am\'ecourt und
seine Dampf-Helicoptere; de la Landelle'{\s} System einer Verbindung
von Schrauben mit geneigten Ebenen und Fallschirmen; Louvri\'e'{\s}
Aero{\s}caph; Esterno'{\s} mechanischer Vogel; Groof'{\s} Apparat mit
durch Hebel bewegten Fl\"ugeln. Jetzt, nachdem der Ansto{\ss} gegeben
ist, erfinden die Erfinder, berechnen die Rechner Alle{\s}, wa{\s}
die willk\"urliche Fortbewegung durch die Luft ihrer praktischen
Anwendung zuzuf\"uhren verspricht. Bourcart, Le Bri{\s}, Kaufmann,
Smyth, Stringfellow, Prigent, Danjard, Pom\`e{\s} und de la Panze,
Moy, R\'enaud, Jobert, Hureau de Villeneuve, Achenbach, Garapon,
Duche{\s}ne, Danduran, Parisel, Dieuaide, Melki{\s}ff, Forlanini,
Brearey, Tatin, Dandrieux, Edison erdenken, construiren, erbauen und
vervollkommnen -- die Einen unter Anwendung von Fl\"ugeln oder
Schrauben, die Anderen unter den von geneigten Ebenen -- Maschinen,
welche bereit sein werden, an dem Tage zu functioniren, wo ihnen ein
gl\"ucklicher Erfinder einen Motor von hinreichender Kraft und
au{\ss}erordentlicher Leichtigkeit hinzuf\"ugt.

Wir bitten wegen diese{\s} etwa{\s} langen Namen{\s}verzeichnisse{\s}
um freundliche Entschuldigung, doch e{\s} schien un{\s} nothwendig,
die Einzelstufen jener Leiter der Luftschifffahrt{\s}erfolge, auf
deren Gipfel jetzt Robur der Sieger erscheint, vorzuf\"uhren. Ohne
die schwachen Versuche und die k\"uhnen Experimente seiner
Vorg\"anger, h\"atte der Ingenieur ja unm\"oglich einen so
vollkommenen Apparat zu construiren vermocht. Und wenn er nur ein
ver\"achtliche{\s} Achselzucken f\"ur Diejenigen hatte, welche noch
immer starrsinnig dabei verharrten, die Lenkbarkeit von Ballon{\s}
erfinden zu wollen, so standen bei ihm die Anh\"anger de{\s}
Grundsatze{\s} {\glqq}schwerer al{\s} die Luft{\grqq} in hohem
Ansehen, ob da{\s} nun Engl\"ander, Amerikaner, Deutsche,
Oesterreicher, Italiener oder Franzosen waren -- vor Allem letztere,
deren durch ihn vervollkommnete Arbeiten ihn in den Stand gesetzt
hatten, seine Flugmaschine, den {\glqq}Albatro{\s}{\grqq}, zu
ersinnen und au{\s}zuf\"uhren, jene{\s} Ungeheuer, da{\s} jetzt
da{\s} Luftmeer durchma{\ss}.

{\glqq}Die Taube fliegt! hatte einer der enthusiastischen Anh\"anger
de{\s} {\glqq}Albatro{\s}{\grqq} gerufen.

-- Man wird noch durch die Luft spazieren fahren, wie jetzt \"uber
die Erde! hatte darauf ein hochbegeisterter Vertheidiger derselben
Theorie hinzugesetzt.

-- Mit der Locomotive, der Aeromotive!{\grqq} hatte der lustigste von
Allen hinau{\s}trompetet, der sich mit Vorliebe der Presse bediente,
um die Alte und die Neue Welt f\"ur die Sache zu interessiren.

In der That steht e{\s} ja durch Rechnung und Erfahrung fest, da{\ss}
die Luft ein sehr widerstand{\s}f\"ahiger St\"utzpunkt sein kann. Ein
Krei{\s} von einem Meter Durchmesser vermag al{\s} Fallschirm nicht
allein da{\s} Eindringen in die Luft zu verlangsamen, sondern den
Fall sogar isochronisch zu machen. Da{\s} war schon l\"angst bekannt.

Ebenso wu{\ss}te man, da{\ss} die Einwirkung der Schwerkraft, wenn
die Fortbewegung eine{\s} K\"orper{\s} eine sehr schnelle ist, etwa
im umgekehrten Verh\"altnisse de{\s} Quadrat{\s} zur Schnelligkeit
abnimmt und zuletzt ziemlich bedeutung{\s}lo{\s} werden kann.

Man wu{\ss}te ferner, da{\ss} sich bei zunehmendem eigenen Gewichte
eine{\s} fliegenden Thiere{\s} die zum Schwebenderhalten de{\s}selben
nothwendige Oberfl\"ache der Fl\"ugel nicht in gleichem Ma{\ss}e
vergr\"o{\ss}ert, obwohl die Bewegungen, die e{\s} au{\s}f\"uhrt,
etwa{\s} langsamer sein m\"ussen.

Ein Aviation{\s}-Apparat mu{\ss} demnach so construirt sein, da{\ss}
er diesen Naturgesetzen entspricht und dem Vogel nachgebildet ist,
{\glqq}dem wunderbaren Typu{\s} der Fortbewegung in der Luft{\grqq},
wie Doctor Marey im franz\"osischen Institut sich au{\s}gedr\"uckt
hat.

Die Apparate nun, welche allein diese{\s} Problem zu l\"osen
verm\"ogen, zerfallen in folgende drei Arten:

1. Die Helicopteren oder Spiraliferen, welche nur au{\s} Schrauben
mit verticalen Achsen bestehen.

2. Die Orthopteren, da{\s} sind Maschinen, welche ganz den
nat\"urlichen Flug der V\"ogel nachzuahmen bestimmt sind.

3. Die Aeroplanen, in Wirklichkeit geneigte Ebenen, wie die
Papierdrachen der Knaben, aber von horizontalen Schrauben getrieben
oder gezogen.

Jede{\s} dieser Systeme hatte und hat selbst noch heute entschiedene
Anh\"anger, welche ihre Ansichten unab\"anderlich al{\s} die
richtigen hinstellen.

Au{\s} mancherlei Gr\"unden hatte Robur jedoch die beiden letzteren
verworfen.

E{\s} unterliegt zwar keinem Zweifel, da{\ss} die Orthoptere, der
mechanische Vogel, gewisse Vorz\"uge aufweist. Die Arbeiten
Renaud'{\s} haben daf\"ur 1884 den Bewei{\s} beigebracht. Doch man
darf, wie dem Genannten auch eingeworfen wurde, die Natur niemal{\s}
sclavisch nachahmen wollen. Die Locomotiven sind auch nicht nach dem
Vorbilde der Hasen und die Dampfschiffe nicht nach dem der Fische
gebaut. Den ersteren gab man R\"ader, welche doch keine Beine sind,
dem letzteren Schrauben, welche gewi{\ss} nicht den Flossen
entsprechen. Und wir meinen, Beide laufen doch recht gut.
Uebrigen{\s} wei{\ss} man noch gar nicht genau, wie der Flug der
V\"ogel, welche sehr complicirte Bewegungen au{\s}f\"uhren,
eigentlich zu Stande kommt. Doctor Marey z.~B. hat die Vermuthung
au{\s}gesprochen, da{\ss} die Federn der Fl\"ugel sich beim Aufschlag
\"offnen, um die Luft durchzulassen -- ein Vorgang, der durch eine
k\"unstliche Maschine schwerlich herzustellen sein m\"ochte.

Andererseit{\s} ist e{\s} nicht zweifelhaft, da{\ss} auch die
Aeroplane einige gute Erfolge aufzuweisen haben. Setzen die Schrauben
den Luftschichten eine schiefe Ebene entgegen, so m\"u{\ss}te
darau{\s} eine ansteigende Bewegung hervorgehen, und kleine
Versuch{\s}-Apparate haben bewiesen, da{\ss} da{\s} di{\s}ponible
Gewicht, da{\s}jenige, \"uber welche{\s} man noch \"uber da{\s}
Eigengewicht de{\s} Apparate{\s} hinau{\s} verf\"ugen k\"onnte, mit
dem Quadrate der Geschwindigkeit zunahm. Hierin liegt offenbar ein
gro{\ss}er Vortheil, der die der Aerostaten, welche aufgetrieben
werden sollen, weit \"ubertrifft.

Nicht{\s}destoweniger glaubte Robur, da{\ss}, wenn e{\s} etwa{\s}
noch Bessere{\s} g\"abe, da{\s} auch noch einfacher sein m\"usse.
Auch die Schrauben -- die ihm im Weldon-Institut durch ein
gl\"uckliche{\s} Wortspiel direct an den Kopf geworfen worden waren
-- konnten ja wohl allen Anspr\"uchen an seine Flugmaschine
gen\"ugen. Die Einen sollten dieselbe in der Luft schwebend erhalten,
die Anderen sie unter den besten Verh\"altnissen der Schnelligkeit
und Sicherheit in der Horizontalen fortbewegen.

Theoretisch m\"u{\ss}te man ja mittelst einer nur kurzen, aber in der
Fl\"ache gro{\ss}en Schraube, wie Victor Tatin e{\s} zuerst
au{\s}gesprochen hatte, dahin gelangen, {\glqq}wenn man e{\s} auf
da{\s} Aeu{\ss}erste triebe, in ungeheure{\s} Gewicht durch
verschwindend kleine Kraft zu heben{\grqq}.

Wenn die Orthoptere -- durch Fl\"ugelschl\"age gleich einem Vogel --
sich erhebt, indem sie sich in normaler Weise gegen die Luft
st\"urzt, so steigt die Helioptere auf, indem sie mit ihren
Schraubenfl\"ugeln schr\"ag dagegen wirkt, al{\s} ob sie eine
geneigte Ebene hinaufklimme. Im Grunde hat sie ja nur
schraubenf\"ormig gewundene, an Stelle der schaufelartigen Fl\"ugel.
Die Schraube bewegt sich nothwendig in der Richtung ihrer Achse fort.
Ist diese vertical, so bewegt sie sich vertical; ist sie horizontal
angebracht, so treibt sie in horizontalem Sinne.

Der gro{\ss}e Flugapparat de{\s} Ingenieur Robur beruhte nur auf
diesen beiden Wirkungen.

Wir geben hier eine genaue Beschreibung de{\s}selben, welche
f\"uglich in drei Theile zerfallen kann, betreffend da{\s} Verdeck,
die Schwebe- und Treibmaschinen und die \"au{\ss}ere Maschinerie.

Verdeck. Da{\s} war ein drei{\ss}ig Meter lange{\s}, vier Meter
breite{\s} Bauwerk, ein wirkliche{\s} Schiff{\s}deck mit Vordersteven
in Form eine{\s} Rammsporn{\s}. Darunter w\"olbte sich der
festgef\"ugte Rumpf, der die Apparate zur Erzeugung der mechanischen
Kraft barg, ferner befanden sich da die Pulverkammer, die Werkzeuge,
da{\s} allgemeine Magazin f\"ur Vorr\"athe aller Art, darunter auch
die Wassergef\"a{\ss}e de{\s} Fahrzeug{\s}. Rund herum standen
leichte Pfosten, die, mit Eisendraht unter einander verbunden, die
Reeling bildeten. Darauf aber erhoben sich drei
Ruff{\s}\footnote[1]{\frakfamily Ruff{\s} nennt man auf Seeschiffen die
auf Deck stehenden kleinen Holzbauwerke, welche al{\s} Wohnung f\"ur
die Mannschaft dienen.}, deren R\"aumlichkeiten zur Wohnung f\"ur
da{\s} Personal und f\"ur die Maschinerie bestimmt sind. Im mittleren
Ruff arbeitet die Maschine, welche da{\s} Schwebewerk in Bewegung
setzt; in dem vorderen die Maschine f\"ur die vordere Treibschraube,
im hinteren die f\"ur die hintere Schraube, so da{\ss} diese drei von
einander v\"ollig unabh\"angig wirken. Nahe dem Vordertheile befindet
sich der Ruff f\"ur die Werkstatt die K\"uche und die
Mannschaft{\s}wohnung. Nahe dem Heck, im letzten Ruff, finden sich
mehrere Cabinen, unter anderen die de{\s} Ingenieur{\s}, ein
Speisezimmer und dar\"uber ein kleiner verglaster Raum, in dem sich
der Steuermann aufh\"alt, der da{\s} Ganze mittelst eine{\s}
gewaltigen Steuerrade{\s} lenkt und leitet. Alle diese Ruff{\s}
werden durch Lichtpforten erhellt, deren Scheiben au{\s} Hartgla{\s}
bestehen, da{\s} eine zehnfach gr\"o{\ss}ere
Widerstand{\s}f\"ahigkeit al{\s} gew\"ohnliche{\s} Gla{\s} hat.
Unterhalb de{\s} Rumpfe{\s} ist ein System von biegsamen Federn
angebracht, dazu bestimmt, etwaige St\"o{\ss}e zu mildern, obwohl
eine Landung ungemein sanft bewerkstelligt werden konnte, so sehr war
der Ingenieur Herr seine{\s} Apparate{\s}.

Auftrieb{\s}- und Treibmaschinen. Auf dem Verdeck erhoben sich
lothrecht siebenunddrei{\ss}ig Achsen, von denen je f\"unfzehn an
Back- und Steuerbord und sieben h\"ohere in der Mitte errichtet
waren, so da{\ss} da{\s} Ganze einem Schiffe mit siebenunddrei{\ss}ig
Masten \"ahnlich wurde; nur trugen diese Masten an Stelle der Segel
jeder zwei horizontale Schrauben von kurzer Steigung und geringem
Durchmesser, denen aber eine ungeheure Umdrehung{\s}geschwindigkeit
ertheilt werden konnte. Jede dieser Achsen empf\"angt ihre Bewegung
unabh\"angig von der anderen, und au{\ss}erdem drehen sich je zwei
und zwei in entgegengesetztem Sinne -- eine nothwendige Ma{\ss}regel,
um den ganzen Apparat nicht in wellenf\"ormige Bewegung kommen zu
lassen. Auf diese Weise aber halten sich alle Schrauben, w\"ahrend
sie eine wie die andere auf verticalen Lufts\"aulen emporzusteigen
streben, gegenseitig da{\s} Gleichgewicht. Der Apparat ist demnach
mit vierundsiebenzig Schwebeschrauben au{\s}ger\"ustet, deren drei
Arme \"au{\ss}erlich durch einen Metallring zusammengehalten werden,
der, al{\s} Schwungrad wirkend, die motorische Kraft besser
au{\s}n\"utzen hilft. Am Vorder- wie am Hintertheile drehen sich, auf
horizontalen Achsen montirt, zwei Treibschrauben mit vier Armen jede,
welche der Fortbewegung de{\s} Ganzen vorstehen. Diese Schrauben,
welche an Durchmesser die Auftriebsschrauben \"ubertreffen, k\"onnen
sich ebenfall{\s} mit der gr\"o{\ss}ten Geschwindigkeit drehen.

Alle{\s} in Allem schlie{\ss}t sich dieser Apparat an die Systeme an,
welche von Cossu{\s}, de la Landelle und de Ponton d'Am\'ecourt
aufgestellt und vom Ingenieur Robur verbessert wurden. Dagegen
geb\"uhrt ihm bez\"uglich der Wahl und der Verwendung der motorischen
Kraft unbedingt der Ruhm de{\s} Erfinder{\s}.

Maschinerie. Weder vom Dampf de{\s} Wasser{\s} oder anderer
Fl\"ussigkeiten, weder von der comprimirten Luft oder anderen
elastischen Gasen und endlich ebenso wenig von explosiven Gemischen,
welche f\"ahig sind, eine mechanische Wirkung au{\s}zu\"uben,
entlehnte Robur die nothwendige Kraft, seinen Apparat schwebend zu
erhalten und fort\/zubewegen, er nahm dazu die Elektrizit\"at in
Anspruch, jene Naturkraft, welche dereinst die Seele der
industriellen Welt zu werden verspricht. Eine dynamo-elektrische
Maschine verwendete er zur Erzeugung derselben jedoch nicht, sondern
nur Batterien und Accumulatoren; nur war e{\s} Robur'{\s}
Geheimni{\ss}, welche Materialien er zur Zusammenstellung seiner
Elemente und welche S\"auren er zur Erregung derselben anwendete;
da{\s}selbe galt f\"ur die Accumulatoren. Niemand wu{\ss}te,
worau{\s} deren positive und negative Platten bestanden. Der
Ingenieur hatte sich au{\s} gewissen Gr\"unden wohl geh\"utet, darauf
ein Patent zu nehmen. Unbestreitbar aber zeigten seine Batterien eine
au{\ss}erordentliche Ergiebigkeit, die S\"auren eine fast
vollst\"andige Widerstand{\s}f\"ahigkeit gegen Verdunstung und
Frieren, seine Accumulatoren eine unverkennbare Ueberlegenheit \"uber
die von Faure, Sellon, Volckmar, und endlich lieferten seine Str\"ome
Amp\`ere{\s} von bi{\s}her unerreichter Anzahl. Darau{\s} aber ergab
sich eine so zu sagen unendliche Menge elektrischer Pferdekr\"afte
zur Bewegung der Schrauben, welche dem Apparate eine seinen
Bed\"urfnissen weit \"uberlegene Schwebe- und Triebkraft unter allen
Umst\"anden verliehen.

Wir wiederholen jedoch, da{\s} war die Sache de{\s} Ingenieur Robur
und dar\"uber bewahrte er ein unverbr\"uchliche{\s} Geheimni{\ss},
und wenn der Vorsitzende und der Schriftf\"uhrer de{\s}
Weldon-Institut{\s} nicht da{\s} Gl\"uck haben, da{\s}selbe zu
durchdringen, so d\"urfte e{\s} wahrscheinlich auf immer f\"ur die
Menschheit verloren sein.

E{\s} versteht sich von selbst, da{\ss} dieser Apparat infolge der
gl\"ucklich gew\"ahlten Lage seine{\s} Schwerpunkte{\s} hinreichende
Stabilit\"at zeigte. Man brauchte also niemal{\s} zu f\"urchten,
da{\ss} er mit der horizontalen bedenkliche Winkel bilden oder gar
umschlagen k\"onnte.

E{\s} er\"ubrigt noch mit\/zutheilen, au{\s} welchem Material der
Ingenieur Robur seinen Aeronef hergestellt hatte, ein Name, der f\"ur
den {\glqq}Albatro{\s}{\grqq} besonder{\s} geeignet erscheint.
Welche{\s} war der so harte Stoff, da{\ss} da{\s} Bowie-Messer Phil
Evan{\s}' ihn nicht zu ritzen und dessen Natur Onkel Prudent nicht zu
erkennen vermochte? Ganz einfach -- Papier!

Schon eine Reihe von Jahren hatte die Fabrikation de{\s}selben
gro{\ss}e Au{\s}dehnung gewonnen. Ungeleimte{\s} Papier, dessen
Bl\"atter mit Dextrin und St\"arkemehl impr\"agnirt wurden, bildet
unter der Wirkung der hydraulischen Presse ein Material von der
H\"arte de{\s} Stahl{\s}. Man erzeugt darau{\s} Rollen, Schienen und
Wagenr\"ader, welche haltbarer und gleichzeitig leichter sind, al{\s}
solche au{\s} Metall. Diese gro{\ss}e Haltbarkeit bei
au{\ss}erordentlicher Leichtigkeit eben hatte Robur bei der Erbauung
seiner Luftlocomotive zu ben\"utzen gesucht. Alle{\s}, Rumpf, Rippen,
Ruff{\s}, Cabinen, bestand au{\s} Strohpapier, da{\s} unter hohem
Drucke metall\"ahnlich geworden war und da{\s} sich -- ein Umstand,
bei einem in gro{\ss}er H\"ohe dahinschwebenden Apparate gewi{\ss}
von Werth -- auch al{\s} unverbrennlich erwie{\s}.

F\"ur die verschiedenen Theile der Schwebe- und Treibmaschinerie, wie
f\"ur die Fl\"ugel der Schrauben hatte gelatinierte{\s} Fasergewebe
al{\s} ebenso widerstand{\s}f\"ahige{\s}, wie biegsame{\s} Material
gedient und von diesem auch, da e{\s} sich allen Formen anpa{\ss}te,
in den meisten Gasen und Fl\"ussigkeiten, S\"auren und Salzen
unl\"o{\s}lich war -- ohne von seinen isolirenden Eigenschaften zu
sprechen -- in der elektrischen Maschinerie de{\s}
{\glqq}Albatro{\s}{\grqq} der au{\s}gedehnteste Gebrauch gemacht
worden.

Der Ingenieur Robur, sein Obersteuermann Tom Turner, ein Mechaniker
mit zwei Gehilfen, zwei Boot{\s}m\"anner und ein Koch -- Alle{\s} in
Allem acht Personen -- bildeten die ganze Besatzung de{\s}
{\glqq}Albatro{\s}{\grqq}, welche f\"ur die bei der Fahrt durch die
Luft n\"othigen Man\"over \"ubrig au{\s}reichte. Jagd- und
Krieg{\s}waffen, Fischerger\"athe, elektrische Lampen,
Beobachtung{\s}instrumente, Boussolen und Sextanten zur Erkennung der
Fahrtrichtung, Thermometer zur Messung der Temperatur; verschiedene
Barometer, die einen, um die erreichte H\"ohe, die anderen, um die
Ver\"anderung de{\s} Luftdrucke{\s} zu messen, ein
{\glqq}Stormgla{\s}{\grqq} zum Erkennen drohender St\"urme, eine
kleine Bibliothek, eine kleine tragbare Druckerei, ein Gesch\"utz auf
Zapfen, in der Mitte de{\s} Deck{\s}, da{\s}, von r\"uckw\"art{\s}
geladen, ein Gescho{\ss} von sech{\s} Centimeter Durchmesser
schleuderte; der n\"othige Vorrath an Pulver, Kugeln,
Dynamitpatronen; eine K\"uche, in der die von Accumulatoren
gelieferten Str\"ome zur Feuerung dienten, ein Vorrath von Conserven,
Fleisch und Gem\"use, die in einer Camb\"use \begin{antiqua}ad
hoc\end{antiqua} neben Gef\"a{\ss}en mit Brandy, Whi{\s}ky und Gin
aufgestapelt waren, endlich Alle{\s}, wa{\s} w\"ahrend einiger Monate
gebraucht werden konnte, ohne landen zu m\"ussen -- da{\s} bildete
da{\s} Material und die Vorr\"athe de{\s} {\glqq}Albatro{\s}{\grqq}
-- ohne die ber\"uhmte Trompete zu rechnen.

Au{\ss}erdem befand sich ein leichte{\s}, unversenkbare{\s}
Kautschukboot an Bord, da{\s} acht Mann auf einem Flusse, einem See
oder auch auf ruhigem Meer tragen konnte. Fallschirme in
Vorau{\s}sicht eine{\s} eintretenden Ungl\"uck{\s} f\"uhrte Robur
nicht mit sich. Er glaubte nicht an Unf\"alle dieser Art. Die Achsen
der Schrauben waren alle von einander unabh\"angig; der Stillstand
der einen blieb auf die \"ubrigen ohne Einflu{\ss}. Selbst wenn nur
da{\s} halbe Triebwerk in Gang blieb, reichte e{\s} schon au{\s}, den
{\glqq}Albatro{\s}{\grqq} in seinem nat\"urlichen Element zu
erhalten.

{\glqq}Und mit ihm, wie Robur der Sieger Gelegenheit fand gegen seine
Passagiere -- Passagiere wider Willen -- zu \"au{\ss}ern, mit ihm bin
ich Herr jene{\s} siebenten Welttheile{\s}, der an Gr\"o{\ss}e
Australien und Afrika, Oceanien, Asien, Amerika und Europa
\"ubertrifft, jene{\s} Icarien{\s} der Luft, da{\s} eine{\s} Tage{\s}
noch Tausende von Icarussen bev\"olkern werden!{\grqq}



\newpage\begin{center}\label{kap07}
{\large \begin{antiqua}VII.\end{antiqua}\\
In welchem der Onkel Prudent und Phil Evan{\s} sich noch immer nicht
\"uberzeugen lassen wollen.\\\bigskip}
\end{center}



Der Vorsitzende de{\s} Weldon-Institut{\s} war h\"ochst erstaunt,
sein Gef\"ahrte geradezu verbl\"ufft. Aber weder der Eine, noch der
Andere wollte sich diese so nat\"urliche Regung anmerken lassen. Der
Diener Frycollin dagegen verheimlichte sein Entsetzen nicht, sich an
Bord einer solchen Maschine in den Luftraum entf\"uhrt zu sehen, im
Gegentheil, er gab da{\s} offen zu erkennen.

Inzwischen drehten sich die Schwebe- oder Auftriebschrauben hastig
\"uber ihren K\"opfen. So schnell diese Bewegung auch vor sich ging,
h\"atte sie doch noch um da{\s} Dreifache gesteigert werden k\"onnen,
im Fall der {\glqq}Albatro{\s}{\grqq} h\"ohere Zonen erreichen
wollte.

Die beiden eigentlichen Propeller, die jetzt einen mittelm\"a{\ss}igen
Gang zeigten, verliehen dem Apparate nur eine Fortbewegung von
zwanzig Kilometern in der Stunde.

Sich \"uber da{\s} Verdeck hinau{\s}beugend, konnten die Passagiere
de{\s} {\glqq}Albatro{\s}{\grqq} ein lange{\s}, gewundene{\s},
fl\"ussige{\s} Band wahrnehmen, da{\s} sich gleich einem B\"achlein
durch wellenf\"ormige{\s} Land schl\"angelte, in dem auch einzelne
kleinere Seen die Strahlen der Sonne gl\"anzend widerspiegelten.
Dieser Bach war \"ubrigen{\s} ein Flu{\ss}, und zwar einer der
bedeutendsten de{\s} betreffenden Gebiet{\s}. An seinem linken Ufer
erhob sich eine Bergkette, deren Fortsetzung sich \"uber
Gesicht{\s}weite hinau{\s} verlor.

{\glqq}Werden Sie un{\s} wohl sagen, wo wir un{\s} befinden? fragte
Onkel Prudent mit einer vor Ingrimm zitternden Stimme.

-- Ich habe dazu gar keine Veranlassung, antwortete Robur.

-- Und werden Sie un{\s} sagen, wohin wir fahren? setzte Phil
Evan{\s} hinzu.

-- Durch den Luftraum.

-- Und da{\s} dauert, wie lange?~...

-- So lange e{\s} Zeit erfordert.

-- Wollen Sie etwa eine Reise um die Erde mit un{\s} machen? fragte
Phil Evan{\s} ironisch.

-- Noch mehr al{\s} da{\s}, erwiderte Robur.

-- Und wenn eine solche Reise un{\s} nicht pa{\ss}t? ... versetzte
Onkel Prudent.

-- So wird sie ihnen eben passen m\"ussen!{\grqq}

Da{\s} gab einen kleinen Vorgeschmack von den Beziehungen, die sich
zwischen dem Herrn de{\s} {\glqq}Albatro{\s}{\grqq} und seinen
G\"asten, um nicht zu sagen, seinen Gefangenen, zu entwickeln
versprachen. Offenbar wollte Jener ihnen jedoch erst Zeit g\"onnen,
sich zu sammeln, den au{\ss}erordentlichen Apparat zu bewundern, der
sie durch die L\"ufte trug, und ohne Zweifel auch, um dem Erfinder
ihre Gl\"uckw\"unsche darbringen zu k\"onnen. Dieser schlenderte
scheinbar ziello{\s} von einem Ende de{\s} Verdeck{\s} zum anderen;
sie dagegen hatten vollst\"andige Freiheit, die Anordnung der
Maschinerie und die Gesammtau{\s}r\"ustung de{\s}
{\glqq}Aeronef{\s}{\grqq} zu betrachten, oder auch ihre
Aufmerksamkeit ungetheilt der Landschaft zuzuwenden, deren Relief
sich unter ihren F\"u{\ss}en so zu sagen aufrollte.

{\glqq}Onkel Prudent, begann da Phil Evan{\s}, wenn ich mich nicht
t\"ausche, m\"ussen wir \"uber den mittleren Gebiet{\s}theilen von
Canada hinschweben. Jener nach Nordwest verlaufende Flu{\ss} ist
wahrscheinlich der St.~Lorenz, und die Stadt, die wir da hinter
un{\s} lassen, ist Quebec.{\grqq}

E{\s} war in der That die alte Stadt Champlain'{\s}, deren
Wei{\ss}blechd\"acher wie Reflectoren in der Sonne gl\"anzten. Der
{\glqq}Albatro{\s}{\grqq} hatte sich bi{\s} zum sech{\s}undvierzigsten
Grade der Breite erhoben -- wa{\s} den vorzeitigen Anbruch de{\s}
Tage{\s} und die abnorme Verl\"angerung de{\s} Morgenrothe{\s}
hinl\"anglich erkl\"arte.

{\glqq}Ja, fuhr Phil Evan{\s} fort, da liegt ja die Stadt in der
Gestalt eine{\s} Amphitheater{\s}, der H\"ugel, der ihre Citadelle
tr\"agt, diese{\s} Gibraltar Nordamerika{\s}! Dort erheben sich die
englische und die franz\"osische Hauptkirche, und da wieder da{\s}
Zollamt mit seiner Kuppel und der englischen Fahne darauf!{\grqq}

Phil Evan{\s} hatte kaum au{\s}gesprochen, al{\s} die Hauptstadt
Canada{\s} schon wieder in der Ferne zu verschwinden begann. Der
Aeronef trat in eine Schicht kleinere Wolken ein, welche den
Au{\s}blick nach der Erde allm\"ahlich verhinderten.

Al{\s} Robur jetzt sah, da{\ss} der Vorsitzende und Schriftf\"uhrer
de{\s} Weldon-Institut{\s} ihre Aufmerksamkeit der \"au{\ss}eren
Construction de{\s} {\glqq}Albatro{\s}{\grqq} zuwendeten, trat er auf
sie zu und sagte:

{\glqq}Nun, meine Herren, glauben Sie endlich an die M\"oglichkeit
einer Fortbewegung durch die Luft mittelst Apparaten, die schwerer
sind al{\s} jene?{\grqq}

E{\s} w\"are ja schwierig gewesen, sich dem augenscheinlichen Beweise
zu widersetzen. Onkel Prudent und Phil Evan{\s} gaben jedoch keine
Antwort.

{\glqq}Sie schweigen? fuhr der Ingenieur fort. Aha, jedenfall{\s}
verhindert Sie der Hunger am Sprechen ... doch, wenn ich e{\s}
unternommen habe, Sie durch die Luft zu tran{\s}portiren, so glauben
Sie nicht, da{\ss} ich Sie auch mit diesem wenig nahrhaften Fluidum
ern\"ahren wollte. Ihr erste{\s} Fr\"uhst\"uck erwartet Sie.{\grqq}

Da Onkel Prudent und Phil Evan{\s} einen schon recht qu\"alenden
Hunger versp\"urten, hatten sie hier keine Veranlassung, Umst\"ande
zu machen. Eine Mahlzeit verpflichtet ja noch zu nicht{\s}, und wenn
Robur sie erst wieder auf der Erde abgesetzt h\"atte, rechneten sie
nach wie vor darauf, ihm gegen\"uber auch ihre ganze
Handlung{\s}freiheit wieder zu erhalten.

Beide wurden nach dem hinteren Ruff geleitet und nach einem kleinen
\begin{antiqua}dining-room\end{antiqua}, in dem sich ein sauber
gedeckter Tisch befand, an welchem sie w\"ahrend der Fahrt speisen
sollten. An Gerichten trug derselbe verschiedene Conserven und unter
Anderem eine Art Brot au{\s} gleichen Theilen Mehl und pulverisirtem
Fleisch, untermischt mit ein wenig Speck, welche{\s}, in Wasser
gekocht, eine vorz\"ugliche nahrhafte Suppe liefert; ferner Schnitte
von ger\"auchertem Schinken und al{\s} Getr\"ank Thee.

Auch Frycollin war nicht vergessen worden. Auf dem Vordertheil
erhielt er eine t\"uchtige Brotsuppe. Wahrlich, er mu{\ss}te
gewaltigen Hunger haben, um essen zu k\"onnen, denn seine Kinnladen
zitterten eigentlich au{\s} Furcht und h\"atten ihm jeden anderen
Dienst versagt.

{\glqq}Wenn da{\s} ent\/zwei ginge! Wenn da{\s} ent\/zwei ginge!{\grqq}
wiederholte der ungl\"uckliche Neger. Da{\s} machte ihm fortw\"ahrend
Angst. Aber man denke nur ... ein Sturz von f\"unfzehnhundert Meter,
der ihn in Pulver verwandelt h\"atte!

Nach Verlauf einer Stunde erschienen Onkel Prudent und Phil Evan{\s}
wieder auf dem Verdeck. Robur war nicht mehr hier. Am Hintertheile
folgte der Steuermann in seinem Gla{\s}h\"au{\s}chen, da{\s} Auge auf
den Compa{\ss} gerichtet, unentwegt dem ihm vom Ingenieur
vorgezeichneten Curse.

Die andere Mannschaft mochte wohl auch durch da{\s} Fr\"uhst\"uck in
ihrem Logi{\s} zur\"uckgehalten werden. Nur ein Hilf{\s}mechaniker,
dem nun die Ueberwachung der Maschinen oblag, wanderte von einem Ruff
zum anderen umher.

War die Geschwindigkeit de{\s} Apparat{\s} jetzt auch eine gro{\ss}e,
so konnten die beiden Collegen dar\"uber doch nur unvollkommen
urtheilen, obgleich der {\glqq}Albatro{\s}{\grqq} au{\s} jener
Wolkenschicht wieder hervorgetreten war und sich der Erdboden
f\"unfzehnhundert Meter unter ihnen deutlich zeigte.

{\glqq}Man kann eigentlich gar nicht daran glauben! bemerkte Phil
Evan{\s}.

-- So glauben wir nicht daran,{\grqq} antwortete Onkel Prudent.

Sie begaben sich hiermit nach dem Vorderdeck und lie{\ss}en die
Blicke \"uber den Horizont im Westen schweifen.

{\glqq}Ah, eine andere Stadt! rief Phil Evan{\s}.

-- K\"onnen Sie dieselbe erkennen?

-- Ja, e{\s} scheint mir Montreal zu sein.

-- Montreal? ... Aber wir haben doch Quebeck vor kaum zwei Stunden
verlassen!

-- Da{\s} beweist, da{\ss} diese Maschine sich mit einer
Geschwindigkeit von mindesten{\s} f\"unfundzwanzig Lieue{\s} die
Stunde bewegt.{\grqq}

Da{\s} war in der That die Gr\"o{\ss}e der Geschwindigkeit de{\s}
{\glqq}Albatro{\s}{\grqq}, und wenn die Passagiere davon keine
Bel\"astigung versp\"urten, lag da{\s} daran, da{\ss} sie mit dem
Winde forttrieben. Bei stillem Wetter h\"atte sie diese Schnelligkeit
schon merkbar genirt, weil sie fast der eine{\s} Expre{\ss}zuge{\s}
gleichkommt. Bei widrigem Winde w\"are dieselbe ganz unertr\"aglich
gewesen.

Phil Evan{\s} t\"auschte sich nicht. Unter dem {\glqq}Albatro{\s}{\grqq}
erschien Montreal, da{\s} an seiner Victoria-Br\"ucke, einer
R\"ohrenbr\"ucke \"uber den St.~Lorenz gleich dem Bahnviaduct \"uber
die Lagunen von Venedig, leicht kenntlich war. Bald unterschied man
auch seine breiten Stra{\ss}en, die ungeheuren Magazine, die
Pal\"aste der Banken, die Kathedrale, eine neuerding{\s} nach dem
Vorbilde de{\s} St.~Peter{\s}-Dome{\s} in Rom erbaute Basilica und
endlich den Mont-Royal, der die ganze Stadt \"uberragt und zu einem
herrlichen Park umgeschaffen ist.

E{\s} war ein Gl\"uck zu nennen, da{\ss} Phil Evan{\s} die Hauptstadt
Canada{\s} schon fr\"uher einmal besucht hatte. Er konnte so
Mehrere{\s} erkennen, ohne Robur erst zu fragen. Nach Montreal kamen
sie etwa einhalb zwei Uhr Nachmittag{\s}, \"uber Ottawa hinweg,
dessen F\"alle, von oben gesehen, einem ungeheuren Siedekessel
glichen, der durch sein furchtbare{\s} Uebersch\"aumen einen
gro{\ss}artigen Effect hervorbrachte.

{\glqq}Da ist der Parlament{\s}-Palast,{\grqq} sagte Phil Evan{\s}.

Er wie{\s} bei diesen Worten nach einer Art N\"urnberger Spielzeug,
da{\s} auf einem H\"ugel verloren war. Diese{\s} Spielzeug mit seiner
vielfarbigen Architektur glich dem Parlamenthau{\s} zu London, wie
die Kathedrale von Montreal der Peter{\s}-Kirche zu Rom. Doch
nicht{\s}destoweniger war und blieb die in Sicht befindliche Stadt
eben Ottawa.

Auch diese schien von dem Standpunkte der Beschauer au{\s} schnell
dem Horizonte zuzueilen und bildete bald nur einen etwa{\s} helleren
Fleck auf der Erde.

E{\s} mochte gegen zwei Uhr sein, al{\s} Robur wieder erschien. Sein
Obersteuermann Tom Turner begleitete ihn. Er sagte zu diesem nur drei
Worte. Letzter \"ubermittelte dieselben den beiden Gehilfen im
Vorder- und im Hinterruff. Auf ein Zeichen ver\"anderte der
Steuermann die Richtung de{\s} {\glqq}Albatro{\s}{\grqq}, so da{\ss}
dieser um zwei Grade nach S\"udwesten abwich. Gleichzeitig konnten
Onkel Prudent und Phil Evan{\s} wahrnehmen, da{\ss} den
Treibschrauben de{\s} Aeronef{\s} eine gr\"o{\ss}ere Schnelligkeit
verliehen wurde.

Diese Schnelligkeit h\"atte in Wirklichkeit noch verdoppelt werden
k\"onnen, und man h\"atte damit eine Maschine erhalten, welche alle
Erdmaschinen weit hinter sich lassen mu{\ss}te.

Man urtheile selbst: Torpedoboote k\"onnen zwanzig Knoten oder
vierzig Kilometer in der Stunde zur\"ucklegen. Die schnellsten
Eisenbahnz\"uge bringen e{\s} wohl bi{\s} auf hundert; Schlittenboote
auf den \"ubereisten Seen der Vereinigten Staaten bi{\s} auf
hundertf\"unfzehn; eine in der Werkstatt Patterson'{\s} erbaute
Maschine mit Zahnr\"adern hat \"uber den Erie-See hinweg
hundertdrei{\ss}ig erreicht und eine andere Locomotive zwischen
Trenton und Jersey gar hundertsiebenunddrei{\ss}ig Kilometer.

Der {\glqq}Albatro{\s}{\grqq} aber konnte beim Maximum seiner
Kraft\"au{\ss}erung mittelst seiner Treibschrauben sich zweihundert
Kilometer in der Stunde, da{\s} hei{\ss}t fast f\"unfzig Meter in der
Secunde, fortbewegen.

Eine solche Schnelligkeit aber ist die de{\s} Orkan{\s}, der B\"aume
entwurzelt, die jene{\s} Windsto{\ss}e{\s}, der bei dem Sturm vom 21.
September 1881 in Cahor{\s} hundertvierundneunzig Kilometer in der
Stunde dahinraste. E{\s} ist die mittlere Geschwindigkeit der
Brieftaube, welche nur noch von der gew\"ohnlichen Schwalbe (mit
siebenundsechzig Metern in der Secunde) und von der Mauerschwalbe
(mit neunundacht\/zig Metern) \"ubertroffen wird.

Mit einem Worte, und wie Robur gesagt, der {\glqq}Albatro{\s}{\grqq}
h\"atte bei Entwickelung der ganzen Kraft seiner Schrauben die Fahrt
um die Erde binnen zweihundert Stunden, d.~h. also in noch nicht acht
Tagen zur\"ucklegen k\"onnen.

Ob die Erde jener Zeit schon 450.000 Kilometer Eisenstra{\ss}en
besa{\ss}, d.~h. eine L\"ange, welche elfmal den Aequator umspannt
h\"atte -- so blieb da{\s} doch f\"ur diese fliegende Maschine ohne
Bedeutung. Stand ihr nicht da{\s} ganze gro{\ss}e Luftmeer offen?

Brauchen wir jetzt noch mehr hinzuzuf\"ugen? Jene{\s} Ph\"anomen,
dessen Erscheinung die ganze Alte und Neue Welt in Aufruhr versetzt
hatte, war der Aeronef de{\s} Ingenieur{\s}. Jene Trompete, welche
die schmetternden Fanfaren in den L\"uften ert\"onen lie{\ss}, war
die de{\s} Obersteuermann{\s} Tom Turner. Die Flaggen, welche man auf
den Hauptbauwerken Europa{\s}, Asien{\s} und Amerika{\s} aufgepflanzt
gefunden hatte, war die Flagge Robur'{\s} de{\s} Sieger{\s} und
seine{\s} {\glqq}Albatro{\s}{\grqq}.

Und wenn der Ingenieur bi{\s}her einige Vorsicht beobachtet hatte, um
nicht erkannt zu werden, wenn er mit Vorliebe nur in der Nacht
gefahren war, die er zuweilen durch jene elektrischen Lichtstr\"ome
erhellte, w\"ahrend er den Tag \"uber jenseit{\s} der Wolken zu
verschwinden trachtete, so schien er sein Geheimni{\ss} doch jetzt
nicht mehr bewahren zu wollen. Denn al{\s} er nach Philadelphia
gekommen war und sich in dem Sitzung{\s}saale de{\s}
Weldon-Institut{\s} vorgestellt hatte, konnte er da etwa{\s}
Andere{\s} beabsichtigen, al{\s} die Bekanntgebung seiner wunderbaren
Entdeckung, um selbst die Ungl\"aubigsten \begin{antiqua}ipso
facto\end{antiqua} zu \"uberzeugen?

Wir wissen, wie er hier aufgenommen wurde, und werden sehen, welche
Repressalien er gegen\"uber dem Vorsitzenden und dem Schriftf\"uhrer
de{\s} bekannten Club{\s} zu ergreifen gedachte.

Inzwischen hatte sich Robur den beiden M\"annern gen\"ahert. Diese
stellten sich noch immer, al{\s} erstaunten sie nicht im geringsten
\"uber da{\s}, wa{\s} sie vor sich sahen: offenbar setzte sich
allm\"ahlich unter diesen beiden angels\"achsischen Sch\"adeln ein
Starrsinn fest, der nur schwierig au{\s}zurotten sein w\"urde.

Robur seinerseit{\s} wollte sich auch nicht den Anschein geben,
al{\s} fiele ihm da{\s} auf, und begann de{\s}halb, al{\s} setze er
nur ein Gespr\"ach fort, da{\s} doch schon seit zwei Stunden
unterbrochen war:

{\glqq}Sie haben sich ohne Zweifel damit befa{\ss}t, meine Herren, ob
dieser f\"ur die Bewegung durch die Luft au{\s}gezeichnete Apparat
auch eine noch gr\"o{\ss}ere Geschwindigkeit annehmen k\"onne. Er
w\"are inde{\ss} nicht w\"urdig, den Luftraum sozusagen besiegt zu
haben, wenn er sich denselben nicht g\"anzlich unterw\"urfig machen
k\"onnte. Ich bin darauf au{\s}gegangen, die Luft al{\s} festen
St\"utz- und Angriff{\s}punkt zu ben\"utzen, und al{\s} solcher dient
sie mir. Ich sah l\"angst ein, da{\ss} man, um gegen den Wind
anzuk\"ampfen, st\"arker sein m\"usse, al{\s} dieser, und ich bin
st\"arker. Ich bedarf keiner Segel, die mich ziehen, keiner Ruder
oder R\"ader, die mich treiben, keiner Schienen, um schneller und
leichter fort\/zukommen -- nur Luft ... nicht{\s} weiter! Luft, die
mich ganz ebenso umgiebt, wie da{\s} Wasser jede{\s} submarine
Fahrzeug, und in der meine Propeller sich drehen, wie die Schrauben
eine{\s} Dampfer{\s}. Da{\s} ist da{\s} ganze Geheimni{\s}, wie ich
da{\s} Problem der Aviation l\"oste; da haben Sie, wa{\s} weder ein
Ballon, noch irgend ein Apparat, der leichter al{\s} die Luft ist,
jemal{\s} leisten wird.{\grqq}

Die beiden Collegen schwiegen still wie da{\s} Grab, ohne da{\ss}
sich der Ingenieur dadurch au{\s} der Fassung bringen lie{\ss}. Er
begn\"ugte sich, verstohlen zu l\"acheln, und fuhr in folgenden
Frages\"atzen fort:

{\glqq}Sie fragen vielleicht, ob der {\glqq}Albatro{\s}{\grqq} mit
dieser Kraft, die ihn horizontal treibt, auch eine gleich
wirkung{\s}volle Kraft verbindet, um sich in verticaler Richtung zu
bewegen, mit einem Worte, ob er, wenn e{\s} sich darum handelte,
gro{\ss}e H\"ohen zu erreichen, werde noch mit einem Aerostaten
wetteifern k\"onnen? Nun, ich w\"urde Ihnen nicht rathen, den
\begin{antiqua}Go a head\end{antiqua} mit ihm um den Prei{\s}
k\"ampfen zu lassen.{\grqq}

Die beiden Collegen zuckten einfach mit den Achseln, da{\s} war
vielleicht der wunde Punkt, an dem sie den Ingenieur fassen zu
k\"onnen glaubten.

Robur gab ein Zeichen. Die Treibschrauben standen sofort still, und
nachdem der {\glqq}Albatro{\s}{\grqq} etwa noch eine Meile in
gleicher Richtung dahin geschwebt war, blieb auch er unbeweglich.

Auf ein zweite{\s} Zeichen Robur'{\s} setzten sich die
Auftrieb{\s}schrauben in Bewegung, und zwar mit einer
Geschwindigkeit, welche man f\"uglich h\"atte mit den zu akustischen
Experimenten ben\"utzten Sirenen vergleichen k\"onnen. Ihr frrr erhob
sich in der Tonleiter um fast eine ganze Octave, w\"ahrend dessen
Intensit\"at wegen der jetzt d\"unneren Luft abnahm, und der Apparat
strebte lothrecht in die H\"ohe, wie eine Lerche, welche ihre hellen
T\"one durch die L\"ufte schmettert.

{\glqq}Herr! Bester Herr! ... rief Frycollin wiederholt, wenn nur
nicht Alle{\s} in St\"ucke geht!{\grqq}

Ein ver\"achtliche{\s} L\"acheln war Robur'{\s} ganze Antwort. Nach
wenigen Minuten hatte der {\glqq}Albatro{\s}{\grqq} eine H\"ohe von
zweitausendsiebenhundert Metern erreicht, wa{\s} den Gesicht{\s}krei{\s}
auf siebenzig Meilen au{\s}dehnte -- dann eine solche von viertausend
Metern, wa{\s} der bi{\s} auf vierhundertacht\/zig Millimeter
herabsinkende Barometer anzeigte.

Nach dieser gelungenen Vorf\"uhrung sank der {\glqq}Albatro{\s}{\grqq}
wieder herab. Die Verminderung de{\s} Druck{\s} in den hohen
Luftschichten bedingt bekanntlich eine starke Abnahme de{\s}
Sauerstoffe{\s} in derselben und de{\s}halb auch im Blute. Da{\s} ist
die Ursache der ernsten Unf\"alle, welche schon Hunderten von
Luftschiffern zugesto{\ss}en sind. Robur aber hielt e{\s} f\"ur
nutzlo{\s}, sich einem solchem au{\s}zusetzen.

Der {\glqq}Albatro{\s}{\grqq} gelangte also nach derjenigen H\"ohe
zur\"uck, die er mit Vorliebe einzuhalten schien, und seine wieder in
Th\"atigkeit gesetzten Treibschrauben f\"uhrten ihn jetzt mit
vermehrter Geschwindigkeit nach S\"udwesten hin.

{\glqq}Jetzt, meine Herren, k\"onnen Sie sich, wenn Sie nur darnach
fragten, selbst Antwort geben.{\grqq}

Hiermit neigte er sich \"uber die Reeling hinau{\s} und blieb so in
Betrachtung versunken stehen.

Al{\s} er den Kopf wieder erhob, waren der Vorsitzende und der
Schriftf\"uhrer de{\s} Weldon-Institut vor ihn hingetreten.

{\glqq}Ingenieur Robur, begann Onkel Prudent, der sich vergeben{\s}
zu bemeistern suchte, da{\s}, wa{\s} Sie zu glauben scheinen, haben
wir un{\s} keine{\s}weg{\s} gefragt. Doch wollen wir Ihnen eine Frage
stellen, auf welche wir jedenfall{\s} Antwort erwarten.

-- Reden Sie!

-- Mit welchem Rechte haben Sie un{\s} in Philadelphia, im
Fairmont-Park \"uberfallen? Mit welchem Rechte un{\s} in jene Zelle
eingeschlossen? Mit welchem Rechte entf\"uhren Sie un{\s} wider
Willen an Bord dieser fliegenden Maschine?

-- Und mit welchem Rechte, meine Herren Ballonisten, entgegnete
Robur, mit welchem Rechte haben Sie mich beleidigt, verspottet; in
Ihrem Club bedroht, und zwar in einer Weise, da{\ss} ich mich selbst
wundere, lebend davon gekommen zu sein.

-- Fragen hei{\ss}t nicht antworten, erwiderte Phil Evan{\s}, und ich
wiederhole Ihnen, mit welchem Rechte handelten Sie?

-- Sie wollen da{\s} wissen?~...

-- Wenn e{\s} Ihnen gef\"allig ist.

-- Nun wohl, mit dem Rechte de{\s} St\"arkeren!

-- Da{\s} ist cynisch!

-- Aber e{\s} ist so.

-- Und wie lange, B\"urger Ingenieur, fragte Onkel Prudent, dem nun
die Geduld au{\s}ging, wie lange denken Sie, diese{\s} Recht un{\s}
gegen\"uber au{\s}zun\"utzen?

-- Aber, meine Herren, antwortete Robur ironisch, wie k\"onnen Sie
nur eine solche Frage stellen, da Sie nur den Blick zu senken
brauchen, um ein Schauspiel zu genie{\ss}en, da{\s} in der Welt nicht
seine{\s} Gleichen findet?{\grqq}

Der {\glqq}Albatro{\s}{\grqq} spiegelte sich eben in der ungeheuren
Fl\"ache de{\s} Ontario-See{\s}, er war eben \"uber da{\s} von Cooper
so hoch poetisch besungene Land gekommen, dann folgte er der
S\"udgrenze diese{\s} weiten Wasserbecken{\s} und wandte sich dem
ber\"uhmten Flusse zu, der ihm die Gew\"asser de{\s} Erie-See{\s},
aber zerst\"aubt im seinen Katarakten, zuf\"uhrt.

Einen Augenblick lang drang ein wahrhaft majest\"atische{\s}
Ger\"ausch, gleich dem Rollen de{\s} Sturme{\s}, bi{\s} zu ihm
hinauf, und al{\s} ob sich ein feuchter Dunst in den L\"uften
verbreitete, wurde die Temperatur merklich k\"uhler.

Tief unten donnerten die ungeheuren Wassermassen in Hufeisenbogen
hinunter. Man glaubte wohl einen gewaltigen Strom von Krystall vor
sich zu sehen, den tausend, durch Refraction au{\s} der Zerlegung
de{\s} Sonnenlichte{\s} entstandene Regenbogen umgl\"anzten. E{\s}
war ein wirklich erhebender Anblick.

Vor diesen F\"allen verband eine, gleich einem Faden au{\s}gespannte
schmale Br\"ucke ein Ufer mit dem anderen. Etwa{\s} weiter unten --
etwa drei Meilen entfernt war eine H\"angebr\"ucke dar\"uber
geschlagen, \"uber welche sich eben ein Bahnzug hinschl\"angelte, der
von dem canadischen Ufer nach dem amerikanischen zu dampfte.

{\glqq}Die Niagaraf\"alle!{\grqq} rief Phil Evan{\s}.

Und dieser Au{\s}ruf entfuhr ihm, w\"ahrend Onkel Prudent sich die
erdenklichste M\"uhe gab, alle Wunder, die sich vor seinen Augen
entrollten, scheinbar nicht zu beachten.

Eine Minute sp\"ater hatte der {\glqq}Albatro{\s}{\grqq} den Strom
\"uberschritten, der die Vereinigten Staaten von der Colonie Canada
scheidet, und er schwebte nun \"uber den unendlich weiten Gebieten
de{\s} n\"ordlichen Amerika.



\newpage\begin{center}\label{kap08}
{\large \begin{antiqua}VIII.\end{antiqua}\\
Worin man sehen wird, da{\ss} Robur sich entschlie{\ss}t, auf 
die ihm vorgelegte wichtige Frage zu antworten.\\\bigskip}
\end{center}



In einer der Cabinen de{\s} Ruff{\s} auf dem Hinterdeck hatten Onkel
Prudent und Phil Evan{\s} zwei vorz\"ugliche Lagerst\"atten,
W\"asche, eine hinreichende Menge Kleidung{\s}st\"ucke zum Wechseln,
nebst M\"anteln und Reisedecken vorgefunden. Kein tran{\s}atlantischer
Dampfer h\"atte ihnen mehr Bequemlichkeiten bieten k\"onnen. Wenn sie
nicht in einem fort schliefen, so lag da{\s} nur in ihrer Absicht,
oder e{\s} hielten sie mindesten{\s} sehr beunruhigende Gedanken
davon zur\"uck. In welch' unberechenbare{\s} Abenteuer waren sie hier
gerathen? Wa{\s} zu erleben und zwar wider ihren Willen zu erleben
stand ihnen Alle{\s} noch bevor? Wie w\"urde die ganze Geschichte
ablaufen und wa{\s} beabsichtigte eigentlich der Ingenieur Robur? --
Da{\s} war gewi{\ss} genug Material, unau{\s}gesetzt ihre Gedanken zu
erf\"ullen.

Der Diener Frycollin war auf dem Verdeck in einer mit der de{\s}
Koch{\s} vom {\glqq}Albatro{\s}{\grqq} zusammensto{\ss}enden Cabine
untergebracht worden. Diese Nachbarschaft mi{\ss}fiel ihm
keine{\s}weg{\s} -- er liebte e{\s}, sich mit den gro{\ss}en dieser
Erde auf guten Fu{\ss} zu stellen. Doch wenn er endlich einschlief,
so tr\"aumte der arme Teufel nur vom Herabst\"urzen durch die Luft,
wa{\s} seinen Schlummer zum fortw\"ahrenden Alpdr\"ucken
verunstaltete.

Und doch konnte e{\s} keine ruhigere Fahrt geben, al{\s} diese{\s}
Dahinschweben durch die Atmosph\"are, deren Str\"omung sich am
Sp\"atabend ganz gelegt hatte. Au{\ss}er dem Schwirren der
Schraubenfl\"ugel drang kein Ger\"ausch nach dieser H\"ohe,
h\"ochsten{\s} zuweilen der schrille Pfiff einer irdischen
Locomotive, die auf ihrer Eisenstra{\ss}e dahinrollte, oder kaum
vernehmbare Laute von Hau{\s}thieren. -- Ein eigenth\"umlicher
Instinct schien diesen Erdengesch\"opfen zu verrathen, da{\ss} die
Flugmaschine \"uber ihnen hinglitt, und da{\s} veranla{\ss}te sie,
einen Angstschrei von sich zu geben.

Am folgenden Tage, am 14. Juni, lustwandelten Onkel Prudent und Phil
Evan{\s} schon fr\"uh f\"unf Uhr auf dem Verdeck de{\s}
{\glqq}Albatro{\s}{\grqq}. Eine Ver\"anderung gegen den Vortag zeigte
sich nicht, der Au{\s}guck stand am vorderen, der Steuermann am
hinteren Theile de{\s}selben. Wozu diente aber hier ein Wachtposten?
F\"urchtete man auch mit der ersten Maschine dieser Art einen
etwaigen Zusammensto{\ss}? Nein, da{\s} gewi{\ss} nicht. Robur hatte
ja noch keine Nachahmer gefunden. Die M\"oglichkeit, einem in den
L\"uften schwebenden Aerostaten zu begegnen, war eine so geringe,
da{\ss} sie f\"uglich au{\ss}er Rechnung gelassen werden konnte.
Jedenfall{\s} w\"are der Aerostat am schlimmsten daran gewesen -- wie
bei einem Zusammensto{\ss} de{\s} eisernen Topfe{\s} mit dem irdenen.
Der {\glqq}Albatro{\s}{\grqq} hatte von einer solchen Collision ja so
gut wie nicht{\s} zu f\"urchten.

Doch konnte eine solche \"uberhaupt vorkommen? Ja. E{\s} war ja nicht
au{\s}geschlossen, da{\ss} der Aeronef unversehen{\s} auf eine
K\"uste zusteuerte, wie ein Schiff, wenn ein Berg, den e{\s} eben
nicht umsegeln kann, ihm den Weg versperrte. Ein solcher Berg w\"are
also eine Klippe in der Luft, und diese galt e{\s} zu vermeiden, wie
da{\s} Schiff die Klippe de{\s} Meere{\s} zu meiden hat.

Wohl hatte der Ingenieur, ganz wie ein Capit\"an die Fahrtrichtung
angegeben unter Ber\"uck\-sich\-tigung der nothwendigen H\"ohe, in der
sich der Apparat halten mu{\ss}te, um auch die h\"ochsten Berggipfel
der betreffenden Gegenden zu \"ubersegeln. Da der Aeronef sich aber
eben im stark gebirgigem Lande befand, war e{\s} gewi{\ss} nur ein
Gebot der Klugheit, sorgsam Au{\s}guck zu halten, wenn er einmal
au{\s} irgend einem Grunde vom richtigen Laufe abwich.

Bei Betrachtung der unter ihnen liegenden Gegend bemerkten Onkel
Prudent und Phil Evan{\s} einen gro{\ss}en Binnensee, dessen nach
S\"uden gelegene Spitze der {\glqq}Albatro{\s}{\grqq} bald erreichen
mu{\ss}te. Sie schlossen darau{\s}, da{\ss} sie w\"ahrend der Nacht
\"uber den Erie-See in seiner ganzen L\"ange weggekommen w\"aren. Da
der Aeronef nun direct nach Westen steuerte, so mu{\ss}ten sie
sp\"ater den \"au{\ss}ersten Theil de{\s} Michigan-See{\s} erreichen.

{\glqq}Hier ist kein Zweifel m\"oglich, sagte Phil Evan{\s}, jene{\s}
Meer von D\"achern am Horizonte ist Chicago!{\grqq}

Er t\"auschte sich nicht, da{\s} war die genannte Stadt, in der
siebzehn Eisenbahnlinien zusammenlaufen, die K\"onigin de{\s}
Westen{\s}, da{\s} ungeheure Magazin, in dem die Erzeugnisse von
Indiana, Ohio, Wi{\s}consin, Missouri und \"uberhaupt die au{\s}
allen Provinzen zusammenstr\"omen, welche den westlichen Theil der
Union bilden.

Bewaffnet mit einem vortrefflichen Marinefernrohr, da{\s} er in
seinem Ruff gefunden, erkannte Onkel Prudent leicht die
Hauptgeb\"aude jener Stadt. Sein College konnte ihm die Kirchen, die
\"offentlichen Bauten, die zahlreichen Elevatoren, ebenso wie da{\s}
gewaltige H\^otel Sheeman zeigen, da{\s} einem gro{\ss}en W\"urfel,
wie man solche zum Spielen gebraucht, glich, an dem freilich die
Fenster al{\s} hundertfache Augen auf jeder Seite erschienen.

{\glqq}Da da{\s} Chicago ist, bemerkte Onkel Prudent, so ist damit
bewiesen, da{\ss} wir etwa{\s} gar zu weit nach Westen entf\"uhrt
worden sind, al{\s} e{\s} w\"unschen{\s}wert w\"are, um nach unserem
Abfahrt{\s}punkt zur\"uckzukehren.{\grqq}

Der {\glqq}Albatro{\s}{\grqq} entfernte sich in der That in gerader
Linie von der Hauptstadt Pennsylvanien{\s}.

H\"atte Onkel Prudent aber Robur auch darum angehen wollen, sie nun
nach Osten zur\"uckzuf\"uhren, so w\"are da{\s} jetzt wenigsten{\s}
unm\"oglich gewesen. Gerade an diesem Morgen schien der Ingenieur gar
keine Eile zu haben, seine Cabine zu verlassen, mochte er darin nun
mit irgend welchen Arbeiten besch\"aftigt sein oder vielleicht nur
noch schlafen. Die beiden Collegen mu{\ss}ten also fr\"uhst\"ucken,
ohne ihn gesehen zu haben.

Die Fahrgeschwindigkeit war seit dem vorigen Tage nicht ver\"andert.
Bei der Richtung de{\s} eben wendenden Winde{\s} wurde dieselbe nicht
l\"astig, und da der Thermometer sich nur um einen Grad bei der
Erhebung um hundertsiebenzig Meter senkte, so war auch die Temperatur
eine ertr\"agliche. In Erwartung de{\s} Ingenieur{\s} gingen Onkel
Prudent und Phil Evan{\s} nachdenklich hin und her unter der Takelage
der Schrauben -- wenn der Au{\s}druck erlaubt ist -- welche immerhin
eine so schnelle Drehbewegung einhielten, da{\ss} die Strahlen ihrer
Fl\"ugel zu einer halbdurchscheinenden Scheibe verschmolzen.

In weniger al{\s} zwei Stunden kamen sie auf diese Weise l\"ang{\s}
seiner Nordgrenze \"uber den Staat Illinoi{\s} hinweg, und dabei
\"uber den Vater der Gew\"asser, den Mississippi, dessen zweietagige
Dampfer nicht gr\"o{\ss}er al{\s} gew\"ohnliche K\"ahne erschienen.
Dann wendete sich der {\glqq}Albatro{\s}{\grqq} nach Iova, nachdem
Iova-City gegen elf Uhr Vormittag{\s} in Sicht gekommen war.

Einzelne H\"ugelketten, die {\glqq}Bluff{\s}{\grqq}, schl\"angelten
sich von S\"uden nach dem Nordwesten durch diese{\s} Gebiet. Ihre
m\"a{\ss}ige H\"ohe machte keine besondere Aufsteigung de{\s}
Aeronef{\s} n\"othig. Diese Bluff{\s} mu{\ss}ten auch bald noch
niedriger werden, um nachher den weiten Ebenen von Iova Platz zu
machen, welche sich \"uber dessen ganzen n\"ordlichen Theil, wie
\"uber Nebra{\s}ka au{\s}dehnen -- ungeheure Prairien, welche bi{\s}
zum Fu{\ss} der Felsengebirge heranreichen. Da und dort gl\"anzten
zahlreiche Rio{\s}, Zufl\"usse und Nebenfl\"usse de{\s} Missouri. An
ihren Ufern lagen St\"adte und D\"orfer, welche jedoch immer seltener
wurden, je nachdem der {\glqq}Albatro{\s}{\grqq} schneller nach dem
Far-West \"uber sie hinwegglitt.

Im Laufe de{\s} Tage{\s} ereignete sich nicht{\s} Besondere{\s}.
Onkel Prudent und Phil Evan{\s} blieben sich g\"anzlich selbst
\"uberlassen. Kaum bemerkten sie einmal Frycollin, der auf dem
Verdeck au{\s}gestreckt lag und die Augen geschlossen hielt, um
lieber gar nicht{\s} zu sehen. Uebrigen{\s} litt er nicht etwa an
Schwindelzuf\"allen, wie man h\"atte glauben k\"onnen. Wegen
Mangel{\s} an Vergleich{\s}objecten h\"atte sich dieser Schwindel
\"uberhaupt nicht in derselben Weise \"au{\ss}ern k\"onnen, wie etwa
auf dem Dache eine{\s} hohen Geb\"aude{\s}; der Abgrund verliert
seine Anziehung{\s}kraft, wenn man in der Gondel eine{\s} Ballon{\s}
oder auf dem Deck eine{\s} Aeronef{\s} \"uber ihm schwebt, oder
vielmehr unter dem Aeronauten g\"ahnt gar kein Abgrund, sondern der
Horizont allein erhebt sich an allen Seiten und umringt denselben.

Um zwei Uhr glitt der {\glqq}Albatro{\s}{\grqq} \"uber Omaha an der
Grenze von Nebra{\s}ka hin, \"uber Omaha-City, den wirklichen Kopf
der Pacific-Bahn, jene{\s} f\"unfzehnhundert Lieue{\s} langen
Schienenstrange{\s}, der New-York und San Franci{\s}co verbindet.
Einen Augenblick lang sah man die gelblichen Fluthen de{\s} Missouri,
nachher die Stadt mit ihren Holz- und Steinh\"ausern, inmitten
diese{\s} reichen Becken{\s} gelegen gleich dem Schlo{\ss} eine{\s}
G\"urtel{\s}, der Nordamerika in der Taille umspannt.

Zweifello{\s} mu{\ss}ten, w\"ahrend die Passagiere de{\s} Aeronef{\s}
alle diese Einzelheiten betrachteten, auch die Bewohner von Omaha den
seltsamen Apparat wahrgenommen haben.

Ihr Erstaunen aber, denselben in den L\"uften hinschweben zu sehen,
konnte gewi{\ss} nicht gr\"o{\ss}er sein, al{\s} da{\s} de{\s}
Vorsitzenden und de{\s} Schriftf\"uhrer{\s} de{\s}
Weldon-Institut{\s}, sich an Bord de{\s}selben zu befinden.

Jedenfall{\s} lag hier eine Thatsache vor, welche durch die Journale
der Union besprochen wurde; eben diese lieferte eine Erkl\"arung
de{\s} Ph\"anomen{\s}, mit dem sich seit einiger Zeit die ganze Welt
besch\"aftigte.

Eine Stunde sp\"ater war der {\glqq}Albatro{\s}{\grqq} schon \"uber
Omaha hinweg. Der {\glqq}Albatro{\s}{\grqq} steuerte jetzt constant
nach Westen, indem er sich vom Platte-River entfernte, dessen Thal
die Pacific-Railway durch die Prairie folgt. Den Onkel Prudent und
Phil Evan{\s} konnte diese Wahrnehmung gerade nicht befriedigen.

{\glqq}Die Sache wird ernsthaft mit diesem sinnlosen Project, un{\s}
zu den Antipoden zu bringen, sagte der Eine.

-- Und noch dazu wider unseren Willen! bemerkte der Andere. O, dieser
Robur soll sich nur in Acht nehmen, ich bin nicht der Mann dazu, mit
mir spielen zu lassen!

-- Ich auch nicht! versicherte Phil Evan{\s}. Doch, folgen Sie meinem
Rathe, Onkel Prudent, versuchen Sie sich zu m\"a{\ss}igen~...

-- Mich m\"a{\ss}igen!~...

-- Und bemeistern Sie Ihre Wuth bi{\s} zu dem Augenblick, wo e{\s} an
der Zeit ist, sie au{\s}brechen zu lassen.{\grqq}

Gegen f\"unf Uhr und nach Ueberschreitung der mit Tannen und Cedern
bedeckten schwarzen Berge flog der {\glqq}Albatro{\s}{\grqq} \"uber
jenen Gebieten hin, die man mit Recht da{\s} {\glqq}schlimme
Land{\grqq} genannt hat -- ein Chao{\s} von ockerfarbigen H\"ugeln,
gleichsam von Bergst\"ucken, welche der Sch\"opfer hatte auf die Erde
fallen lassen und die dabei in Tr\"ummer gegangen waren. Von ferne
gesehen, nahmen diese Bl\"ocke die phantastischesten Formen an. Da
und dort inmitten dieser ungeheuren Ansammlung von Bruchst\"ucken
erblickte man Ruinen von mittelalterlichen St\"adten mit Fort{\s},
Wartth\"urmen, Laufgr\"aben und Schanzen. Heut\/zutage bildet diese{\s}
{\glqq}schlimme oder b\"ose Land{\grqq} aber nicht{\s} al{\s} ein
gewaltige{\s} Beinhau{\s}, in dem die Reste von Pachydermen,
Chelonien und der Sage nach sogar von fossilen Menschen bleichen,
welche durch eine unbekannte Erdrevolution in grauer Vorzeit hierher
geworfen wurden.

Mit einbrechendem Abend war schon da{\s} gro{\ss}e Becken de{\s}
Platte-River \"ubersegelt. Jetzt dehnte sich vor dem
{\glqq}Albatro{\s}{\grqq} eine weite Ebene bi{\s} zu dem, durch
dessen hohen Standpunkt sehr erweiterten Horizonte au{\s}.

W\"ahrend der Nacht waren e{\s} nicht mehr die scharfen Pfiffe der
Locomotive oder die heulenden T\"one von Dampfbooten, welche die Ruhe
de{\s} gestirnten Firmament{\s} st\"orten. Lang anhaltende{\s}
Grunzen und Bl\"ocken drang manchmal bi{\s} zu dem, \"ubrigen{\s}
jetzt der Erde n\"aheren Aeronef hinauf. Da{\s}selbe r\"uhrte von den
Bisonheerden her, welche bei Aufsuchung von Wasserl\"aufen und
Weideland durch die Prairie trotteten. Und wenn jene schwiegen, dann
erzeugte da{\s} Rascheln de{\s} Grase{\s} unter ihren Hufen ein
dumpfe{\s} Ger\"ausch, \"ahnlich dem Rauschen einer Ueberschwemmung,
und sehr verschieden von dem Schwirren und Sausen der Schrauben.

Von Zeit zu Zeit lie{\ss} sich wohl auch da{\s} Heulen von W\"olfen,
da{\s} Gebell und Geschrei von F\"uchsen und Wildkatzen h\"oren oder
da{\s} scharfe Bellen von Coyot{\s}, jene{\s} \begin{antiqua}canis
latrans\end{antiqua}, dessen Name sich schon durch die gellenden
T\"one de{\s} Thiere{\s} rechtfertigt.

Daneben verbreitete sich ein durchdringender Duft von Minze, Salbei
und Absinth, vermischt mit dem kr\"aftigen Harzgeruch von Coniferen,
in der reinen Nachtluft.

Endlich h\"orte man, um alle vom Erdboden kommenden Ger\"ausche zu
erw\"ahnen, auch eine Art recht unheimlichen Bellen{\s}, da{\s} aber
nicht von den Coyot{\s} herr\"uhrte; da{\s} war der Schrei einer
Rothhaut, welchen kein Pionnier de{\s} fernen Westen{\s} mit dem
Geschrei eine{\s} Raubthiere{\s} verwechseln k\"onnte.

Am 15. Juni verlie{\ss} Phil Evan{\s} gegen f\"unf Uhr Morgen{\s}
seine Cabine. Vielleicht sollte er an diesem Tage den Ingenieur Robur
endlich wiedersehen.

Begierig, zu erfahren, warum Jener sich am vergangenen Tage gar nicht
gezeigt haben m\"oge, wandte er sich an den Obersteuermann Tom
Turner.

Tom Turner, von englischer Herkunft und etwa f\"unfundvierzig Jahre
alt, breit in der Brust, untersetzt von Gestalt und mit Knochen von
Eisen, hatte einen jener charakteristischen K\"opfe \begin{antiqua}\`a
la\end{antiqua} Hogarth, wie sie dieser Maler der angels\"achsischen
H\"a{\ss}lichkeiten au{\s} seinem Pinsel hervorgezaubert hat. Wer die
Tafel \begin{antiqua}IV\end{antiqua} von Harlot{\s} Progre{\ss}
genauer betrachtet, der wird auf derselben den Kopf Tom Turner'{\s}
auf den Schultern de{\s} Gef\"angni{\ss}w\"arter{\s} wiederfinden
und wird erkennen, da{\ss} dessen Physiognomie nicht eben viel
Ermuthigende{\s} hat.

{\glqq}Werden wir heute den Ingenieur Robur sehen? fragte Phil Evan{\s}.

-- Wei{\ss} nicht, antwortete Tom Turner.

-- Ich frage Sie nicht, ob er etwa weggegangen ist.

-- Vielleicht.

-- Auch nicht, wann er zur\"uckkehren k\"onnte.

-- Vermutlich, wenn er mit seiner Cur{\s}bestimmung fertig ist.{\grqq}

Hiermit verschwand Tom Turner schon wieder in seinem Ruff.

Phil Evan{\s} mu{\ss}te sich wohl oder \"ubel mit dieser Antwort
begn\"ugen, welche umso weniger beruhigend erschien, al{\s} eine
fortgesetzte Beobachtung de{\s} Compasse{\s} ihm lehrte, da{\ss} der
{\glqq}Albatro{\s}{\grqq} noch immer nach S\"udwesten weiter
steuerte. Welcher Unterschied aber zwischen dem seit der Nacht
verlassenen Gebiete de{\s} schlimmen Lande{\s} und der Landschaft,
die sich jetzt unten auf der Erde entrollte!

Nachdem der Aeronef tausend Kilometer von Omaha au{\s}
zur\"uckgelegt, befand er sich \"uber einer Gegend, welche Phil
Evan{\s} au{\s} dem Grunde nicht zu erkennen vermochte, weil er sie
vorher noch niemal{\s} besucht hatte. Einige Fort{\s}, mit dem
Zwecke, die Indianer im Schach zu halten, bekr\"onten die Bluff{\s}
mit ihren geometrischen Linien, welche mehr au{\s} Palissaden, al{\s}
au{\s} Mauerwerk bestanden; D\"orfer gab e{\s} nur wenige und ebenso
wenig Bewohner in diesem von dem goldf\"uhrenden, einige Grade
s\"udlicher liegenden Gebiete Colorado{\s} so auf\/fallend
verschiedenen Landstriche.

In der Ferne erhob sich, vorl\"aufig nur in verschwindenden Umrissen,
eine Reihe von Bergk\"ammen, welche die aufsteigende Sonne mit feurig
leuchtendem Kranze schm\"uckte.

Da{\s} waren die Felsengebirge.

Zum ersten Male an diesem Morgen beobachteten Onkel Prudent und Phil
Evan{\s} eine empfindliche K\"alte. Die Erniedrigung der Temperatur
war aber nicht etwa einem Witterung{\s}umschlage zuzuschreiben, denn
die Sonne leuchtete fortw\"ahrend in hellem Glanze.

{\glqq}Da{\s} wird von der Erhebung de{\s} {\glqq}Albatro{\s}{\grqq}
in der Atmosph\"are herkommen,{\grqq} meinte Phil Evan{\s}.

In der That war der an der \"au{\ss}eren Seite der Th\"ur de{\s}
mittleren Ruff{\s} angebrachte Barometer bi{\s} auf
f\"unfhundertvierzig Millimeter gesunken -- wa{\s} einer Erhebung von
etwa dreitausend Metern entspricht. Der Aeronef hielt sich also in
einer bedeutenden, \"ubrigen{\s} durch die gebirgige
Bodenbeschaffenheit bedingten H\"ohe.

Eine Stunde vorher hatte er sogar eine H\"ohe von viertausend Metern
\"ubersteigen m\"ussen, denn hinter ihm erhoben sich viele, mit
ewigem Schnee bedeckte Bergh\"aupter.

Weder Onkel Prudent noch sein Gef\"ahrte konnten sich erinnern,
welche{\s} Land da{\s} wohl w\"are. Im Laufe der Nacht hatte der
{\glqq}Albatro{\s}{\grqq} ja einen anderen Weg nach Norden oder
S\"uden einhalten k\"onnen, und bei seiner \"uberm\"a{\ss}igen
Schnelligkeit gen\"ugte da{\s}, sie schon sehr weit zu verschlagen.

Nachdem sie verschiedene mehr oder weniger annehmbare Hypothesen
besprochen, einigten sie sich dar\"uber, da{\ss} da{\s} vorliegende,
von einem krei{\s}f\"ormigen Bergwall umrahmte Gebiet da{\s}selbe
sein werde, welche{\s} durch Congre{\ss}acte vom M\"arz 1872 zum
Nationalpark der Vereinigten Staaten erkl\"art worden war.

Sie hatten hiermit Recht, und jene{\s} Gebiet verdient vollst\"andig
den Namen eine{\s} Park{\s}, aber eine{\s} solchen mit Bergen statt
der H\"ugel, mit Seen statt der Teiche, mit Str\"omen statt der
B\"ache, mit tiefen W\"aldern statt k\"unstlich angelegter
Labyrinthe, und al{\s} Springbrunnen schm\"uckten denselben wirkliche
Geyser von erstaunlicher M\"achtigkeit.

Nach wenig Minuten glitt der {\glqq}Albatro{\s}{\grqq}, den
Stevensonberg recht{\s} liegen lassend, \"uber den
Yellowstone-Flu{\ss} hin und gelangte nach dem gro{\ss}en See, der
den Namen jene{\s} Flusse{\s} tr\"agt. Welch' reiche Abwech{\s}lung
im Zuge der Ufer diese{\s} Wasserbecken{\s}, deren flachere, mit
Obsidianen und kleinen Krystallen bes\"aete R\"ander die
Sonnenstrahlen in unz\"ahligen Facetten widerspiegelten! Wie
launenhaft liegen die Inseln \"uber seine Oberfl\"ache zerstreut! Wie
wunderbar blau wirft dieser Riesenspiegel die Farbe de{\s} Himmel{\s}
zur\"uck! Und ring{\s} um diesen See -- \"ubrigen{\s} einer der
h\"ochstgelegenen der ganzen Erde -- schwammen und flatterten ganze
Wolken verschiedener V\"ogel, wie Pelikane, Schw\"ane, M\"oven,
G\"anse, Rothg\"anse und Taucherv\"ogel. Einige der Strecken de{\s}
steiler abfallenden Uferlande{\s} trugen ein immergr\"une{\s} Gewand
von Fichten- und L\"archenb\"aumen, w\"ahrend am Fu{\ss}e seiner
B\"oschungen unz\"ahlige wei{\ss}e Dampfquellen emporwirbelten.
Dieser Dampf entsteigt dem Erdboden wie au{\s} einem ungeheuren
Kessel, in dem da{\s} Wasser durch da{\s} Feuer de{\s} Erdinneren in
fortw\"ahrendem Sieden erhalten wird.

F\"ur den Koch w\"are jetzt oder niemal{\s} eine g\"unstige
Gelegenheit gewesen, sich mit reichlichem Vorrathe von Forellen zu
versorgen, welche Fischart die einzige ist, die der Yellow-See, aber
auch zu Myriaden, ern\"ahrt. Der {\glqq}Albatro{\s}{\grqq} hielt sich
jedoch stet{\s} in einer solchen H\"ohe, da{\ss} ein Fischzug, der
unzweifelhaft von eintr\"aglichem Erfolge gewesen w\"are, sich nicht
h\"atte au{\s}f\"uhren lassen.

Uebrigen{\s} wurde der See schon binnen dreiviertel Stunden und wenig
sp\"ater da{\s} Gebiet der Geyser, die mit den sch\"onsten in
I{\s}land wetteifern, \"uberschritten. Ueber da{\s} Verdeck hinau{\s}
gebeugt, beobachteten Onkel Prudent und Phil Evan{\s} die fl\"ussigen
S\"aulen, die hoch aufstiegen, al{\s} sollten sie dem Aeronef noch
ein neue{\s} Kraftelement zuf\"uhren. E{\s} waren da{\s} {\glqq}der
F\"acher{\grqq}, dessen D\"ampfe sich krei{\s}f\"ormig au{\s}breiten;
{\glqq}da{\s} befestigte Schlo{\ss}{\grqq}, da{\s} sich gleichsam
durch Trombensch\"usse zu vertheidigen scheint; {\glqq}der alte
Treue{\grqq} mit seiner von Regenbogen begrenzten
Fl\"ussigkeit{\s}s\"aule, und {\glqq}der Riese{\grqq}, durch den der
innere Druck einen lothrechten Strom von f\"unfundzwanzig Fu{\ss}
Umfang auf mehr al{\s} zweihundert Fu{\ss} H\"ohe emporschleudert.

Robur schien die Wunder diese{\s} unvergleichlichen Schauspiel{\s},
da{\s} gewi{\ss} in der Welt einzig dasteht, schon zur Gen\"uge zu
kennen, denn er erschien nicht auf dem Verdeck.

Sollte er den Aeronef nur zum Vergn\"ugen seiner G\"aste \"uber
diese{\s} National-Eigenthum hingef\"uhrt haben? Wenn diese
Vorau{\s}setzung auch vielleicht zutraf, so ent\/zog er sich doch ihren
Danke{\s}bezeugungen. Er lie{\ss} sich nicht einmal durch die k\"uhne
Fahrt quer durch die Felsengebirge, welche der {\glqq}Albatro{\s}{\grqq}
gegen sieben Uhr Morgen{\s} erreichte, au{\s} seiner Ruhe st\"oren.

E{\s} ist bekannt, da{\ss} diese{\s} orographische System sich gleich
einem gewaltigen R\"uckgrat von den Lenden Nordamerika{\s} bi{\s} zu
dessen Halse hin au{\s}dehnt, indem e{\s} eine Fortsetzung der
mexikanischen Anden bildet. Da{\s} Ganze erreicht eine L\"ange von
dreitausendf\"unfhundert Kilometern und hat seinen h\"ochsten Punkt
im Pic-Jame{\s}, der bi{\s} fast zw\"olftausend Fu{\ss} hoch aufragt.

Gewi{\ss} h\"atte der {\glqq}Albatro{\s}{\grqq} durch Vermehrung
seiner Fl\"ugelschl\"age, gleich einem im Aether dahineilenden Vogel,
auch die h\"ochsten Gipfel dieser Ketten \"uberfliegen k\"onnen, um
dann wie mit Riesenschwingen nach Oregon und Utah hinabzusteigen.
Diese{\s} Man\"over war aber nicht einmal nothwendig, da e{\s} hier
P\"asse giebt, um durch die Bergkette zu gelangen, ohne deren Kamm zu
\"ubersteigen. Man findet verschiedene solcher {\glqq}Ca\~non{\s}{\grqq},
eine Art mehr oder weniger enger Schluchten, durch welche man nur
schwer gelangen kann -- die einen, wie der Bridger-Pa{\ss}, dem auch
die Pacific-Bahn folgt, um in da{\s} Mormonengebiet einzudringen, die
anderen etwa{\s} weiter im Norden oder im S\"uden.

In einen dieser Ca\~non{\s} lenkte der {\glqq}Albatro{\s}{\grqq} ein,
nachdem er seine Geschwindigkeit vermindert hatte, um jedenfall{\s}
ein Ansto{\ss}en an die W\"ande der Schlucht zu vermeiden. Der
Steuermann, dessen ungemein sichere Hand die vorz\"ugliche
Wirksamkeit de{\s} Steuerruder{\s} in besonder{\s} helle{\s} Licht
setzte, lenkte denselben, wie er e{\s} mit einem Boote ersten
Range{\s} beim Wettfahren de{\s} Royal Thame{\s} Club gethan h\"atte.
E{\s} war in der That bewunderung{\s}w\"urdig anzusehen. Und trotz
de{\s} Widerwillen{\s}, den die beiden Feinde de{\s} {\glqq}Schwerer,
al{\s} die Luft{\grqq} noch immer empfanden, mu{\ss}ten sie doch
ent\/z\"uckt sein \"uber die Vollkommenheit dieser sich durch den
Luftraum bewegenden Maschine.

Binnen weniger al{\s} zweiundeinerhalben Stunde wurde die gewaltige
Bergkette durchfahren und der {\glqq}Albatro{\s}{\grqq} nahm seine
gew\"ohnliche Geschwindigkeit von hundert Kilometer (in der Stunde)
wieder an. Er steuerte jetzt auf'{\s} Neue dem S\"udwesten zu, um
nicht gar zu hoch \"uber dem Erdboden da{\s} Gebiet von Utah schr\"ag
zu durchschneiden. Dabei war er bi{\s} auf wenige hundert Meter
gesunken, al{\s} die T\"one einer Pfeife die Aufmerksamkeit de{\s}
Onkel Prudent und Phil Evan{\s}' erregten.

Diese kamen von einem Zuge der Pacific-Bahn her, welcher der Stadt am
gro{\ss}en Salzsee zudampfte.

In diesem Augenblick senkte sich auf geheimen Befehl der
{\glqq}Albatro{\s}{\grqq} noch weiter, um dem mit voller Dampfkraft
dahinfahrenden Zuge zu folgen. Er wurde sofort bemerkt. Einige
K\"opfe erschienen an den Th\"uren der Waggon{\s}. Dann dr\"angten
sich bald zahlreiche Passagiere auf den kleinen Laufbr\"ucken, welche
die amerikanischen {\glqq}Car{\s}{\grqq} mit einander verbinden.
Einzelne wagten e{\s} sogar, die Doppelwagen de{\s} Train{\s} zu
erklettern, um die Flugmaschine besser sehen zu k\"onnen. Laute
Hipp{\s} und Hurrah{\s} dr\"ohnten durch die Luft, hatten aber nicht
den Erfolg, Robur erscheinen zu lassen.

Da{\s} Spiel seiner Schrauben weiter verlangsamend, stieg der
{\glqq}Albatro{\s}{\grqq} noch immer tiefer hinunter und verminderte
auch seine horizontale Schnelligkeit, um den Bahnzug, den er bequem
h\"atte \"uberholen k\"onnen, nicht hinter sich zu lassen. So flog er
\"uber diesen hin, wie ein ungeheurer K\"afer, w\"ahrend er doch
h\"atte einem riesenhaften Raubvogel gleichen k\"onnen. Jetzt
schwenkte er wie spielend nach recht{\s} und nach link{\s} ab,
scho{\ss} einmal vorw\"art{\s} und kehrte auf demselben Wege wieder
zur\"uck, auch hatte er stolz die schwarze Flagge mit der goldenen
Sonne gehi{\ss}t, worauf der Zugf\"uhrer al{\s} Antwort da{\s} Banner
mit den siebenunddrei{\ss}ig Sternen der amerikanischen Union
schwenkte.

Vergeblich versuchten die beiden Gefangenen, die sich jetzt
darbietende Gelegenheit zu ben\"utzen, um Kunde davon zu geben,
wa{\s} au{\s} ihnen geworden w\"are. Vergeben{\s} rief der
Vorsitzende de{\s} Weldon-Institut{\s} mit Stentorstimme:

{\glqq}Ich bin Onkel Prudent au{\s} Philadelphia!{\grqq}

Und der Schriftf\"uhrer.

{\glqq}Ich bin Phil Evan{\s}, sein College!{\grqq}

Ihre Rufe verhallten in den tausend Hurrah{\s}, mit denen die
Passagiere de{\s} Zug{\s} die merkw\"urdige Erscheinung de{\s}
Luftschiffe{\s} begr\"u{\ss}ten.

Inzwischen waren drei bi{\s} vier Mann vom Aeronef auf dem Verdeck
de{\s}selben erschienen und Einer von ihnen lie{\ss} -- wie e{\s}
Seeleute zu thun pflegen, wenn sie ein langsamer fahrende{\s} Schiff
\"uberholen -- nach dem Zuge ein St\"uck Tau hinab -- ein
ironische{\s} Angebot, ihn in'{\s} Schlepptau zu nehmen.

Dann nahm der {\glqq}Albatro{\s}{\grqq} sofort seinen gew\"ohnlichen
Gang wieder an und nach einer halben Stunde hatte er jenen
Expre{\ss}zug, dessen letzte Dampfw\"olkchen bald au{\s} dem
Gesicht{\s}kreise verschwanden, schon weit hinter sich gelassen.

Gegen ein Uhr Mittag{\s} wurde eine sehr gro{\ss}e Scheibe sichtbar,
welche die Sonnenstrahlen gleich einem ungeheuren Reflector zur\"uckwarf.

{\glqq}Da{\s} mu{\ss} die Hauptstadt der Mormonen, Salt-Lake-City,
sein!{\grqq} sagte Onkel Prudent.

In der That war e{\s} die gro{\ss}e Salzsee-Stadt und jene convexe
Scheibe war da{\s} Dach de{\s} Tabernakel{\s}, da{\s} bequem
zehntausend Heilige aufnehmen kann. Wie ein erhabener Spiegel
zerstreute derselbe die Strahlen der Sonne nach allen Richtungen hin.

Hier dehnte sich die gro{\ss}e Stadt au{\s} am Fu{\ss}e der
Wasatsh-Berge, welche bi{\s} zur halben H\"ohe mit Cedern und Fichten
bedeckt sind, und am Ufer jene{\s} Jordan, der die Gew\"asser von
Utah in den gro{\ss}en Salzsee ergie{\ss}t. Unter dem Aeronef
breitete sich da{\s} Damenbrett au{\s}, welche{\s} die meisten
amerikanischen St\"adte bilden -- hier ein Damenbrett mit {\glqq}mehr
Damen al{\s} Feldern{\grqq}, da die Polygamie bei den Mormonen in so
hoher Bl\"uthe steht. Die Landschaft im Umkreise zeigte sich jedoch
gut bestellt und cultivirt, auch reich an Spinnfaserpflanzen,
w\"ahrend sich Schafheerden von mehr al{\s} tausend K\"opfen vielfach
umhertummelten.

Aber da{\s} Ganze verbla{\ss}te wie ein Schatten, und der
{\glqq}Albatro{\s}{\grqq} flog jetzt nach S\"udwest mit gesteigerter
Geschwindigkeit, welche ziemlich f\"uhlbar wurde, weil sie die de{\s}
Winde{\s} \"ubertraf.

Bald darauf schwebte der Aeronef \"uber dem Staate Nevada und seinen
silberf\"uhrenden Gebieten, die nur die Sierra von den
goldf\"uhrenden L\"andereien Kalifornien{\s} trennt.

{\glqq}Wir k\"onnen auf jeden Fall erwarten, San Franci{\s}co noch
vor dem Abend zu sehen, sagte Phil Evan{\s}.

-- Und dann?...{\grqq} antwortete Onkel Prudent.

E{\s} war jetzt um sech{\s} Uhr Nachmittag{\s}, al{\s} die Sierra
Nevada durch denselben Einschnitt von Truckie \"uberschritten wurde,
der auch der Bahn al{\s} Bergpa{\ss} dient. Von hier au{\s} hatte man
nur noch dreihundert Kilometer zur\"uckzulegen, um, wenn nicht San
Franci{\s}co, so doch mindesten{\s} Sacramento, die Hauptstadt von
Californien, zu erreichen.

Die dem {\glqq}Albatro{\s}{\grqq} jetzt verliehene Geschwindigkeit
war eine so gro{\ss}e, da{\ss} noch vor acht Uhr die Kuppel de{\s}
Capitol{\s} am westlichen Horizonte auftauchte, nur um bald wieder am
entgegengesetzten zu verschwinden.

Eben jetzt zeigte sich Robur auf dem Verdeck. Die beiden Collegen
gingen auf ihn zu.

{\glqq}Ingenieur Robur, begann Onkel Prudent, wir befinden un{\s} nun
an den Grenzen Amerika{\s}. Wir meinen, dieser Scherz k\"onnte nun
sein Ende finden.

-- Ich scherze nie,{\grqq} antwortete Robur.

Er gab ein Zeichen; der {\glqq}Albatro{\s}{\grqq} senkte sich schnell
abw\"art{\s}, doch gleichzeitig nahm er eine solche Schnelligkeit an,
da{\ss} sich Alle in die Ruff{\s} fl\"uchten mu{\ss}ten.

Kaum hatte sich die Th\"ur der Cabine hinter den beiden Collegen
geschlossen, al{\s} Onkel Prudent rief:

{\glqq}Nur noch etwa{\s} mehr und ich erw\"urge ihn!

-- Wir m\"ussen versuchen, zu entfliehen, rieth Phil Evan{\s}.

-- Ja ... e{\s} koste, wa{\s} e{\s} wolle!{\grqq}

Da klang ein lange{\s} Gemurmel bi{\s} zu ihnen herein.

Da{\s} war da{\s} Grollen de{\s} Meere{\s}, da{\s} gegen die
K\"ustenfelsen brandete. E{\s} war der Pacifische Ocean.



\newpage\begin{center}\label{kap09}
{\large \begin{antiqua}IX.\end{antiqua}\\
In dem der {\glqq}Albatro{\s}{\grqq} fast zehntausend Kilometer 
zur\"ucklegt und da{\s} mit einem merkw\"urdigen Sprunge
endigt.\\\bigskip}
\end{center}



Onkel Prudent und Phil Evan{\s} waren fest entschlossen, zu fliehen.
H\"atten sie e{\s} nur zu dreien mit den acht, allerding{\s} sehr
kr\"aftigen M\"annern zu thun gehabt, welche die Besatzung de{\s}
Aeronef{\s} bildeten, so w\"urden sie den Kampf vielleicht gewagt
haben. Ein k\"uhner Handstreich h\"atte sie zu Herren an Bord gemacht
und ihnen die M\"oglichkeit gegeben, an einem beliebigen Punkte der
Vereinigten Staaten niederzugehen. Zu Zweien aber -- denn Frycollin
konnte ja nur al{\s} verschwindende Gr\"o{\ss}e gez\"ahlt werden --
war daran nicht wohl zu denken; da jede Gewaltanwendung also
au{\s}geschlossen blieb, mu{\ss}ten sie, sobald der
{\glqq}Albatro{\s}{\grqq} einmal zur Erde hinabging, zur List ihre
Zuflucht nehmen. Da{\s} bem\"uhte sich auch Phil Evan{\s} seinem
wuthschnaubenden Collegen beizubringen, da er von diesem immer noch
eine gewaltth\"atige Uebereilung f\"urchtete, welche ihre Lage nur
verschlimmern konnte.

Jedenfall{\s} war jetzt kein g\"unstiger Augenblick. Der Aeronef
glitt in schnellster Gangart eben \"uber den Nordpacifischen Ocean
hin. Schon am n\"achsten Morgen, dem de{\s} 16. Juni, sah man
nicht{\s} von der K\"uste, und da diese von der Insel Vancouver
bi{\s} zur Gruppe der Al\"euten -- da{\s} ist der fr\"u\-heren
russischen Besitzung in Amerika, welche 1867 an die Vereinigten
Staaten abgetreten wurde -- in einem gro{\ss}en Bogen verl\"auft,
so hatte e{\s} den Anschein, al{\s} ob der {\glqq}Albatro{\s}{\grqq}
letztere an dem vorspringendsten Bogentheile kreuzen sollte,
wenigsten{\s} wenn die jetzt eingehaltene Fahrtrichtung nicht
ver\"andert wurde.

Wie lang erschienen die N\"achte jetzt den beiden Collegen! Sie
beeilten sich auch jeden Morgen, ihre Cabine zu verlassen. Al{\s} sie
heute nach dem Deck kamen, war der Horizont im Osten schon
vollst\"andig hell. Man n\"aherte sich ja der Sommersonnenwende, dem
l\"angsten Tage auf der n\"ordlichen Halbkugel, an dem e{\s} unter
dem 60. Breitengrade eigentlich kaum Nacht wird.

Der Ingenieur Robur dagegen schien -- ob au{\s} Gewohnheit oder mit
Absicht -- keine besondere Eile zu haben, seinen Ruff zu verlassen;
und al{\s} da{\s} heute endlich geschah, begn\"ugte er sich, seine
beiden G\"aste zu begr\"u{\ss}en, al{\s} er auf dem Hintertheile
de{\s} Aeronef ihren Weg kreuzte.

Inzwischen hatte sich auch Frycollin mit vor Schlaflosigkeit
ger\"otheten Augen, glanzlosem Blicke und schlotternden Beinen au{\s}
seiner Cabine gewagt. Er ging dahin wie Einer, dessen Fu{\ss} e{\s}
empfindet, da{\ss} dem Boden darunter nicht recht zu trauen ist. Sein
erster Blick richtete sich nach der Auftrieb{\s}maschinerie, die,
ohne sich zu beeilen, mit beruhigender Regelm\"a{\ss}igkeit
arbeitete.

Danach begab sich der immerfort schwankende Neger nach der Reeling
und ergriff diese mit beiden H\"anden, um sich, mehr Gleichgewicht zu
sichern. Offenbar w\"unschte er einen Ueberblick \"uber da{\s} Land
zu gewinnen, da{\s} der {\glqq}Albatro{\s}{\grqq} jetzt in der H\"ohe
von h\"ochsten{\s} zweihundert Metern \"uberflog.

Frycollin hatte sich t\"uchtig zusammennehmen m\"ussen, um einen
solchen Versuch zu wagen. E{\s} bedurfte ja, seiner Meinung nach,
einer gewissen K\"uhnheit, seine werthe Person einer solchen Gefahr
au{\s}zusetzen.

Vor der Reeling stehend, hielt Frycollin erst den K\"orper nach
r\"uckw\"art{\s} geneigt, dann sch\"uttelte er an derselben, um ihre
Haltbarkeit zu pr\"ufen; nachher richtete er sich auf, beugte sich
etwa{\s} nach vorw\"art{\s} und steckte endlich den Kopf ein wenig
hinau{\s}. Wir brauchen wohl nicht zu bemerken, da{\ss} er w\"ahrend
der Dauer diese{\s} Experimente{\s} beide Augen fest geschlossen
hielt. Endlich \"offnete er dieselben.

Hei, wie schrie er da laut, wie flog er eiligst zur\"uck und wie
verkroch sich sein Kopf zwischen den Schultern!

Unter dem Abgrunde hatte er den ungeheuren Ocean erblickt. W\"aren
seine Haare nicht gar zu krank gewesen, sie h\"atten sich gewi{\ss}
\"uber der Stirn gestr\"aubt.

{\glqq}Da{\s} Meer! Da{\s} Meer!~...{\grqq} schrie er auf.

Frycollin w\"are lang auf da{\s} Verdeck hingest\"urzt, wenn der Koch
nicht die Arme au{\s}gebreitet h\"atte, ihn aufzufangen.

Dieser Koch war ein Franzose, vielleicht ein Ga{\s}cogner, obwohl er
sich Fran\c{c}oi{\s} Tapage nannte. Wenn er nicht Ga{\s}cogner war,
so mu{\ss}te er w\"ahrend seiner Kindheit die Brisen der Garonne
eingesaugt haben. Wie dieser Fran\c{c}oi{\s} Tapage aber in die
Dienste de{\s} Ingenieur{\s} gekommen, durch welche Reihe von
Zuf\"alligkeiten er unter die Mannschaft de{\s}
{\glqq}Albatro{\s}{\grqq} gerathen war, da{\s} wu{\ss}te kein Mensch.
Jedenfall{\s} sprach dieser Schlaukopf englisch trotz jedem Yankee.

{\glqq}Heda, aufrecht, zum Teufel, herauf! rief er, den Neger mit
kr\"aftigem Handgriffe aufrichtend.

-- Master Tapage! ... antwortete der arme Teufel, einen
verzweiflung{\s}vollen Blick nach den Schrauben werfend.

-- Wa{\s} willst Du denn, Frycollin?

-- Geht da{\s} manchmal ent\/zwei?

-- Manchmal nicht, aber e{\s} wird einmal ent\/zwei gehen.

-- Warum? ... Warum denn?~...

-- Weil zuletzt Alle{\s} einmal zum Kuckuk geht, wie man bei mir zu
Hause sagt.

-- Ja, aber da ist ja da{\s} Meer darunter?~...

-- Im Fall eine{\s} Sturze{\s} ist da{\s} viel besser.

-- Doch da mu{\ss} man ertrinken!

-- Man ertrinkt freilich, aber man beh\"alt seine Knochen{\grqq},
erwiderte Fran\c{c}oi{\s} Tapage zuversichtlich.

Wie eine Schlange dahinkriechend, war Frycollin gleich darauf tief
hinein in seine Cabine geschlichen.

Im Laufe de{\s} 16. Juni hielt der Aeronef nur eine mittlere
Geschwindigkeit ein. Er schien an der Oberfl\"ache diese{\s} so
ruhigen Meere{\s}, da{\s} im vollen Sonnenschein gl\"anzte, fast
hinzustreichen, da er sich kaum hundert Fu{\ss} \"uber demselben
hielt. Heute nun waren Onkel Prudent und sein Gef\"ahrte in ihrer
Cabine zur\"uckgeblieben, um Robur nicht zu begegnen, der rauchend,
bald allein, bald mit seinem Obersteuermann Tom Turner, auf dem Deck
umherging. Nur die halbe Anzahl Schrauben war in Th\"atigkeit, doch
gen\"ugte schon, den Apparat in den niedrigeren Zonen der
Atmosph\"are zu erhalten.

Unter diesen Verh\"altnissen h\"atte die Mannschaft au{\ss}er dem
Vergn\"ugen eine{\s} Fischzug{\s} sich noch die Befriedigung bereiten
k\"onnen, in ihren gewohnten Speisezettel eine Abwech{\s}lung zu
bringen, wenn da{\s} Wasser de{\s} Stillen Ocean{\s} fischreich genug
w\"are. Auf dessen Oberfl\"ache zeigten sich aber nur einzelne
Walfische, von der Art mit gelbem Bauche, welche gegen
f\"unfundzwanzig Meter in der L\"ange mi{\ss}t. Gerade diese kennt
man al{\s} die furchtbarsten Cetaceer der n\"ordlichen Meere. Die
Fischer von Beruf h\"uten sich wei{\s}lich, dieselben anzugreifen, so
gef\"ahrlich k\"onnen die Thiere werden.

Immerhin konnte man wohl die Harpunirung eine{\s} jener Walfische
entweder mit der Flechter'schen Rakete oder mit der Wurfbombe
versuchen, und von beiden hatte man eine Au{\s}wahl an Bord.

Wozu aber diese unn\"utze Schl\"achterei? Wahrscheinlich wollte Robur
nur den beiden Mitgliedern de{\s} Weldon-Institut{\s} zeigen, wozu er
seinen Aeronef Alle{\s} verwenden k\"onne, und de{\s}halb sollte auf
einen der gewaltigen Cetaceer Jagd gemacht werden.

Auf den Ruf: {\glqq}Walfische! Walfische!{\grqq} eilten Onkel Prudent
und Phil Evan{\s} au{\s} ihren Cabinen. Vielleicht war ein Schiff,
ein sogenannter Walfischfahrer, in Sicht. In diesem Falle w\"aren
Beide, um ihrem Gef\"angnisse zu entfliehen, entschlossen gewesen,
sich in'{\s} Meer zu st\"urzen, auf die schwache Hoffnung hin, von
einem Fahrzeug aufgenommen zu werden.

Schon stand die ganze Mannschaft de{\s} {\glqq}Albatro{\s}{\grqq}
geordnet und jede{\s} Befehl{\s} gew\"artig auf dem Verdeck und
wartete.

{\glqq}Wir wollen'{\s} also versuchen, Master Robur? fragte der
Obersteuermann Tom Turner.

-- Ja, Tom,{\grqq} antwortete der Ingenieur.

In den Ruff{\s} f\"ur die Maschinerie standen der Mechaniker und
seine Gehilfen auf Posten, um jede{\s} Man\"over au{\s}zuf\"uhren,
da{\s} ihnen durch Zeichen anbefohlen wurde. Der
{\glqq}Albatro{\s}{\grqq} senkte sich sofort nach dem Meere zu und
hielt etwa f\"unfzig Fu{\ss} dar\"uber an.

Wie die beiden Collegen sich \"uberzeugen konnten, war hier kein
Schiff in Sicht, so wenig wie eine K\"uste, welche sie h\"atten
schwimmend erreichen k\"onnen, vorau{\s}gesetzt, da{\ss} Robur sie
nicht wieder ergreifen lie{\ss}.

Mehrere Dunst- und Wasserstrahlen, welche sie durch die Nasenl\"ocher
au{\s}trieben, verk\"undeten die Anwesenheit von Walfischen, welche,
um zu athmen, einmal auf die Oberfl\"ache kamen.

Tom Turner hatte sich, unterst\"utzt von einem seiner Kameraden, am
Vordertheil aufgestellt. Ihm nahe zur Hand lag eine jener Wurfbomben
californischen Fabrikat{\s}, welche mit einer Art B\"uchse
abgeschossen werden. Jene besteht au{\s} einem Metallcylinder, der
mit einer ebenso geformten Bombe endigt, welche in eine Stange mit
widerhakigen Spitzen au{\s}l\"auft. Von dem Vordercastell au{\s},
da{\s} er eben bestieg, gab Robur mit der rechten Hand dem Mechaniker
und mit der linken Hand dem Steuermann die n\"othigen Zeichen, wie
sie man\"ovriren sollten; so beherrschte er den Aeronef sowohl in
wagrechter, wie in senkrechter Richtung.

{\glqq}Ein Walfisch! ... Ein Walfisch!{\grqq} rief Tom Turner noch
einmal.

Eben tauchte wirklich der R\"ucken eine{\s} solchen Cetaceer{\s} etwa
vier Kabell\"angen vor dem {\glqq}Albatro{\s}{\grqq} auf.

Der Aeronef st\"urzte gleichsam auf ihn zu und hielt, al{\s} er sich
kaum noch sechzig Fu{\ss} \"uber dem Thiere befand, schnell an.

Tom Turner hatte seine, in einer an der Reeling befestigten Gabel
liegende B\"uchse angeschlagen. Der Schu{\ss} krachte und da{\s}
Gescho{\ss}, da{\s} eine lange, mit ihrem Ende am Verdeck angebundene
Leine mit sich ri{\ss}, schlug in den K\"orper de{\s} Walfische{\s}
ein. Die mit leicht ent\/z\"undlichen Stoffen gef\"ullte Bombe
explodirte und schleuderte dabei eine Art kleinere, zweiarmige
Harpune, die sich in da{\s} Fleisch de{\s} Thiere{\s} einkrallte.

{\glqq}Achtung!{\grqq} rief Tom Turner. Trotz ihrer herzlich
schlechten Laune betrachteten Onkel Prudent und Phil Evan{\s}
diese{\s} Schauspiel doch mit aufrichtigem Interesse.

Der schwer verwundete Walfisch hatte da{\s} Meer mit dem Schwanze so
furchtbar gepeitscht, da{\ss} da{\s} Wasser bi{\s} zum Vordertheil
de{\s} Aeronef{\s} hinaufspritzte; dann tauchte derselbe bi{\s} zu
gro{\ss}er Tiefe hinab, w\"ahrend man die Leine schnell nachgleiten
lie{\ss}; letztere war \"ubrigen{\s} in einem mit Wasser gef\"ullten
Fasse zusammengelegt, um durch die Reibung nicht Feuer zu fangen.
Al{\s} der Walfisch wieder an die Oberfl\"ache kam, suchte er so
schnell al{\s} m\"oglich in der Richtung nach Norden zu entfliehen.

Der Leser kann sich leicht vorstellen, mit welch' rasender
Schnelligkeit der {\glqq}Albatro{\s}{\grqq} dabei geschleppt wurde,
denn die Triebschrauben waren vorher angehalten worden. Man
\"uberlie{\ss} da{\s} Thier ganz sich selbst und hielt sich nur im
gleicher Linie mit ihm. Tom Turner stand bereit, die Leine zu kappen,
wenn ein erneute{\s} Tauchen diese{\s} Schleppen gef\"ahrlich machte.

So wurde der {\glqq}Albatro{\s}{\grqq} etwa eine halbe Stunde lang
und vielleicht eine Entfernung von sech{\s} Meilen weit hingezerrt;
dann merkte man aber, da{\ss} der Cetaceer zu erlahmen anfing.

Jetzt lie{\ss}en die Hilf{\s}maschinisten da{\s} Triebwerk nach
r\"uckw\"art{\s} arbeiten und die Triebschrauben setzten dem
Walfisch, der sich dem Bord mehr und mehr n\"aherte einen gewissen
Widerstand entgegen.

Bald schwebte der Aeronef nur noch f\"unfundzwanzig Fu{\ss} \"uber
demselben; noch immer peitschte sein Schweif da{\s} Wasser mit fast
unglaublicher Gewalt, und wenn er sich vom Bauch auf den R\"ucken
drehte, w\"uhlte da{\s} Thier eine wirkliche Brandung auf.

Pl\"otzlich richtete e{\s} sich, so zu sagen, gerade in die H\"ohe
und tauchte mit solcher Schnelligkeit unter, da{\ss} Tom Turner kaum
Zeit hatte, ihm die Leine geh\"orig nachschie{\ss}en zu lassen.

Mit einem Male wurde der Aeronef bi{\s} zur Wasserfl\"ache
herabgezerrt; an der Stelle, wo da{\s} Thier verschwunden war, hatte
sich ein vollst\"andiger Wirbel gebildet, und \"uber die Reeling
hinein schlug da{\s} Wasser, wie e{\s} \"uber den Bug eine{\s}
Schiffe{\s} geht, gegen Wind und Wellen l\"auft.

Gl\"ucklicher Weise trennte Tom Turner noch recht\/zeitig mit einem
Axthiebe die Leine, und der nun befreite {\glqq}Albatro{\s}{\grqq}
stieg unter dem Drucke seiner Auftriebschrauben zweihundert Meter
empor.

Auch w\"ahrend diese{\s} aufregenden Zwischenfall{\s} hatte Robur den
Apparat geleitet, ohne da{\ss} ihn seine Kaltbl\"utigkeit nur einen
Augenblick verlassen h\"atte.

Einige Minuten sp\"ater kam der Walfisch wieder an die Oberfl\"ache
-- die{\s}mal aber todt.

Von allen Seiten flatterten die Seev\"ogel herzu, um sich de{\s}
Cadaver{\s} zu bem\"achtigen, und stie{\ss}en Schreie au{\s}, welche
einen sich zankenden Congre{\ss} taub gemacht h\"atten.

Der {\glqq}Albatro{\s}{\grqq}, der mit der todten Beute doch
nicht{\s} beginnen konnte, setzte seinen Weg nach Westen fort.

Am folgenden Tage, am 17. Juni, Morgen{\s} um 6 Uhr, erstreckte sich
am Horizonte Land hin. E{\s} war die Halbinsel Ala{\s}ka und die
lange Klippenreihe der Al\"euten.

Der {\glqq}Albatro{\s}{\grqq} zog \"uber dieser Barri\`ere hin, an
der e{\s} von Pelzseehunden wimmelte, welche die Al\"eutier f\"ur
Rechnung der russisch-amerikanischen Gesellschaft jagen. Der Fang
dieser sech{\s} bi{\s} sieben Fu{\ss} langen, fast rosenrothen und
zwei-, drei- bi{\s} f\"unfhundert Pfund wiegenden Amphibien ist f\"ur
sie ein sehr gute{\s} Gesch\"aft. Dieselben lagen hier in endloser
Reihe wie in Schlachtordnung und in Abertausenden von Exemplaren.

Wenn sie sich durch da{\s} Vor\"uberkommen de{\s}
{\glqq}Albatro{\s}{\grqq} nicht in ihrer phlegmatischen Ruhe st\"oren
lie{\ss}en, so war da{\s} nicht der Fall mit den Taucherv\"ogeln,
Polarenten und Ei{\s}tauchern, deren heisere{\s} Geschrei die Luft
erf\"ullte und welche unter dem Wasser verschwanden, al{\s} ob ein
entsetzliche{\s} Luftungeheuer sie bedrohte.

Die zweitausend Kilometer de{\s} Bering-Meere{\s} von den ersten
Al\"euten bi{\s} zur \"au{\ss}ersten Spitze von Kamtschatka wurden
w\"ahrend der vierundzwanzig Stunden diese{\s} Tage{\s} und der
folgenden Nacht zur\"uckgelegt. Um ihren Fluchtplan in'{\s} Werk zu
setzen, befanden sich Onkel Prudent und Phil Evan{\s} nicht gerade in
g\"unstigen Verh\"altnissen, denn weder an dem \"oden Strande de{\s}
n\"ordlichsten Asien{\s}, noch \"uber dem Ochot{\s}kischen Meere
h\"atten sie mit auch nur einiger Au{\s}sicht auf gl\"ucklichen
Erfolg entweichen k\"onnen.

Allem Anscheine nach wandte sich der {\glqq}Albatro{\s}{\grqq} nach
der Gegend von Japan oder China zu. Wenn e{\s} auch nicht sehr weise
sein mochte, sich auf die Unterst\"utzung von Chinesen oder Japanesen
zu verlassen, waren die beiden Collegen doch fest entschlossen, zu
fliehen, wenn der Aeronef an irgend einem Punkte dieser L\"ander
anhalten sollte.

Doch w\"urde er denn Halt machen? E{\s} lag ja bei ihm nicht so, wie
bei einem Vogel, der durch langen Flug endlich erm\"udet, oder wie
bei einem Ballon, der wegen Ga{\s}mangel gen\"othigt wird, einmal
niederzugehen. Der Aeronef besa{\ss} noch f\"ur mehrere Wochen
au{\s}haltende Vorr\"ate aller Art, und seine Organe von wunderbarer
Solidit\"at straften jede Erwartung auf Schw\"ache oder Tr\"agheit
L\"ugen.

Nach scharfer Fahrt \"uber die Halbinsel Kamtschatka, von der man
kaum die Niederlassung von Petropaulow{\s}k und den Vulcan von
Klutschew sah, und nach der weiteren, \"uber da{\s} Ochot{\s}kische
Meer, nahezu in der H\"ohe der Kurilen, welche darin einen von
Hunderten von Can\"alen unterbrochenen Damm bilden, erreichte der
{\glqq}Albatro{\s}{\grqq} am 19. Juni die La~P\'erouse-Stra{\ss}e
zwischen der Nordspitze von Japan und der Insel Sachalien an dem
kleinen Einschnitt, in welchen sich der gro{\ss}e sibirische Strom,
der Amur, ergie{\ss}t.

Nachher erhob sich ein dichter Nebel, den der Aeronef unter sich
lassen wollte, wenn er auch nicht gezwungen war, denselben zu meiden,
um weiter zu fahren, denn in der von ihm jetzt eingenommenen H\"ohe
hatte er kein Hinderni{\ss} zu f\"urchten, weder h\"ohere Bauwerke,
an welche er h\"atte ansto{\ss}en k\"onnen, noch Berge, an welchen er
sich im Fluge zu zertr\"ummern Gefahr gelaufen w\"are. Da{\s} Land
war kaum wellenf\"ormiger Natur. Die D\"unste machten sich aber doch
zu unangenehm f\"uhlbar, da sie Alle{\s} an Bord durchn\"a{\ss}ten.

E{\s} bedurfte ja nicht{\s} weiter, al{\s} sich \"uber diese
Nebelschicht, welche drei- bi{\s} vierhundert Meter stark sein
mochte, zu erheben. Die Schrauben wurden also in schnelle Umdrehung
versetzt, und oberhalb de{\s} Nebel{\s} fand der
{\glqq}Albatro{\s}{\grqq} wieder den reinen, vom Sonnenlicht
gebadeten Himmel.

Unter diesen Verh\"altnissen h\"atten Onkel Prudent und Phil Evan{\s}
M\"uhe gehabt, ihren Fluchtversuch au{\s}zuf\"uhren, selbst wenn sie
den Aeronef h\"atten verlassen k\"onnen.

An diesem Tage blieb Robur, al{\s} er einmal an ihnen vor\"uberkam,
wie zuf\"allig stehen und sagte, ohne \"au{\ss}erlich seinen Worten
besondere Bedeutung beizulegen:

{\glqq}Meine Herren, ein Segel- oder Dampfschiff, da{\s} in einen
Nebel gerieth, dem e{\s} nicht entrinnen kann, ist immer sehr genirt,
e{\s} f\"ahrt nur unter fortw\"ahrendem Pfeifen oder unter den
T\"onen de{\s} Nebelhorn{\s} weiter. E{\s} mu{\ss} seine Fortbewegung
verlangsamen und hat trotz aller Vorsicht jeden Augenblick eine
Collision zu bef\"urchten. Der {\glqq}Albatro{\s}{\grqq} kennt solche
Sorgen nicht. Wa{\s} k\"ummern ihn die Nebel, da er sich ihnen
ent\/ziehen kann? Ihm geh\"ort da{\s} Luftmeer, die ganze weite
Atmosph\"are!{\grqq}

Nach diesen Worten ging Robur ruhig weiter, ohne eine Antwort
abzuwarten, die er auch gar nicht verlangte, und die blauen
W\"olkchen seiner Pfeife zerflossen im Azur.

{\glqq}Onkel Prudent, begann da Phil Evan{\s}, e{\s} scheint, al{\s}
ob dieser merkw\"urdige {\glqq}Albatro{\s}{\grqq} ganz und gar
nicht{\s} zu f\"urchten habe.

-- Da{\s} werden wir noch sehen!{\grqq} antwortete der Vorsitzende
de{\s} Weldon-Institut{\s}.

Der Nebel hielt drei Tage lang, den 19., 20. und 21. Juni, mit
beklagen{\s}werther Z\"ahigkeit an. Man hatte hoch steigen m\"ussen,
um die japanesischen Gebirge von Fuji-Yama zu vermeiden. Al{\s}
dieser Nebelvorhang aber zerrissen war, gewahrte man eine ungeheure
Stadt mit Pal\"asten, Villen, Th\"urmchen, G\"arten und Park{\s}.
Selbst ohne dieselben zu sehen, h\"atte Robur sie schon erkannt an
dem Gebell der Tausende von Hunden, an dem Schreien der Raubv\"ogel
und vor Allem an dem Leichengeruch, den die K\"orper von
Hingerichteten in weitem Umkreise verbreiteten.

Die beiden Collegen befanden sich auf dem Deck, al{\s} der Ingenieur
eben da{\s} Besteck machte, f\"ur den Fall, da{\ss} er seine Fahrt
wieder im Nebel fort\/zusetzen gezwungen w\"are.

{\glqq}Meine Herren, begann er, ich habe keinen Grund, Ihnen zu
verheimlichen, da{\ss} diese Stadt Yeddo, die Hauptstadt von Japan
ist.{\grqq}

Onkel Prudent antwortete nicht. In Gegenwart de{\s} Ingenieur{\s}
keuchte er nur, al{\s} wenn e{\s} seinen seinen Lungen an Luft
fehlte.

{\glqq}Dieser Anblick Yeddo{\s} ist wirklich recht merkw\"urdig.

-- So merkw\"urdig er auch sein mag ... versetzte Phil Evan{\s}.

-- So bleibt er doch hinter dem von Peking zur\"uck, unterbrach ihn
der Ingenieur. Da{\s} ist meine Meinung auch, -- und Sie werden
binnen Kurzem selbst dar\"uber urtheilen k\"onnen.{\grqq}

Unm\"oglich h\"atte der Mann lieben{\s}w\"urdiger sein k\"onnen.

Der {\glqq}Albatro{\s}{\grqq}, der bi{\s}her auf S\"udost zuhielt,
ver\"anderte jetzt seine Richtung um vier Compa{\ss}striche, um im
Osten eine neue Route aufzusuchen.

W\"ahrend der Nacht zerstreute sich der Nebel, dagegen erschienen
Anzeichen eine{\s} nicht weit entfernten Typhon{\s}, denn der
Barometer fiel sehr rasch, alle Dunstmassen verschwanden, am fast
kupferfarbenen Grunde de{\s} Himmel{\s} ballten sich gro{\ss}e
elliptische Wolken zusammen und am entgegensetzten Horizont gl\"uhten
lange, carminrothe Streifen, die sich vom schieferblauen Hintergrunde
abhoben, im Norden aber war ein Theil de{\s} Himmel{\s} v\"ollig
klar. Da{\s} Meer lag zwar still; sein Wasser nahm jedoch mit
Sonnenuntergang eine dunkle Scharlachfarbe an.

Zum Gl\"uck entfesselte sich dieser Typhon mehr im S\"uden und hatte
hier keine weiteren Folgen, al{\s} da{\ss} er die seit drei Tagen
angeh\"auften Nebelmassen zertheilte.

Binnen einer Stunde hatte man die zweihundert Kilometer der Meerenge
von Korea und nachher die vorspringendste Spitze dieser Halbinsel
\"uberschritten; w\"ahrend der Typhon an den S\"udostk\"usten von
China w\"uthete, wiegte sich der {\glqq}Albatro{\s}{\grqq} \"uber dem
Gelben Meere, und w\"ahrend de{\s} 22. und 23. \"uber dem Golf von
Petscheli; am 24. glitt er da{\s} Thal de{\s} Pei-Ho hinauf und
gelangte endlich \"uber die Hauptstadt de{\s} Himmlischen Reiche{\s}.

Ueber die Reling hinau{\s}gebeugt, konnten die beiden Collegen -- wie
e{\s} der Ingenieur vorau{\s}gesagt -- sehr deutlich die ungeheure
Stadt sehen, die Mauer, welche sie in zwei ungleiche H\"alften, die
Mandschu- und die Chinesenstadt, theilt, ebenso wie die zw\"olf sie
umgebenden Vorst\"adte, die breiten, nach dem Mittelpunkte zu
verlaufenden Alleestra{\ss}en, die Tempel, deren gelbe oder gr\"une
D\"acher in der aufgehenden Sonne ergl\"anzten, die Park{\s}, welche
sich um die Pal\"aste der Mandarinen au{\s}dehnen; ferner, inmitten
der Mandschustadt, die sech{\s}hundertachtundsechzig Hektar (= 1/8
geogr. Quadratmeile) gro{\ss}e Gelbe Stadt mit ihren Pagoden, ihren
kaiserlichen G\"arten, k\"unstlichen Seen, dem die ganze Stadt
\"uberragenden Kohlenberge, und endlich unterschieden sie in der
Mitte der Gelben Stadt, gleich einer jener wunderbaren chinesischen
in einander geschachtelten Arbeiten, die Rothe Stadt, d.~i. den
eigentlichen Kaiserpalast, mit allen Phantasien seiner fast
unglaublichen Architektur.

Eben jetzt ert\"onte die Luft unter dem {\glqq}Albatro{\s}{\grqq} von
einer seltsamen Harmonie; man h\"atte ein Concert von Aeol{\s}harfen
zu h\"oren vermeint. In der Luft schwankten n\"amlich gegen hundert
verschieden geformte Drachen au{\s} Palmen- oder Pandanu{\s}papier
umher, deren oberen Theil eine Art leichten h\"olzernen Bogen{\s}
bildete, welcher durch ein ganz d\"unne{\s} Bambu{\s}st\"abchen
gespannt erhalten wurde. Unter dem schwachen Windhauche erzeugten
alle diese saitenartigen, verschiedene, denen einer Harmonika
\"ahnliche T\"one gebenden St\"abchen ein leise{\s} Gesumme von
h\"ochst melancholischer Wirkung. E{\s} machte den Eindruck, al{\s}
ob man hier in der H\"ohe -- musikalischen Sauerstoff einathme.

Da fiel e{\s} Robur ein, sich diesem Luftorchester zu n\"ahern, und
langsam tauchte der {\glqq}Albatro{\s}{\grqq} in die t\"onenden
Wellen herab, welche die Drachen nach der Atmosph\"are entsandten.

Pl\"otzlich entstand in der fast zahllosen Bev\"olkerung tief unten
eine au{\ss}erordentliche Aufregung. Tamtamschl\"age und andere
entsetzliche Instrumente de{\s} chinesischen Orchester{\s}
erschallten, Flintensch\"usse krachten und hundertfach h\"ammerten
die Leute auf gro{\ss}en M\"orsern herum, Alle{\s} in der Absicht,
den Aeronef zu verjagen. Wenn die Sternkundigen de{\s} chinesischen
Reiche{\s} an diesem Tage vielleicht erkannten, da{\ss} diese
Flugmaschine die veranlassende Ursache zu so vielen Streitigkeiten
der ganzen gelehrten Welt gewesen sein m\"ochte, so hielten die
Millionen Chinesen vom niedrigsten Manne bi{\s} zum vielkn\"opfigen
Mandarin sie jedenfall{\s} f\"ur ein apokalyptische{\s} Ungeheuer,
da{\s} am Himmel Buddha{\s} erschien.

In dem unnahbaren {\glqq}Albatro{\s}{\grqq} k\"ummerte sich
nat\"urlich Niemand um jene l\"armenden Kundgebungen. Die Bindf\"aden
aber, welche die Drachen an kleinen in den kaiserlichen G\"arten
eingerahmten Pf\"ahlen festhielten, wurden entweder zerschnitten oder
schnell eingezogen. Die leichten {\glqq}Spielzeuge{\grqq}, wie wir
sagen w\"urden, kamen dadurch, einen nur noch lauteren Ton gebend,
entweder rasch zur Erde, oder sie fielen herab, gleich fl\"ugellahm
geschossenen V\"ogeln, deren Gesang mit dem letzten Athemzuge
verstummt.

Da dr\"ohnte eine gewaltige Fanfare au{\s} der Trompete Tom
Turner'{\s} \"uber der Hauptstadt und \"ubert\"aubte die letzten
Kl\"ange jene{\s} Lufttonwerk{\s}, doch da{\s} machte dem Gewehrfeuer
unten kein Ende. Al{\s} aber eine Sprengkugel nur einige zwanzig
Fu{\ss} vom Verdeck de{\s} {\glqq}Albatro{\s}{\grqq} platzte, stieg
dieser nach den unerreichbaren Zonen de{\s} Himmel{\s} empor.

Im Laufe der n\"achstfolgenden Tage ereignete sich kein Zwischenfall,
den sich die Gefangenen h\"atten zu nutze machen k\"onnen. Die
Richtung de{\s} Aeronef{\s} blieb unab\"anderlich eine
s\"udwestliche, wa{\s} darauf hindeutete, da{\ss} er sich Hindostan
n\"ahern sollte. Uebrigen{\s} bemerkte man, da{\ss} der fortw\"ahrend
h\"oher aufsteigende Erdboden den {\glqq}Albatro{\s}{\grqq}
n\"othigte, sich nach den Linien seine{\s} Profil{\s} zu richten.
Etwa zehn Stunden nach der Weiterfahrt von Peking konnten Onkel
Prudent und Phil Evan{\s} an der Grenze von Chen-Si einen Theil der
Gro{\ss}en Mauer erkennen. Dann kamen sie, unter Umgehung der
Bung-Berge, \"uber und durch da{\s} Thal von Wany-Ho und
\"uberschritten die Grenze de{\s} chinesischen Kaiserreich{\s}, da,
wo diese mit Tibet zusammenst\"o{\ss}t.

Tibet bildet eine vegetation{\s}lose Hochebene, da und dort mit
schneebedeckten Gipfeln, trockenen Schluchten oder von Gletschern
gen\"ahrten Bergstr\"omen, mit Abgr\"unden, au{\s} welchen m\"achtige
Salzlager heraufschimmern, und mit vielen, von gr\"unenden Forsten
eingerahmten Seebecken.

Da{\s} auf 450 Millimeter gesunkene Wettergla{\s} zeigte jetzt eine
H\"ohe von viertausend Metern \"uber dem Meere an. In dieser H\"ohe
\"uberschritt die Temperatur, obgleich man sich jetzt in den
w\"armsten Monaten der n\"ordlichen Halbkugel befand, nicht den
Gefrierpunkt. Diese starke Abk\"uhlung im Verein mit der
Schnelligkeit de{\s} {\glqq}Albatro{\s}{\grqq} machte die Situation
fast unertr\"aglich, und obwohl die beiden Collegen warme Reisedecken
zur Verf\"ugung hatten, zogen sie e{\s} doch vor, in ihre Ruff{\s}
zur\"uckzukehren.

Selbstverst\"andlich mu{\ss}te den Auftrieb{\s}schrauben eine
au{\ss}erordentliche Schnelligkeit ertheilt werden, um den Aeronef in
der hier schon recht verd\"unnten Luft zu erhalten. Diese arbeiteten
jedoch in vorz\"uglichstem Zusammenwirken, und e{\s} schien, al{\s}
ob die Insassen de{\s} Apparat{\s} durch da{\s} Schwirren ihrer
Fl\"ugel gewiegt w\"urden.

An diesem Tage sah Garlok, eine Stadt de{\s} n\"ordlichen Tibet und
der Hauptort der Provinz Gavi-Khorsum, den {\glqq}Albatro{\s}{\grqq}
etwa in der Gr\"o{\ss}e einer Brieftaube vor\"uberschweben.

Am 27. Juni bemerkten Onkel Prudent und Phil Evan{\s} einen
gewaltigen Damm mit verschiedenen, in ewigem Schnee verlorenen
Spitzen, der den Horizont begrenzte.

An da{\s} Ruff auf dem Vordertheil gelehnt, um dem Luftdruck bei der
so schnellen Fortbewegung widerstehen zu k\"onnen, sahen Beide die
colossalen Bergmassen, welche dem Aeronef vorau{\s}zulaufen schienen.

{\glqq}Jedenfall{\s} der Himalaya, sagte Phil Evan{\s},
wahrscheinlich wird Robur nur den unteren Theil de{\s}selben
umkreisen, ohne nach Indien einzudringen.

-- Desto schlimmer, antwortete Onkel Prudent, auf diesem ungeheuren
Gebiete h\"atten wir vielleicht Gelegenheit~--

-- Wenigsten{\s}, wenn er um die Bergkette nicht \"uber Birma im
Osten oder \"uber Nepal im Westen f\"ahrt.

-- Ich m\"ochte darauf wetten, da{\ss} er \"uber dieselben gehen
wird.

-- Jedenfall{\s}!{\grqq} lie{\ss} sich da eine Stimme vernehmen.

Am folgenden Tage, am 28. Juni, befand sich der {\glqq}Albatro{\s}{\grqq}
\"uber der Provinz Zyang gegen\"uber jenen gewaltigen Bergmassen. An
der anderen Seite de{\s} Himalaya lag da{\s} Gebiet von Nepal.

Wenn man von Norden kommt, schneiden nacheinander drei
Gebirg{\s}ketten den Weg nach Indien. Die beiden n\"ordlichen,
zwischen denen der {\glqq}Albatro{\s}{\grqq} wie ein Schiff zwischen
ungeheuren Klippen dahinglitt, sind die ersten Stufen de{\s}
Grenzwalle{\s} im S\"uden von Central-Asien. Der Kuen-L\"un, und nach
diesem der Karakorum bezeichnen zuerst diese{\s} l\"angliche und mit
dem Himalaya parallel verlaufende Thal, ungef\"ahr in jener
H\"ohenlage, in welcher sich die Stromgebiete de{\s} Indu{\s} im
Westen und de{\s} Brahmaputra im Osten abgabeln.

Welch' wunderbare{\s} orographische{\s} System! Hier ragen \"uber
zweihundert schon gemessene Gipfel auf, von denen siebzehn
f\"unfundzwanzigtausend Fu{\ss} \"ubersteigen! Vor dem
{\glqq}Albatro{\s}{\grqq} erhob sich der Mount Everest auf
achttausendachthundertvierzig Meter H\"ohe; ihm zur Rechten der
Dawalaghiri, achttausendzweihundert Meter hoch; zur Linken der
Kinahanjunga, achttausendf\"unfhundert\/zweiundneunzig Meter, der also
seit den letzten genaueren Messungen de{\s} Mount Everest nur noch
die zweite Stelle einnimmt.

Offenbar hatte Robur nicht die Absicht, \"uber jene Gipfel
hinwegzugehen, sondern er kannte zweifel{\s}ohne schon die
verschiedenen P\"asse de{\s} Himalaya, unter Anderen den
Ibi-Yamin-Pa{\ss}, den die Gebr\"uder Schlagintweit 1856 in einer
H\"ohe von sech{\s}tausendachthundert Metern \"uberschritten haben;
wenigsten{\s} hielt er entschlossen auf diesen zu.

Jetzt kamen einige \"angstliche, selbst sehr beschwerliche Stunden,
und wenn die Verd\"unnung der Luft auch nicht einen solchen Grad
erreichte, da{\ss} man zu eigen{\s} daf\"ur construirten Apparaten
h\"atte greifen m\"ussen, den Sauerstoff in den Cabinen zu erneuern,
so wurde die K\"alte doch h\"ochst beschwerlich.

Auf dem Vordertheile stehend und die kr\"aftige Gestalt in einen
Mantel geh\"ullt, leitete Robur alle Man\"over. Tom Turner hielt die
Barre de{\s} Steuerruder{\s} fest in der Hand. Der Maschinist
\"uberwachte aufmerksam seine Batterien, von deren S\"auren
gl\"ucklicher Weise ein Einfrieren nicht zu f\"urchten war. Die zur
allergr\"o{\ss}ten Umdrehung{\s}geschwindigkeit angetriebenen
Schrauben gaben einen immer sch\"arfer werdenden Ton, der trotz der
h\"ochst d\"unnen Luft laut vernehmbar blieb. Der Barometer fiel auf
zweihundertneunzig Millimeter, wa{\s} eine H\"ohe von siebentausend
Metern anzeigte.

Wie prachtvoll lag diese{\s} Chao{\s} von Bergriesen hier vor dem
erstaunten Blicke au{\s}gebreitet! Ueberall wei{\ss}gl\"anzende
Gipfel, keine Seen, aber gewaltige schimmernde Gletscher, die bi{\s}
auf zehntausend Fu{\ss} H\"ohe hinabreichen. Kein Gra{\s}, au{\ss}er
einigen d\"urftigen Kryptogamen an der Grenze de{\s} vegetabilischen
Leben{\s}, nicht{\s} von jenen wundersch\"onen Fichten und Cedern,
die sich an den unteren Abh\"angen der Kette in herrlichen W\"aldern
vorfinden; nicht{\s} von gigantischen Farren und endlosen
Schmarotzerpflanzen, die sich, wie im Unterholz der Dschungeln, von
Baum zu Baum hinziehen. Kein Thier, weder wilde Pferde, noch Yak{\s}
oder tibetanische Rinder; dann und wann nur eine Gazelle, die sich
bi{\s} nach diesen Oeden hinein verirrt hatte; keine V\"ogel,
au{\ss}er einzelnen jener P\"archen Raben, welche sich bi{\s} zu den
letzten Schichten der athembaren Luft erheben.

Nachdem er diesen Pa{\ss} durchschritten, begann der
{\glqq}Albatro{\s}{\grqq} wieder hinabzusteigen. Al{\s} sie dessen
Au{\s}gang passirten, hatten die Reisenden, jenseit{\s} der Region
der Bergwaldung, eine grenzenlose Landschaft vor sich, die sich in
weitem Umkreise vor ihnen au{\s}dehnte.

Jetzt trat Robur an seine G\"aste heran und sagte mit
lieben{\s}w\"urdigem Tone:

{\glqq}Da haben Sie Indien, meine Herren!{\grqq}



\newpage\begin{center}\label{kap10}
{\large \begin{antiqua}X.\end{antiqua}\\
Worin man sehen wird, wie und warum der Diener Frycollin in'{\s}
Schlepptau genommen wurde.\\\bigskip}
\end{center}



Der Ingenieur hatte nicht die Absicht, seinen Apparat \"uber die
wundervollen Gefilde von Hindostan hinwegzuf\"uhren. Jedenfall{\s}
wollte er nur den Himalaya \"ubersteigen, um zu beweisen, \"uber
welch' au{\ss}erordentliche Fortbewegung{\s}maschine er verf\"ugte,
und um davon selbst Diejenigen zu \"uberzeugen, welche nicht
\"uberzeugt sein wollten. Bedeutete da{\s} wohl so viel wie die
Behauptung, da{\ss} der {\glqq}Albatro{\s}{\grqq} vollkommen sei,
obgleich die Vollkommenheit nicht von dieser Welt ist? Da{\s} wird
sich sp\"ater zeigen.

Wenn Onkel Prudent und sein College auch nicht umhin konnten,
innerlich anzuerkennen, da{\ss} die Kraft dieser Flugmaschine eine
ganz au{\ss}erordentliche war, so lie{\ss}en sie sich davon
wenigsten{\s} nicht{\s} merken. Sie suchten nur die Gelegenheit, zu
entfliehen; ja, sie bewunderten nicht einmal da{\s} prachtvolle
Schauspiel, welche{\s} sich ihren Augen bot, al{\s} der
{\glqq}Albatro{\s}{\grqq} den reizenden Landschaften de{\s} Pendjab
folgte.

Wohl giebt e{\s} am Himalaya einen Strich sumpfigen Lande{\s}, von
dem gesundheit{\s}sch\"adliche D\"unste aufsteigen, jene{\s} Terrain,
in dem Fieberkrankheiten epidemisch herrschen. Doch da{\s} ging den
{\glqq}Albatro{\s}{\grqq} ja nicht{\s} an und konnte da{\s}
Wohlbefinden seiner Insassen nicht gef\"ahrden, er erhob sich ohne
gro{\ss}e Eile nach dem Winkel zu, den Hindostan in seinem
Vereinigung{\s}punkt mit Turkestan und China bildet. Am 29. Juni
\"offnete sich vor ihm schon in den ersten Morgenstunden da{\s}
herrliche Thal von Kaschmir.

Ja, sie ist ohne Gleichen, diese Hohlkehle, welche der Himalaya
zwischen sich frei l\"a{\ss}t! Gefurcht von Hunderten von
Einzelvorspr\"ungen, welche die ungeheure Kette bi{\s} zum Becken
de{\s} Hydaspi{\s} entsendet, wird dieselbe bew\"assert von den
launischen Windungen de{\s} Flusse{\s}, der die Heers\"aulen
Poru{\s}' und Alexander{\s}, d.~h. Indien und Griechenland, in
Central-Asien zum Kampfe zusammensto{\ss}en sah. Er f\"ullt noch
immer sein Bett, dieser Hydaspi{\s}, w\"ahrend die von dem Macedonier
zur Erinnerung an seinen Sieg gegr\"undeten beiden St\"adte so
vollst\"andig verschwunden sind, da{\ss} man nicht einmal im Stande
ist, die Stelle derselben wieder zu finden.

W\"ahrend diese{\s} Vormittag{\s} schwebte der {\glqq}Albatro{\s}{\grqq}
\"uber Srinagar -- mehr bekannt unter dem Namen Kaschmir -- hin.

Onkel Prudent und sein Gef\"ahrte sahen eine sehr sch\"one, an beiden
Flu{\ss}ufern sich hinziehende Stadt mit ihren Br\"ucken gleich
au{\s}gespannten F\"aden, den Sennh\"utten mit ihren geschnitzten
Balkon{\s}, ihren von hohen Pappeln beschatteten Geb\"auden mit
berasten D\"achern, welche fast da{\s} Au{\s}sehen gro{\ss}er
Maulwurf{\s}haufen haben, ihren vielfachen Can\"alen mit Barken
gleich Nu{\ss}schalen und Boot{\s}leuten gleich Ameisen darauf, mit
ihren Pal\"asten, Tempeln, Kio{\s}k{\s}, Moscheen und den
Bungalow{\s} am Eingange der Vorst\"adte -- da{\s} Ganze auch noch
verdoppelt durch die Widerspiegelung de{\s} Wasser{\s}; endlich die
alte Citadelle Hari-Parvata, die auf einem H\"ugel angelegt ist, wie
da{\s} st\"arkste Fort von Pari{\s} auf dem Mont-Val\'erien.

{\glqq}Da{\s} w\"are Venedig, wenn wir un{\s} in Europa
bef\"anden,{\grqq} sagte Phil Evan{\s}.

-- Und wenn wir in Europa w\"aren, w\"urden wir den R\"uckweg nach
Amerika schon zu finden wissen,{\grqq} antwortete Onkel Prudent.

Der {\glqq}Albatro{\s}{\grqq} verweilte nicht \"uber dem See, den der
Flu{\ss} durchflie{\ss}t, sondern setzte seinen Flug durch da{\s}
Thal de{\s} Hydaspi{\s} fort.

Nur eine halbe Stunde blieb er, bi{\s} auf zehn Meter \"uber dem
Flusse hinabsteigend, einmal an ein und derselben Stelle. W\"ahrend
dessen versorgten sich Tom Turner und seine Leute mittelst eine{\s}
Kautschukschlauche{\s} mit neuem Wasservorrathe, der durch eine Pumpe
aufgesaugt wurde, welche die Str\"ome der Accumulatoren in Bewegung
setzten.

Onkel Prudent und Phil Evan{\s} hatten sich dabei bedeutung{\s}voll
angesehen, da ein und derselbe Gedanke in ihnen aufstieg. Sie
befanden sich nur wenige Meter \"uber der Oberfl\"ache de{\s}
Hydaspi{\s} und nahe dem Ufer de{\s}selben. Beide waren ge\"ubte
Schwimmer. Ein Sprung konnte ihnen jetzt die Freiheit wiedergeben,
und wenn sie dann ein St\"uck unter dem Wasser fortschwammen, wie
h\"atte Robur sie wieder ergreifen lassen k\"onnen? Um den
Treibschrauben ihre Beweglichkeit zu sichern, mu{\ss}te er sie ja mit
seinem Apparate mindesten{\s} zwei Meter \"uber dem Seebecken halten.

In einem Augenblicke hatten sie alle g\"unstigen und ung\"unstigen
Um\-st\"an\-de eine{\s} solchen Versuch{\s} gegen einander abgewogen
und schon waren sie im Begriff, sich von dem Verdeck de{\s}
Luftschiffe{\s} hinabzust\"urzen, al{\s} sich mehrere H\"ande fest
auf ihre Schultern legten.

Sie wurden beobachtet und erkannten die Unm\"oglichkeit, zu
entfliehen.

Immerhin ergaben sie sich nicht ohne einigen Widerstand und
bem\"uhten sich, die, welche sie hielten, zur\"uckzusto{\ss}en --
aber e{\s} waren handfeste Burschen, diese Leute de{\s}
{\glqq}Albatro{\s}{\grqq}!

{\glqq}Meine Herren, begn\"ugte sich der Ingenieur zu sagen, wenn man
da{\s} Vergn\"ugen hat, in Gesellschaft mit Robur dem Sieger zu
reisen, wie Sie ihn selbst so passend bezeichnet haben, und an Bord
seine{\s} wunderbaren {\glqq}Albatro{\s}{\grqq}, so verl\"a{\ss}t man
diesen nicht so ... franz\"osisch. Ja, ich sage Ihnen, Sie verlassen
denselben \"uberhaupt nicht wieder!{\grqq}

Phil Evan{\s} zerrte seinen Gef\"ahrten, der sich schon zu einem
Gewaltacte hinrei{\ss}en lassen wollte, noch zur\"uck. Beide begaben
sich nach ihrem Ruff, noch immer entschlossen, zu fliehen und wenn
e{\s} ihnen, gleichviel wo, auch da{\s} Leben kosten sollte.

Der {\glqq}Albatro{\s}{\grqq} hatte wieder seinen Cur{\s} nach Westen
eingeschlagen. W\"ahrend diese{\s} Tage{\s} \"uberschritt er bei
mittlerer Geschwindigkeit da{\s} Gebiet von Kabulistan, die Grenze
de{\s} K\"onigreich{\s} Herat.

In diesen noch immer so bestrittenen L\"andern und auf diesem Wege,
der den Russen nach den englischen Besitzungen in Indien offen steht,
erschienen gro{\ss}e Haufen von Menschen, Colonnen, Gep\"ackwagen,
mit einem Worte Alle{\s}, wa{\s} da{\s} Personal und Material einer
auf dem Marsche befindlichen Armee bildet. Man h\"orte wohl auch
Kanonendonner und da{\s} Knattern von Gewehren; der Ingenieur mischte
sich aber niemal{\s} in die Angelegenheiten Anderer, so lange diese
f\"ur ihn nicht eine Frage de{\s} Ehrgeize{\s} oder der Humanit\"at
bildeten. War Herat, wie man sagt, wirklich der Schl\"ussel
Central-Asien{\s}, so k\"ummerte e{\s} ihn doch gar nicht, ob dieser
Schl\"ussel in eine englische oder eine mo{\s}kowitische Tasche kam.
Irdische Interessen ber\"uhrten den furchtlosen Mann nicht, der
da{\s} Luftmeer zu seinem au{\s}schlie{\ss}lichen Gebiete erkoren
hatte.

Uebrigen{\s} schwand da{\s} Land sehr bald unter einem wahrhaften
Orkan von Sand, wie er in diesen Gegenden so h\"aufig vorkommt.
Dieser Sturmwind, der hier {\glqq}Tebbad{\grqq} genannt wird, tr\"agt
manche Fieberkeime mit dem unw\"agbar feinen Sand oft sehr weit mit
fort, und manche Caravane ist schon in seinen w\"uthenden Wirbeln zu
Grunde gegangen.

Um diesem harten Staube zu entgehen, der die Feinheit seiner
Zahngetriebe h\"atte gef\"ahrden k\"onnen, erhob sich der
{\glqq}Albatro{\s}{\grqq} um zweitausend Meter nach einer reineren
Zone.

Damit schwand auch die Grenze Persien{\s} au{\s} den Augen und
blieben dessen weite Ebenen fast ganz unsichtbar. Die Gangart war
dabei eine sehr gem\"a{\ss}igte, obwohl eine Felsenklippe nirgend{\s}
zu f\"urchten war. Wenn eine Landkarte dieser Gegend auch einige
Berge zeigte, so steigen diese doch nur zu mittlerer H\"ohe an. Bei
der Ann\"aherung an die Hauptstadt freilich galt e{\s}, den Demawend
zu vermeiden, der fast sech{\s}tausendsech{\s}hundert Meter
emporragt, und auch die Elbru{\s}kette, an deren Fu{\ss} Teheran
erbaut ist.

Mit dem ersten Tage{\s}grauen de{\s} 2. Juli tauchte jener Demawend
au{\s} dem Sand-Samum auf.

Der {\glqq}Albatro{\s}{\grqq} steuerte so, um \"uber die Stadt
hinwegzugehen, welche der Wind durch eine Wolke feinen Staube{\s}
verh\"ullte.

Gegen zehn Uhr Morgen{\s} konnte man inde{\ss} die breiten Gr\"aben
erkennen, welche die Umwallung einschlie{\ss}en, und in der Mitte den
Palast de{\s} Schah, dessen Mauern mit Fayenceplatten bedeckt sind
und dessen Wasserbecken au{\s} ungeheuren T\"urkisen von leuchtendem
Blau geschnitten scheinen.

Da{\s} sch\"one Bild verrann leider nur zu bald. Von hier au{\s}
schlug der {\glqq}Albatro{\s}{\grqq} nun eine andere Richtung ein und
steuerte ziemlich genau nach Norden. Einige Stunden sp\"ater befanden
sie sich \"uber einer kleinen Stadt im n\"ordlichen Winkel der
persischen Grenze und am Strande einer au{\s}gedehnten
Wasserfl\"ache, deren Ende weder nach Norden, noch nach Osten zu
erkennbar war.

Diese Stadt war der Hafen Aschuarda, die s\"udlichste Station
Ru{\ss}land{\s}; die Wasserfl\"ache aber fast ein Meer, n\"amlich der
Kaspi-See.

Hier wirbelte kein Staub mehr umher. Man sah bequem einen Haufen nach
europ\"aischer Art gebauter H\"auser, welche, mit einem sie
\"uberragenden Glockenthurm, l\"ang{\s} eine{\s} Vorgebirge{\s}
lagen.

Der {\glqq}Albatro{\s}{\grqq} senkte sich \"uber diese{\s} Meer,
dessen Gew\"asser dreihundert Fu{\ss} unter dem Niveau de{\s}
Mittelmeere{\s} liegen. Gegen Abend glitt er l\"ang{\s} der fr\"uher
turkestanischen, jetzt aber russischen K\"uste hin, die nach dem Golf
de{\s} Becken{\s} zu aufsteigt, und am n\"achsten Tage, dem 3. Juli,
schwebte er etwa hundert Meter \"uber dem Kaspi-See.

Weder an der asiatischen, noch an der europ\"aischen Seite war hier
Land in Sicht; nur auf dem Meer bemerkte man einzelne, von schwacher
Brise geschwellte Segel, an deren Form man erkannte, da{\ss} e{\s}
Fahrzeuge von Eingeborenen, Kesebeg{\s} mit zwei Masten, Kajik{\s},
da{\s} sind Piratenschiffe mit nur einem Maste, und Teimil{\s},
einfache, zur K\"ustenfahrt oder zum Fischfang ben\"utzte Boote
waren. Dann und wann wirbelten wohl auch die Au{\s}l\"aufer von
Rauchs\"aulen bi{\s} zum {\glqq}Albatro{\s}{\grqq} empor, welche
au{\s} den Schornsteinen der Dampfer von Aschuarda quollen, die
Ru{\ss}land zu Polizeizwecken auf den turkomanischen Gew\"assern
unterh\"alt.

An diesem Morgen plauderte der Obersteuermann Tom Turner mit dem Koch
Fran\c{c}oi{\s} Tapage und gab auf eine Frage de{\s} Letzteren
Antwort:

{\glqq}Ja, wir werden gegen achtundvierzig Stunden \"uber dem
Kaspi-See verweilen.

-- Sch\"on, erwiderte der Koch, da haben wir doch einmal Gelegenheit,
zu fischen?

-- Ganz gewi{\ss}.{\grqq}

Da \"uber vierzig Stunden darauf verwendet werden sollten, die
sech{\s}hundertf\"unfundzwanzig Meilen, welche jene{\s} Binnenmeer
bei zweihundert (englischen) Meilen Breite mi{\ss}t, mu{\ss}te die
Geschwindigkeit de{\s} {\glqq}Albatro{\s}{\grqq} nat\"urlich stark
gem\"a{\ss}igt und letzterer w\"ahrend eine{\s} vorzunehmenden
Fischfange{\s} ganz still gehalten werden.

Jene Antwort Tom Turner'{\s} wurde auch von Phil Evan{\s} geh\"ort,
der sich grade auf dem Vordertheil befand.

Eben begann Frycollin wieder mit seinen unaufh\"orlichen Klagen und
bat ihn, bei seinem Herrn ein gute{\s} Wort einzulegen, da{\ss} er
ihn {\glqq}auf der Erde absetzen{\grqq} lasse.

Ohne auf diese{\s} sinnlose Verlangen zu antworten, begab sich Phil
Evan{\s} nach dem Hintertheil, um den Onkel Prudent zu treffen.
Diesem theilte er unter gr\"o{\ss}ter Vorsicht, von Niemand geh\"ort
zu werden, die wenigen zwischen Tom Turner und dem Koche gewechselten
Worte mit.

{\glqq}Phil Evan{\s}, meinte Onkel Prudent, ich denke, wir machen
un{\s} doch keine Illusionen \"uber die letzten Absichten diese{\s}
Elenden?

-- Gewi{\ss} nicht, antwortete Phil Evan{\s}. Er wird un{\s} die
Freiheit nur wiedergeben, wenn ihm da{\s} pa{\ss}t -- und wenn er sie
un{\s} \"uberhaupt wieder giebt.

-- In diesem Falle m\"ussen wir Alle{\s} wagen, um den
{\glqq}Albatro{\s}{\grqq} zu verlassen.

-- Ein wundervoller Apparat, da{\s} mu{\ss} man wohl zugestehen!

-- Da{\s} ist wohl m\"oglich, rief Onkel Prudent, aber e{\s} ist der
Apparat eine{\s} Schurken, der un{\s} gegen alle{\s} Recht und Gesetz
hier zur\"uckh\"alt. Uebrigen{\s} bildet dieser Apparat f\"ur un{\s}
und die Unsrigen eine unau{\s}gesetzte Gefahr. Gelingt e{\s} un{\s}
also nicht, denselben zu vernichten~...

-- Beginnen wir damit, un{\s} zu retten! ... antwortete Phil
Evan{\s}, wir werden ja sp\"ater sehen.

-- Zugegeben, antwortete Onkel Prudent, und ben\"utzen wir jede sich
bietende Gelegenheit. Allem Anscheine nach f\"ahrt der
{\glqq}Albatro{\s}{\grqq} \"uber den Kaspi-See, um sich dann im
Norden oder im S\"uden von Ru{\ss}land nach Europa zu begeben. Nun,
wohin wir auch den Fu{\ss} setzen m\"ogen, bi{\s} zum Atlantischen
Ocean hin w\"are unsere Rettung gesichert. Wir m\"ussen un{\s} also
jede Stunde bereit halten.

-- Aber, fragte Phil Evan{\s}, wie sollten wir fliehen k\"onnen?

-- H\"oren Sie mich an, antwortete Onkel Prudent. E{\s} kommt
zuweilen vor, da{\ss} der {\glqq}Albatro{\s}{\grqq} w\"ahrend der
Nacht nur wenige hundert Fu{\ss} \"uber dem Erdboden hinschwebt. An
Bord befinden sich verschiedene Kabel von dieser L\"ange, und mit
einiger K\"uhnheit k\"onnte man sich wohl hinabgleiten lassen~...

-- Ja, stimmte Phil Evan{\s} bei, im gegebenen Falle w\"urde ich
nicht zaudern~...

-- Ich auch nicht, versicherte Onkel Prudent. Ich f\"uge noch hinzu,
da{\ss} w\"ahrend der Nacht au{\ss}er dem Steuermann auf dem
Hintertheile Niemand wach ist. Eine{\s} jener Kabel liegt nun
gew\"ohnlich auf dem Verdeck, und ohne gesehen und geh\"ort zu
werden, d\"urfte e{\s} m\"oglich sein, da{\s}selbe aufzurollen~...

-- Gut, gut, unterbrach ihn Phil Evan{\s}; ich sehe mit Vergn\"ugen,
Onkel Prudent, da{\ss} Sie jetzt weit ruhiger sind; da{\s} ist
besser, wenn man handeln will. Augenblicklich freilich befinden wir
un{\s} auf dem Kaspi-See; verschiedene Fahrzeuge sind in Sicht. Der
{\glqq}Albatro{\s}{\grqq} wird noch tiefer hinabgehen und w\"ahrend
de{\s} Fischzuge{\s} anhalten ... K\"onnten wir darau{\s} keinen
Vortheil ziehen?~...

-- Ah, man \"uberwacht un{\s}, selbst wenn wir nicht glauben,
\"uberwacht zu sein, antwortete Onkel Prudent. Sie haben'{\s} ja
gesehen, al{\s} wir versuchten, un{\s} in den Hydaspi{\s} zu
st\"urzen.

-- Und wer sagt, da{\ss} wir nicht auch in der Nacht beobachtet sind?
erwiderte Phil Evan{\s}.

-- Einerlei, wir m\"ussen ein Ende machen, rief Onkel Prudent, ein
Ende machen mit diesem {\glqq}Albatro{\s}{\grqq} und seinem
Besitzer!{\grqq}

Man sieht, da{\ss} die beiden Collegen -- und vorz\"uglich Onkel
Prudent -- unter der Aufregung de{\s} Zorne{\s} leicht dazu
verf\"uhrt werden konnten, die waghalsigsten und f\"ur ihre eigene
Sicherheit vielleicht gef\"ahrlichsten Handlungen zu begehen.

Da{\s} Gef\"uhl ihrer Ohnmacht, der ver\"achtliche Spott, mit dem
Robur sie behandelte, die derben Antworten, welche er ihnen
ertheilte, Alle{\s} trug dazu bei, die Spannung ihrer Lage zu
erh\"ohen, deren Druck jeden Tag deutlicher hervortrat.

An jenem Tage h\"atte \"ubrigen{\s} ein neuer Auftritt bald einen
h\"ochst bedauerlichen Wortwechsel zwischen Robur und den beiden
Collegen herbeigef\"uhrt, und Frycollin ahnte wohl kaum, da{\ss} er
dazu die Veranlassung geben sollte.

Al{\s} er sich einmal \"uber diesem Meere ohne Grenzen sah,
bem\"achtigte sich de{\s} Hasenfu{\ss}e{\s} wieder ein furchtbarer
Schrecken. Wie ein Kind -- und wie ein Neger, der er ja war -- fing
er an zu jammern zu klagen und zu protestiren und machte die tollsten
Verrenkungen und Grimassen.

{\glqq}Ich will fort! ... Ich will weg von hier! rief er. Ich bin
kein Vogel! ... Ich bin nicht geschaffen zum Fliegen! ... Ich will,
da{\ss} ich auf der Erde abgesetzt werde, und da{\s} sogleich!{\grqq}

Selbstverst\"andlich bem\"uhte sich Onkel Prudent keine{\s}weg{\s},
ihn zu beruhigen, im Gegentheil. Da{\s} Heulen de{\s} Schwarzen
erregte denn auch die Ungeduld Robur'{\s}.

Da Tom Turner und die Anderen sich eben zum Fischfang anschickten,
befahl der Ingenieur, um sich Frycollin{\s} zu entledigen, diesen in
sein Ruff einzusperren. Der Neger setzte da{\s} vorige Unwesen fort,
donnerte an die Wand und heulte au{\s} Leibe{\s}kr\"aften.

E{\s} war jetzt Mittag. Der {\glqq}Albatro{\s}{\grqq} schwebte eben
nur f\"unf oder sech{\s} Meter \"uber der Oberfl\"ache de{\s}
Meere{\s}. Einige bei seiner Ann\"aherung erschreckte Boote waren
eiligst davongefahren. Dieser Theil de{\s} Kaspi-See{\s} mu{\ss}te
also bald ganz verlassen sein.

Man begreift leicht, da{\ss} die beiden Collegen unter diesen
Verh\"altnissen, wo sie gelegentlich nur h\"atten mit dem Kopfe zu
nicken brauchen, der Gegenstand erh\"ohter Aufmerksamkeit sein
mu{\ss}ten und wirklich waren.

Doch selbst angenommen, da{\ss} sie sich \"uber Bord gest\"urzt
h\"atten, so w\"are e{\s} doch leicht gewesen, sie mit Hilfe de{\s}
Kautschukboote{\s} de{\s} {\glqq}Albatro{\s}{\grqq} wieder
einzufangen. W\"ahrend diese{\s} Fischzuge{\s} war also nicht{\s} zu
thun, und Phil Evan{\s} betheiligte sich lieber selbst th\"atig
dabei, w\"ahrend Onkel Prudent im Zustand fortw\"ahrend kochender
Wuth sich in seine Cabine zur\"uckzog.

Bekanntlich bildet der Kaspi-See eine betr\"achtliche Bodendepression
wahrscheinlich vulcanischen Ursprunge{\s}. In diese{\s} Becken
ergie{\ss}en sich die Gew\"asser sehr gro{\ss}er Str\"ome, wie der
Wolga, de{\s} Ural, de{\s} Kur, der Kuma, Jemba u.~A. Ohne die starke
Verdunstung, welche dem Wasserbecken den Wasser\"uberflu{\ss} wieder
entf\"uhrt, h\"atte diese{\s} siebzehntausend Quadratmeilen gro{\ss}e
Loch von f\"unf- bi{\s} sech{\s}hundert Fu{\ss} mittlerer Tiefe schon
l\"angst die niedrigen und sumpfigen K\"usten im Norden und Osten
\"uberfluthet. Obgleich diese Schale weder mit dem Schwarzen, noch
mit dem Aral-Meer in Verbindung steht, deren Niveau weit h\"oher
liegt, so ern\"ahrt e{\s} doch eine gro{\ss}e Menge Fische -- wohl zu
bemerken aber nur solche, welche die stark hervortretende Bitterkeit
seine{\s} Wasser{\s}, eine Folge der Naphthaquellen am S\"udende
de{\s}selben, vertragen.

Bei dem Gedanken an die Abwech{\s}lung, welche dieser Fischzug ihrem
gewohnten Speisezettel zu verleihen versprach, gab die Mannschaft
de{\s} {\glqq}Albatro{\s}{\grqq} die Befriedigung, welche er
derselben gew\"ahrte, deutlich genug zu erkennen.

{\glqq}Achtung!{\grqq} rief Tom Turner, der eben einen Fisch von
ziemlich bedeutender Gr\"o{\ss}e und \"ahnlich einem Haifisch
harpunirt hatte.

E{\s} war da{\s} ein pr\"achtiger, gegen sieben Fu{\ss} langer
St\"or, von der Art, welche die Russen Belonga nennen, dessen mit
Salz, Essig und Wei{\ss}wein zugerichtete Eier den Caviar darstellen.
Vielleicht sind die in den Fl\"ussen gefangenen St\"ore noch
schmackhafter al{\s} die au{\s} dem Meere. Doch wurden letztere an
Bord de{\s} {\glqq}Albatro{\s}{\grqq} mit gro{\ss}em Jubel
begr\"u{\ss}t.

Noch weit ergiebiger gestaltete sich dieser Fischzug aber durch
Anwendung von Schleppnetzen, in welchen e{\s} bald von Karpfen,
Brachsen und Seehechten, vorz\"uglich von jenen mittelgro{\ss}en
Sterlet{\s} wimmelte, welche reiche Feinschmecker lebend von
Astrachan nach Mo{\s}kau und Peter{\s}burg bringen lassen. Diese hier
wanderten -- ohne alle Tran{\s}portkosten -- unmittelbar au{\s} ihrem
nat\"urlichen Element in die Siedekessel der Mannschaft{\s}k\"uche.

Die Leute Robur'{\s} zogen mit gro{\ss}em Vergn\"ugen die Leine ein,
nachdem der {\glqq}Albatro{\s}{\grqq} sie mehrere Stunden lang
langsam dahingef\"uhrt hatte. Der Ga{\s}cogner Tapage (der Name
bedeutet deutsch: L\"armen, Get\"ose) machte durch sein Jubelgeschrei
seinem Namen alle Ehre. Eine Stunde gen\"ugte, alle Beh\"alter de{\s}
{\glqq}Albatro{\s}{\grqq} mit jenem Nahrung{\s}material zu f\"ullen,
und dieser fuhr darauf nach Norden zu weiter.

W\"ahrend diese{\s} Aufenthalte{\s} hatte Frycollin nicht
aufgeh\"ort, zu schreien, an die Wand seiner Cabine zu h\"ammern, mit
einem Worte, einen unau{\s}stehlichen L\"arm zu machen.

{\glqq}Wird dieser verdammte Nigger denn nicht Ruhe halten lernen!
sagte Robur, dem die Geduld nun wirklich zu Ende ging.

-- Mir scheint, Herr Robur, da{\ss} er v\"ollig Recht hat, sich zu
beklagen, bemerkte Phil Evan{\s}.

-- Ja, ganz wie ich da{\s} Recht habe, meinen Ohren diese Qual zu
ersparen, erwiderte Robur.

-- Ingenieur Robur! ... lie{\ss} sich da der eben auf dem Verdeck
erscheinende Onkel Prudent vernehmen.

-- Herr Pr\"asident de{\s} Weldon-Institut{\s}?{\grqq}~...

Beide waren auf einander zugetreten und sahen sich eine Zeit lang in
die Augen.

Dann zuckte Robur ein wenig die Achseln.

{\glqq}An da{\s} Ende de{\s} Taue{\s}!{\grqq} sagte er.

Tom Turner hatte ihn verstanden; Frycollin wurde au{\s} seiner Cabine
geholt.

Aber wie j\"ammerlich schrie er auf, al{\s} der Obersteuermann und
einer von dessen Kameraden ihn ergriffen und in einer Art Korb
festbanden, an dem sie sorgsam da{\s} Ende eine{\s} Taue{\s}
festkn\"upften.

E{\s} war da{\s} eine{\s} jener Taue, welche Onkel Prudent zu dem
un{\s} bekannten Zwecke ben\"utzen wollte.

Der Neger hatte zuerst geglaubt, er solle gehenkt werden ... Nein, er
sollte nur aufgeh\"angt werden.

Da{\s} Tau wurde n\"amlich au{\ss}en in der L\"ange von etwa hundert
Fu{\ss} abgerollt und Frycollin schwebte damit frei in der Luft.

Jetzt stand e{\s} in seinem Belieben, zu schreien, so viel er wollte;
der Schrecken schn\"urte ihm jedoch den Kehlkopf zu -- er blieb
stumm.

Onkel Prudent und Phil Evan{\s} hatten sich dem barbarischen
Verfahren widersetzen wollen -- sie wurden einfach
zur\"uckgesto{\ss}en.

{\glqq}Da{\s} ist abscheulich! ... Da{\s} ist Barbarei! rief Onkel
Prudent, der dar\"uber ganz au{\ss}er sich war.

-- Freilich! antwortete Robur.

-- Da{\s} ist ein Mi{\ss}brauch der Gewalt, gegen den ich noch
ander{\s} al{\s} durch Worte allein Einspruch erheben werde!

-- Immer zu!

-- Ich werde mich r\"achen, Ingenieur Robur!

-- R\"achen Sie sich getrost, Pr\"asident de{\s} Weldon-Institut{\s}.

-- An Ihnen und Ihren Leuten!{\grqq}

Die Mannschaft de{\s} {\glqq}Albatro{\s}{\grqq} hatte sich in nicht
besonder{\s} wohlwollender Haltung gen\"ahert. Robur gab den Leuten
ein Zeichen, sich zu entfernen.

{\glqq}Ja, an Ihnen und Ihren Leuten ... wiederholte Onkel Prudent,
den sein College vergeben{\s} zu beruhigen suchte.

-- Ganz wie e{\s} Ihnen beliebt, erwiderte der Ingenieur.

-- Und ohne R\"ucksicht auf die Mittel!

-- Genug, sagte jetzt Robur in drohendem Tone, genug! E{\s} giebt
noch mehr Taue an Bord! Schweigen Sie ... oder ... der Herr ganz wie
der Diener.{\grqq}

Onkel Prudent schwieg, aber nicht au{\s} Furcht, sondern weil ihn
eine wahre Erstickung beklemmte, so da{\ss} Phil Evan{\s} ihn in
seine Cabine f\"uhren mu{\ss}te.

Seit einer Stunde hatte sich da{\s} Wetter sehr merkbar ver\"andert
und e{\s} traten einzelne Zeichen hervor, welche keine Mi{\ss}deutung
zulie{\ss}en -- ein Unwetter war im Anzug. Die elektrische
S\"attigung der Atmosph\"are hatte einen so hohen Grad erreicht,
da{\ss} Robur gegen zweieinhalb Uhr Zeuge einer bi{\s}her von ihm nie
beobachteten Erscheinung wurde.

Im Norden, von wo da{\s} Unwetter herkam, stiegen dicht geballte,
fast leuchtende D\"unste auf -- wa{\s} jedenfall{\s} von der
verschiedenen und wechselnden elektrischen Spannung der
Wolkenschichten herr\"uhrte.

Der Reflex von diesen Ansammlungen lie{\ss} Myriaden von Lichtern auf
der Oberfl\"ache de{\s} Meere{\s} hintanzen, deren Intensit\"at um so
lebhafter wurde, je mehr der Himmel sich verfinsterte.

Der {\glqq}Albatro{\s}{\grqq} und jene{\s} Meteor mu{\ss}ten bald
zusammentreffen, da sie sich auf einander zu bewegten.

Und Frycollin? -- Nun Frycollin folgte noch immer im Schlepptau --
ja, da{\s} ist da{\s} richtige Wort, denn jene{\s} Tau bildete einen
weit offenen Winkel gegen den mit der Geschwindigkeit von hundert
Kilometern hinfliegenden Apparat, wodurch der Korb nicht unerheblich
zur\"uckblieb.

Da{\s} Entsetzen de{\s} armen Teufel{\s} wird man sich unschwer
au{\s}malen k\"onnen, al{\s} die Blitze jetzt um ihn her aufzuckten
und der Donner mit gewaltiger Macht durch die Himmel{\s}r\"aume
rollte. Da{\s} ganze Personal bem\"uhte sich angesicht{\s} diese{\s}
Unwetter{\s} so zu man\"ovriren, da{\ss} sie entweder h\"oher al{\s}
da{\s}selbe hinaufkamen oder in den unteren Luftschichten bald
jene{\s} im R\"ucken lie{\ss}en.

Der {\glqq}Albatro{\s}{\grqq} befand sich eben ungef\"ahr in
mittlerer H\"ohe -- etwa tausend Meter -- al{\s} ein Donnerschlag von
ungeheurer Heftigkeit \"uber ihn hereinbrach, dem ein furchtbarer
Windsto{\ss} folgte. Binnen wenigen Sekunden st\"urzten sich die
feurigen Wolken auf den Aeronef.

Da raffte sich Phil Evan{\s} zusammen, um zu Gunsten Frycollin{\s}
ein gute{\s} Wort einzulegen und zu erkl\"aren, da{\ss} dieser wieder
an Bord herangezogen w\"urde.

Robur hatte eine solche Vermittlung aber gar nicht erst abgewartet
und schon den n\"othigen Befehl ertheilt. Jetzt waren die Leute
bereit{\s} mit dem Einziehen de{\s} Taue{\s} besch\"aftigt, al{\s}
sich eine pl\"otzliche Verlangsamung der Rotation der
Auftriebschrauben bemerkbar machte.

Robur sprang nach dem mittleren Ruff.

{\glqq}Kraft! ... Volle Kraft! rief er dem Maschinisten zu. Wir
m\"ussen schnell h\"oher, al{\s} da{\s} Unwetter steht, emporsteigen.

-- E{\s} ist unm\"oglich, Herr Ingenieur.

-- Warum?

-- Die Str\"ome sind gest\"ort ... E{\s} treten Unterbrechungen
ein.{\grqq}

In der That senkte sich der {\glqq}Albatro{\s}{\grqq} schon recht
merkbar.

Ganz wie da{\s} bei Gewittern mit den Str\"omen in den
Telegraphendr\"ahten vorkommt, so versagten jetzt auch die
Accumulatoren de{\s} Apparat{\s} den regelm\"a{\ss}igen Dienst;
wa{\s} aber nur eine Unbequemlichkeit ist, wenn e{\s} sich um
Absendung von Depeschen handelt, wurde hier zur furchtbarsten Gefahr,
da{\s} mu{\ss}te damit enden, da{\ss} der Aeronef, ohne da{\ss} man
seiner ferner Herr war, in'{\s} Meer hinabst\"urzte.

{\glqq}Lass' ihn sich senken, rief Robur, damit wir au{\s} der
elektrischen Zone herau{\s}kommen. Vorw\"art{\s}, Jungen, bewahrt
Euer kalte{\s} Blut!{\grqq}

Der Ingenieur hatte seine Commandobr\"ucke bestiegen. Die Mannschaft
war an ihrer Stelle und hielt sich bereit, jeder Anordnung ihre{\s}
Herrn eiligst nachzukommen.

Obwohl der {\glqq}Albatro{\s}{\grqq} sich nur einige hundert Fu{\ss}
gesenkt hatte, schwebte er doch immer noch in der dichten
Wolkenschicht inmitten von Blitzen, die sich wie Raketen eine{\s}
Feuerwerk{\s} kreuzten. Man mu{\ss}te jeden Augenblick f\"urchten,
da{\ss} ihn ein Blitzstrahl treffe. Die Bewegung der Schrauben
verlangsamte sich noch mehr, und wa{\s} bi{\s}her ein etwa{\s}
Schnellere{\s} Herabsinken war, drohte jetzt ein gef\"ahrlicher Sturz
zu werden.

Zuletzt lag e{\s} auf der Hand, da{\ss} er in weniger al{\s} einer
Minute auf der Meere{\s}fl\"ache angelangt sein mu{\ss}te, und einmal
in'{\s} Wasser getaucht, h\"atte keine Macht ihn darau{\s} zu
befreien vermocht.

Pl\"otzlich lagerte sich die elektrische Wolke dicht \"uber ihnen.
Der {\glqq}Albatro{\s}{\grqq} war jetzt nicht mehr al{\s} sechzig
Fu{\ss} vom Kamm der Wellen entfernt. Binnen zwei bi{\s} drei
Secunden drohten diese da{\s} Verdeck zu \"uberfluthen.

Da ben\"utzte Robur noch den letzten Moment, st\"urzte nach dem
mittleren Ruff hin und packte hier die Hebel f\"ur die
Vorw\"art{\s}bewegung, wodurch die von den Batterien kommenden
Str\"ome geschlossen wurden, auf welche die elektrische Spannung der
umgebenden Atmosph\"are keinen Einflu{\ss} \"au{\ss}erte ... in einem
Augenblick hatte er den Schrauben ihre normale Schnelligkeit wieder
gegeben, den Sturz aufgehalten, und der {\glqq}Albatro{\s}{\grqq}
hielt sich in geringer H\"ohe, entfloh jetzt aber mit rasender Eile
dem Unwetter, da{\s} er bald hinter sich zur\"ucklie{\ss}.

E{\s} bedarf wohl nicht besonderer Bemerkung, da{\ss} Frycollin, wenn
auch nur f\"ur wenige Secunden, ein unfreiwillige{\s} Bad genommen
hatte. Al{\s} er an Bord zur\"uckkam, war er durchn\"a{\ss}t, al{\s}
h\"atte er die Tiefe de{\s} Meere{\s} gemessen. Man wird e{\s} kaum
glauben, aber er schrie nicht mehr.

Am n\"achsten Tage, am 4. Juli, hatte der {\glqq}Albatro{\s}{\grqq}
die Nordgrenze de{\s} Kaspi-See{\s} \"uberschritten.



\newpage\begin{center}\label{kap11}
{\large \begin{antiqua}XI.\end{antiqua}\\
In dem die Wuth de{\s} Onkel Prudent mit dem Quadrat der
Geschwindigkeit zunimmt.\\\bigskip}
\end{center}



Wenn Onkel Prudent und Phil Evan{\s} je auf die Hoffnung, entfliehen
zu k\"onnen, verzichten mu{\ss}ten, so war da{\s} w\"ahrend der nun
folgenden f\"unfzig Stunden der Fall. Bef\"urchtete Robur, da{\ss}
die Ueberwachung seiner Gefangenen bei der Fahrt \"uber Europa
weniger leicht sein m\"ochte? Vielleicht. Er wu{\ss}te ja
\"ubrigen{\s}, da{\ss} sie zu Allem entschlossen waren, um zu
entweichen.

Doch, wie dem auch sein mochte, jeder Versuch w\"are jetzt einem
Selbstmorde gleichgekommen. Wenn Einer von einem Expre{\ss}zuge, der
mit der Geschwindigkeit von hundert Kilometern in der Stunde
dahinfliegt, herabspringt, so setzt er vielleicht sein Leben in
Gefahr; wer da{\s} aber von einem zweihundert Kilometer in der Stunde
dahinrasenden Blitzzuge versuchte, der kann nur den Tod wollen.

Eben diese Geschwindigkeit, die gr\"o{\ss}te, die er anzunehmen im
Stande war, war jetzt dem {\glqq}Albatro{\s}{\grqq} ertheilt worden.
Er \"uberholte noch den Flug der Schwalbe, die hundertacht\/zig
Kilometer in der Stunde zur\"ucklegen kann.

Hier mag auch bemerkt sein, da{\ss} bi{\s}her nord\"ostliche Winde in
einer der Fortbewegung de{\s} {\glqq}Albatro{\s}{\grqq} sehr
g\"unstigen Au{\s}dauer anhielten, da dieser in derselben Richtung,
d.~h. im Allgemeinen nach Westen zu flog. Dieser Wind begann aber
allm\"ahlich abzuflauen, so da{\ss} e{\s} nachgerade unm\"oglich
wurde, sich auf dem Verdeck zu halten, ohne die Athmung durch die
Schnelligkeit der Bewegung fast aufgehoben zu sehen. Die beiden
Collegen w\"aren auch beinahe \"uber Bord geschleudert worden, wenn
sie nicht der Luftdruck an ihrem Ruff so zu sagen festgenagelt
h\"atte.

Zum Gl\"uck bemerkte sie der Steuermann durch die Lichtpforten seiner
H\"utte, und eine elektrische Klingel setzte die auf dem Verdeck
eingeschlossenen Mannschaften von ihrer Nothlage in Kenntni{\ss}.

Ueber da{\s} Verdeck hinkriechend, glitten vier Mann davon nach dem
Hintertheil zu.

Diejenigen, welche sich in einem Sturm auf einem vor dem Winde
liegenden Schiffe befunden haben, werden verstehen, welchen Druck der
Wind dabei au{\s}zu\"uben vermag. Hier war e{\s} jedoch der
{\glqq}Albatro{\s}{\grqq} selbst, der diesen durch seine ma{\ss}lose
Geschwindigkeit hervorrief.

Man mu{\ss}te wirklich seinen Gang verlangsamen, wa{\s} Onkel Prudent
und Phil Evan{\s} gestattete, ihre Cabine wieder zu erreichen.

Im Inneren seiner Ruff{\s} f\"uhrte der {\glqq}Albatro{\s}{\grqq},
ganz wie der Ingenieur da{\s} versichert hatte, eine vollkommen
athembare Atmosph\"are mit sich.

Welche erstaunliche Festigkeit besa{\ss} aber dieser Apparat, um
einer so schnellen Fortbewegung den n\"othigen Widerstand leisten zu
k\"onnen. Die Triebschrauben am Bug und am Heck sah man gar nicht
mehr sich drehen; sie pfiffen nur mit scharfem, durchdringendem Ton
durch die Luft.

Die letzte, vom Bord au{\s} gesehene Stadt war Astrachan gewesen,
da{\s} ziemlich am n\"ordlichsten Ende de{\s} Kaspi-See{\s} lag.

Der Stern der W\"uste -- jedenfall{\s} hat ein russischer Dichter
e{\s} so genannt -- ist jetzt von der ersten Gr\"o{\ss}e zur
f\"unften oder sech{\s}ten zur\"uckgegangen. Dieser sehr einfache
Hauptort de{\s} Gouvernement{\s} hatte einen Augenblick seine alten,
mit unn\"utzen Zinnen gekr\"onten Mauern gezeigt, ebenso wie seine
alten Th\"urme in der Mitte der Stadt, seine an Kirchen in modernem
Stil angrenzenden Moscheen, seine Kathedrale mit f\"unf vergoldeten
und mit blauen Sternen \"ubers\"aeten Kuppeln, die einem
au{\s}geschnittenen St\"uck Firmament glichen -- da{\s} Ganze fast im
Niveau der hier zwei Kilometer breiten Wolgam\"undung.

Von diesem Punkt au{\s} war der Flug de{\s} {\glqq}Albatro{\s}{\grqq}
schon mehr eine Art Ritt durch die H\"ohen de{\s} Himmel{\s}, al{\s}
w\"urde er von fabelhaften Hippogryphen fortgetragen, welche eine
Meile mit jedem Fl\"ugelschlage zur\"ucklegen.

E{\s} war gegen zehn Uhr Morgen{\s} am 4. Juli, al{\s} der Aeronef,
etwa dem Thale der Wolga folgend, nach Nordwesten weiter steuerte. An
beiden Strome{\s}ufern hin dehnten sich die Steppen de{\s} Don und
de{\s} Ural. W\"are e{\s} m\"oglich gewesen, einen Blick auf diese
ungeheuren Gebiete zu werfen, so h\"atte man die St\"adte und
D\"orfer darin kaum z\"ahlen k\"onnen. Am Abend endlich zog der
Aeronef \"uber Mo{\s}kau weg, ohne die auf dem Kreml flatternde
Flagge zu salutiren. Binnen zehn Stunden hatte er die zweitausend
Kilometer, welche Astrachan von der Hauptstadt aller Russen trennen,
zur\"uckgelegt.

Von Mo{\s}kau nach Peter{\s}burg ist die Eisenbahnlinie nicht
l\"anger al{\s} zw\"olfhundert Kilometer, konnte also mehr Zeit
al{\s} einen halben Tag nicht beanspruchen. So erreichte denn auch
der {\glqq}Albatro{\s}{\grqq} mit der P\"unktlichkeit eine{\s}
Expre{\ss}zuge{\s} Peter{\s}burg und die Ufer der Newa gegen zwei Uhr
Morgen{\s}. Die Helligkeit der Nacht, in der in so hoher Breite die
Sonne nicht tief unter den Horizont nieder taucht, gestattete einen
Augenblick, da{\s} Gesammtbild dieser gro{\ss}en Stadt zu
\"uberschauen.

Nachher folgte der finnische Meerbusen, da{\s} Inselgewirr von Abo,
die Ostsee, Schweden in der Breite von Stockholm, Norwegen in der von
Christiania -- zweitausend Kilometer in nur zehn Stunden! Wahrlich,
man h\"atte glauben k\"onnen, da{\ss} keine menschliche Macht
fernerhin im Stande w\"are, die Geschwindigkeit de{\s}
{\glqq}Albatro{\s}{\grqq} zu hemmen, al{\s} ob die Resultante seiner
Treibkraft und der Anziehung der Erde ihn in unver\"anderlichem
Krei{\s}laufe um die Erde gefesselt hielte.

Danach unterbrach er seinen Lauf, und zwar genau \"uber dem
ber\"uhmten Wasserfall de{\s} Rjukanfo{\s} in Norwegen. Der Gusta,
dessen Gipfel diesen herrlichen Theil von Telemarken beherrscht,
erschien gleich einem riesenhaften Grenzwall, den er nach Westen
nicht \"uberschreiten durfte.

Von hier au{\s} n\"aherte sich der {\glqq}Albatro{\s}{\grqq} auch,
ohne Verminderung seiner Geschwindigkeit, wieder mehr dem Erdboden.

Und wa{\s} begann wohl Frycollin w\"ahrend dieser Fahrt ohne
Gleichen?

Frycollin blieb stumm in seiner Cabine und schlief, mit Au{\s}nahme
der Zeit, wo gegessen wurde, so gut er konnte.

Fran\c{c}oi{\s} Tapage leistete ihm dann Gesellschaft und erg\"otzte
sich weidlich an seiner ewigen Angst.

{\glqq}He, he, mein Junge, sagte er, Du heulst ja gar nicht mehr?
Brauchst Dich gar nicht zu geniren! ... Mit zwei Stunden aufgeh\"angt
sein ist Alle{\s} quitt gemacht! ... He, bei der Schnelligkeit, mit
der wir jetzt fahren, m\"u{\ss}te da{\s} ein vortreffliche{\s}
Luftbad gegen den Rheumati{\s}mu{\s} abgeben!

-- Mir kommt e{\s} vor, al{\s} ob Alle{\s} in kurze und kleine
St\"ucke ginge, antwortete Frycollin.

-- Da{\s} ist wohl m\"oglich, mein wackerer Frycollin; aber wir
fliegen so schnell dahin, da{\ss} wir gar nicht mehr fallen
k\"onnten. Da{\s} ist doch auch eine Beruhigung.

-- Glauben Sie?

-- Bei meiner Ga{\s}cogner-Ehre!{\grqq}

Um die Wahrheit zu sagen und nicht zu \"ubertreiben, wie
Fran\c{c}oi{\s} Tapage, so lag die Sache so, da{\ss} die Arbeit der
Auftrieb{\s}schrauben infolge jener ungeheuren Geschwindigkeit jetzt
ein wenig vermindert war, der Aeronef glitt auf den Luftschichten
etwa hin, wie eine Congr\`eve'sche Rakete.

{\glqq}Und da{\s} wird noch lange so fortdauern? sagte Frycollin.

-- Lange? ... O nein! antwortete der Koch, nur da{\s} ganze Leben
lang.

-- Ach! seufzte der Neger, wieder mit seinen Klagen beginnend.

-- Nimm Dich in Acht, Frycollin, nimm Dich in Acht! rief da
Fran\c{c}oi{\s} Tapage, denn, wie man bei mir zu Hause sagt, der Herr
k\"onnte Dich auf die Schaukel hinau{\s}setzen.{\grqq}

Und mit den Bissen, die er gleich doppelt in den Mund steckte,
w\"urgte Frycollin auch seine Seufzer hinunter.

W\"ahrend dessen entwarfen Onkel Prudent und Phil Evan{\s}, welche
nicht dazu angethan waren, sich unn\"utz zu beklagen, einen wohl
durchdachten Plan. An Au{\s}f\"uhrung eine{\s} Fluchtversuche{\s} war
ja unm\"oglich zu denken. Doch wenn sie den Fu{\ss} auch nicht auf
die Erde setzen konnten, war e{\s} nicht denkbar, den Erdenbewohnern
mit\/zutheilen, wa{\s} nach ihrem Verschwinden au{\s} dem Vorsitzenden
und dem Schriftf\"uhrer de{\s} Weldon-Institut{\s} geworden war, wer
sie geraubt hatte, auf welcher fliegenden Maschine sie sich befanden,
um vielleicht -- aber, lieber Gott, auf welche Weise? -- einen
k\"uhnen Versuch ihrer Freunde, sie den H\"anden Robur'{\s} zu
entrei{\ss}en, herbeif\"uhren zu k\"onnen?

Doch wie sollten sie von sich Nachricht geben? H\"atte e{\s} dazu
hingereicht, die Methode der Seeleute nachzuahmen, welche ein
Schriftst\"uck mit Bezeichnung der Stelle de{\s} Schiffbruch{\s} in
eine Flasche stecken und diese in'{\s} Meer werfen?

Hier vertrat die Stelle de{\s} Meere{\s} aber die Atmosph\"are. Die
Flasche konnte darauf nat\"urlich nicht schwimmen. Fiel dieselbe
nicht gerade auf einen zuf\"allig Vor\"ubergehenden, dem sie recht
gut den Sch\"adel zerschmettern konnte, so lag die Vermuthung nahe,
da{\ss} sie niemal{\s} aufgefunden wurde.

Die beiden Collegen hatten leider kein andere{\s} Mittel zur
Verf\"ugung und sie standen schon im Begriff, eine Flasche de{\s}
Luftfahrzeuge{\s} zu opfern, al{\s} dem Onkel Prudent noch ein
anderer Gedanke kam. Er schnupfte, wie wir wissen, und diese kleine
Untugend darf man einem Amerikaner, der weit schlimmere Unsitten
h\"atte an sich haben k\"onnen, wohl nachsehen. Al{\s} Schnupfer
besa{\ss} er nat\"urlich auch eine Dose, die jetzt schon l\"angst
leer war. Diese Dose war au{\s} Aluminium gearbeitet. Warf er
dieselbe hinau{\s}, so durfte man hoffen, da{\ss} jeder ehrbare
B\"urger, der sie fand, sie auch aufheben werde. Hob er sie auf, so
lieferte er sie auch bei der Polizei ab, und hier w\"urde man
Kenntni{\ss} nehmen von dem Document, welche{\s} dazu dienen sollte,
die Lage der beiden Opfer Robur de{\s} Sieger{\s} kund zu geben.

Da{\s} wurde denn auch au{\s}gef\"uhrt. Da{\s} kurze
einzuschlie{\ss}ende Schriftst\"uck sagte Alle{\s} und trug daneben
die Adresse de{\s} Weldon-Institut{\s} mit der Bitte, da{\s}selbe
dahin zu bef\"ordern.

Nachdem Onkel Prudent da{\s} Papier eingelegt, umwickelte er die Dose
sorgsam mit einem dicken wollenen Band, um zu verh\"uten, da{\ss}
dieselbe sich w\"ahrend de{\s} Falle{\s} schon \"offne und durch
da{\s} Aufschlagen nicht in St\"ucke gehe. Jetzt galt e{\s} nur noch
eine g\"unstige Gelegenheit abzuwarten.

Da{\s} Schwerste bei der ganzen Sache war e{\s} aber w\"ahrend dieser
merkw\"urdigen Fahrt \"uber Europa, da{\s} Ruff zu verlassen, \"uber
da{\s} Verdeck zu kriechen, auf die Gefahr hin, fortgerissen zu
werden, und da{\s} ganz heimlich durchzuf\"uhren. Andererseit{\s} kam
e{\s} darauf an, da{\ss} die Dose nicht in ein Meer, einen Golf, See
oder einen anderen Wasserlauf fiel, denn damit -- w\"are sie ja
verloren gewesen.

Jedenfall{\s} schien e{\s} aber nicht unm\"oglich, da{\ss} die beiden
Collegen sich durch diese{\s} Mittel mit der bewohnten Welt in'{\s}
Einvernehmen setzen konnten.

Eben jetzt wurde e{\s} jedoch Tag und e{\s} schien rathsamer, die
Nacht abzuwarten und entweder eine Verminderung der Geschwindigkeit
oder einen Halt zu ben\"utzen, um da{\s} Ruff zu verlassen.
Vielleicht konnten sie dann die Reeling erreichen und die kostbare
Dose genau \"uber einer Stadt herunterfallen lassen.

Doch selbst bei dem Zusammentreffen aller g\"unstigen Umst\"ande
h\"atte da{\s} Vorhaben nicht gleich zur Au{\s}f\"uhrung gebracht
werden k\"onnen -- wenigsten{\s} nicht am heutigen Tage.

Nachdem der Aeronef n\"amlich Norwegen in der H\"ohe de{\s} Gusta
verlassen, hatte er sich nach dem S\"uden zu gewendet und folgte
jetzt genau dem franz\"osischen Meridian Null, der bekanntlich \"uber
Pari{\s} verl\"auft. Er schwebte also \"uber die Nordsee hinweg,
nicht ohne an Bord der Tausende von K\"ustenfahrern, welche zwischen
dem Festlande und England verkehren, da{\s} gr\"o{\ss}te Aufsehen zu
erregen. Fiel die Dose hier nicht gerade auf da{\s} Deck eine{\s}
solchen Schiffe{\s}, so hatte sie die gegr\"undete Au{\s}sicht, auf
Nimmerwiedersehen in der Tiefe zu versinken.

Onkel Prudent und Phil Evan{\s} sahen sich also gen\"othigt, einen
g\"unstigeren Augenblick abzuwarten. Da sollte sich ihnen, wie wir
sehen werden, bald eine besonder{\s} geeignete Gelegenheit darbieten.

Gegen zehn Uhr Abend{\s} erreichte der {\glqq}Albatro{\s}{\grqq} die
K\"uste Frankreich{\s}, nahezu in der H\"ohe von Dunkerque. Die Nacht
war ziemlich dunkel. Einen Augenblick konnte man den Leuchtthurm von
Gri{\s}-Nez seine elektrischen Lichtstrahlen mit denen von Dover
kreuzen sehen, die also den Canal in seiner ganzen Breite erhellten.
Dann steuerte der {\glqq}Albatro{\s}{\grqq} \"uber Frankreich hin und
hielt sich dabei in einer mittleren H\"ohe von etwa tausend Metern.

Seine Geschwindigkeit hatte sich freilich nicht ge\"andert. Wie eine
Bombe flog er \"uber St\"adte, Schl\"osser und D\"orfer hinweg, die
so zahlreich in den fruchtbaren Provinzen de{\s} n\"ordlichen
Frankreich{\s} zerstreut liegen. E{\s} waren da{\s} unter dem
Meridiane von Pari{\s} nach Dunkerque, Doullon{\s}, Amien{\s}, Creil,
St.~Deni{\s} ... Immer hielt jener dabei eine gerade Linie ein. So
gelangte er gegen Mitternacht \"uber die {\glqq}Stadt de{\s}
Licht{\s}{\grqq}, welche diesen Namen wenigsten{\s} verdient, wenn
ihre Einwohner schlafen oder doch schlafen sollten.

Welch' sonderbare Laune veranla{\ss}te nun den Ingenieur gerade
\"uber der Stadt Pari{\s} einmal anzuhalten? Niemand wei{\ss} e{\s}.
Jedenfall{\s} aber senkte sich hier der {\glqq}Albatro{\s}{\grqq} so
weit, da{\ss} er nur wenige hundert Fu{\ss} \"uber derselben
schwebte. Robur trat au{\s} seiner Cabine hervor und auch die ganze
Mannschaft erschien, um lustwandelnd einmal frische Luft zu
sch\"opfen, auf dem Verdeck.

Onkel Prudent und Phil Evan{\s} achteten wohl darauf, die sich jetzt
bietende au{\s}gezeichnete Gelegenheit nicht vor\"uber gehen zu
lassen. Beide suchten sich, sobald sie au{\s} ihrem Ruff getreten
waren, von den Anderen entfernt zu halten, um den rechten Augenblick
zur Au{\s}f\"uhrung ihre{\s} Vorhaben{\s} au{\s}w\"ahlen zu k\"onnen.
Auf keinen Fall sollten die Anderen etwa{\s} davon merken.

Einem gigantischen K\"afer \"ahnlich zog der
{\glqq}Albatro{\s}{\grqq} so langsam \"uber die Stadt hin. Er
\"uberschritt die Linie der Boulevard{\s}, welche durch Edison'sche
Lampen in hellem Tage{\s}lichte lagen. Da{\s} Ger\"ausch der Wagen,
welche noch durch die Stra{\ss}en jagten, und da{\s} Rollen der
Z\"uge auf den vielen, in Pari{\s} zusammentreffenden Bahnlinien
drang bi{\s} zu ihm hinauf.

Dann glitt er in der H\"ohe der h\"ochsten Bauwerke hin, al{\s}
h\"atte er die Kugel vom Pantheon oder da{\s} Kreuz vom Invalidendom
abstreifen wollen. Er steuerte zwischen den beiden Minaret{\s} de{\s}
Trocadero hindurch nach dem eisernen Thurm de{\s} Mar{\s}felde{\s},
dessen ungeheurer Reflector die ganze Hauptstadt mit elektrischem
Lichte \"uberstr\"omte.

Diese Luftpromenade, diese{\s} Flaniren in der Nacht, w\"ahrte etwa
eine Stunde. E{\s} glich einer Station in den L\"uften vor
Fortsetzung der endlosen Reise.

Der Ingenieur Robur wollte dabei den Parisern offenbar den Anblick
eine{\s} Meteor{\s} bereiten, da{\s} ihre Astronomen noch niemal{\s}
gesehen oder nur geahnt hatten. Die Signallichter de{\s}
{\glqq}Albatro{\s}{\grqq} wurden in Function gesetzt. Zwei
gl\"anzende Strahlenb\"undel ergossen sich \"uber die Pl\"atze, die
H\"auservierecke, G\"arten und die sechzigtausend H\"auser der Stadt,
indem sie ungeheure Lichtmassen von einem Horizont zum anderen
schweifen lie{\ss}en.

Gewi{\ss} -- die{\s}mal war der {\glqq}Albatro{\s}{\grqq} gesehen
worden, und nicht allein gesehen, sondern auch geh\"ort worden, denn
Tom Turner hatte die Trompete hervorgeholt und eine schmetternde
Fanfare \"uber die Stadt hin ert\"onen lassen. In diesem Augenblick
beugte sich Onkel Prudent ein wenig \"uber die Reeling, \"offnete die
Hand und lie{\ss} die Dose fallen.

Fast gleichzeitig erhob sich der {\glqq}Albatro{\s}{\grqq} wieder
sehr schnell.

Da schallte durch die H\"ohe de{\s} Pariser Himmel{\s} ein
vieltausendf\"altige{\s} Hurrah au{\s} der Menge, da{\s} sich \"uber
die Boulevard{\s} fortpflanzte -- ein Hurrah der Verwunderung \"uber
da{\s} unvorhergesehene, phantastische Meteor.

Pl\"otzlich erloschen die Lichtquellen de{\s} Aeronef{\s} und rund um
ihn wurde e{\s} wieder dunkel und still; darauf nahm er seine Fahrt
mit der Geschwindigkeit von zweihundert Kilometern in der Stunde
wieder auf.

Da{\s} war Alle{\s} gewesen, wa{\s} seine Insassen von der Hauptstadt
Frankreich{\s} hatten sehen sollen.

Um vier Uhr Morgen{\s} hatte der {\glqq}Albatro{\s}{\grqq} da{\s}
ganze Land schon \"uberflogen. Um keine Zeit mit der Ueberschreitung
der Pyren\"aen oder der Alpen zu verlieren, glitt er jetzt an der
Oberfl\"ache der Provence bi{\s} zur Spitze de{\s} Cap d'Antibe{\s} hin.

Um neun Uhr blieben die auf der Terrasse de{\s} St.~Peter{\s}-Dome{\s}
versammelten R\"omer verbl\"ufft stehen, al{\s} sie ihn \"uber die
Ewige Stadt hinwegschweben sahen. Zwei Stunden sp\"ater schwankte er
hoch \"uber dem Golf von Neapel, einen Augenblick in der Rauchs\"aule
de{\s} Vesuv{\s}. Nachdem er dann da{\s} Mittelmeer in schr\"ager
Richtung \"uberschritten, wurde er in der ersten Nachmittag{\s}stunde
von den Wachtposten in La Golette an der tunesischen K\"uste
beobachtet.

Von Amerika \"uber Asien! Von Asien \"uber Europa! Mehr al{\s}
drei{\ss}igtausend Kilometer hatte der wunderbare Apparat in weniger
al{\s} dreiundzwanzig Tagen zur\"uckgelegt.

Und jetzt zog er majest\"atisch \"uber die bekannten und unbekannten
Landmassen Afrika{\s} dahin!

\begin{center}
\makebox[15em]{\hrulefill}\bigskip
\end{center}

Vielleicht w\"unscht der Leser zu erfahren, wa{\s} nach dem
Herabfallen au{\s} der ber\"uhmten Schnupf\-tabak{\s}\-dose geworden
war?

Die Dose war in der Rivoli-Stra{\ss}e vor dem Hause Nummer 210 zu
einer Zeit niedergefallen, wo diese Stra{\ss}e gerade ziemlich leer
war. Am folgenden Morgen wurde sie von einer ehrlichen
Stra{\ss}enkehrerin aufgefunden, welche sich beeilte, dieselbe auf
der Polizei-Pr\"afectur abzuliefern.

Hier hielt man sie zuerst f\"ur einen explodirenden K\"orper und
wickelte sie mit derselben allergr\"o{\ss}ten Vorsicht auf, wie sie
zuletzt ge\"offnet wurde.

Da trat wirklich eine Art Explosion ein ... Ein furchtbare{\s}
Niesen, dessen sich der Sicherheit{\s}chef nicht zu erwehren
vermochte.

Dann zog man da{\s} Schriftst\"uck au{\s} der Dose und la{\s} zum
allgemeinen Erstaunen wie folgt:
\bigskip

{\glqq}Onkel Prudent und Phil Evan{\s}, Vorsitzender und
Schriftf\"uhrer de{\s} Weldon-Institut{\s} zu Philadelphia,
entf\"uhrt durch den Aeronef de{\s} Ingenieur Robur.

Freunden und Bekannten davon Nachricht zu geben.

\hfill Onkel Prudent und Phil Evan{\s}.{\grqq}
\bigskip

Hiermit war die unerkl\"arliche Erscheinung den Bewohnern beider
Welten endlich erkl\"art, und die vielen Gelehrten an den Observatorien,
welche e{\s} auf der Erde giebt, gewannen die l\"angst verlorene Ruhe
endlich wieder.



\newpage\begin{center}\label{kap12}
{\large \begin{antiqua}XII.\end{antiqua}\\
In dem der Ingenieur Robur handelt, al{\s} ob er sich um einen der
Monthyon-Preise bewerben wollte.\\\bigskip}
\end{center}



Bei dieser Erdumkreisung de{\s} {\glqq}Albatro{\s}{\grqq} dr\"angen
sich wohl von selbst ganz verschiedene Fragen auf; zum Beispiel:

Wer ist \"uberhaupt dieser Robur, von dem bi{\s}her nicht{\s} al{\s}
der Name bekannt ist? Verbringt er sein Leben ganz in der Luft? Ruht
sein Aeronef niemal{\s} au{\s}? Hat er nicht vielleicht eine Zuflucht
an unzug\"anglichem Orte, an dem er selbst wenn er der Ruhe nicht
bed\"urfte, sich wenigsten{\s} mit neuen Vorr\"athen versorgt? E{\s}
w\"are doch merkw\"urdig, wenn da{\s} nicht so sein sollte. Auch die
m\"achtigsten Segler der L\"ufte haben ja irgendwo einen Horst oder
ein Nest.

Und weiter: Wa{\s} gedenkt der Ingenieur mit den beiden, ihn doch nur
bel\"astigenden Gefangenen zu beginnen? Beabsichtigt er sie in seiner
Gewalt zu behalten und f\"ur ewig zu verdammen, mit ihm umherzufliegen?
Oder wird er ihnen, nachdem er sie \"uber Afrika, S\"udamerika,
Austral-Asien, den Indischen, den Atlantischen und den Stillen Ocean
hinweggef\"uhrt, um sie wider Willen zu seinen Anschauungen zu
\"uberzeugen, die Freiheit wieder schenken, etwa mit den Worten:

{\glqq}Jetzt, meine Herren, hoffe ich, werden Sie sich bez\"uglich
de{\s} Grundsatze{\s}: {\glqq}Schwerer, al{\s} die Luft{\grqq}, nicht
mehr so ungl\"aubig, zeigen!~--~?{\grqq}

Auf diese Fragen l\"a{\ss}t sich vorl\"aufig noch keine Antwort
geben; die{\s} ist ein Geheimni{\ss} der Zukunft; vielleicht wird
da{\s}selbe eine{\s} Tage{\s} entschleiert werden.

Auf keinen Fall schickte der Vogel Robur sich aber an, jene{\s}
angedeutete Nest an der Nordk\"uste Afrika{\s} aufzusuchen. Im Laufe
de{\s} Tage{\s} strich er noch, je nach Laune, bald dahinrasend, bald
langsamer schwebend, vom Cap Bon bi{\s} zum Cap Carthago \"uber die
Regentschaft Tuni{\s} hin. Darauf wandte er sich mehr dem
Lande{\s}inneren zu und schlug den Weg durch da{\s} wundervolle Thal
der Medjerda ein, indem er dem gelblichen, unter Cactu{\s} und
Rosenb\"uschen verborgenen Wasserlauf derselben folgte. Zu vielen
Hunderten flogen V\"ogel auf, die in langen Reihen auf den
Telegraphendr\"ahten sa{\ss}en, al{\s} wollten sie die Depeschen beim
Durchgang abfangen und auf ihren Fl\"ugeln weiter tragen.

Mit Einbruch der Nacht schwebte der {\glqq}Albatro{\s}{\grqq} \"uber
den Grenzen von Krumirien, und wenn noch ein Krumir wach war, so
unterlie{\ss} er e{\s} gewi{\ss} nicht, da{\s} Gesicht auf die Erde
niederzuwerfen und Allah bei der Erscheinung diese{\s} riesenhaften
Adler{\s} um Schutz und Hilfe anzuflehen.

Am folgenden Morgen waren Bona und die sch\"onen H\"ugel seiner
Umgebung in Sicht, sp\"ater Philippeville, jetzt ein kleine{\s}
Algier, mit seinen bogenf\"ormigen Quai{\s}, seinen herrlichen
Weing\"arten, deren gr\"unende Reben der ganzen Landschaft ihren
Charakter verleihen, einer Landschaft, welche au{\s} Bordelai{\s} und
den gesegneten Gebieten von Burgund herau{\s}geschnitten zu sein
scheint.

Diese Spazierfahrt von f\"unfhundert Kilometern \"uber Gro{\ss}- und
Klein-Kabylien hinweg endigte gegen Mittag in der H\"ohe der
Ka{\s}bah von Algier. Welch' sch\"one{\s} Bild bot sich da den
Passagieren de{\s} Aeronef{\s}! Die offene Rhede zwischen Cap Matifu
und der Pe{\s}cade-Spitze, da{\s} mit Pal\"asten, Maravut{\s} und
Landh\"ausern bes\"aete Uferland; die launenhaft gewundenen Th\"aler
mit ihrem Mantel von Weinstocken; da{\s} tiefblaue Mittelmeer, da{\s}
die hier kleinen Booten gleichenden tran{\s}atlantischen Dampfer
durchfurchen. So ging e{\s} weiter bi{\s} zu dem malerischen Oran,
dessen in den Gartenanlagen der Citadelle versammelte Bewohner den
{\glqq}Albatro{\s}{\grqq} mit den ersten aufleuchtenden Sternen
verschmelzen sahen.

Wenn Onkel Prudent und Phil Evan{\s} sich fragten, welcher Laune der
Ingenieur Robur nachgebe, al{\s} er ihr fliegende{\s} Gef\"angni{\ss}
\"uber Algerien -- die Fortsetzung Frankreich{\s} an der S\"udk\"uste
de{\s} Mittelmeere{\s} -- hinf\"uhrte, so mu{\ss}ten sie die
Ueberzeugung gewinnen, da{\ss} diese Laune zwei Stunden nach
Sonnenuntergang befriedigt sei. Eine Wendung de{\s} Steuerruder{\s}
lenkte den {\glqq}Albatro{\s}{\grqq} nach S\"udosten ab, und dieser
sah am folgenden Tage, nachdem er die bergige Gegend de{\s} Tell
\"uberstiegen, die Sonne \"uber dem W\"ustensande der Sahara
aufgehen.

Am 8. Juli wurde nun folgende Reiseroute zur\"uckgelegt: Zuerst
erblickte man den kleinen Flecken G\'eryville, der, wie Laghuat, an
der Grenze der W\"uste gegr\"undet wurde, um die endliche Eroberung
von Kabylien zu erleichtern; nachher passirte man den Kamm von
Stillen, und zwar bei dem herrschenden heftigen Gegenwinde nicht ohne
Schwierigkeit. Weiter ging e{\s} \"uber die W\"uste hin, bald langsam
oberhalb der gr\"unenden Oasen oder Ksar{\s}, bald mit wilder
Schnelligkeit, welche den Flug der L\"ammergeier \"uberholte.
Manchmal mu{\ss}te sogar auf diese gewaltigen Raubv\"ogel Feuer
gegeben werden, die sich, zu zw\"olf und f\"unfzehn vereinigt, selbst
nicht scheuten, zum gr\"o{\ss}ten Schrecken Frycollin{\s} sich auf
den Aeronef zu st\"urzen.

Wenn diese L\"ammergeier nur durch furchtbare{\s} Geschrei, durch
Schnabelhiebe und Krallenschl\"age zu antworten vermochten, so
verschonten die nicht minder wilden Eingeborenen ihn nicht mit
Flintensch\"ussen, vorz\"uglich al{\s} er \"uber die Berge von Sel
gekommen war, deren gr\"une und violette Grundmasse da und dort durch
den wei{\ss}en Mantel blickte. Jetzt schwebte da{\s} Luftschiff schon
\"uber der gro{\ss}en Sahara, wo an verschiedenen Stellen noch Reste
der Lagerst\"atten Abd-el-Kader'{\s} zu bemerken waren. Hier -- und
vorz\"uglich unter den Verb\"undeten Beni-Myal -- bietet da{\s} Land
f\"ur den europ\"aischen Reisenden noch immer ernste Gefahren.

Jetzt mu{\ss}te der {\glqq}Albatro{\s}{\grqq} wieder in h\"ohere
Zonen fl\"uchten, um einem daherrasenden Samum zu entgehen, der eine
gewaltige Welle r\"othlichen Sande{\s} auf der Erde vor sich
hintrieb, wie die steigende Fluth die Brandung{\s}welle im Ocean.
Weiterhin entluden die \"oden Hochplateau{\s} der Chebka ihre
schw\"arzlichen Lavamassen bi{\s} herunter zu dem frischen, gr\"unen
Thale de{\s} Ain-Massin. Schwerlich verm\"ochte sich Jemand
gr\"o{\ss}ere Mannigfaltigkeit der Landschaften vorzustellen, welche
der Blick hier in weitem Umfange umfa{\ss}te. Auf baum- und
buschbedeckte H\"ugel folgten da lange graue Bodenwellen, gleich den
Falten eine{\s} arabischen Burnu{\s}. In der Ferne erschienen
{\glqq}Oued{\s}{\grqq} mit brausenden Bergstr\"omen, W\"alder von
Palmen, kleine Ansammlungen von H\"utten, welche entweder einen
H\"ugel kr\"onten oder eine Moschee umrahmten, unter anderen Metliti,
wo ein religi\"oser H\"auptling, der gro{\ss}e Marabut Sidi Scheik,
seinen Sitz hat.

W\"ahrend der Nacht wurden mehrere hundert Kilometer \"uber ein
ziemlich ebene{\s}, nur von D\"unen unterbrochene{\s} Gebiet
zur\"uckgelegt. H\"atte der {\glqq}Albatro{\s}{\grqq} hier Halt
machen wollen, so w\"urde er in der Niederung der, unter einem
ungeheuren Palmenwald versteckten Oase Uargla die Erde erreicht
haben. Sehr deutlich zeigte sich die Stadt mit ihren drei bestimmt
unterschiedenen Quartieren, mit dem alten Palast de{\s} Sultan{\s},
einer Art befestigter Ka{\s}bah, ihren H\"ausern au{\s} Backsteinen,
welche erst die gl\"uhende Sonne hart brennt, und mit ihren im Thale
erbohrten artesischen Brunnen, an denen der Aeronef seinen
Wasservorrath h\"atte erneuern k\"onnen. Dank seiner
au{\ss}erordentlichen Schnelligkeit aber f\"ullte da{\s} im Thale von
Kaschmir au{\s} dem Hydaspi{\s} gesch\"opfte Wasser noch immer die
Vorrath{\s}tonnen, selbst in den W\"usten von Afrika, an.

Der {\glqq}Albatro{\s}{\grqq} wurde von den Arabern, den Mozabiten
und den Negern, welche sich in die Oasen von Uargla theilen,
unzweifelhaft bemerkt, denn e{\s} begr\"u{\ss}ten ihn von hier au{\s}
Hunderte von Gewehrsch\"ussen, ohne da{\ss} die Kugeln ihn h\"atten
erreichen k\"onnen.

Dann kam die Nacht, die grabe{\s}stille W\"ustennacht, deren
Geheimnisse Felicien David so hochpoetisch geschildert hat.

W\"ahrend der folgenden Stunden kehrte man wieder nach S\"udwesten
zur\"uck und kreuzte die Stra{\ss}en von El Golea, deren eine im
Jahre 1859 durch den unerschrockenen Duveyrier entdeckt worden war.

Ring{\s} herrschte tiefe Finsterni{\ss}. Nicht{\s} war zu sehen von
der nach den Pl\"anen Duponchel'{\s} zu erbauenden Sahara-Bahn, dem
langen Eisenbande, da{\s} Algier mit Timbuctu \"uber Leghuat und
Gardaia verkn\"upfen und sp\"ater bi{\s} zum Golf von Guinea
fortgesetzt werden soll.

Der {\glqq}Albatro{\s}{\grqq} gelangte nun in die \"aquatorialen
Gebiete jenseit{\s} de{\s} Wendekreise{\s} de{\s} Krebse{\s}. Tausend
Kilometer von der Nordgrenze der Sahara \"uberschritt er die
Stra{\ss}e, wo der Major Loiny 1846 den Tod fand; er kreuzte den Weg
der Caravane von Marokko nach dem Sudan, und \"uber dem Theile der
W\"uste, in dem die Tuary{\s} hausen, h\"orte er, wa{\s} man den
{\glqq}Gesang de{\s} Sande{\s}{\grqq} zu nennen pflegt, ein
sanfte{\s}, klagende{\s} Murmeln, da{\s} dem Erdboden zu entsteigen
scheint.

Nur ein einziger Zwischenfall ereignete sich hier; eine Wolke von
Heuschrecken zog in gro{\ss}er H\"ohe daher und au{\s} derselben fiel
nun eine so gro{\ss}e Menge an Bord, da{\ss} da{\s} Luftschiff davon
unterzugehen drohte. Die Mannschaft beeilte sich jedoch, die
unerw\"unschte Last wieder abzuwerfen, bi{\s} auf mehrere hundert
St\"uck, welche Fran\c{c}oi{\s} Tapage f\"ur sich in Anspruch nahm.
Er richtete dieselben auf so au{\s}gezeichnet schmackhafte Weise zu,
da{\ss} Frycollin dar\"uber sogar einmal seine eigene Angst
verga{\ss}.

{\glqq}Da{\s} schmeckt so gut, wie die besten Krabben!{\grqq} sagte er.

Man befand sich jetzt tausendachthundert Kilometer von der Oase
Uargla entfernt, fast auf der Nordgrenze de{\s} ungeheuren
K\"onigreich{\s} Sudan.

Gegen zwei Uhr Nachmittag{\s} wurde auch am Knie eine{\s} gro{\ss}en
Strome{\s} eine Stadt sichtbar. Dieser Strom war der Niger -- die
Stadt war Timbuctu.

Wenn diese{\s} afrikanische Mekka bi{\s}her nur von k\"uhnen
Reisenden der Alten Welt, von einem Batouta, Khazan, Imbert,
Mungo-Park, Adam{\s}, Loiny, Laill\'e, Barth, Lenz und Anderen
besucht worden war, so konnten von heute ab, und zwar Dank den
Zuf\"alligkeiten eine{\s} Abenteuer{\s} ohne Gleichen, auch zwei
Amerikaner \begin{antiqua}de visu, de auditu\end{antiqua} und
obendrein \begin{antiqua}de olfactu\end{antiqua} davon bei ihrer
Heimkehr nach Amerika reden -- wenn sie \"uberhaupt einmal dahin
zur\"uckgelangten.

\begin{antiqua}De visu\end{antiqua}, weil sie alle Ecken de{\s}
f\"unf bi{\s} sech{\s} Kilometer gro{\ss}en Dreieck{\s}, da{\s} die
Stadt bildet, \"ubersehen konnten; -- \begin{antiqua}de
auditu\end{antiqua}, weil an diesem Tage gro{\ss}er Markt abgehalten
wurde, bei dem e{\s} ohne einen Heidenl\"armen nicht abgeht; --
\begin{antiqua}de olfactu\end{antiqua}, weil der Geruch{\s}nerv sehr
unangenehm erregt werden mu{\ss}te durch die D\"unste de{\s}
Yubu-Kamo-Platze{\s}, auf dem sich dicht neben dem alten Palaste der
K\"onige die Fleischverkauf{\s}halle erhebt.

Jedenfall{\s} glaubte der Ingenieur den Vorsitzenden und den
Schriftf\"uhrer de{\s} Weldon-Institut{\s} darauf aufmerksam machen
zu sollen, da{\ss} e{\s} die h\"ochste Zeit sei, sich die K\"onigin
de{\s} Sudan{\s} zu betrachten, die sich jetzt in den H\"anden der
Touareg{\s} von Laganet befindet.

{\glqq}Timbuctu, meine Herren!{\grqq} sagte er zu ihnen in demselben
Tone, in dem er zw\"olf Tage fr\"uher zu ihnen {\glqq}Indien, meine
Herren!{\grqq} gesagt hatte.

Dann fuhr er fort:

{\glqq}Timbuctu, unter 18 Grad n\"ordlicher Breite und 5 Grad 56
Minuten westlicher L\"ange von Pari{\s}, zweihundertf\"unfundvierzig
Meter \"uber dem mittleren Niveau de{\s} Meere{\s} gelegen. Eine
bedeutende Stadt von zw\"olf- bi{\s} dreizehntausend Einwohnern, die
sich ehedem durch Kunst und Wissenschaft au{\s}zeichnete. --
Vielleicht hatten Sie den Wunsch, hier einige Tage Halt zu
machen?{\grqq}

Ein solche{\s} Angebot de{\s} Ingenieur{\s} konnte nur ironisch
gemeint sein.

{\glqq}Inde{\ss}, fuhr er fort, e{\s} m\"ochte f\"ur Fremde
einigerma{\ss}en gef\"ahrlich werden, inmitten von Negern, Berbern,
Fullah{\s} und Arabern, welche hier wohnen, vorz\"uglich wenn wir
bedenken, da{\ss} die Ankunft de{\s} Aeronef{\s} ihr Mi{\ss}fallen
erregt haben d\"urfte.

-- Mein Herr, erwiderte Phil Evan{\s} in derselben Tonart, f\"ur
da{\s} Vergn\"ugen, Sie verlassen zu k\"onnen, w\"urden wir gern die
Gefahr auf un{\s} nehmen, von den Eingeborenen hier \"ubel empfangen
zu werden. Ein Kerker ist so gut wie der andere, aber Timbuctu immer
noch besser, al{\s} der {\glqq}Albatro{\s}{\grqq}.

-- Da{\s} sind Geschmack{\s}sachen, versetzte der Ingenieur. Auf
keinen Fall m\"ochte ich da{\s} Abenteuer wagen, denn ich bin
verantwortlich f\"ur die Sicherheit der G\"aste, welche mir die Ehre
anthun, mit mir zu reisen~...

-- Sie begn\"ugen sich also nicht mehr, Ingenieur Robur, platzte
jetzt Onkel Prudent, dem die Galle \"uberlief, herau{\s}, mit der
Rolle unsere{\s} Kerkermeister{\s} -- nein, Sie m\"ussen un{\s} auch
noch beleidigen?

-- O, da{\s} war h\"ochsten{\s} eine erlaubte Ironie!

-- Giebt e{\s} denn keine Waffen an Bord?

-- Gewi{\ss}, ein ganze{\s} Arsenal.

-- Zwei Revolver w\"urden gen\"ugen, wenn ich den einen nehme und
Sie, mein Herr, den anderen.

-- Ein Duell, rief Robur, ein Duell, da{\s} Einem von un{\s} da{\s}
Leben kosten k\"onnte!

-- Nein, ihm gewi{\ss} kosten w\"urde!

-- Nein, nein, mein Herr Pr\"asident de{\s} Weldon-Institut{\s}, ich
ziehe e{\s} vor, Sie am Leben zu erhalten.

-- Um sicherer zu sein, da{\ss} Sie selbst leben bleiben. Da{\s} ist
sehr klug.

-- Klug oder nicht, mir pa{\ss}t e{\s} eben. E{\s} steht Ihnen
v\"ollig frei, dar\"uber ander{\s} zu denken und Klage zu erheben,
wenn Sie e{\s} k\"onnen.

-- Da{\s} ist schon geschehen, Ingenieur Robur!

-- Wirklich?

-- War e{\s} denn bei unserer Fahrt \"uber die bewohnten Gegenden
Europa{\s} so schwierig, ein Schriftst\"uck hinunterfallen zu
lassen~...

-- Da{\s} h\"atten Sie gethan? unterbrach ihn Robur, in dem der Zorn
hell aufloderte.

-- Und wenn wir e{\s} gethan h\"atten?

-- Wenn Sie e{\s} gethan h\"atten, verdienten Sie~...

-- Wa{\s} denn, mein Herr Ingenieur?

-- Da{\ss} man Sie Ihrem Schreiben \"uber Bord nachfliegen lie{\ss}e!

-- So werfen Sie un{\s} \"uber Bord ... Wir haben e{\s}
gethan!{\grqq} rief Onkel Prudent.

Robur trat auf die beiden Collegen zu. Auf ein Zeichen von ihm waren
Tom Turner und einige seiner Kameraden herzugelaufen. Ja, der
Ingenieur hatte verzweifelte Lust, seine Drohung zur Au{\s}f\"uhrung
zu bringen, und ohne Zweifel zog er sich nur au{\s} Besorgni{\ss},
ihr nicht widerstehen zu k\"onnen, pl\"otzlich in seine Cabine
zur\"uck.

{\glqq}Sehr sch\"on! sagte Phil Evan{\s}.

-- Und wa{\s} er zu thun nicht wagte, erkl\"arte Onkel Prudent,
da{\s} werde ich wagen, ich, ja, ich werde e{\s} thun!{\grqq}

In diesem Augenblick liefen die Bewohner von Timbuctu auf den
Pl\"atzen und Stra{\ss}en der Stadt zusammen und sammelten sich auf
den Terrassen der amphitheatralisch erbauten H\"auser.

In den reichen Vierteln von Sankore und Sarahama, wie in den elenden
kugelf\"ormigen H\"utten de{\s} Quartier{\s} Raguidi donnerten die
Priester von den Spitzen der Minaret{\s} die schlimmsten Fl\"uche und
Verw\"unschungen gegen da{\s} Ungeheuer in der Luft. Da{\s} war
inde{\ss} unsch\"adlicher, al{\s} Flintenkugeln.

Und auch bi{\s} zum Hafen von Kabara an der scharfen Biegung de{\s}
Niger war Alle{\s}, wa{\s} sich auf Schiffen und Booten befand, in
lebhafterer Bewegung. Wenn der {\glqq}Albatro{\s}{\grqq} hier zur
Erde niedergegangen w\"are, die Leute h\"atten ihn in St\"ucke
gerissen.

W\"ahrend einiger Stunden folgten ihm schreiend und an Schnelligkeit
wetteifernd l\"armende Schaaren von St\"orchen, Haselh\"uhnern und
Ibissen; sein rascher Flug hatte dieselben aber bald hinter sich
zur\"uckgelassen.

Gegen Abend ert\"onte ein dumpfe{\s} Grollen und Murren von
zahlreichen Elephanten und B\"uf\-fel\-heerden, welche in diesen, durch
ganz besondere Fruchtbarkeit au{\s}gezeichneten Gebieten umherirrten.

W\"ahrend vierundzwanzig Stunden entrollte sich die ganze zwischen
dem Meridian 0 und dem 2. Grade der L\"ange zwischen dem Knie de{\s}
Strome{\s} gelegene Gegend unter dem {\glqq}Albatro{\s}{\grqq} gleich
einem Wandelpanorama.

Ja, wenn ein Geograph einen solchen Apparat zur Verf\"ugung gehabt
h\"atte, wie leicht w\"are e{\s} ihm dann nicht gewesen, eine
topographische Aufnahme de{\s} Lande{\s} au{\s}zuf\"uhren, die
h\"ochsten Punkte zu messen, den Lauf der Str\"ome und ihrer
Nebenfl\"usse zu bestimmen und die Lage der St\"adte und D\"orfer
fest\/zusetzen. Dann g\"abe e{\s} in den Karten von Inner-Afrika nicht
mehr so viel leere Stellen, so viel nur mit blassen Farben markirte
L\"ander -- und keine punktirte Linien und unsichere Abgrenzungen
mehr, welche die Kartographen zur Verzweiflung bringen.

Am Morgen de{\s} 11. \"uberschritt der {\glqq}Albatro{\s}{\grqq} die
Berge de{\s} n\"ordlichen Guinea zwischen dem Sudan und dem Golf, der
dessen Namen tr\"agt. Am Horizont erhoben sich schon in undeutlicher
Linie die Kong-Berge de{\s} K\"onigreich{\s} Dahomey.

Seit der Abfahrt von Timbuctu hatten Onkel Prudent und Phil Evan{\s}
beobachten k\"onnen, da{\ss} sie stet{\s} die Richtung von Norden
nach S\"uden eingehalten hatten. Sie schlossen darau{\s}, da{\ss}
sie, wenn hierin keine Aenderung eintrat, sech{\s} Grade weiter die
Aequinoctiallinie erreichen mu{\ss}ten. Sollte der
{\glqq}Albatro{\s}{\grqq} sich wirklich vom Festland ganz wegwenden
und hinau{\s}, nicht auf da{\s} Behring-Meer, den Kaspi{\s}-See, die
Nordsee und da{\s} Mittelmeer, sondern auf den Atlantischen Ocean
wagen wollen?

Diese Au{\s}sicht war f\"ur die beiden Collegen, welche damit jede
Gelegenheit, zu entfliehen, verloren, freilich keine besonder{\s}
angenehme.

Der {\glqq}Albatro{\s}{\grqq} bewegte sich jetzt jedoch nur langsam
vorw\"art{\s}, al{\s} z\"ogere er noch, da{\s} afrikanische Gebiet zu
verlassen. Der Ingenieur dachte inde{\ss} keine{\s}weg{\s} an eine
Umkehr, nur fesselte da{\s} Land, \"uber welche{\s} sie kamen, seine
Aufmerksamkeit im h\"ochsten Grade.

E{\s} ist allgemein und war ihm nicht minder bekannt, da{\ss} da{\s}
K\"onigreich Dahomey an der Westk\"uste Afrika{\s} eine{\s} der
m\"achtigsten ist. Stark genug, um sich mit dem benachbarten Reiche
der Aschanti{\s} im Kampfe messen zu k\"onnen, sind seine Grenzen
doch sehr beschr\"ankt, denn e{\s} mi{\ss}t nur f\"unfundzwanzig
(englische) Meilen von Nord nach S\"ud und gegen sechzig Meilen von
Ost nach West; seine Einwohnerzahl bel\"auft sich jedoch auf
7--800.000 Seelen, seitdem e{\s} die bi{\s}her unabh\"angigen Gebiete
von Ardrah und Wydoch annectirt hat.

Wenn diese{\s} K\"onigreich Dahomey also auch nicht gro{\ss} ist, so
hat e{\s} doch recht oft von sich reden gemacht. E{\s} wurde zeitig
ber\"uhmt durch die entsetzlichen Grausamkeiten, welche daselbst beim
Jahre{\s}wechsel begangen werden, durch die Menschenopfer, die
furchtbaren Hekatomben, welche gew\"ohnlich dem verstorbenen und dem
seine Stelle ersetzenden K\"onige dargebracht werden. Ja, e{\s}
geh\"ort so zu sagen zum guten Ton, da{\ss} der K\"onig von Dahomey,
wenn er den Besuch einer hohen Person oder etwa eine{\s} Gesandten
erh\"alt, diesem zu Ehren einem Dutzend Gefangenen die K\"opfe
abschlagen l\"a{\ss}t -- abschlagen durch seinen Minister der Justiz,
den {\glqq}Minghan{\grqq}, der sich seiner Aufgabe al{\s} Henker
vortrefflich entledigt.

Zur Zeit, al{\s} der {\glqq}Albatro{\s}{\grqq} die Grenze von Dahomey
\"uberschritt, war eben der K\"onig Lahadu verstorben und die ganze
Bev\"olkerung schritt zur Feier der Thronbesteigung seine{\s}
Nachfolger{\s}. Daher herrschte im ganzen Lande eine gro{\ss}e
Aufregung und Bewegung, welche Robur nicht hatte entgehen k\"onnen.

Lange Z\"uge von Landbewohnern Dahomey{\s} dr\"angten sich nach
Abomey, der Hauptstadt de{\s} Reiche{\s}, hin. Ueberall zeigte da{\s}
Land wohlunterhaltene Stra{\ss}en, welche durch weite, mit sehr hohem
Grase bewachsene Ebenen verlaufen. Ungeheure Maniocfelder, W\"alder
voll herrlicher Palmen, Coco{\s}nu{\ss}b\"aume, Mimosen, Orangen- und
Mangob\"aume, deren D\"ufte bi{\s} zum {\glqq}Albatro{\s}{\grqq}
hinaufstiegen, w\"ahrend Tausende von Papageien und Cardin\"alen
au{\s} dem dunklen Gr\"un auf\/flatterten.

Ueber die Reeling gebeugt und in Gedanken versunken, wechselte der
Ingenieur nur wenige Worte mit Tom Turner.

E{\s} schien \"ubrigen{\s} nicht, al{\s} ob der
{\glqq}Albatro{\s}{\grqq} von vornherein die Aufmerksamkeit der sich
fortbewegenden Menschenmasse erweckte, welche auch selbst unter den
dichten Baumkronen meist nicht sichtbar war. Haupts\"achlich kam
da{\s} jedoch wohl daher, da{\ss} er sich in gro{\ss}er H\"ohe und
zwischen leichten Wolken hielt.

Gegen elf Uhr Vormittag{\s} erschien die Stadt mit ihrem
Mauerg\"urtel, den noch ein zw\"olf Meilen im Umfang messender Graben
vertheidigt, mit ihren breiten, regelm\"a{\ss}igen, sehr eben
verlaufenden Stra{\ss}en und dem gro{\ss}en Platz, den der Palast
de{\s} K\"onig{\s} einnimmt. Alle die vielen Baulichkeiten \"uberragt
noch eine Terrasse, nicht weit vom gew\"ohnlichen Opferplatz.
W\"ahrend der gr\"o{\ss}ten Feste werden dem Volke von der H\"ohe
derselben au{\s} die in Weidenk\"orben angebundenen Gefangenen
zugeworfen, und man kann sich schwer eine Vorstellung von der Wuth
machen, mit welcher diese Ungl\"ucklichen in St\"ucke gerissen
werden.

In einem Theile der H\"ofe, welche den Palast de{\s} Herrscher{\s}
umschlie{\ss}en, sind viertausend Krieger einquartirt, eine der
Abtheilungen der k\"oniglichen Armee, und nat\"urlich nicht die
schlechteste.

Wenn e{\s} auch zweifelhaft ist, da{\ss} e{\s} jemal{\s} Amazonen auf
dem Strome diese{\s} Namen{\s} gegeben habe, so liegt da{\s} in
Dahomey ander{\s}. Die Einen tragen hier ein blaue{\s} Hemd, roth und
blaue Sch\"arpe, wei{\ss}e, blaugestreifte Beinkleider, wei{\ss}e
kurze Beinkleider dar\"uber und die Patronentasche im G\"urtel; die
Anderen, die Elephanten-J\"agerinnen, sind bewaffnet mit einer
plumpen Flinte, einem Dolch mit kurzer Klinge, und auf dem Kopfe
tragen sie zwei mit einem Eisenringe befestigte Antilopenh\"orner;
die Artilleristen haben einen halb rothen und halb blauen Ueberwurf
und al{\s} Waffe die Donnerb\"uchse mit alten gu{\ss}eisernen Rohren,
noch Andere endlich, ein Bataillon jener M\"adchen, tr\"agt eine Art
blauer M\"antel mit kurzem wei{\ss}en Beinkleid; da{\s} sind
wirkliche Vestalinnen, keusch wie Diana und wie diese mit Pfeilen und
Bogen au{\s}ger\"ustet.

Rechnet man zu diesen Amazonen noch f\"unf- bi{\s} sech{\s}tausend
Mann in Baumwollhemden und mit einem G\"urtel um die Taille, so hat
man die ganze Armee von Dahomey Revue passiren lassen.

Abomey selbst war an diesem Tage v\"ollig menschenleer, der K\"onig,
da{\s} ganze Personal, die m\"annliche wie die weibliche Armee, sowie
die Einwohner, Alle hatten die Hauptstadt verlassen, um einige Meilen
entfernt auf einem gro{\ss}en, von pr\"achtigem Baumschlag
eingerahmten Platze zusammenzustr\"omen.

E{\s} war da{\s} die Ebene, auf der die Huldigung de{\s} neuen
K\"onig{\s} stattfinden sollte, und hier harrten Tausende, bei
Gelegenheit der letzten Razzia{\s} eingebrachte Gefangene zur Ehre
de{\s}selben ihre{\s} letzten Augenblick{\s}.

Gegen zwei Uhr Nachmittag{\s} begann der jetzt \"uber derselben Ebene
schwebende {\glqq}Albatro{\s}{\grqq} au{\s} einer leichten
Dunstschicht, die ihn bi{\s}her den Augen der Bev\"olkerung von
Dahomey verh\"ullt hatte, etwa{\s} mehr niederzusinken.

Hier befanden sich jetzt wohl gegen sechzigtausend Menschen, die
au{\s} allen Gegenden de{\s} Reiche{\s}, au{\s} Midah, Karapay,
Ardrah, Tombory und au{\s} allen St\"adten und D\"orfern gekommen
waren.

Der neue K\"onig -- ein kr\"aftiger Kerl, Namen{\s} Bu-Stadi und
f\"unfundzwanzig Jahre alt -- thronte auf einer kleinen Anh\"ohe,
welche eine Gruppe von B\"aumen mit langen Aesten beschattete. Vor
ihm dr\"angte sich der neue Hofstaat, seine m\"annliche Armee, seine
Amazonen und da{\s} ganze Volk hin und her.

Am Fu{\ss}e diese{\s} Erdh\"ugel{\s} spielten etwa f\"unfzig Musiker
auf ihren barbarischen Instrumenten, bliesen auf Elephantenz\"ahnen,
die einen rauhen Ton gaben, wirbelten auf gro{\ss}en, mit einer
Hirschkuhhaut bespannten Trommeln, oder hatten Flaschenk\"urbisse,
Guitarren, Glocken, die mit einem Eisenstabe angeschlagen wurden, und
Fl\"oten au{\s} Bambu{\s}rohr, deren scharfer Klang da{\s} ganze
Orchester \"ubert\"onte. Jeden Augenblick krachten die Flinten,
Donnerb\"uchsen und zuweilen die alten Kanonen, deren Lafetten dabei
zur\"ucksprangen, da{\ss} die Artilleristen in Leben{\s}gefahr kamen;
dazu herrschte ein solcher Heidenl\"arm und so w\"uste{\s} Geschrei,
da{\ss} man kaum einen Donnerschlag h\"atte h\"oren k\"onnen.

In einer Ecke der freien Ebene standen, von Soldaten \"uberwacht, die
Gefangenen, welche dem verstorbenen K\"onige da{\s} Geleit in die
andere Welt geben sollten, denn durch sein Ableben darf ein solcher
noch keine Einbu{\ss}e an seiner hohen W\"urde erleiden. Bei der
Leichenfeier Ghozo'{\s}, de{\s} Vater{\s} Bahadu'{\s}, hatte dessen
Sohn ihm dreitausend Diener mitgegeben. Bu-Stadi konnte seinem
Vorg\"anger hierin doch nicht nachstehen. Der Todte brauchte ja eine
Menge Sendboten, nicht allein, um die Geister seiner Ahnen
herbeizurufen, sondern auch, um alle Bewohner de{\s} Himmel{\s} zu
versammeln, welche da{\s} Gefolge de{\s} verewigten K\"onig{\s}
bilden sollten.

Eine Stunde verging mit Gespr\"achen, Vortr\"agen und Ansprachen,
unterbrochen von T\"anzen, welche nicht allein die eigentlichen
Bajaderen auf\/f\"uhrten, sondern auch die Amazonen, die dabei viel
kriegerische Grazie entwickelten.

Inzwischen kam die Zeit zur Hinrichtung heran. Robur, der die
blutigen Gewohnheiten von Dahomey schon kannte, verlor die gefangenen
M\"anner, Frauen und Kinder, welche abgeschlachtet werden sollten,
niemal{\s} au{\s} dem Auge.

Der Minghan verweilte am Fu{\ss}e de{\s} Erdh\"ugel{\s}. Er schwang
da{\s} Richtschwert mit gebogener Klinge, auf der auch noch ein
metallener Vogel sa{\ss}, dessen Gewicht ihm noch mehr Schwung
verlieh. Diese{\s} Mal war er nicht allein; er w\"are mit der Arbeit
auch nicht fertig geworden. In seiner Umgebung befanden sich noch
hundert Scharfrichter, die alle einge\"ubt waren, einen Kopf mit
einem einzigen Hieb vom Rumpfe zu l\"osen.

Inzwischen n\"aherte sich der {\glqq}Albatro{\s}{\grqq} allm\"ahlich
in schr\"ager Richtung und lie{\ss} seine Auftrieb{\s}- und
Treibschrauben mit verminderter Geschwindigkeit spielen. Bald trat er
au{\s} der Wolkenschicht hervor, die ihn bi{\s} wenigsten{\s} hundert
Meter von der Erde verh\"ullt hatte, und wurde jetzt zum ersten Male
sichtbar.

Ganz entgegen den gew\"ohnlichen Erfahrungen sahen die wilden
Eingeborenen in ihm nur ein himmlische{\s} Wesen, da{\s} ganz allein
zu dem Zwecke herabgestiegen sei, dem K\"onige Bahadu zu huldigen.

Da{\s} gab einen Enthusia{\s}mu{\s} ohne Gleichen, unendliche Zurufe,
laute{\s} Jubeln und allgemeine Gebete, gerichtet an diesen
\"ubernat\"urlichen Hippogryph, der ohne Zweifel jetzt kam, um den
K\"orper de{\s} verstorbenen K\"onig{\s} in die H\"ohe de{\s}
Dahomey'schen Himmel{\s} zu tragen.

Eben da fiel der erste Kopf unter dem Schwerte de{\s} Minghan; dann
wurden hundert andere Gefangene ihren schrecklichen Henkern
zugef\"uhrt.

Pl\"otzlich krachte vom {\glqq}Albatro{\s}{\grqq} ein Schu{\ss}. Der
Justizminister st\"urzte getroffen zur Erde.

{\glqq}Gut gezielt, Tom! sagte Robur.

-- Bah! ... E{\s} war ein Schu{\ss} mitten in den Haufen!{\grqq}
antwortete bescheiden der Obersteuermann.

Seine ebenfall{\s} bewaffneten Kameraden standen bereit, auf da{\s}
erste Zeichen de{\s} Ingenieur{\s} Feuer zu geben.

Die Volk{\s}menge hatte durch diesen Vorfall aber ihre Anschauungen
schnell gewechselt; diese{\s} gefl\"ugelte Ungeheuer war kein guter,
sondern ein dem guten Volk von Dahomey feindlicher Geist. Nachdem der
Minghan gefallen, erhob sich ein wilde{\s} Geheul. Gleichzeitig
knatterten viele Gewehre, die nach dem {\glqq}Albatro{\s}{\grqq}
gerichtet waren.

Diese Drohungen hinderten letzteren jedoch nicht, bi{\s} auf etwa
hundertf\"unfzig Fu{\ss} \"uber der Erde niederzusinken. Trotz ihrer
dem Ingenieur Robur gewi{\ss} ung\"unstigen Stimmung konnten sich
Onkel Prudent und Phil Evan{\s} doch nicht versagen, an diesem
menschenfreundlichen Werke theilzunehmen.

{\glqq}Ja, la{\ss}t un{\s} die Gefangenen befreien! riefen sie.

-- Da{\s} ist meine Absicht!{\grqq} antwortete der Ingenieur.

Schon begannen die Repetirgewehre de{\s} {\glqq}Albatro{\s}{\grqq} in
den H\"anden der beiden Collegen, wie in denen der Mannschaft, ein
Schnellfeuer, von dem doch keine Kugel inmitten der gro{\ss}en
Menschenmasse verloren ging. Und selbst da{\s} kleine Gesch\"utz an
Bord, da{\s} so tief al{\s} m\"oglich herabgerichtet wurde, sandte
einige Kart\"atschenladungen hinunter, welche wahre Wunder wirkten.

Sofort sprengten die Gefangenen, ohne etwa{\s} von der ihnen au{\s}
der H\"ohe gekommenen Hilfe zu begreifen, ihre Fesseln, w\"ahrend die
Soldaten auf den Aeronef Feuer gaben. Die vordere Schraube wurde von
einer Kugel durchl\"ochert, w\"ahrend einige andere an den Rumpf
de{\s} Fahrzeuge{\s} schlugen. Frycollin, der sich im Hintergrunde
seiner Cabine verkrochen hatte, w\"are fast noch durch die Wand
de{\s} Ruff{\s} getroffen worden.

{\glqq}Aha, sie haben Appetit auf etwa{\s} mehr!{\grqq} rief Tom
Turner.

Er begab sich nach der Munition{\s}kammer und kehrte von dort mit
einem Dutzend Dynamitpatronen zur\"uck, die er an die Kameraden
vertheilte. Auf ein Zeichen Robur'{\s} wurden dieselben \"uber den
H\"ugel hinabgeworfen und durch da{\s} Aufschlagen auf den Erdboden
zersprangen sie wie kleine Bomben.

Da{\s} gab aber eine wilde Flucht! Der K\"onig, der Hof, die Armee
und da{\s} ganze Volk st\"urzte, von gewi{\ss} nicht
ungerechtfertigter Furcht ergriffen, auf und davon! Alle suchten
unter den B\"aumen Schutz, w\"ahrend die Gefangenen entflohen und
Niemand daran dachte, sie zu verfolgen.

So wurden die Festlichkeiten zu Ehren de{\s} neuen K\"onig{\s} von
Dahomey unterbrochen. Auch Onkel Prudent und Phil Evan{\s} mu{\ss}ten
zugestehen, \"uber wie gro{\ss}e Machtmittel dieser Apparat
verf\"ugte, und welchen hohen Nutzen er der Menschheit h\"atte
gew\"ahren k\"onnen.

Der {\glqq}Albatro{\s}{\grqq} erhob sich darauf zu mittlerer H\"ohe;
er glitt \"uber Mydah hinweg und hatte bald diese ungastliche
K\"uste, welche die Westwinde mit unnahbarer Brandung peitschten,
au{\s} dem Gesichte verloren.

Er schwebte nun \"uber dem Atlantischen Weltmeere.



\newpage\begin{center}\label{kap13}
{\large \begin{antiqua}XIII.\end{antiqua}\\
In dem Onkel Prudent und Phil Evan{\s} einen ganzen Ocean
durchfahren, ohne die Seekrankheit zu bekommen.\\\bigskip}
\end{center}



Ja, da{\s} Atlantische Meer! Die Bef\"urchtungen der beiden Collegen
hatten sich bewahrheitet. E{\s} schien \"ubrigen{\s} nicht, al{\s} ob
Robur hier \"uber dem unendlichen Ocean irgend welche Unruhe
empf\"ande. Da{\s} k\"ummerte ihn so wenig wie seine Leute, welche an
derartigen Fahrten gew\"ohnt sein mochten. Dieselben waren schon
wieder in ihre Wohnung zur\"uckgekehrt. Kein Alpdr\"ucken sollte
ihren Schlummer st\"oren.

Wohin steuerte nun der {\glqq}Albatro{\s}{\grqq}? Sollte er wirklich
noch mehr al{\s} eine Reise um die Erde au{\s}f\"uhren? Auf jeden
Fall mu{\ss}te diese Fahrt doch irgendwo ein Ende nehmen. Da{\ss}
Robur sein ganze{\s} Leben in den L\"uften, an Bord de{\s}
Aeronef{\s} zubringen sollte, ohne jemal{\s} zur Erde hinunter zu
gehen, war doch nicht wohl annehmbar, denn wie h\"atte er seine
Vorr\"athe an Munition und Leben{\s}mitteln erneuern sollen, ohne
da{\s} f\"ur die Functionirung der Maschine nothwendige Material zu
erw\"ahnen? Unbedingt mu{\ss}te er also einen Zuflucht{\s}ort, eine
Art Nothhafen haben, und wahrscheinlich auf einen unbekannten und
schwer erreichbaren Punkt der Erde, wo der {\glqq}Albatro{\s}{\grqq}
sich mit allen Bed\"urfnissen frisch versehen konnte. Mit den
Bewohnern der Erde mochte er jeden Verkehr abgebrochen haben, mit der
Erde al{\s} solcher aber gewi{\ss} nicht.

Doch wenn da{\s} der Fall war, wo lag dieser Punkt? Wie mochte der
Ingenieur dazu gelangt sein, ihn zu erw\"ahlen? Erwartete ihn eine
kleine Colonie etwa al{\s} ihren Herrn? Konnte er von da neue
Mannschaften erhalten? Und zun\"achst, wie war er \"uberhaupt dazu
gekommen, seine, au{\s} den verschiedensten L\"andern stammenden
Leute an sein Schicksal zu binden? Ueber welche Mittel verf\"ugte er
ferner, um einen so kostspieligen Apparat erbauen zu k\"onnen, dessen
ganze Construction so geheim gehalten worden war? Seine Unterhaltung
freilich schien nicht besonder{\s} viel zu beanspruchen. An Bord
f\"uhrte man fast ein gemeinsame{\s} Leben, wie in einer Familie oder
wie gl\"uckliche Leute, die kein Geheimni{\ss} vor einander haben.
Doch, wer war eigentlich jener Robur? Woher kam er? Welcher Art war
seine Vergangenheit? Da{\s} waren ebenso viele unl\"o{\s}bare
R\"athsel, und der, auf den sie Bezug hatten, w\"urde gewi{\ss} der
Letzte sein, eine Erkl\"arung dar\"uber abzugeben.

E{\s} ist gewi{\ss} nicht zu verwundern, wenn diese Situation voller
unenth\"ullbarer Probleme die beiden Collegen mehr und mehr erregte.
Sich so in'{\s} Unbekannte hinau{\s} entf\"uhrt und den endlichen
Au{\s}gang eine{\s} solchen Abenteuer{\s} nicht im geringsten
vorau{\s}zusehen, selbst daran zu zweifeln, da{\ss} da{\s}selbe
\"uberhaupt jemal{\s} ein Ende nehme, zum ewigen Umherfliegen
verurtheilt zu sein -- mu{\ss}te da{\s} den Vorsitzenden und den
Schriftf\"uhrer de{\s} Weldon-Institut{\s} nicht auf'{\s}
Aeu{\ss}erste treiben?

Inzwischen schwebte der {\glqq}Albatro{\s}{\grqq} am Abend de{\s}
elften Juli \"uber den Atlantischen Ocean hin. Al{\s} am n\"achsten
Morgen die Sonne aufging, erhob sie sich \"uber die krei{\s}f\"ormige
Linie, in der Himmel und Wasser zusammen zu treffen scheinen. Trotz
de{\s} weit au{\s}gedehnten Gesicht{\s}felde{\s} war doch nirgend{\s}
ein Land in Sicht und Afrika schon vollst\"andig hinter dem
n\"ordlichen Horizont verschwunden.

Al{\s} Frycollin sich einmal au{\s} seiner Cabine wagte und da{\s}
weite Meer unter sich sah, wurde er sofort von der grimmigsten Angst
gepackt. Unter sich ist eigentlich nicht der richtige Au{\s}druck,
e{\s} w\"are besser zu sagen, {\glqq}um sich{\grqq}, denn f\"ur einen
auf sehr hohem Punkte befindlichen Beobachter erscheint e{\s}, al{\s}
ob der Abgrund ihn von allen Seiten umg\"abe, und der Horizont weicht
dabei gleichsam zur\"uck, ohne da{\ss} man je seine Grenzen erreichen
k\"onnte.

Physikalisch erkl\"arte sich Frycollin diese Erscheinung sicherlich
nicht, aber er f\"uhlte sie moralisch. Da{\s} gen\"ugte aber schon,
um in ihm die {\glqq}Angst vor der Leere{\grqq} zu erzeugen, deren
sich manche, sonst ganz muthige Naturen nicht ent\/ziehen k\"onnen.

Jedenfall{\s} erging sich der Neger au{\s} Klugheit nicht in den
gewohnten Klagen. Mit geschlossenen Augen tastete er sich nach seiner
Cabine zur\"uck, entschlossen, diese auf lange Zeit nicht wieder zu
verlassen.

Von den 373,895.343 Quadratkilometern\footnote[1]{\frakfamily Die
Oberfl\"ache de{\s} festen Lande{\s} betr\"agt 136,055.371
Quadratkilometer.}, welche die Oberfl\"ache der Meere einnehmen,
f\"allt \"uber ein Viertel auf den Atlantischen Ocean. E{\s} schien
aber gar nicht, al{\s} ob der Ingenieur jetzt besondere Eile habe,
wenigsten{\s} hatte er nicht Befehl gegeben, den Aeronef mit voller
Geschwindigkeit arbeiten zu lassen. Uebrigen{\s} h\"atte dieser auch
die Fahrtschnelligkeit wie \"uber Europa hin nicht erreichen
k\"onnen. In den Gegenden, in denen der S\"udwestwind vorherrscht,
lief er diesem fast entgegen, und obwohl derselbe nur schwach zu
nennen war, so bot der Apparat ihm doch eine gro{\ss}e
Angriff{\s}fl\"ache.

Die neuesten und auf eine gro{\ss}e Anzahl von Beobachtungen
gest\"utzten meteorologischen Arbeiten haben eine gewisse Convergenz
der Passate, entweder nach der Sahara oder nach dem Golf von Mexiko,
erkennen lassen. Au{\ss}erhalb der Region der Calmen kommen sie
entweder von Westen und str\"omen nach Afrika zu, oder sie kommen von
Osten her und ziehen nach der Neuen Welt zu -- wenigsten{\s}
w\"ahrend der w\"armeren Jahre{\s}zeit.

Der {\glqq}Albatro{\s}{\grqq} versuchte also gar nicht, gegen den ihm
widrigen Wind mit der ganzen Kraft seiner Treibschrauben anzuk\"ampfen.
Er begn\"ugte sich mit einer gem\"a{\ss}igten Gangart, welche
\"ubrigen{\s} die der tran{\s}atlantischen Dampfer immer noch
\"uberholte.

Am 13. Juli \"uberschritt der Aeronef den Aequator, wa{\s} der ganzen
Mannschaft besonder{\s} angemeldet wurde.

Onkel Prudent und Phil Evan{\s} erfuhren also dabei auch, da{\ss} sie
nun die n\"ordliche Halbkugel verlassen hatten und nach der
s\"udlichen gekommen waren. Diese Passirung der Linie wurde jedoch
nicht durch die tollen Ceremonien gefeiert, welche auf vielen
Krieg{\s}- und Handel{\s}schiffen gebr\"auchlich sind.

Nur Fran\c{c}oi{\s} Tapage lie{\ss} e{\s} sich nicht nehmen,
Frycollin eine gro{\ss}e Pinte Wasser \"uber den Kopf zu gie{\ss}en,
da dieser Taufe aber einige Gl\"aser Gin nachfolgten, erkl\"arte der
Neger sich bereit, die Linie so oft passiren zu wollen, wie man
w\"unschte, vorau{\s}gesetzt, da{\ss} da{\s} nicht auf dem R\"ucken
eine{\s} mechanischen Vogel{\s} zu geschehen brauche, der ihm nun
einmal kein Vertrauen einfl\"o{\ss}te.

Am Morgen de{\s} 15. schwebte der {\glqq}Albatro{\s}{\grqq} \"uber
den Inseln Ascension und St.~Helena, aber n\"aher der letzteren hin,
deren h\"ohere Theile sich einige Stunden lang am Horizonte zeigten.

H\"atte zur Zeit, al{\s} Napoleon sich in der Gewalt der Engl\"ander
befand, ein Apparat, \"ahnlich dem de{\s} Ingenieur{\s} Robur,
existirt, gewi{\ss} w\"urde Hudson Lowe trotz seiner oft geradezu
beleidigenden Vorsicht{\s}ma{\ss}regeln seinen ber\"uhmten Gefangenen
auf dem Wege durch die L\"ufte haben entweichen sehen.

W\"ahrend der beiden Abende de{\s} 16. und 17. Juli zeigten sich mit
Abnahme de{\s} Tage{\s}lichte{\s} h\"ochst eigenth\"umliche
D\"ammerung{\s}erscheinungen. Unter h\"oherer Breite h\"atte man bei
ihrem Anblick an ein Nordlicht denken k\"onnen. Die Sonne warf
n\"amlich bei ihrem Niedergang \"uber den Himmel vielfarbige
Strahlen, von denen einige in leuchtendem Gr\"un erschienen.

War da{\s} eine Wolke ko{\s}mischen Staube{\s}, welche an der Erde
vor\"uber zog und jetzt den letzten Schimmer de{\s} Tage{\s}
wiederstrahlte? Einige Beobachter haben solche
D\"ammerung{\s}erscheinungen in dieser Weise allerding{\s} erkl\"art;
sie w\"aren aber gewi{\ss} zu anderer Anschauung gekommen, wenn sie
sich an Bord de{\s} Aeronef{\s} befunden h\"atten.

Eine aufmerksame Pr\"ufung ergab n\"amlich, da{\ss} in der Luft feine
Pyroxen-Krystalle schwebten, gla{\s}artige K\"ugelchen, n\"amlich
zarte Theilchen magnetischen Eisen{\s}, ganz entsprechend den
Stoffen, welche feuerspeiende Berge au{\s}werfen. E{\s} schwand damit
also jeder Zweifel, da{\ss} diese Wolke von einer vulcanischen
Eruption herr\"uhrte, deren krystallinische Au{\s}wurf{\s}stoffe die
beobachtete Erscheinung erzeugten -- eine Wolke, welche die
Luftstr\"omungen auch noch \"uber dem Atlantischen Ocean schwebend
erhielten.

W\"ahrend diese{\s} Theile{\s} der Reise wurden \"ubrigen{\s} auch
noch andere Erscheinungen wahrgenommen. Wiederholt verliehen gewisse
Wolken dem Himmel eine wei{\ss}graue F\"arbung von eigenth\"umlichem
Au{\s}sehen; gelangte man dann durch einen solchen Dunstvorhang, so
erschien dessen Oberfl\"ache ganz \"ubers\"aet von gl\"anzend
wei{\ss}en K\"orperchen, zwischen denen einzelne gr\"o{\ss}ere
besonder{\s} hervorleuchteten, wa{\s} sich unter dieser Breite durch
nicht{\s} Andere{\s}, al{\s} durch eine Hagelbildung erkl\"aren
lie{\ss}.

In der Nacht vom 17. zum 18. bildete sich ein gr\"unlichgelber
Mondregenbogen infolge der Stellung de{\s} Aeronef{\s} zwischen dem
Vollmonde und einem Netz von fernem Regen, der schon in Dunst
\"uberging, ehe er da{\s} Meer erreichte. Vielleicht lie{\ss} sich
au{\s} diesen verschiedenen Erscheinungen schon auf einen
bevorstehenden Witterung{\s}umschlag schlie{\ss}en. Jedenfall{\s}
hatte der Wind, der seit der Abfahrt von der afrikanischen K\"uste
stet{\s} au{\s} S\"udwesten wehte, sich in der N\"ahe de{\s}
Aequator{\s} ganz gelegt. Hier in der Tropenzone herrschte dazu eine
fast unertr\"agliche Hitze. Robur suchte daher K\"uhlung in h\"oheren
Luftschichten, und doch mu{\ss}te man sich auch noch hier vor den
directen Sonnenstrahlen sch\"utzen, welche Niemand h\"atte
au{\s}halten k\"onnen.

Dieser Wechsel in den Luftstr\"omungen lie{\ss} schon ahnen, da{\ss}
jenseit{\s} de{\s} Aequatorialgebiet{\s} auch andere klimatische
Verh\"altnisse herrschen w\"urden; e{\s} darf hierbei auch nicht
vergessen werden, da{\ss} der Monat Juli der s\"udlichen Halbkugel
der Januar der n\"ordlichen ist, also dem tiefsten Winter entspricht.
Wenn der {\glqq}Albatro{\s}{\grqq} noch weiter nach S\"uden vordrang,
mu{\ss}te er die Folgen davon bald sp\"uren.

Da{\s} Meer aber {\glqq}empfand da{\s}{\grqq}, wie die Seeleute
sagen. Am 18. Juli zeigte sich jenseit{\s} de{\s} Wendekreise{\s}
de{\s} Steinbock{\s} ein andere{\s} Ph\"anomen, welche{\s} gewi{\ss}
jeden Schiffer erschreckt h\"atte.

Mit einer auf mindesten{\s} sechzig Meilen in der Stunde zu
sch\"atzenden Geschwindigkeit zog \"uber da{\s} Meer weg eine
merkw\"urdige Reihe von leuchtenden Wellen, die einander in der
Entfernung von etwa acht\/zig Fu{\ss} folgten und lang schimmernde
Streifen zur\"ucklie{\ss}en. Mit einbrechender Nacht strahlte der
Widerschein davon sogar bi{\s} zum {\glqq}Albatro{\s}{\grqq} hinauf,
so da{\ss} dieser jetzt wirklich h\"atte f\"ur einen gl\"uhenden
kleinen Himmel{\s}k\"orper angesehen werden k\"onnen. Noch nie war
e{\s} Robur vorgekommen, \"uber ein Meer in Flammen hinwegzusteuern
-- \"uber Flammen ohne Hitze, denen zu entfliehen er nicht n\"othig
hatte.

Die Elektricit\"at mu{\ss}te offenbar die Ursache dieser Erscheinung
sein, denn etwa einer Fischlaichbank oder einem von jenen kleinen
Gesch\"opfen gebildeten Zuge, welche zuweilen die Fl\"ache de{\s}
Meere{\s} bedecken, konnte man dieselbe nicht zuschreiben.

Da{\s} lie{\ss} vermuthen, da{\ss} die elektrische Spannung der Luft
jetzt eine sehr hohe sein m\"usse.

Am folgenden Tage, am 19. Juli w\"are ein Schiff auf diesem Meere
wohl dem Untergang geweiht gewesen. Der {\glqq}Albatro{\s}{\grqq}
dagegen spielte mit Wind und Wellen, wie der gewaltige Vogel, dessen
Namen er trug. Wenn e{\s} ihm nicht beliebte, wie ein Sturmvogel
\"uber der Meere{\s}fl\"ache hinzugleiten, so konnte er wie der Adler
in h\"oheren Schichten Ruhe und Sonnenschein aufsuchen.

Man hatte jetzt den 47. Grad s\"udlicher Breite \"uberschritten. Der
Tag dauerte nur noch sieben bi{\s} acht Stunden, und er mu{\ss}te mit
der Ann\"aherung an die antarktischen Gegenden noch immer k\"urzer
werden.

Gegen ein Uhr Nachmittag{\s} hatte sich der {\glqq}Albatro{\s}{\grqq},
um eine g\"unstige Luftstr\"omung aufzufinden, sehr tief gesenkt. Er
schwebte h\"ochsten{\s} noch hundert Fu{\ss} \"uber der Oberfl\"ache
de{\s} Meere{\s}.

Da{\s} Wetter war still. An einzelnen Stellen de{\s} Himmel{\s} zogen
dicke, dunkle Wolken mit au{\s}gezackten R\"andern auf, welche oben
eine genau horizontale Linie bildeten. Au{\s} diesen Wolken quollen
langgezogene Protuberanzen hervor, deren Ende da{\s} Wasser
anzuziehen schien, da{\s} darunter in Form eine{\s} fl\"ussigen
Strau{\ss}e{\s} aufbrodelte.

Pl\"otzlich stieg da{\s} Wasser in Form einer ungeheuren Sanduhr hoch
empor.

In einem Augenblick wurde der {\glqq}Albatro{\s}{\grqq} in den Wirbel
einer riesigen Trombe hineingezogen, der bald zwanzig andere von
Tintenschw\"arze da{\s} Geleite gaben. Zum Gl\"uck vollzog sich die
Drehung dieser Trombe entgegen der der Auftrieb{\s}schrauben, sonst
h\"atten diese ihre Wirkung ganz eingeb\"u{\ss}t und der Aeronef
w\"are in'{\s} Meer gefallen; jetzt wurde er nur mit erschreckender
Schnelligkeit um sich selbst gedreht.

Immerhin war die Gefahr gro{\ss} und schien unm\"oglich abwendbar, da
der Aeronef sich nicht au{\s} der Trombe lo{\s} machen konnte, deren
Anziehung ihn trotz der Treibschrauben zur\"uckhielt. Durch die
Centrifugalkraft wurde die Mannschaft nach beiden Enden de{\s}
Verdeck{\s} geschleudert, und mu{\ss}te sich hier an den
Schraubenmasten anhalten, um nicht \"uber Bord zu fallen.

{\glqq}Ruhig Blut!{\grqq} rief Robur.

Und da{\s} brauchten sie wirklich, und Geduld obendrein.

Onkel Prudent und Phil Evan{\s}, die au{\s} ihrer Cabine
herau{\s}traten, wurden nach dem Hintertheil getrieben und hatten die
gr\"o{\ss}te M\"uhe, sich noch fest zu klammern.

Und w\"ahrend sich der {\glqq}Albatro{\s}{\grqq} in dieser Weise um
sich selbst drehte, folgte er auch der Lagever\"anderung der Tromben,
die mit einer Schnelligkeit, auf welche die Schrauben de{\s}selben
h\"atten eifers\"uchtig werden k\"onnen, sich weiterhin wanden.
Sobald er der einen entgangen, wurde er von einer anderen gepackt und
war stet{\s} in Gefahr, in St\"ucke zerrissen zu werden.

{\glqq}Einen Kanonenschu{\ss}!{\grqq} rief der Ingenieur.

Der Obersteuermann Tom Turner verstand v\"ollig diesen an ihn
gerichteten Befehl; er lehnte eben an dem kleinen Bordgesch\"utz
mittschiff{\s}, wo die Centrifugalkraft minder wirksam war.
Schneller, al{\s} wir e{\s} beschreiben k\"onnen, hatte er die
Schwanzschraube de{\s} Rohr{\s} ge\"offnet und f\"uhrte in diese eine
scharfe Patrone ein, von denen ein kleiner Vorrath in einem an der
Lafette befestigten Kasten vorhanden war. Der Schu{\ss} krachte und
sofort sanken einige Tromben zusammen.

Die Luftersch\"utterung hatte hingereicht, da{\s} Meteor zu
zerrei{\ss}en und die ungeheure Dunstmasse l\"oste sich in einen
Sturzregen auf, der den Himmel mit dicken Wasserstreifen \"uberzog,
die Meer und Himmel verbanden.

Endlich befreit, beeilte sich der {\glqq}Albatro{\s}{\grqq}, um
einige hundert Meter aufzusteigen.

{\glqq}Nicht{\s} zerbrochen an Bord?{\grqq} fragte der Ingenieur.

-- Nein, antwortete Tom Turner; aber da{\s} war denn doch ein
etwa{\s} gar zu tolle{\s} Kreiseln, da{\s} wir un{\s} nicht zum
zweiten Male w\"unschen m\"ochten.{\grqq}

In der That, in der Zeit von zehn Minuten war der
{\glqq}Albatro{\s}{\grqq} in gr\"o{\ss}ter Gefahr gewesen, und ohne
seine solide Bauart d\"urfte er dem Wirbeln der Tromben schwerlich
widerstanden haben.

Wie lang wurden die Stunden bei dieser Fahrt \"uber den Ocean, wenn
nicht{\s} die Eint\"onigkeit derselben unterbrach. Die Tage nahmen
immer mehr ab und die K\"alte wurde allm\"ahlich f\"uhlbar. Onkel
Prudent und Phil Evan{\s} sahen Robur nur wenig. In seine Cabine
eingeschlossen, besch\"aftigte er sich damit, den Cur{\s} zu
bestimmen, auf seinen Karten die zur\"uckgelegten Strecken
einzutragen und sich, wenn e{\s} irgend anging, Gewi{\ss}heit zu
verschaffen, wo sie sich eben befanden, ferner die Barometer,
Thermometer und Chronometer zu beobachten und endlich alle
Zwischenf\"alle der Reise in da{\s} Schiff{\s}buch einzutragen.

Sorgsam verh\"ullt, bem\"uhten sich die beiden Collegen unabl\"assig,
im S\"uden Land zu entdecken.

Frycollin seinerseit{\s} versuchte, gem\"a{\ss} einem besonderen
Auftrage de{\s} Onkel Prudent, den Koch bez\"uglich de{\s}
Ingenieur{\s} au{\s}zuforschen. Wie h\"atte aber Jemand au{\s} dem,
wa{\s} der Ga{\s}cogner Fran\c{c}oi{\s} Tapage zur Antwort gab, klug
werden k\"onnen? Nach ihm war Robur bald ein ehemaliger Minister der
Republik Argentina, ein Chef der Admiralit\"at, ein abgetretener
Pr\"asident der Vereinigten Staaten, ein auf Wartegeld gesetzter
spanischer General, oder auch ein Vicek\"onig von Indien, der in den
L\"uften eine noch h\"ohere Stellung gesucht hatte. Bald besa{\ss}
er, Dank der mit Hilfe seiner Maschine au{\s}gef\"uhrten Razzia{\s},
Millionen und war er allgemein in die Acht erkl\"art; bald hatte er
sich wieder durch die Herstellung diese{\s} Apparat{\s} ruinirt und
gezwungen gesehen, \"offentlich aufzusteigen, um sein Geld wieder zu
gewinnen. Auf die Frage nach einem Ruheplatz de{\s}selben war keine
Au{\s}kunft zu erhalten, au{\ss}er der, da{\ss} er nach dem Mond zu
gehen beabsichtige, um dort zu bleiben, wenn er eine ihm passende
Oertlichkeit antr\"afe.

{\glqq}He, Fry ... mein Kamerad! ... Nicht wahr, e{\s} w\"urde Dir
Vergn\"ugen machen, zu sehen, wie e{\s} da oben zugeht?

-- Ich gehe nicht mit! Ich weigere mich! ... erwiderte der
Schwachkopf, der alle diese Faseleien f\"ur Ernst nahm.

-- Und we{\s}halb, Fry, we{\s}halb? Wir verheiraten Dich dort mit
einer h\"ubschen, jungen Mondbewohnerin ... Du wirst da der
Stammvater der Neger!{\grqq}

Al{\s} Frycollin da{\s}, wa{\s} er geh\"ort, seinem Herrn
hinterbrachte, erkannte dieser wohl, da{\ss} \"uber Robur keine
Au{\s}kunft zu erlangen sei. Er dachte also nur noch daran, sich zu
r\"achen.

{\glqq}Phil, begann er eine{\s} Tage{\s} zu seinem Collegen, e{\s}
liegt nun auf der Hand, da{\ss} eine Flucht f\"ur un{\s} unm\"oglich
ist.

-- Unm\"oglich, Onkel Prudent!

-- Zugegeben, ein Mann ist aber stet{\s} sein eigener Herr, und wenn
e{\s} sein mu{\ss}, indem er sein Leben opfert~...

-- Wenn ein solche{\s} Opfer nothwendig ist, dann wird e{\s} so
schnell al{\s} m\"oglich gebracht! antwortete Phil Evan{\s}, dessen
sonst so k\"uhle{\s} Temperament nun doch die Grenze de{\s}
Ertr\"aglichen erreicht hatte. Ja, e{\s} ist Zeit, ein Ende zu
machen! ... Wohin geht der {\glqq}Albatro{\s}{\grqq}? ... Jetzt
fliegt er schr\"ag \"uber den Atlantischen Ocean, und wenn er diese
Richtung beibeh\"alt, mu{\ss} er nach den K\"usten von Patagonien,
dann nach denen de{\s} Feuerlande{\s} kommen ... Aber nachher? ...
Wird er auch noch \"uber den Stillen Ocean hinau{\s}schweben? Oder
steuert er dann nach dem S\"udpolarlande? ... Diesem Robur ist
Alle{\s} zuzutrauen! ... Dann w\"aren wir verloren! ... Wir befinden
un{\s} also in der Zwang{\s}lage berechtigter Nothwehr, und wenn wir
einmal zu Grunde gehen m\"ussen~...

-- So geschehe e{\s} nicht, fiel Onkel Prudent ein, ohne da{\ss} wir
un{\s} ger\"acht, ohne da{\ss} wir diesen Apparat mit Allen, die er
tr\"agt, zerst\"ort haben!{\grqq}

Bi{\s} zu solchen Anschauungen hatte der ohnm\"achtige Zorn, die in
ihnen aufgeh\"aufte Wuth die beiden Collegen schon gebracht! Ja, weil
e{\s} nicht ander{\s} ging, wollten sie sich opfern, um den Erfinder
sammt seinem Geheimni{\ss} zu vernichten. Nur wenige Monate h\"atte
dann dieser wunderbare Aeronef erlebt, dessen unbestreitbare
Ueberlegenheit bez\"uglich der Fortbewegung durch die Luft sie
anzuerkennen sich gezwungen sahen.

Diese Vorstellung hatte in ihren K\"opfen so fest Wurzel geschlagen,
da{\ss} sie an gar nicht{\s} Andere{\s} mehr dachten. Doch wie
sollten sie zu Werke gehen? O, sie wollten sich nur einer der an Bord
vorhandenen Explosion{\s}maschinen bem\"achtigen, um damit den ganzen
Apparat in tausend St\"ucke zu zersprengen; freilich mu{\ss}ten sie
dazu erst Gelegenheit finden, in die Munition{\s}kammer einzudringen.

Gl\"ucklicher Weise ahnte Frycollin nicht{\s} von ihren Absichten.
Bei dem Gedanken, da{\ss} der {\glqq}Albatro{\s}{\grqq} in die Luft
gesprengt werden sollte, w\"urde er sich nicht entbl\"odet haben,
seinen eigenen Herrn zu verrathen.

Am 23. Juli war e{\s}, wo in S\"udwesten wieder Land sichtbar wurde,
und zwar nahe dem Cap der Jungfrauen am Eingange der
Magellan-Stra{\ss}e. Jenseit{\s} de{\s} 54. Breitengrade{\s} w\"ahrte
die Nacht in dieser Jahre{\s}zeit fast neunzehn Stunden und die
Mitteltemperatur der Luft blieb fortw\"ahrend unter 0 Grad zur\"uck.

Statt jetzt noch weiter nach S\"uden vorzudringen, folgte der
{\glqq}Albatro{\s}{\grqq} zun\"achst den Windungen jener Meerenge,
al{\s} ob er dem Stillen Ocean zutriebe. Nachdem er \"uber die Bai
von Loma{\s} hinweggekommen, den Gregory-Berg im Norden und die
Brecknock{\s}-Berge im Westen hinter sich gelassen, kam er in Sicht
von Punta Arena, einem kleinen chilenischen D\"orfchen, gerade al{\s}
daselbst da{\s} volle Kirchengel\"aute erklang, und einige Stunden
sp\"ater in die N\"ahe der alten Niederlassung im sogenannten
Hungerhafen.

Wenn die Patagonier, deren Feuer man da und dort aufleuchten sah,
wirklich eine da{\s} mittlere Menschenma{\ss} \"ubertreffende
K\"orpergr\"o{\ss}e haben, so konnten doch die Passagiere de{\s}
Aeronef{\s} dar\"uber nicht urtheilen, da sie ihnen, von dieser
H\"ohe gesehen, al{\s} Zwerge erschienen.

Doch welch' Schauspiel bot sich hier w\"ahrend der kurzen Stunden
de{\s} s\"udlichen Tage{\s}! Steile, zerkl\"uftete Berge, mit ewigem
Schnee bedeckte Spitzen, deren Seiten mit dichten W\"aldern bedeckt
waren; Binnenseen, Buchten zwischen den Vorgebirgen und Inseln
diese{\s} Archipel{\s}; daneben \mbox{Clarence-,} Dawson- und
Desolation{\s}land, Can\"ale und Furthen, unz\"ahlige Cap{\s} und
Halbinseln -- all' diese{\s} undurchdringliche Gewirr, und jetzt
durch da{\s} Ei{\s} zu einer festen Masse verschmolzen, vom Cap
Forward, am Ende de{\s} amerikanischen Festlande{\s}, bi{\s} zum Cap
Horn am letzten Au{\s}l\"aufer der Neuen Welt!

Nachdem er jedoch den Hungerhafen erreicht, nahm der
{\glqq}Albatro{\s}{\grqq} wieder eine v\"ollig s\"udliche Richtung
an. Zwischen dem Tarnberge der Halbinsel Brun{\s}wick und dem
Grawe{\s}-Berge hindurchsteuernd, wandte er sich in gerader Linie
nach dem Sarmiento-Berge, einem gewaltigen, dick \"ubereisten
Spitzberge, welcher die Meerenge der Magellan-Stra{\ss}e in einer
H\"ohe von zweitausend Metern beherrscht.

Hier zeigte sich den Blicken der Reisenden da{\s} Land der
Pescher\"a{\s} oder Fuegier, jener Ureinwohner, welche noch im
Feuerland siedeln.

Wie herrlich und fruchtbar h\"atten sich diese Gebiete --
vorz\"uglich deren mittlerer Theil -- im Sommer gezeigt, wo die Tage
f\"unfzehn bi{\s} sechzehn Stunden lang dauern! Ueberall bieten sie
n\"amlich Th\"aler und Weidepl\"atze, welche Abertausende von Thieren
ern\"ahren k\"onnten, nebst jungfr\"aulichen W\"aldern mit
riesenhaften B\"aumen, mit Weiden, Buchen, Eschen, Cypressen und
bl\"uhenden Farrenkr\"autern; dann wieder Ebenen, welche gro{\ss}e
Heerden von Guanaquen, Vigogneschafen und Strau{\ss}en bewohnen.
Al{\s} der {\glqq}Albatro{\s}{\grqq} seine elektrischen Lichter
ergl\"uhen lie{\ss}, flatterten auch Papageientaucher, Enten und
G\"anse -- hundertmal mehr, al{\s} Fran\c{c}oi{\s} Tapage'{\s}
Speisekammer fassen konnte, an Bord.

Der Koch, welcher diese{\s} Federwild so vortrefflich zuzurichten
verstand, da{\ss} e{\s} seinen thranigen Geschmack ganz verlor,
erhielt dadurch pl\"otzlich weit mehr Arbeit, al{\s} gew\"ohnlich;
mehr Arbeit machte da{\s} aber auch Frycollin, der e{\s} nicht
abschlagen konnte, von diesem interessanten Gefl\"ugel ein Dutzend
nach dem anderen wenigsten{\s} zu rupfen.

Am n\"amlichen Tage zeigte sich, al{\s} die Sonne eben versinken
wollte, auch noch ein ziemlich gro{\ss}er, von pr\"achtigen Waldungen
eingerahmter See. Jetzt lagerte \"uber dem See eine feste Ei{\s}decke
und einige Eingeborene glitten auf ihren langen Schneeschuhen
pfeilschnell \"uber dessen Oberfl\"ache hin.

Beim Erblicken der Flugmaschine entflohen diese Fuegier n\"amlich
nach allen Seiten, und wenn sie nicht fliehen konnten, so versteckten
sie sich doch und gruben sich wie Thiere in die Erde ein.

Der {\glqq}Albatro{\s}{\grqq} steuerte noch immer nach S\"uden \"uber
dem Beagle-Canal hinau{\s} und weiter fort, al{\s} die Insel Navarin,
deren griechischer Name unter den gew\"ohnlichen Bezeichnungen dieser
entlegenen Landstrecken etwa{\s} auf\/f\"allig erscheint, weiter,
al{\s} die Insel Wollaston, die sich schon in den Wogen de{\s}
Stillen Ocean{\s} badet. Endlich, nachdem er von Dahomey{\s} K\"uste
au{\s} \"uber siebentausendf\"unfhundert Kilometer zur\"uckgelegt,
schwebte er \"uber die letzten Inseln de{\s} Magellan-Archipel{\s}
hinweg und endlich ganz im S\"uden \"uber da{\s} schreckliche Cap
Horn, an da{\s} eine unaufh\"orliche wilde Brandung donnert.



\newpage\begin{center}\label{kap14}
{\large \begin{antiqua}XIV.\end{antiqua}\\
In dem der {\glqq}Albatro{\s}{\grqq} etwa{\s} au{\s}f\"uhrt, wa{\s}
man vielleicht niemal{\s} d\"urfte au{\s}f\"uhren k\"onnen.\\\bigskip}
\end{center}



Der n\"achstfolgende Tag war der 24. Juli. Der 24. Juli der
s\"udlichen Halbkugel entspricht bekanntlich aber dem 24. Januar der
n\"ordlichen Hemisph\"are; au{\ss}erdem war jetzt auch schon der 56.
Breitegrad \"uberschritten, der im Norden Europa{\s} Schottland in
der H\"ohe von Edinburgh und S\"udschweden in der von Helsingborg
durchschneidet.

Der Thermometer hielt sich auch fortw\"ahrend unter 0 Grad, so
da{\ss} e{\s} sich n\"othig machte, die Ruff{\s} durch k\"unstliche
W\"arme etwa{\s} wohnlicher zu machen.

E{\s} versteht sich von selbst, da{\ss} der Tag, wenn er seit dem 21.
Juni de{\s} s\"udlichen Winter{\s} auch schon zunahm, doch immer noch
merklich k\"urzer wurde, da der {\glqq}Albatro{\s}{\grqq} einen
Cur{\s} nach den Polarregionen einhielt.

Die Folge davon war eine sehr geringe Helligkeit \"uber jenem Theile
de{\s} Stillen Ocean{\s}, der an da{\s} antarktische Ei{\s}meer
grenzt. Man hatte also nur wenig Au{\s}sicht und w\"ahrend der Nacht
recht empfindliche K\"alte. Um ihr zu widerstehen, mu{\ss}te man sich
nach Art der E{\s}kimo{\s} und Fuegier kleiden; doch da e{\s} an Bord
an warmen Ueberkleidern nicht fehlte, konnten die beiden, wohl
eingepackten Collegen doch auf dem Verdeck au{\s}harren, wobei sie
freilich immer nur an ihr Vorhaben dachten und eine dazu g\"unstige
Gelegenheit zu ersp\"ahen suchten. Uebrigen{\s} sahen sie Robur setzt
sehr wenig, und seit jenem Wortwechsel mit beiderseitigen Drohungen,
der \"uber Timbuctu stattfand, sprachen der Ingenieur und sie nicht
mit einander.

Frycollin kam jetzt kaum noch au{\s} der K\"uche Fran\c{c}oi{\s}
Tapage'{\s} herau{\s}, der ihm hier wohlwollende Gastfreundschaft
gew\"ahrte -- unter der Bedingung, da{\ss} er sich al{\s}
Hilf{\s}koch n\"utzlich machte. Da da{\s} nicht ohne Vortheil f\"ur
ihn abging, hatte der Neger sich mit Erlaubni{\ss} seine{\s} Herrn
gern dazu verpflichtet. So eingeschlossen, sah er ja auch nicht,
wa{\s} drau{\ss}en vorging, und konnte sich gegen jede Gefahr
gesch\"utzt glauben. Glich er nicht v\"ollig dem th\"orichten
Strau{\ss}e, nicht allein physisch durch seinen vortrefflichen Magen,
sondern auch geistig durch seine kindische Beschr\"anktheit?

Doch nach welchem Punkte der Erde sollte der
{\glqq}Albatro{\s}{\grqq} sich nun wenden? Konnte man wohl annehmen,
da{\ss} er sich im tiefen Winter \"uber diese s\"udlichen Meere oder
\"uber da{\s} Festland de{\s} Pol{\s} hinau{\s}wage? Selbst wenn die
Chemikalien in den Batterien nicht durch die furchtbare K\"alte
erstarrten, drohte in dieser eisigen Atmosph\"are doch Allen der Tod
-- der schreckliche Tod de{\s} Erfrieren{\s}. Da{\ss} Robur e{\s}
unternommen h\"atte, in der warmen Jahre{\s}zeit \"uber den Pol zu
fahren, m\"ochte wohl angehen; inmitten der ewigen Nacht de{\s}
antarktischen Winter{\s} erschien die{\s} dagegen wie der Streich
eine{\s} Tollh\"au{\s}ler{\s}.

Diesen Gedankengang hatten der Vorsitzende und der Schriftf\"uhrer
de{\s} Weldon-Institut{\s}, al{\s} sie sich jetzt nach dem
\"au{\ss}ersten Ende der Neuen Welt entf\"uhrt sahen, nach Gegenden,
welche zwar zu Amerika, aber freilich nicht zu den Vereinigten
Staaten geh\"oren.

Ja, wa{\s} hatte dieser unergr\"undliche Robur noch Alle{\s} vor? War
jetzt nicht der Zeitpunkt gekommen, die Reise durch Zerst\"orung
de{\s} fliegenden Apparat{\s} zu beendigen?

Fiel hier\"uber die Antwort auch noch unbestimmt au{\s}, so stand
dagegen fest, da{\ss} der Ingenieur im Laufe de{\s} 24. Juli
wiederholt mit seinem Obersteuermann verhandelte. Mehrere Male
beobachteten auch Tom Turner und er selbst den Barometer, die{\s}mal
aber nicht zur Absch\"atzung der H\"ohe, sondern um verschiedene
Anzeichen eine{\s} drohenden Wetterumschlag{\s} zu erkennen.

Onkel Prudent glaubte auch zu bemerken, da{\ss} Robur in Erfahrung zu
bringen suchte, wie viel er noch an Vorr\"athen aller Art, ebenso
derjenigen zur Unterhaltung der Treib- und Auftrieb{\s}maschinen
de{\s} Aeronef{\s}, wie derjenigen f\"ur die menschlichen Maschinen
besa{\ss}, da e{\s} darauf ankam, auch diese und ihre Arbeit{\s}kraft
in bestem Zustand zu erhalten.

Alle{\s} da{\s} schien auf eine geplante Umkehr hinzudeuten.

{\glqq}Umkehr? ... sagte Phil Evan{\s}, aber wohin?

-- Dahin, wo sich Robur frisch verproviantiren kann, antwortete Onkel
Prudent.

-- Da{\s} m\"u{\ss}te irgend eine im Stillen Ocean verlorene Insel
sein, mit einer, ihre{\s} Chef{\s} ganz w\"urdigen Colonie von
Verbrechern.

-- Da{\s} ist auch meine Ansicht, Phil Evan{\s}. Ich glaube wirklich,
er wird nach Westen zu wenden lassen, und bei der Schnelligkeit,
\"uber die er verf\"ugt, d\"urfte er sein Ziel bald genug erreicht
haben.

-- Wir w\"urden jedoch unseren Plan nicht mehr zur Au{\s}f\"uhrung
bringen k\"onnen ... wenn er daselbst ankommt~...

-- Er wird nicht ankommen, Phil Evan{\s}!{\grqq}

Wie sich zeigte, hatten die beiden Collegen die Absichten de{\s}
Ingenieur{\s} wenigsten{\s} zum Theil errathen. Im Laufe de{\s}
Tage{\s} noch schwand jeder Zweifel, da{\ss} der
{\glqq}Albatro{\s}{\grqq}, nachdem er die Grenzen de{\s} s\"udlichen
Ei{\s}meere{\s} gestreift, entschieden wieder r\"uckw\"art{\s}
steuerte.

Wenn die Ei{\s}schollen auf dem Wasser bi{\s} zum Cap Horn
hintrieben, mu{\ss}ten sich die s\"udlicheren Theile de{\s} Stillen
Weltmeere{\s} ganz mit Ei{\s}feldern und Ei{\s}bergen bedecken, und
da{\s} Packei{\s} bildete dann einen undurchdringlichen Wall f\"ur
die festesten Schiffe, wie f\"ur die tollk\"uhnsten Reisenden.

Gewi{\ss} h\"atte der {\glqq}Albatro{\s}{\grqq}, wenn er seinen
Fl\"ugelschlag beschleunigte, diese \"uber den Ocean aufgeth\"urmten
Ei{\s}berge ebenso \"uberfliegen k\"onnen, wie die auf dem Festlande
de{\s} Polarkreise{\s} aufragenden Gebirge -- wenn e{\s} \"uberhaupt
ein Festland ist, wa{\s} da{\s} S\"udende der Erdachse \"uberdeckt.
Doch entschieden w\"urde er nicht gewagt haben, inmitten der
finsteren Polarnacht auch einer Polarluft Trotz zu bieten, welche
sich zuweilen bi{\s} 60 Grad unter Null abk\"uhlen kann.

Nachdem er also etwa hundert Kilometer nach S\"uden vorgedrungen,
wandte sich der {\glqq}Albatro{\s}{\grqq} nach Westen, so, al{\s} ob
er die Richtung nach einer unbekannten Insel de{\s} Archipel{\s}
de{\s} Stillen Ocean{\s} einschl\"uge.

Unter ihm breitete sich die fl\"ussige Ebene au{\s}, welche zwischen
die L\"andermasse Amerika{\s} und Asien{\s} geworfen ist. Jetzt
hatten die Fluthen derselben jene eigenth\"umliche F\"arbung
angenommen, die zum Theil dem Ocean den Namen de{\s}
{\glqq}Milchmeere{\s}{\grqq} erworben haben. In dem Halbdunkel,
welche{\s} die matten Sonnenstrahlen niemal{\s} wirklich
durchdrangen, erschien die ganze Oberfl\"ache de{\s} Stillen
Weltmeere{\s} wirklich milchig wei{\ss}. Man h\"atte ein
ungeheure{\s} Schneefeld zu erblicken geglaubt, dessen Bodensenkungen
und Erhebungen au{\s} dieser H\"ohe nicht zu erkennen w\"aren. W\"are
auch dieser ganze Meere{\s}theil durch die K\"alte zu einem einzigen
Ei{\s}feld erstarrt gewesen, der Anblick de{\s}selben h\"atte kaum
ein anderer sein k\"onnen.

Jetzt wei{\ss} man, da{\ss} e{\s} Myriaden leuchtender Theilchen,
phosphore{\s}cirende K\"orperchen sind, welche diese Erscheinung
erzeugen. Merkw\"urdig blieb nur der Umstand, diesen opalisirenden
Massen au{\ss}erhalb de{\s} indischen Ocean{\s} zu begegnen.

Nachdem der Barometer sich in den ersten Stunden de{\s} Tage{\s}
ziemlich hoch gehalten hatte, fiel er pl\"otzlich sehr rasch, ein
Anzeichen, da{\s} f\"ur jede{\s} Schiff von ernster Bedeutung gewesen
w\"are, w\"ahrend der Aeronef e{\s} au{\ss}er Acht lassen konnte.
Jedenfall{\s} lie{\ss} da{\s}selbe aber erkennen, da{\ss} in letzter
Zeit ein furchtbarer Sturm die Gew\"asser de{\s} Pacifischen
Ocean{\s} aufgewirbelt haben mochte.

E{\s} war gegen zwei Uhr Mittag{\s}, al{\s} Tom Turner auf den
Ingenieur zutrat.

{\glqq}Master Robur, begann er, sehen Sie da den schwarzen Punkt am
Horizont? ... Dort, gerade im Norden vor un{\s} ... ein Felsen kann
da{\s} nicht sein?

-- Nein, Tom, nach dieser Seite zu liegt kein Land.

-- Dann mu{\ss} e{\s} ein Schiff sein, oder doch irgend ein
Fahrzeug.{\grqq}

Onkel Prudent und Phil Evan{\s} blickten nach dem von Tom Turner
bezeichneten Punkt hinau{\s}.

Robur lie{\ss} sich sein Seefernrohr reichen und betrachtete scharf
den fraglichen Gegenstand.

{\glqq}E{\s} ist ein Boot, sagte er, und ich m\"ochte behaupten,
da{\ss} sich Menschen in demselben befinden.

-- Schiffbr\"uchige? rief Tom.

-- Ja, Schiffbr\"uchige, welche gezwungen gewesen sein werden, ihr
Schiff zu verlassen, erkl\"arte Robur; Ungl\"uckliche, welche nicht
wissen, wo sie Land finden sollen und die vielleicht vor Hunger und
Durst umkommen. Wohlan, e{\s} soll Niemand sagen k\"onnen, der
{\glqq}Albatro{\s}{\grqq} h\"atte nicht den Versuch gemacht, ihnen
Hilfe zu bringen!{\grqq}

Der Maschinist und der Gehilfe erhielten dem entsprechend Befehl und
der Aeronef begann langsam hinabzusinken. In hundert Meter H\"ohe
hielt er damit ein und seine Propeller trieben ihn rasch nach Norden.

E{\s} war in der That ein Boot, an dessen Mast ein Segel schlaff
herabhing und da{\s} wegen Mangel{\s} an Wind nicht vorw\"art{\s}
kommen konnte. Die an Bord befindlichen Leute hatten offenbar nicht
mehr Kraft genug, ein Ruder zu handhaben.

Auf dem Boden de{\s}selben lagen f\"unf Menschen, die eingeschlafen
oder vor Entkr\"aftung regung{\s}lo{\s} geworden waren, wenn nicht
gar der Tod sie schon ereilt hatte.

Ueber ihnen angekommen, ging der {\glqq}Albatro{\s}{\grqq} langsam
nach unten. Am Heck de{\s} Boote{\s} konnte man noch den Namen de{\s}
Schiffe{\s} lesen, zu dem e{\s} geh\"ort hatte; e{\s} war die
{\glqq}Jeannette{\grqq} von Nante{\s} gewesen, ein franz\"osische{\s}
Schiff, da{\s} seine Besatzung hatte aufgeben m\"ussen.

{\glqq}Aoh!{\grqq} rief Tom Turner.

Die Leute mu{\ss}ten ihn wohl h\"oren, denn sie befanden sich kaum
acht\/zig Fu{\ss} unter ihm.

Keine Antwort.

{\glqq}Schie{\ss}t ein Gewehr ab!{\grqq} sagte Robur.

Der Befehl wurde au{\s}gef\"uhrt und der Knall de{\s} Schusse{\s}
verbreitete sich weithin \"uber die Oberfl\"ache de{\s} Wasser{\s}.

Da sah man einen der Schiffbr\"uchigen sich m\"uhsam aufrichten,
seine Augen lagen tief in ihren H\"ohlen, so da{\ss} da{\s} Gesicht
mehr dem eine{\s} Skelet{\s} \"ahnelte.

Al{\s} er den {\glqq}Albatro{\s}{\grqq} bemerkte, malte sich in
seinen Z\"ugen erst der helle Schrecken.

{\glqq}F\"urchtet nicht{\s}! rief Robur ihm franz\"osisch zu. Wir
kommen Euch zu retten. Wer seid Ihr?{\grqq}

-- Matrosen von der {\glqq}Jeannette{\grqq}, einer Dreimastbark,
deren zweiter Officier ich war, antwortete der Mann. Vor nun vierzehn
Tagen ... mu{\ss}ten wir dieselbe verlassen ... weil sie schon im
Sinken war ... Wir haben weder Wasser, noch Leben{\s}mittel
mehr!~...{\grqq}

Die vier \"ubrigen Schiffbr\"uchigen hatten sich nach und nach
etwa{\s} erhoben. Elend, kraftlo{\s} und entsetzlich abgemagert,
streckten sie die H\"ande nach dem Aeronef empor.

{\glqq}Achtung!{\grqq} rief Robur.

Vom Verdeck au{\s} sank ein Tau hernieder und ein Eimer mit
S\"u{\ss}wasser wurde zu dem Boote hinabgesendet.

Die Ungl\"ucklichen st\"urzten dar\"uber her und tranken mit einer
Hast, welche fast widerlich mit anzusehen war.

{\glqq}Brot! ... Brot! ...{\grqq} riefen sie.

Sofort stieg auch ein Korb mit einigen Leben{\s}mitteln, mit
Conserven, einem Fl\"aschchen Brandy und mehreren Pinten Kaffee zu
ihnen herunter. Der zweite Officier hatte alle M\"uhe, die Leute bei
der Stillung ihre{\s} Hunger{\s} nur einigerma{\ss}en im Zaum zu
halten.

{\glqq}Wo sind wir denn? fragte er dann.

-- F\"unfzig Meilen von der K\"uste von Chili und dem
Chona{\s}-Archipel, antwortete Robur.

-- Ich danke, doch wir haben keinen Wind, und~...

-- Wir werden Sie in'{\s} Schlepptau nehmen.

-- Wer sind Sie?

-- Leute, die sich gl\"ucklich sch\"atzen, da{\ss} sie im Stande
waren, Euch Hilfe zu bringen,{\grqq} erwiderte einfach Robur.

Der Mann begriff, da{\ss} er hier ein Incognito zu respectiren habe.
Doch war e{\s} wirklich m\"oglich, da{\ss} diese Maschine Kraft genug
besa{\ss}, sie zu schleppen?

Ja; durch Vermittlung eine{\s} hundert Fu{\ss} langen Kabel{\s} wurde
da{\s} Boot von dem m\"achtigen Apparat nach Osten hingezogen.

Um zehn Uhr Abend{\s} war Land in Sicht, oder man sah wenigsten{\s}
die Leuchtfeuer, welche dessen Lage bezeichneten. Sie war wirklich
zur rechten Zeit gekommen, diese Hilfe vom Himmel f\"ur die
Schiffbr\"uchigen der {\glqq}Jeannette{\grqq}, und sie hatten
gewi{\ss} einige{\s} Recht, ihre Rettung f\"ur ein Wunder zu halten.

Al{\s} sie dann bi{\s} zum Eingang der Wasserstra{\ss}e zwischen den
Chona{\s}-Inseln gebracht worden waren, rief ihnen Robur zu, da{\s}
Tau schie{\ss}en zu lassen, wa{\s} sie denn auch, ihre Retter
segnend, thaten, und der {\glqq}Albatro{\s}{\grqq} steuerte wieder
auf die offene See hinau{\s}.

Entschieden besa{\ss} er doch gute Eigenschaften, dieser Aeronef, der
auf diese Weise im weiten Weltmeer verirrten Seeleuten Beistand
leisten konnte. Welcher noch so vervollkommnete Ballon w\"are im
Stande gewesen, e{\s} ihm nachzuthun? Unter sich mu{\ss}ten auch
Onkel Prudent und Phil Evan{\s} da{\s} anerkennen, obwohl sie mehr in
der Laune waren, die Wahrheit de{\s} ganzen Zwischenfall{\s} zu
leugnen.

Da{\s} Meer blieb immer aufgeregt und e{\s} traten weitere
beunruhigende Vorzeichen auf. Der Barometer sank noch um einige
Millimeter, dann und wann brausten sehr heftige B\"oen daher, welche
in den helikopterischen Maschinen de{\s} {\glqq}Albatro{\s}{\grqq}
ein laute{\s} Pfeifen verursachten und diesen merkbar aufhielten.
Unter solchen Umst\"anden h\"atte ein Segelschiff schon die
Mar{\s}segel zweimal und da{\s} Focksegel einmal reefen m\"ussen.
Alle{\s} deutete darauf, da{\ss} der Wind nach Nordwesten umschlagen
werde. Da{\s} Rohr de{\s} Sturmglase{\s} fing an, sich beunruhigend
zu tr\"uben. Um ein Uhr Morgen{\s} erlangte der Wind eine
ungew\"ohnliche Heftigkeit. Obgleich der Aeronef denselben ganz von
vorne hatte, so konnten seine Propeller ihn doch noch forttreiben, so
da{\ss} er etwa vier bi{\s} f\"unf Meilen in der Stunde
zur\"ucklegte. Mehr konnte man jedoch nicht von ihm verlangen.

Ganz entschieden war jetzt ein Cyclon im Anzuge, wa{\s} unter diesen
Breiten sehr selten vorkommt. Ob man diesen nun Hurracan im
Atlantischen, Typhon im Chinesischen Meere, Samum in der Sahara und
Tornado an der Westk\"uste nennt, immer ist e{\s} ein Wirbelsturm,
der gro{\ss}e Gefahren mit sich bringt. Ja, Gefahren f\"ur jede{\s}
Fahrzeug, da{\s} von seiner drehenden Bewegung gepackt wird, die nach
dem Centrum hin zunimmt und nur eine Stelle ruhig l\"a{\ss}t, den
innersten Mittelpunkt diese{\s} Maelstrome{\s} der L\"ufte.

Robur wu{\ss}te da{\s}. Er wu{\ss}te auch, da{\ss} e{\s} rathsam war,
einem Cyclon zu entfliehen, indem er au{\s} dem Bereiche seiner
Anziehung nach h\"oheren Luftschichten aufstieg. Bi{\s}her hatte er
da{\s} immer vermocht. Aber e{\s} war keine Stunde, vielleicht keine
Minute mehr zu verlieren.

In der That wuch{\s} die Gewalt de{\s} Sturme{\s} zusehend{\s}. Die
an ihren K\"ammen zerrissenen Wellen trugen einen wei{\ss}lichen
Staub \"uber die Meere{\s}fl\"ache hin. E{\s} war auch zu erkennen,
da{\ss} der Cyclon beim Fortschreiten mit rasender Schnelligkeit sich
den Polargebieten n\"ahern mu{\ss}te.

{\glqq}Hinauf! befahl Robur.

-- Hinauf!{\grqq} wiederholte Tom Turner.

Dem Aeronef wurde die \"au{\ss}erste Auftrieb{\s}kraft ertheilt und
er erhob sich in schiefer Richtung, al{\s} steige er eine schiefe
Ebene empor, die sich nach S\"udwesten hin senkte.

Da fiel der Barometer noch weiter -- die Quecksilbers\"aule sank
schnell um acht, dann um zw\"olf Millimeter. Pl\"otzlich h\"orte die
aufsteigende Bewegung de{\s} {\glqq}Albatro{\s}{\grqq} vollst\"andig
auf.

Wa{\s} veranla{\ss}te diesen Halt? Offenbar war e{\s} der Druck der
Luft infolge einer sehr starken Str\"omung, die von oben nach unten
zu stattfand und den Widerstand seine{\s} Angriff{\s}punkte{\s}
herabsetzte.

Wenn ein Dampfer einem Strome entgegenf\"ahrt, erzeugt seine Schraube
eine um so weniger wirksame Arbeit, je schneller da{\s} str\"omende
Wasser zwischen ihren Fl\"ugeln hindurchflie{\ss}t. Dann bleibt er
zur\"uck und da{\s} kann so weit gehen, da{\ss} er mit der Str\"omung
zur\"uckgeht. In dieser Lage befand sich jetzt der
{\glqq}Albatro{\s}{\grqq}.

Robur gab seine Sache aber damit noch nicht auf. Seine mit
erstaunlichster Uebereinstimmung arbeitenden Schrauben wurden in die
schnellstm\"ogliche Umdrehung versetzt; doch unwiderstehlich von dem
Cyclon angezogen, konnte der Apparat ihm nicht entgehen. W\"ahrend
kurzer Stillen stieg er wieder etwa{\s} in die H\"ohe. Dann zog ihn
der schwerere Luftdruck auf'{\s} Neue hernieder und er sank, wie ein
Schiff im Untergehen. Und konnte man da{\s} nicht ein Untergehen im
Luftmeere nennen, inmitten einer Nacht, welche die Blendlichter
de{\s} {\glqq}Albatro{\s}{\grqq} nur in geringem Umfange
unterbrachen?

Wenn die Kraft de{\s} Cyclon{\s} immer zunahm, wurde der
{\glqq}Albatro{\s}{\grqq} gewi{\ss} bald zum unlenkbaren Strohhalm,
der von den Wirbeln hinweggetragen wurde, welche B\"aume entwurzeln,
D\"acher abdecken und oft ganze Mauern umwerfen.

Robur und Tom konnten sich nur noch durch Zeichen verst\"andlich
machen. An die Reeling geklammert, fragten sich Onkel Prudent und
Phil Evan{\s}, ob da{\s} schauerliche Meteor nicht f\"ur sie
eintreten und den Aeronef mit seinem Erfinder, und mit dem Erfinder
da{\s} ganze Geheimni{\ss} der Erfindung vernichten w\"urde.

Da e{\s} nun dem {\glqq}Albatro{\s}{\grqq} nicht gelang, sich in
lothrechter Richtung diesem Cyclon zu ent\/ziehen, blieb ihm nur noch
der eine Au{\s}weg, den verh\"altni{\ss}m\"a{\ss}ig stilleren
Mittelpunkt de{\s}selben aufzusuchen, wo er mehr Herr seiner
Man\"over war. Gewi{\ss}; doch um zu jenem vorzudringen, mu{\ss}te er
die Krei{\s}str\"ome \"uberwinden, die ihn an ihren Peripherien mit
fort\/zerrten. Besa{\ss} er wirklich genug mechanische Kr\"afte, sich
diesen zu entrei{\ss}en?

Pl\"otzlich barsten jetzt die Wolken \"uber ihm; die D\"unste
verdichteten sich zu einem furchtbaren Platzregen.

E{\s} war um zwei Uhr Morgen{\s}. Der um zw\"olf Millimeter auf- und
abschwankende Barometer war bi{\s} auf 709 gefallen, wobei
allerding{\s} die H\"ohe, welche der Aeronef \"uber dem Meere
einnahm, in Rechnung zu ziehen ist.

Seltsamer Weise hatte sich dieser Cyclon au{\ss}erhalb der Zonen, die
er sonst durchzieht, gebildet, d.~h. zwischen dem 30. Grade
n\"ordlicher und dem 27. Grade s\"udlicher Breite. Vielleicht
erkl\"art sich hierdurch, warum dieser Wirbelsturm sehr bald in einen
gew\"ohnlichen, ziemlich geradlinig verlaufenden \"uberging. Doch
welcher Orkan w\"uthete daf\"ur! Der Windsto{\ss} von Connecticut am
22. M\"arz 1882 h\"atte ihm etwa verglichen werden k\"onnen, dessen
Geschwindigkeit hundertsechzehn Meter in der Secunde, d.~h. \"uber
hundert Meilen in der Stunde, erreichte.

E{\s} blieb jetzt also nicht{\s} \"ubrig, al{\s} nach r\"uckw\"art{\s}
zu entfliehen, wie ein Schiff vor dem Sturm, oder sich vielmehr von
dieser Str\"omung mit forttragen zu lassen, die der
{\glqq}Albatro{\s}{\grqq} nicht \"uberwinden und au{\s} der er sich
nicht befreien konnte. Doch wenn er dieser ihm aufgezwungenen
Stra{\ss}e folgte, floh er nach dem S\"uden hin und wurde nach den
Polargebieten verschlagen, welche Robur hatte vermeiden wollen. Doch
da er nicht mehr Herr seiner Bewegungen war, mu{\ss}te er ja
hingehen, wohin der Orkan ihn trug.

Tom Turner hielt noch immer getreulich am Steuerruder au{\s}. E{\s}
bedurfte aller seiner Gewandtheit, um nicht immer von einem Bord zum
anderen geschleudert zu werden.

Mit den ersten Stunden de{\s} Tage{\s} -- wenn man den schwachen
Schein, der den Horizont f\"arbte, so nennen konnte -- hatte der
{\glqq}Albatro{\s}{\grqq} vom Cap Horn her f\"unfzehn Breitengrade
hinter sich gelassen, d.~h. \"uber vierhundert Meilen, und er
\"uberschritt nun den Polarkrei{\s}.

Hier dauert die Nacht im Monat Juli noch neunzehn Stunden lang, die
kalte Sonnenscheibe erscheint nur schwach leuchtend am Horizont, um
fast sogleich wieder zu verschwinden. Am Pole selbst verl\"angert
sich diese Nacht bi{\s} auf hundertneunundsiebzig volle Tage.
Alle{\s} deutete darauf hin, da{\ss} sich der {\glqq}Albatro{\s}{\grqq}
wie in einen Abgrund in dieselbe st\"urzen m\"usse.

An diesem Tage h\"atte eine Beobachtung, wenn eine solche m\"oglich
gewesen w\"are, die s\"udliche Breite von 66 Grad 40 Minuten ergeben.
Der Aeronaut befand sich also nun vierzehnhundert Meilen vom
antarktischen Pol.

Unwiderstehlich nach diesem sonst unerreichbaren Punkt der Erdkugel
hingezogen, {\glqq}verzehrte{\grqq} seine Geschwindigkeit so zu sagen
seine ganze Schwere, obwohl diese infolge der Abplattung der Erde am
Pol hier eine noch gr\"o{\ss}ere war. Seine Auftrieb{\s}schrauben
konnten sich da{\s} ja wohl gefallen lassen. Bald wurde die Gewalt
de{\s} Sturme{\s} eine so ungeheure, da{\ss} Robur die Umdrehung{\s}zahl
der Propeller auf ein Minimum zu reduciren beschlo{\ss}, um diese vor
ernsten Besch\"adigungen zu sch\"utzen und doch ein wenig bei der
geringsten m\"oglichen eigenen Geschwindigkeit durch da{\s}
Steuerruder wirken zu k\"onnen.

Inmitten dieser Gefahren ertheilte der Ingenieur seine Befehle mit
gr\"o{\ss}ter Kaltbl\"utigkeit, und die Mannschaft gehorchte ihm,
al{\s} ob die Seele de{\s} Chef{\s} auch in ihr lebte.

Onkel Prudent und Phil Evan{\s} hatten da{\s} Verdeck, wo sie
\"ubrigen{\s} ohne alle Schwierigkeit verweilen konnten, nicht einen
Augenblick verlassen. Die Luft bot ja keinen oder nur sehr schwachen
Widerstand. Der Aeronef befand sich eben in derselben Lage, wie jeder
Aerostat, der sich mit dem Fluidum, in dem er ganz eintaucht,
fortbewegt.

Da{\s} s\"udliche Polargebiet umfa{\ss}t der gew\"ohnlichen Angabe
nach eine Fl\"ache von viereinhalb Millionen (englische)
Quadratmeilen, doch wei{\ss} man nicht, ob da{\s}selbe ein Festland,
einen Archipel oder nur ein Meer enth\"alt, dessen Ei{\s} selbst
w\"ahrend der langen Sommer{\s}zeit nicht zum Schmelzen kommt.
Bekannt ist dagegen, da{\ss} der S\"udpol noch k\"alter ist, al{\s}
der Nordpol, eine Erscheinung, welche von der Stellung der Erde in
ihrer Bahn w\"ahrend de{\s} Winter{\s} der antarktischen Region
abgeleitet wird.

W\"ahrend de{\s} Tage{\s} trat kein Anzeichen ein, da{\ss} der Sturm
abnehmen werde. Der {\glqq}Albatro{\s}{\grqq} gelangte unter den 75.
Grad westlicher L\"ange nach dem Polargebiete. Wer h\"atte wissen
k\"onnen, unter welchem Meridian er wieder au{\s} demselben
herau{\s}treten sollte?

Je mehr er nach S\"uden hinabkam, desto mehr verminderte sich die
Tage{\s}l\"ange. Binnen Kurzem mu{\ss}te er sich in der
fortw\"ahrenden Nacht befinden, welche nur vom Mondschein oder von
dem schwachen Leuchten der S\"udlichter erhellt wird. Jetzt war aber
Neumond, und die Gef\"ahrten Robur'{\s} liefen Gefahr, gar nicht{\s}
von jenen Gegenden zu sehen, deren Geheimni{\ss} der Mensch noch
nicht zu entschleiern vermocht hat.

H\"ochst wahrscheinlich kam der {\glqq}Albatro{\s}{\grqq} nahe dem
Polarkreise \"uber einzelne schon bekannte Punkte hinweg, so im
Westen \"uber da{\s} Graham-Land, welche{\s} Bi{\s}co\"e 1832
entdeckt hat, und \"uber da{\s} Loui{\s} Philipp-Land, da{\s} Dumont
d'Urville al{\s} \"au{\ss}ersten Au{\s}l\"aufer de{\s} unbekannten
Festlande{\s} 1838 entdeckte.

An Bord litt man zwar nicht besonder{\s} von der K\"alte, welche
nicht so arg war, al{\s} man eigentlich f\"urchten mu{\ss}te. Der
Orkan schien eine Art Golfstrom der Luft zu bilden, der eine gewisse
Menge W\"arme mit sich f\"uhrte.

Wie bedauerlich war e{\s}, da{\ss} diese ganze Gegend in
Finsterni{\ss} lag! Hierzu kommt noch, da{\ss} selbst bei vollem
Mondschein jede Beobachtung sehr beschr\"ankt war, denn zu dieser
Jahre{\s}zeit bedeckt eine ungeheure Schneelage und ein dicker
Ei{\s}panzer die ganze Oberfl\"ache de{\s} Polargebiete{\s}. Man
gewahrt dann nicht einmal jenen {\glqq}Blink{\grqq} der Ei{\s}massen,
den wei{\ss}lichen Schein, der sich an dem dunklen Horizonte nicht
widerspiegeln kann. Wie h\"atte Jemand unter diesen Umst\"anden die
Gestalt von L\"andern, die Au{\s}dehnung der Meere oder die
Vertheilung von Inseln zu erkennen vermocht? Wie h\"atte man sich
\"uber da{\s} hydrographische Netz de{\s} Lande{\s} unterrichten,
oder selbst dessen orographische Anordnung aufnehmen k\"onnen, da
jetzt alle H\"ugel, alle Berge mit den Ei{\s}bergen und dem Packeise
zu einer einzigen Masse verschmolzen?

Kurz vor Mitternacht erhellte ein S\"udpolarlicht einmal die tiefe
Finsterni{\ss}. Mit seinen silbernen Au{\s}l\"aufern, seinen weit
hinau{\s}reichenden Strahlen, bildete da{\s} Meteor die Gestalt
eine{\s} ungeheuren F\"acher{\s}, der etwa die H\"alfte de{\s}
Himmel{\s} einnahm. Die letzten elektrischen Effluvien de{\s}selben
verloren sich am s\"udlichen Kreuz, dessen vier Sterne im Zenith
brannten. Diese Erscheinung war von wirklich unvergleichlicher
Gro{\ss}artigkeit und ihre Helligkeit reichte hin, einen Ueberblick
\"uber diese in grenzenlose{\s} Wei{\ss} verh\"ullte Gegend zu
gew\"ahren.

E{\s} versteht sich von selbst, da{\ss} der Compa{\ss} in diesen, dem
magnetischen S\"udpol so nahe gelegenen Gebieten g\"anzlich gest\"ort
erschien und \"uber die eingehaltene Richtung keinerlei Aufkl\"arung
geben konnte. Die Inclination der Nadel wurde aber gelegentlich eine
so bedeutende, da{\ss} Robur annehmen mu{\ss}te, \"uber diesen
magnetischen Pol, der ziemlich genau im achtundsiebenzigsten Meridian
zu suchen ist, hinweggekommen zu sein.

Und sp\"ater, gegen ein Uhr de{\s} Morgen{\s}, rief er nach
Beobachtung de{\s} Winkel{\s}, den die Nadel mit der Verticalen
machte, laut:

{\glqq}Jetzt ist der S\"udpol unter unseren F\"u{\ss}en!{\grqq}

Wohl sah man eine ganz wei{\ss}e Fl\"ache, aber nicht{\s} verrieth,
wa{\s} sie unter ihrem Ei{\s}panzer bergen mochte.

Da{\s} S\"udpolarlicht erlosch bald nachher, und jener ideale Punkt,
in dem sich alle Meridiane kreuzen, ist noch immer erst zu entdecken.

Hatten Onkel Prudent und Phil Evan{\s} die Absicht, den Aeronef und
Alle, die er trug, in der geheimni{\ss}vollsten Ein\"ode zu begraben,
so war jetzt die beste Gelegenheit dazu. Wenn sie e{\s} doch nicht
thaten, so lag e{\s} daran, da{\ss} ihnen die dazu nothwendige
Sprengmaschine mangelte.

Indessen raste der Orkan mit einer solchen Wuth weiter, da{\ss} der
{\glqq}Albatro{\s}{\grqq}, wenn er auf seinem Wege einen Berg
getroffen h\"atte, daran unbedingt ebenso zerschellt w\"are, wie ein
Schiff, da{\s} auf eine felsige K\"uste geworfen wird.

Augenblicklich vermochte er sich n\"amlich ebenso wenig horizontal
fort\/zubewegen, wie er Herr \"uber sein Auf- und Absteigen war.

Einzelne Gipfel aber erheben sich bekanntlich auch in den
antarktischen Gebieten. Jeden Augenblick war ein Zusammensto{\ss}
m\"oglich, der die Vernichtung de{\s} ganzen Apparate{\s} h\"atte
herbeif\"uhren m\"ussen.

Eine solche Katastrophe schien um so mehr zu f\"urchten, al{\s} der
Wind, der nach dem Meridian 0 mehr zur\"uckging, weiter nach Osten
umschlug. Schon zeigten sich auch, etwa hundert Kilometer vom
{\glqq}Albatro{\s}{\grqq} entfernt, zwei leuchtende Punkte.

E{\s} waren da{\s} die beiden Vulcane, welche zu dem gewaltigen
Gebiete der Ro{\ss}-, Erebu{\s}- und Terrorberge geh\"oren.

Sollte der {\glqq}Albatro{\s}{\grqq} in den Flammen gleich einem
riesigen Schmetterling verbrennen?

E{\s} war eine Stunde voll entsetzlicher Angst; der eine der Vulcane,
der Erebu{\s}, schien ordentlich auf den Aeronef, der sich au{\s} dem
Bett de{\s} Vulcan{\s} nicht befreien konnte, lo{\s}zust\"urzen ...
Sein Flammenb\"uschel wuch{\s} zusehend{\s}, ein Feuernetz versperrte
den Weg, da{\s} die Luft weithin erleuchtete. Die an Bord jetzt
deutlich sichtbaren Gestalten nahmen ein halb teuflische{\s}
Au{\s}sehen an. Alle erwarteten regung{\s}lo{\s}, ohne einen Schrei,
ohne mit den Mu{\s}keln zu zucken, die entsetzliche Minute, in der
dieser Hochofen sie mit seinen Flammen umh\"ullen w\"urde.

Der Orkan aber, der den {\glqq}Albatro{\s}{\grqq} mit sich
fortri{\ss}, rettete ihn auch vor dieser schrecklichen Katastrophe.
Die von dem Sturm niedergedr\"uckten Flammen de{\s} Erebu{\s} gaben
ihm den gef\"ahrlichen Weg frei, und inmitten eine{\s} Hagel{\s} von
Lavast\"ucken, welche durch die centrifugale Bewegung der
Auftrieb{\s}schrauben gl\"ucklicher Weise weggeschleudert wurden, kam
er gl\"ucklich \"uber diesen in voller Eruption begriffenen Krater
hinweg.

Eine Stunde sp\"ater verdeckte schon der Horizont die beiden
colossalen Flammen, welche da{\s} Ende der Welt w\"ahrend der langen
Polarnacht erleuchten.

Um zwei Uhr Morgen{\s} kam man an der Insel Ballery und zwar am Rande
der K\"uste der Entdeckung vor\"uber, ohne diese jedoch erkennen zu
k\"onnen, da auch sie mit den Polarl\"andern durch feste{\s} Ei{\s}
verkettet war.

Mit dem Au{\s}tritt au{\s} dem Polarkreise, den der
{\glqq}Albatro{\s}{\grqq} unter dem f\"unfundsiebenzigsten Meridian
durchschnitten, trug ihn der Orkan \"uber die Packei{\s}massen und
die Ei{\s}berge hinweg, an welchen er sich hundertmal zu
zertr\"ummern drohte. Er war eben nicht mehr in der Hand seine{\s}
Steuermanne{\s}, sondern nur in der Hand Gotte{\s}, und Gott ist ja
der beste Pilot.

Der Aeronef folgte nun wieder dem Meridian von Pari{\s}, der mit dem,
unter welchem er die antarktische Welt betreten, einen Winkel von 105
Grad bildet.

Endlich, jenseit{\s} de{\s} 60. Breitengrade{\s}, schien die Kraft
de{\s} Orkan{\s} zu erlahmen. Seine Schnelligkeit nahm merklich ab.
Der {\glqq}Albatro{\s}{\grqq} wurde wieder mehr seiner selbst Herr.
Ferner kam er jetzt, wa{\s} eine gro{\ss}e Erleichterung gew\"ahrte,
wieder in die erleuchteten Theile der Erdkugel, und um acht Uhr
Morgen{\s} brach der Tag an.

Nachdem Robur und die Seinen dem Wirbelsturm de{\s} Cap Horn
gl\"ucklich entgangen waren, hatten sie nun auch diesen Orkan
\"uberstanden. Sie waren \"uber da{\s} ganze S\"udpolargebiet weg,
nachdem sie binnen neunzehn Stunden gegen siebentausend Kilometer
zur\"uckgelegt, wieder nach dem Stillen Ocean getrieben worden, und
da sie zu einer Meile nur eine Minute gebraucht hatten, war ihre
Schnelligkeit doppelt so gro{\ss} gewesen, al{\s} sie der
{\glqq}Albatro{\s}{\grqq} unter gew\"ohnlichen Umst\"anden h\"atte
entwickeln k\"onnen.

Infolge der St\"orung de{\s} Magneti{\s}mu{\s} seiner Compa{\ss}nadel
im Polargebiete, wu{\ss}te Robur nun aber nicht mehr, wo er sich
befand. Er mu{\ss}te also warten, bi{\s} die Sonne unter hinreichend
g\"unstigen Verh\"altnissen schien, um eine directe Beobachtung zu
gestatten. Ungl\"ucklicher Weise bedeckten dichte Wolken an diesem
Tage den Himmel und die Sonne wurde \"uberhaupt nicht sichtbar.

Da{\s} war f\"ur Alle desto betr\"ubender, weil die beiden
Treibschrauben w\"ahrend de{\s} Sturme{\s} einige Besch\"adigungen
erlitten hatten.

Sehr verstimmt durch diesen Unfall, konnte Robur w\"ahrend de{\s}
ganzen Tage{\s} nur mit stark verminderter Geschwindigkeit
weiterfahren. Al{\s} er \"uber den Antipoden von Pari{\s} schwebte,
legte er nur sech{\s} Meilen in der Stunde zur\"uck, denn er
mu{\ss}te sich wohl h\"uten, die Besch\"adigungen zu verschlimmern.
Versagten seine beiden Treibschrauben etwa ganz vollst\"andig den
Dienst, so wurde die Lage de{\s} Aeronef{\s} \"uber dem ungeheuren
Stillen Ocean eine sehr mi{\ss}liche. Der Ingenieur fragte sich auch
schon, ob er die n\"othigen Au{\s}besserungen nicht sollte an Ort und
Stelle vornehmen lassen, um die Fortsetzung der Reise zu sichern.

Am Morgen de{\s} 27. Juli wurde da ein Land im Norden gemeldet.

Man erkannte bald, da{\ss} da{\s} eine Insel war; doch welche von den
Tausenden, die im Pacifischen Ocean verstreut liegen?
Nicht{\s}destoweniger beschlo{\ss} Robur hier Halt zu machen, doch
ohne auf die Erde selbst zu gehen. Seiner Ansicht nach mu{\ss}te ein
Tag hinreichen, die Havarien au{\s}zubessern, und er meinte dann
denselben Abend wieder weiter fahren zu k\"onnen.

Der Wind hatte sich -- ein g\"unstiger Umstand zur Au{\s}f\"uhrung
jene{\s} Vorhaben{\s} -- fast vollst\"andig gelegt. Da er nun
anhalten sollte, konnte der {\glqq}Albatro{\s}{\grqq} wenigsten{\s}
nicht nach unbekannten Gegenden verschlagen werden.

Man lie{\ss} also ein mit einem Anker versehene{\s} hundertf\"unfzig
Fu{\ss} lange{\s} Kabel von dem Luftschiff herunter. Al{\s} der
Aeronef an den Rand der Insel kam, fa{\ss}te der Anker an den ersten
Klippen de{\s}selben und legte sich schnell zwischen zwei Felsen
fest. Da{\s} Kabel spannte sich unter der Wirkung der
Auftrieb{\s}schrauben straff an und der {\glqq}Albatro{\s}{\grqq}
blieb unbeweglich, wie ein Schiff, da{\s} am Strande festgelegt
wurde.

E{\s} war da{\s} erste Mal, da{\ss} er seit der Abfahrt au{\s}
Philadelphia \"uberhaupt mit der Erde in Ber\"uhrung kam.



\newpage\begin{center}\label{kap15}
{\large \begin{antiqua}XV.\end{antiqua}\\
Worin Dinge vorgehen, deren Schilderung sich gewi{\ss} der M\"uhe
lohnt.\\\bigskip}
\end{center}



Al{\s} der {\glqq}Albatro{\s}{\grqq} noch in gen\"ugend hoher Luftschicht
schwebte, konnte man erkennen, da{\ss} diese Insel von mittlerer
Gr\"o{\ss}e war. Doch welcher Breitengrad durchschnitt wohl dieselbe?
Bi{\s} zu welcher Mittag{\s}linie war man gelangt? War jene eine
Insel de{\s} Stillen Ocean{\s}, Austral-Asien{\s} (Neu-Holland{\s})
oder de{\s} Indischen Meere{\s}? Da{\s} blieb so lange unbestimmt,
bi{\s} Robur sein Besteck gemacht hatte. Obgleich dieser nun nicht im
Stande gewesen war, Compa{\ss}angaben zu Rathe zu ziehen, hatte er
doch Ursache, zu glauben, da{\ss} er sich \"uber dem Stillen Ocean
befinde. Sobald nur die Sonne erschien, mu{\ss}ten die Umst\"ande zu
einer genauen Beobachtung h\"ochst g\"unstige sein.

Von dieser H\"ohe -- von etwa f\"unfhundert Fu{\ss} -- au{\s} zeigte
sich die, gegen f\"unfzehn (englische) Meilen im Durchmesser haltende
Insel in der Form eine{\s} dreispitzigen Seestern{\s}.

Vor deren S\"udspitze tauchte noch ein Eiland auf, an da{\s} sich ein
Felsengewirr anschlo{\ss}. Am Strande verrieth sich kein von Ebbe und
Fluth zur\"uckgebliebene{\s} Merkmal, wa{\s} die Ansicht Robur'{\s}
bez\"uglich seiner augenblicklichen Lage zu best\"arken schien, da
die Gezeiten im Stillen Ocean fast gleich Null sind.

An der Nordweste erhob sich ein nahezu kegelf\"ormiger Berg, dessen
H\"ohe auf zw\"olfhundert Fu{\ss} zu sch\"atzen war.

Von einem Eingeborenen sah man nicht{\s}; vielleicht wohnten solche
jedoch am entgegengesetzten Ufer. Jedenfall{\s} hatte der Aeronef,
wenn sie diesen \"uberhaupt bemerkten, sie erschreckt und
veranla{\ss}t, sich zu verbergen oder zu entfliehen.

Der {\glqq}Albatro{\s}{\grqq} hatte die S\"udostspitze der Insel
ber\"uhrt. Unfern derselben schl\"angelte sich in beschr\"ankter
Bucht ein Fl\"u{\ss}chen durch die Felsen. Weiterhin zeigten sich
gewundene Th\"aler mit verschiedenen Baumarten und zahlreichem wilden
Gefl\"ugel, vorz\"uglich Rebh\"uhner und Trappen. Wenn die Insel
nicht bewohnt war, so erschien sie danach also mindesten{\s}
bewohnbar. Unzweifelhaft h\"atte Robur hier an'{\s} Land gehen
k\"onnen, und wenn er e{\s} doch nicht that, so kam da{\s} nur daher,
da{\ss} der sehr unebene Boden ihm keinen geeigneten Platz zur
Auflagerung de{\s} Aeronef{\s} zu bieten schien.

Vor dem Wiederaufsteigen lie{\ss} der Ingenieur die nothwendigsten
Au{\s}besserungen vornehmen, welche er im Laufe de{\s} Tage{\s}
beendet zu sehen hoffte. Die in vollkommen gutem Zustande
befindlichen Schwebeschrauben hatten selbst gegen\"uber der
gr\"o{\ss}ten Heftigkeit de{\s} Orkan{\s} vortrefflich functionirt,
da letzterer, wie e{\s} sich zeigte, die Arbeit derselben sogar
erleichterte. Augenblicklich war nur die H\"alfte de{\s}
Auftrieb{\s}mechani{\s}mu{\s} in Th\"atigkeit, doch hinreichend viel,
um da{\s} lothrecht am Ufer befestigte Tau gespannt zu erhalten.

Dagegen hatten die beiden eigentlichen Propeller gelitten, und zwar
mehr, al{\s} Robur selbst vorau{\s}setzte. Mindesten{\s} mu{\ss}ten
deren Schaufeln wieder aufgerichtet und da{\s} Zahngetriebe, durch
welche{\s} sie die Drehbewegung erhielten, au{\s}gebessert werden.

Die Mannschaft besch\"aftigte sich unter der Leitung Robur'{\s} und
Tom Turner'{\s} mit der vorderen Schraube. E{\s} erschien da{\s}
vortheilhafter f\"ur den Fall, da{\ss} der {\glqq}Albatro{\s}{\grqq}
au{\s} irgend welchem Grunde gen\"othigt sein k\"onnte, vor
v\"olliger Vollendung der Arbeit wieder weiter zu treiben und man
schon mit diesem Propeller allein n\"othigenfall{\s} eine gen\"ugende
Fahrgeschwindigkeit erreichen konnte.

Inzwischen hatten sich Onkel Prudent und sein College, die vorher auf
der Plattform umherspaziert waren, auf dem Hinterdeck niedergelassen.

Frycollin zeigte sich jetzt merkw\"urdig beruhigt. Welcher
Unterschied, nur noch hundertf\"unfzig Fu{\ss} \"uber dem Erdboden zu
schweben!

Die Arbeiten wurden nur unterbrochen, al{\s} die Erhebung der Sonne
\"uber dem Horizonte zun\"achst einen Stundenwinkel zu messen und
dann zur Zeit ihrer Culmination die Mittag{\s}linie de{\s} Ort{\s} zu
bestimmen gestattete.

Al{\s} Resultat der mit gr\"o{\ss}ter Sorgfalt au{\s}gef\"uhrten
Beobachtung ergab sich da eine

\begin{quote}
L\"ange von 176${}^{\circ}$ 17' westlich von Greenwich, \\
Breite von 43${}^{\circ}$ 37' s\"udlich vom Aequator.
\end{quote}

Dieser Punkt auf der Karte entsprach der Insel Chatam und dem Eilande
Viff, welche Gruppe gew\"ohnlich mit dem Namen der Brougthon-Inseln
bezeichnet wird. Dieselbe findet sich etwa f\"unfzehn Grade \"ostlich
von Tawai-Pomanu, der s\"udlich gelegenen Inselh\"alfte
Neuseeland{\s} im S\"uden de{\s} Stillen Ocean{\s}.

{\glqq}Da{\s} stimmt nahezu mit meiner Vorau{\s}setzung \"uberein,
sagte Robur zu Tom Turner.

-- Und wir befinden un{\s} demnach~...?

-- Sech{\s}undvierzig Grade s\"udlich der Insel X, da{\s} hei{\ss}t
gegen zweitausendachthundert Seemeilen von dieser entfernt.

-- Ein Grund mehr, unsere Propeller wieder in Ordnung zu bringen,
antwortete der Obersteuermann. Bei der Fahrt dahin k\"onnten wir
leicht widrige Winde antreffen, und mit R\"ucksicht auf unsere jetzt
nur geringen Proviantvorr\"athe ist e{\s} von Wichtigkeit, die Insel
X so schnell al{\s} m\"oglich wieder anzulaufen.

-- Gewi{\ss}, Tom, und ich hoffe auch, schon in der Nacht wieder
aufzubrechen, schlimmsten Fall{\s} mit einer einzigen Triebschraube,
w\"ahrend die zweite dann unterweg{\s} wieder in Ordnung gebracht
w\"urde.

-- Master Robur, fragte da Tom Turner, aber die beiden Herren und
deren Diener~...?

-- Nun, Tom Turner, erwiderte der Ingenieur, h\"atten sie dar\"uber
sich zu beklagen, Colonisten der Insel X zu werden?{\grqq}

Wa{\s} war denn eigentlich diese Insel X? -- Eine in dem grenzenlosen
Stillen Ocean verlorene Insel zwischen dem Aequator und dem
Wendekrei{\s} de{\s} Krebse{\s}; eine Insel, welche da{\s}
algebraische Zeichen, da{\s} Robur zu ihrem Namen erw\"ahlt hatte,
vollkommen rechtfertigte. Sie entstieg dem weiten Meere der Marquisen
au{\ss}erhalb aller Wege de{\s} interoceanischen Verkehr{\s}. Da
hatte Robur seine kleine Colonie begr\"undet, da rastete der
{\glqq}Albatro{\s}{\grqq}, wenn er von seinem Fluge erm\"udet war,
und da versah er sich auch mit allem Nothwendigen f\"ur seine fast
unaufh\"orlichen Reisen. Auf dieser Insel X hatte Robur, der \"uber
reichliche Hilf{\s}mittel verf\"ugte, eine Werft errichten und seinen
Aeronef erbauen k\"onnen. Hier konnte er denselben au{\s}bessern,
selbst ganz neu wiederherstellen. Seine Magazine strotzten von
Materialien, Nahrung{\s}mitteln und Vorr\"athen aller Art, welche
hier mit Unterst\"utzung der gegen f\"unfzig K\"opfe z\"ahlenden
Einwohnerschaft aufgeh\"auft wurden.

Al{\s} Robur vor wenig Tagen da{\s} Cap Horn umschiffte, war e{\s}
seine Absicht gewesen, sich schr\"ag \"uber den Stillen Ocean nach
der Insel X zu begeben. Da hatte aber die Cyklone den
{\glqq}Albatro{\s}{\grqq} in ihren Wirbel gerissen und nachher der
wilde Orkan ihn nach s\"udlicheren Zonen verschlagen. Kurz, er war
dadurch wieder mehr in seine urspr\"ungliche Fahrtrichtung gedr\"angt
worden, und abgesehen von den Besch\"adigungen seiner Triebschrauben,
w\"are dieser Verz\"ogerung keine besondere Bedeutung beizumessen
gewesen.

Jetzt wollte man sich also nach der Insel X zur\"uckbegeben, doch
war, wie der Obersteuermann Tom Turner vorhergesagt hatte, der Weg
dahin ein recht weiter, und h\"ochst wahrscheinlich hatte man dabei
auch noch gegen widrige Winde anzuk\"ampfen. Jedenfall{\s} bedurfte
e{\s} de{\s} Aufwande{\s} aller mechanischen Kraft de{\s}
{\glqq}Albatro{\s}{\grqq}, um jene{\s} Ziel zur bestimmten Zeit zu
erreichen. Bei einigerma{\ss}en guter Witterung und bei der
gew\"ohnlichen Fahrtgeschwindigkeit h\"atte da{\s} sonst nur drei
bi{\s} vier Tage beansprucht.

De{\s}halb hatte sich Robur auch zum Anlegen an der Insel Chatam
entschlossen, wo er wenigsten{\s} die vordere Triebschraube unter
g\"unstigeren Verh\"altnissen wieder au{\s}bessern konnte. Er
f\"urchtete dann nicht mehr, selbst im Fall sich eine ganz
entgegengesetzte Brise erhob, nach S\"uden hin verschlagen zu werden,
wenn er nach Norden zu fahren wollte. Mit Einbruch der Nacht war
diese Reparatur vollendet. Er traf also Anstalt, seinen Anker zu
lichten. Sollte dieser zwischen den Uferfelsen gar zu fest
eingegriffen haben, so war er entschlossen, einfach da{\s} Ankertau
zu kappen und den Flug gegen den Aequator zu beginnen.

E{\s} liegt auf der Hand, da{\ss} da{\s} die einfachste Methode war
und entschieden auch die beste, um schnell fort\/zukommen, und sie
wurde denn auch sogleich verfolgt.

Im Bewu{\ss}tsein, da{\ss} jetzt keine Zeit mehr zu verlieren sei,
ging die Mannschaft de{\s} {\glqq}Albatro{\s}{\grqq} entschlossen an
diese Arbeit.

Und w\"ahrend die Anderen am Vordertheil de{\s} Aeronef besch\"aftigt
waren, f\"uhrten Onkel Prudent und Phil Evan{\s} eine Unterhaltung,
deren Folgen von ganz au{\ss}erordentlicher Bedeutung sein sollten.

{\glqq}Phil Evan{\s}, sagte Onkel Prudent, sind Sie gleich mir
entschlossen, da{\s} Leben zum Opfer zu bringen?

-- Ja, gleich Ihnen!

-- Und noch einmal, e{\s} liegt auf der Hand, da{\ss} wir von diesem
Robur nicht{\s} zu hoffen haben.

-- Nicht{\s}!

-- Nun wohl, Phil Evan{\s}, mein Entschlu{\ss} ist gefa{\ss}t. Da der
{\glqq}Albatro{\s}{\grqq} noch heute sp\"at Abend{\s} abfahren soll,
wird die Nacht nicht vergehen, ohne da{\ss} unser Werk vollbracht
w\"are. Wir werden dem Vogel de{\s} Ingenieur Robur die Fl\"ugel
zerbrechen. Diese Nacht wird er mitten in der Luft zersprengt!

-- Haben Sie auch alle{\s} dazu N\"othige in Bereitschaft?

-- Gewi{\ss}. Letztvergangene Nacht, al{\s} sich Robur und seine
Leute nur um die Rettung de{\s} Aeronef{\s} bem\"uhten, gelang e{\s}
mir, in die Munition{\s}kammer zu schleichen und eine Dynamitpatrone
mit\/zunehmen.

-- So lassen Sie un{\s} unverz\"uglich an'{\s} Werk gehen, Onkel
Prudent~...

-- Nein, erst heute am Sp\"atabend. Wenn e{\s} dunkel geworden ist,
ziehen wir un{\s} nach unserer Wohnung zur\"uck und Sie \"ubernehmen
die Wache, da{\ss} mich bei den Vorbereitungen Niemand
\"uberrascht.{\grqq}

Gegen sech{\s} Uhr speisten die beiden Genossen in hergebrachter
Weise. Zwei Stunden sp\"ater hatten sie sich nach ihrer Cabine
zur\"uckgezogen, al{\s} wollten sie sich f\"ur die letzte schlaflose
Nacht schadlo{\s} halten.

Weder Robur, noch irgend einer seiner Leute konnte im entferntesten
ahnen, welche Katastrophe den {\glqq}Albatro{\s}{\grqq} bedrohte.

Onkel Prudent aber dachte nach folgender Art zur Au{\s}f\"uhrung zu
schreiten:

Wie schon erw\"ahnt, war e{\s} ihm gelungen in die Munition{\s}kammer
einzudringen, welche in einer der unteren Rumpfabtheilungen de{\s}
Aeronef{\s} angelegt war. Dort hatte er sich einer gewissen Menge
Pulver{\s} und einer Patrone bem\"achtigt, die ganz mit denen
\"ubereinstimmte, deren sich der Ingenieur fr\"uher in Dahomey
bediente. Nach seiner Cabine zur\"uckgekehrt, hatte er die Patrone
sorgf\"altig versteckt, mit der der {\glqq}Albatro{\s}{\grqq}, wenn
er in der Nacht seinen Flug durch die L\"ufte wieder begonnen,
gesprengt werden sollte.

Eben jetzt besichtigte Phil Evan{\s} den von seinem Genossen
geraubten Explosion{\s}k\"orper.

Dieser bestand au{\s} einer l\"angeren H\"ulse, deren metallene Wand
etwa ein Kilogramm de{\s} explosiven Stoffe{\s} enthielt, also
vorau{\s}sichtlich eine hinreichende Menge, um den Aeronef
au{\s}einander zu rei{\ss}en und sein vielf\"altige{\s}
Steigschrauben-Getriebe zu zerst\"oren. Vernichtete ihn die Explosion
aber nicht mit einem Schlage, so mu{\ss}te der Sturz in die Tiefe
da{\s} Zerst\"orung{\s}werk vollenden. Die Form und Gr\"o{\ss}e der
Patrone beg\"unstigten e{\s} \"ubrigen{\s} au{\ss}erordentlich, diese
in einer Ecke der Cabine so anzubringen, da{\ss} sie die Plattform
unbedingt durchschlug und ihre Wirkung auch da{\s} Rippenwerk de{\s}
Rumpfe{\s} erreichte. Die Explosion konnte nun aber nur durch da{\s}
Z\"undh\"utchen, mit dem die Patrone au{\s}ger\"ustet war,
bewerkstelligt werden, da{\s} war der heiklichste Theil de{\s} ganzen
Vorhaben{\s}, denn diese{\s} Z\"undh\"utchen sollte nur nach
vorau{\s}berechneter Zeit in Brand gesetzt werden.

Onkel Prudent hatte sich den Verlauf folgenderma{\ss}en gedacht: Gleich
nach Vollendung der Reparaturarbeiten in der Vordertriebschraube
sollte der Aeronef den Weg nach Norden wieder aufnehmen: wenn
Obige{\s} aber geschehen war, lag die Wahrscheinlichkeit nahe,
da{\ss} Robur mit seinen Leuten nach dem Hinterdeck kommen w\"urde,
um auch die hintere Triebschraube wieder in guten Stand zu setzen.
Die Anwesenheit der gesammten Mannschaft an der N\"ahe der Cabine
aber konnte Onkel Prudent leicht bei seiner Th\"atigkeit st\"oren.
De{\s}halb hatte er sich zur Verwendung einer Lunte entschlossen, um
mittelst derselben die Explosion zu einer bestimmten Zeit eintreten
zu lassen.

Er sprach sich dar\"uber gegen Phil Evan{\s} mit folgenden Worten
au{\s}:

{\glqq}Al{\s} ich mir die Patrone holte, habe ich gleichzeitig auch
etwa{\s} Pulver mitgenommen. Mit diesem Pulver denke ich eine Lunte
herzustellen, deren L\"ange mit ihrer gew\"unschten Brenndauer in
Uebereinstimmung zu bringen sein wird und deren Ende ich in dem
Z\"undh\"utchen zu befestigen gedenke. Meine Absicht geht nun dahin,
dieselbe um Mitternacht anzuz\"unden und die Explosion zwischen drei
und vier Uhr fr\"uh erfolgen zu lassen.

-- Gut au{\s}gedacht!{\grqq} antwortete Phil Evan{\s}.

Die beiden Collegen waren, wie man hierau{\s} ersieht, schon dahin
gelangt, mit gr\"o{\ss}ter Kaltbl\"utigkeit da{\s} schreckliche
Vernichtung{\s}werk zu besprechen, durch da{\s} auch sie mit
untergehen sollten. Sie trugen eben eine solche Summe von Ha{\ss}
gegen Robur und seine Leute in sich, da{\ss} ihnen selbst die
Aufopferung de{\s} eigenen Leben{\s} nicht zu gro{\ss} erschien, nur
um den {\glqq}Albatro{\s}{\grqq} und Alle, die er mit durch die
L\"ufte f\"uhrte, mit einem Schlage zu vernichten. Zugegeben, da{\ss}
diese That ein sinnlose{\s}, ein verruchte{\s} Unterfangen war; nach
vollen f\"unf Wochen eine{\s} nie zum Au{\s}bruch gekommenen
Zorne{\s}, einer nie bes\"anftigten stillen Wuth lie{\ss}en sie sich
durch eine solche Kleinigkeit nicht mehr abhalten.

{\glqq}Und Frycollin? warf Phil Evan{\s} noch ein; steht un{\s}
da{\s} Recht zu, ohne ihn zu fragen, auch \"uber sein Leben zu
verf\"ugen?

-- Wir opfern ja auch da{\s} unsrige,{\grqq} entgegnete Onkel
Prudent.

E{\s} d\"urfte zweifelhaft sein, da{\ss} Frycollin da{\s} al{\s}
stichhaltigen Grund angesehen h\"atte.

Onkel Prudent ging also sofort an'{\s} Werk, w\"ahrend Phil Evan{\s}
vor dem Ruff Wache hielt.

Die Mannschaft war noch immer am Vordertheil besch\"aftigt und eine
Ueberraschung vorl\"aufig also kaum zu f\"urchten.

Onkel Prudent begann damit, eine geringe Menge Pulver zu Mehl zu
verreiben. Nachdem er da{\s}selbe leicht angefeuchtet, f\"ullte er
e{\s}, um eine Lunte zu erhalten, in einen engen, au{\s} Leinwand
herstellten Schlauch. Durch eine vorl\"aufige Probe \"uberzeugte er
sich, da{\ss} diese binnen zehn Minuten f\"unf Centimeter weit
verglimmte, bei der L\"ange von einem Meter also drei und einhalb
Stunden au{\s}reichen mu{\ss}te. Er l\"oschte die Lunte nun wieder
au{\s}, umwand sie fest mit Bindfaden und f\"uhrte da{\s} Ende
derselben in da{\s} Z\"undh\"utchen ein.

Alle{\s} da{\s} war, ohne den geringen Argwohn Anderer zu erwecken,
um zehn Uhr Abend{\s} vollendet.

Da trat Phil Evan{\s} wieder zu seinem Collegen in die Cabine.

W\"ahrend derselben Zeit war die Au{\s}besserung der vorderen
Triebschraube eifrig gef\"ordert worden; man hatte diese aber ganz
hereinnehmen m\"ussen, um die jetzt falsch gebogenen Fl\"ugel abheben
zu k\"onnen.

Weder Batterien, noch Accumulatoren, \"uberhaupt nicht{\s}, wa{\s}
zur Erzeugung der mechanischen Kraft de{\s} {\glqq}Albatro{\s}{\grqq}
geh\"orte, hatte durch die Wuth der Cyklone Schaden gelitten, und
jedenfall{\s} war noch f\"ur vier bi{\s} f\"unf Tage hinreichender
Kraftvorrath vorhanden.

E{\s} war schon Nacht geworden, al{\s} Robur und seine Leute ihre
Arbeit unterbrachen, ohne die vordere Triebschraube bi{\s}her wieder
an richtiger Stelle eingesetzt zu haben, da e{\s} noch einer etwa
dreist\"undigen Reparatur bedurfte, ehe dieselbe wieder functioniren
konnte. Nach kurzer R\"ucksprache mit Tom Turner entschied sich der
Ingenieur daf\"ur, seinen von der gehabten Anstrengung ersch\"opften
Leuten einige Erholung zu g\"onnen und auf den folgenden Morgen zu
verschieben, wa{\s} noch zu thun \"ubrig blieb. Uebrigen{\s} brauchte
man zu dieser, die peinlichste Sorgfalt erfordernden Arbeit die volle
Tage{\s}helle, w\"ahrend die Position{\s}laternen dazu nur
ungen\"ugende{\s} Licht h\"atten liefern k\"onnen.

Hiervon wu{\ss}ten nun Onkel Prudent und Phil Evan{\s} freilich
nicht{\s}. Nach den ihnen zu Ohren gekommenen Aeu{\ss}erungen
Robur'{\s} mu{\ss}ten sie vorau{\s}setzen, da{\ss} die vordere
Triebschraube vor Einbruch der Nacht schon wieder v\"ollig in Stand
gesetzt sei und der {\glqq}Albatro{\s}{\grqq} seine Fahrt nach Norden
unverz\"uglich angetreten habe. Sie hielten diesen also f\"ur
lo{\s}gel\"ost von der Insel, an der sein Anker ihn doch noch
festhielt. Dieser Umstand aber sollte der ganzen Angelegenheit eine
von ihnen gar nicht geahnte Wendung geben.

E{\s} war eine dunkle, monde{\s}lose Nacht, deren Finsterni{\ss}
schwere Wolken nur noch tiefer machten. Von S\"udwesten her wehte
dann und wann ein leichter Lufthauch; dieser bewegte aber den
{\glqq}Albatro{\s}{\grqq} nicht von der Stelle, sondern letzterer
hielt sich an seinem Anker still, dessen senkrecht gespannte{\s} Tau
ihn an die Erde fesselte.

In ihre Cabine eingeschlossen, wechselten Onkel Prudent und sein
College nur wenige Worte; sie lauschten auf da{\s} Schwirren der
Auftrieb{\s}schrauben, da{\s} jede{\s} andere Ger\"ausch an Bord
\"ubert\"onte, und erwarteten nun den Augenblick zum Handeln.

Kurz vor Mitternacht begann Onkel Prudent:

{\glqq}E{\s} ist nun Zeit!{\grqq}

Unter den Lagerst\"atten der Cabine befand sich ein schubladenartiger
Koffer, in den Onkel Prudent die mit der Lunte versehene
Dynamitpatrone gelegt hatte, damit die Lunte verglimmen konnte, ohne
sich durch auf\/f\"alligen Geruch oder etwaige{\s} Knistern zu
verrathen. Onkel Prudent z\"undete da{\s} freie Ende derselben an und
schob den Koffer wieder unter da{\s} Bett zur\"uck.

{\glqq}Nun nach dem Hinterdeck, sagte er, dort wollen wir
warten.{\grqq}

Beide traten herau{\s} und verwunderten sich nicht wenig, den
Steuermann nicht an seinem gewohnten Platze zu sehen.

Da bog sich Phil Evan{\s} \"uber da{\s} Deck hinau{\s}.

{\glqq}Der {\glqq}Albatro{\s}{\grqq} schwebt noch am n\"amlichen
Orte, sagte er leise. Die Arbeiten sind offenbar noch unvollendet. Er
hat nicht abfahren k\"onnen.{\grqq}

Ueber Onkel Prudent'{\s} Gesicht lief ein Zug der Entt\"auschung.

{\glqq}So m\"ussen wir die Lunte l\"oschen, sagte er.

-- Nein ... aber un{\s} retten! erwiderte Phil Evan{\s}.

-- Un{\s} retten?

-- Ja, mittelst de{\s} Ankertaue{\s}, da e{\s} jetzt finster ist.
Hundertf\"unfzig Fu{\ss} hinabzuklettern hat ja nicht{\s} zu
bedeuten.

-- Nicht{\s}, best\"atigte Onkel Prudent, und wir w\"aren reine
Thoren, eine so unerwartet g\"unstige Gelegenheit unben\"utzt zu
lassen.{\grqq}

Vorher kehrten sie jedoch noch einmal nach der Cabine zur\"uck und
versahen sich mit Allem, wa{\s} sie in Vorau{\s}sicht eine{\s}
k\"urzeren oder l\"angeren Verweilen{\s} auf der Insel Chatam
glaubten bed\"urfen zu k\"onnen. Nachdem sie die Th\"ur wieder
geschlossen, schlichen sie m\"oglichst ger\"auschlo{\s} nach dem
Vorderdeck.

Sie wollten auch Frycollin wecken und diesen zur gleichzeitigen
Flucht mit ihnen veranlagen.

Ring{\s} herrschte tiefe{\s} Dunkel. Die Wolkenstr\"omung von
S\"udwesten wurde etwa{\s} schneller. Der Aeronef schlingerte ein
wenig vor seinem Anker, indem er, so weit e{\s} da{\s} gespannte
Kabel zulie{\ss}, leicht in verticaler Richtung schwankte. Der
Abstieg drohte also etwa{\s} mehr Schwierigkeiten zu bieten. Da{\s}
war aber nicht dazu angethan, zwei M\"anner abzuschrecken, die eben
noch entschlossen gewesen waren, ihr Leben geradezu wegzuwerfen.

Beide schlichen also \"uber da{\s} Deck hin und standen zuweilen,
gesch\"utzt durch die Bauten darauf, still, um zu lauschen, ob irgend
ein Ger\"ausch vernehmbar werde. Nein ... Alle{\s} still. Kein Schein
zitterte durch die Lichtpforten. Der Aeronef lag nicht allein
schweigend da, er war vielmehr in Schlaf versunken.

Onkel Prudent und sein Begleiter n\"aherten sich schon der Cabine
Frycollin'{\s}, al{\s} Phil Evan{\s} pl\"otzlich stehen blieb.

{\glqq}Der Wachtposten!{\grqq} sagte er.

Wirklich lag ein Mann in der N\"ahe eine{\s} der Ruff{\s}. Offenbar
konnte derselbe, wie man zu sagen pflegt, kaum eingenickt sein. Wenn
dieser L\"arm schlug, mu{\ss}te die Flucht unm\"oglich werden.

Nahe hierbei lagen einige Stricke, Leinwandst\"ucke und Werg, wa{\s}
Alle{\s} bei Au{\s}besserung der Schraube gebraucht worden war.

Eine Minute sp\"ater war der Mann geknebelt, \"uber und \"uber
eingewickelt und an einen Pfosten de{\s} Vordercastell{\s} gebunden,
so da{\ss} er weder einen Laut von sich geben, noch eine Bewegung
machen konnte.

Alle{\s} da{\s} vollzog sich fast ohne jede{\s} Ger\"ausch.

Onkel Prudent und Phil Evan{\s} horchten gespannt ... Selbst im
Inneren der Ruff{\s} lie{\ss} sich kein Laut h\"oren. Wa{\s} an Bord
war, lag in festem Schlafe.

Die beiden Fl\"uchtlinge -- denn diesen Namen darf man ihnen wohl
geben -- kamen nach der von Frycollin eingenommenen Cabine.
Fran\c{c}oi{\s} Tapage lie{\ss} ein h\"ochst beruhigende{\s}
Schnarchen vernehmen.

Zur gr\"o{\ss}ten Ueberraschung brauchte Onkel Prudent die Th\"ur
Frycollin'{\s} nicht einmal aufzuklinken, denn diese stand offen. Er
trat einen Schritt in die Cabine ein, zog sich aber gleich wieder
zur\"uck.

{\glqq}Da ist Niemand d'rin, fl\"usterte er.

-- Niemand ... Wo k\"onnte er sein?{\grqq} murmelte Phil Evan{\s}.

Beide begaben sich nun weiter nach vorn, in der Meinung, Frycollin
m\"ochte in irgend einem Winkel eingeschlafen sein.

Auch hier fand sich Niemand.

{\glqq}Sollte der Spitzbube un{\s} schon vorau{\s}gegangen sein? ...
fragte Onkel Prudent.

-- Mag da{\s} der Fall sein oder nicht, antwortete Phil Evan{\s}, wir
k\"onnen unbedingt nicht l\"anger warten. Vorw\"art{\s}!{\grqq}

Ohne Z\"ogern packten die Fl\"uchtlinge einer nach dem anderen da{\s}
Tau mit den H\"anden und hielten sich auch mit den F\"u{\ss}en daran
fest, dann glitten sie daran herab und kamen heil und gesund zur Erde
nieder.

Welche{\s} Ent\/z\"ucken f\"ur sie, den Erdboden zu betreten, der ihnen
so lange gefehlt hatte, auf fester Grundlage dahin zu gehen und nicht
mehr ein Spielball der Luft zu sein!

Sie suchten eben, l\"ang{\s} de{\s} kleinen Wasserlauf{\s}
hinwandernd, nach dem Inneren der Insel zu gelangen, al{\s} sich
pl\"otzlich vor ihnen ein Schatten erhob.

Da{\s} war Frycollin.

Ja, der Neger hatte denselben Gedanken gehabt, der seinem Herrn
gekommen war, und sogar die K\"uhnheit, denselben ohne jede Meldung
vorher zur Au{\s}f\"uhrung zu bringen.

Jetzt war freilich keine Zeit zu Au{\s}einandersetzungen, und Onkel
Prudent dr\"angte e{\s} weit mehr, einen Zuflucht{\s}ort in entfernteren
Theilen der Insel zu finden, al{\s} Phil Evan{\s} stehen blieb.

{\glqq}H\"oren Sie mich an, Onkel Prudent, begann er. Wir sind jetzt
au{\ss}er dem Machtbereich jene{\s} Robur. Er und seine Begleiter
sind einem schrecklichen Tode geweiht, und ich gebe zu, da{\ss} er
ihn verdient hat. Wenn er aber nun bei seiner Ehre schw\"oren wollte,
von jedem Versuche, un{\s} wieder mit sich zu schleppen,
abzugehen~...

-- Die Ehre eine{\s} solchen Manne{\s}~...{\grqq}

Onkel Prudent konnte den Satz nicht vollenden. An Bord de{\s}
{\glqq}Albatro{\s}{\grqq} entstand eine auf\/f\"allige Bewegung.

Allem Anscheine nach war Alarm geschlagen und die Flucht entdeckt
worden.

{\glqq}Hierher, hierher,{\grqq} rief eine Stimme.

Diese kam von dem Wachthabenden, der seine Umh\"ullung doch hatte
abstreifen k\"onnen. Fast gleichzeitig warfen die Bordlichter ihre
elektrischen Strahlen \"uber einen weiten Umkrei{\s}.

{\glqq}Da sind sie! Da unten!{\grqq} rief Tom Turner.

Die Fl\"uchtlinge waren erkannt worden.

Gleichzeitig wurde auf einen laut ertheilten Befehl Robur'{\s} hin
die Bewegung der Auftrieb{\s}\-schrauben verlangsamt und durch
Einziehung de{\s} Ankertaue{\s} begann der {\glqq}Albatro{\s}{\grqq}
sich der Erde zu n\"ahern.

In diesem Augenblick lie{\ss} sich deutlich die Stimme Phil Evan{\s}'
vernehmen:

{\glqq}Ingenieur Robur! rief er, verpflichten Sie sich auf Ehre,
un{\s} hier auf dieser Insel frei zu lassen?

-- Niemal{\s}!{\grqq} entgegnen Robur bestimmt.

Diese Antwort begleitete \"uberdie{\s} der Knall eine{\s}
Gewehre{\s}, dessen Gescho{\ss} die Schulter Phil Evan{\s}' streifte.

{\glqq}Ah, diese Schurken!{\grqq} rief Onkel Prudent.

Sein Messer in der Hand, st\"urzte er damit schon nach dem Felsen,
zwischen denen der Anker eingegriffen hatte. Der Aeronef befand sich
nur noch f\"unfzig Fu{\ss} \"uber der Erde.

Binnen wenigen Secunden war da{\s} Tau durchschnitten, und die
inzwischen merklich aufgefrischte Brise, die den
{\glqq}Albatro{\s}{\grqq} in schiefer Richtung traf, f\"uhrte diesen
nach Nordosten \"uber da{\s} Meer hinau{\s}.



\newpage\begin{center}\label{kap16}
{\large \begin{antiqua}XVI.\end{antiqua}\\
Welche{\s} den Leser in einer vielleicht beklagen{\s}werthen
Ungewi{\ss}heit l\"a{\ss}t.\\\bigskip}
\end{center}



E{\s} war jetzt zwanzig Minuten nach Mitternacht. Noch f\"unf bi{\s}
sech{\s} Flintensch\"usse krachten von dem Aeronef herunter. Phil
Evan{\s} unterst\"utzend, hatten sich Onkel Prudent und Frycollin
unter den Schutz der Felsen gefl\"uchtet, ohne von einer Kugel
verletzt zu werden. F\"ur den Augenblick hatten sie nicht{\s} mehr zu
f\"urchten.

Zun\"achst wurde der {\glqq}Albatro{\s}{\grqq}, w\"ahrend er sich
gleichzeitig von der Insel Chatam entfernte, zu einer H\"ohe von
neunhundert Metern emporgetrieben. Er hatte seine
Aufstieg{\s}schnelligkeit vergr\"o{\ss}ern m\"ussen, um nicht in'{\s}
Meer zu fallen.

In dem Augenblick, al{\s} der von seiner Emballage befreite
Wachtposten den ersten Au{\s}ruf au{\s}gesto{\ss}en hatte, waren
Robur und Tom Turner auf ihn zugeeilt und hatten ihn vollend{\s} von
der den Kopf umschlie{\ss}enden Leinwandh\"ulle befreit und seine
Fesseln gel\"ost. Darauf st\"urzte der Obersteuermann gleich nach der
Cabine de{\s} Onkel Prudent und Phil Evan{\s}, fand diese aber leer.

Fran\c{c}oi{\s} Tapage hatte inzwischen die Cabine Frycollin'{\s}
durchsucht; auch in dieser war Niemand mehr.

Al{\s} Robur die Ueberzeugung gewann, da{\ss} seine Gefangenen ihm
entronnen waren, ergriff ihn der heftigste Zorn. Mit dem Entweichen
de{\s} Onkel Prudent und Phil Evan{\s}' war sein Geheimni{\ss} und
seine Pers\"onlichkeit aller Welt offenbart. Wegen jene{\s} bei der
Fahrt \"uber Europa herabgeworfenen Schriftst\"ucke{\s} hatte er sich
de{\s}halb weit weniger Sorge gemacht, weil er annehmen durfte,
da{\ss} da{\s}selbe beim Niederfallen \"uberhaupt verloren gegangen
sei ... Jetzt lag die Sache aber ganz ander{\s}.

Dann suchte er sich wieder zu beruhigen.

{\glqq}Sie sind vorl\"aufig zwar entflohen, sagte er sich; da sie von
der Insel Chatam aber vor Ablauf einiger Tage nicht wegkommen
k\"onnen, so werde ich dahin zur\"uckkehren. Ich werde sie suchen ...
sie wieder einfangen ... und dann~...{\grqq}

In der That konnten sich die drei Fl\"uchtlinge noch keine{\s}weg{\s}
al{\s} gerettet betrachten. Erlangte der {\glqq}Albatro{\s}{\grqq}
erst seine Man\"ovrirf\"ahigkeit wieder, so erschien er sicherlich
wieder bei der Insel Chatam, von der Jene schwerlich zeitig genug zu
entkommen vermochten. Schon vor Verlauf von zw\"olf Stunden konnten
sie ung\"unstigen Falle{\s} dem Ingenieur wieder in die H\"ande
gerathen sein.

Vor Verlauf von zw\"olf Stunden? Aber binnen zwei Stunden sollte der
{\glqq}Albatro{\s}{\grqq} ja vernichtet sein! Glich jene
Dynamitpatrone nicht einem an seiner Wand befestigten Torpedo, der
da{\s} Zerst\"orung{\s}werk mitten in der Luft vollf\"uhren sollte?

Inzwischen wurde der Aeronef von der noch mehr sich versteifenden
Brise weiter nach Nordost hin getrieben und mu{\ss}te mit
Sonnenaufgang die Insel Chatam unbedingt au{\s} dem Gesichte verloren
haben.

Um gegen den Wind aufkommen zu k\"onnen, h\"atten seine
Triebschrauben, mindesten{\s} die eine am Vordertheil, zu
functioniren im Stande sein m\"ussen.

{\glqq}Tom, rief der Ingenieur, la{\ss}t die Signallaternen so hell
al{\s} m\"oglich leuchten.

-- Sogleich, Master Robur.

-- Und dann Alle an die Arbeit!

-- Alle!{\grqq} wiederholte der Obersteuermann.

Jetzt konnte nicht mehr davon die Rede sein, die Vollendung der
n\"othigen Reparaturen bi{\s} zum anderen Morgen aufzuschieben. Auf
dem {\glqq}Albatro{\s}{\grqq} gab e{\s} keinen Mann, der nicht den
Eifer seine{\s} Chef{\s} getheilt, keinen einzigen, der nicht da{\s}
Verlangen gehabt h\"atte, die Fl\"uchtlinge wieder zu ergreifen.
Sobald die vordere Triebschraube richtig eingesetzt war, wollte man
nach Chatam umkehren, sich daselbst vor Anker legen und die Spur der
Entflohenen verfolgen. Erst nachher sollte die Au{\s}besserung der
hinteren Schraube vorgenommen werden, damit der Aeronef dann mit
aller Sicherheit seine Reise \"uber den Stillen Ocean und nach der
Insel X au{\s}f\"uhren k\"onne.

Jedenfall{\s} erschien e{\s} von Bedeutung, da{\ss} der
{\glqq}Albatro{\s}{\grqq} nicht allzu weit nach Nordost verschlagen
w\"urde. Leider wurde der Wind aber immer st\"arker und er konnte
gegen denselben jetzt nicht aufkommen, ja, sich nicht einmal an ein
und derselben Stelle erhalten. Seiner Triebschrauben beraubt, war er
eben zum unlenkbaren Aerostaten geworden. Die noch an der K\"uste
weilenden Fl\"uchtlinge konnten sich noch \"uberzeugen, da{\ss} er
vollst\"andig au{\ss}er Sicht gekommen war, ehe die vorbereitete
Explosion ihn in St\"ucke ri{\ss}.

Der jetzige Zustand der Dinge fl\"o{\ss}te Robur doch einige
Beunruhigung ein, da er nur mit ziemlich bedeutender Verz\"ogerung
nach der Insel Chatam zur\"uckzukehren hoffen durfte. Er
entschlo{\ss} sich de{\s}halb, w\"ahrend alle H\"ande mit der so
nothwendigen Au{\s}besserung besch\"aftigt waren, sich tiefer
niederzulassen, in der Erwartung, damit eine schw\"achere
Luftstr\"omung anzutreffen. Vielleicht konnte sich der
{\glqq}Albatro{\s}{\grqq} in diesen Schichten wenigsten{\s} an der
Stelle erhalten, bi{\s} er wieder eigene Kraft genug zu \"au{\ss}ern
vermochte, um gegen die Brise mit Erfolg anzuk\"ampfen.

Diese{\s} Man\"over wurde auch sogleich au{\s}gef\"uhrt. Wenn jetzt
ein Schiff in der N\"ahe gewesen w\"are, wie w\"urde dessen
Mannschaft erschrocken sein beim Anblick der Evolutionen diese{\s}
gewaltigen Apparate{\s}?

Al{\s} der {\glqq}Albatro{\s}{\grqq} nur noch wenige hundert Fu{\ss}
\"uber der Meere{\s}fl\"ache schwebte, wurde sein Niedergang
aufgehalten.

Leider mu{\ss}te sich Robur \"uberzeugen, da{\ss} der Wind in diesen
niederen Zonen nur noch heftiger wehte und der Aeronef also mit noch
gr\"o{\ss}erer Schnelligkeit dahin getrieben wurde. Er lief hiermit
nat\"urlich Gefahr, sehr weit nach Nordost verschlagen zu werden,
wa{\s} die R\"uckkehr nach der Insel Chatam noch mehr verz\"ogern
mu{\ss}te.

Nach diesen vergeblichen Versuchen wurde daher wieder beschlossen,
sich mehr in den oberen Lagen der Atmosph\"are zu erhalten, wo da{\s}
Luftmeer in besserem Gleichgewicht und de{\s}halb weniger bewegt war.
Der {\glqq}Albatro{\s}{\grqq} stieg also wieder zu einer mittleren
H\"ohe von dreitausend Meter empor. Blieb er hier auch nicht
station\"ar, so trieb er doch wenigsten{\s} nur langsam weiter. Der
Ingenieur konnte also hoffen, da{\ss} er bei Tage{\s}anbruch und von
dieser H\"ohe au{\s} die Insel, deren geographische Lage er
\"ubrigen{\s} mit vollkommener Sicherheit aufgenommen hatte, noch
werde sehen k\"onnen.

Darum, ob die Fl\"uchtlinge Seiten{\s} der Eingeborenen -- im Fall
die Insel bewohnt war -- einen freundlichen Empfang gefunden hatten,
oder nicht, machte Robur sich keine weitere Sorge. Ob ihnen die
Inselbewohner hilfreich beistanden, war f\"ur ihn ziemlich
belanglo{\s}. Durch die Angriff{\s}waffen, \"uber die der
{\glqq}Albatro{\s}{\grqq} verf\"ugte, w\"urden sie sicherlich
erschreckt und schnell zerstreut werden. Die Wiedererlangung der
Gefangenen war also eine leichte Aufgabe, und einmal ergriffen~...

{\glqq}Nun, von der Insel X entflieht Niemand!{\grqq} sagte Robur.

Eine Stunde nach Mitternacht war die vordere Triebschraube wieder in
Stand gesetzt. E{\s} er\"ubrigte nun blo{\ss} noch die Montirung
derselben, d.~h. die geh\"orige Anbringung derselben an der Welle,
wa{\s} eine weitere Stunde Arbeit erforderte. Nachher sollte der
{\glqq}Albatro{\s}{\grqq}, den Bug nach S\"udwest gerichtet, sogleich
abfahren und die Reparatur der hinteren Triebschraube in Angriff
genommen werden.

Aber die Lunte, welche in der verlassenen Cabine glimmte, diese
Lunte, von der schon der dritte Theil aufgezehrt war! ... Und jener
Funken, der sich mehr und mehr der Dynamitpatrone n\"aherte!~...

W\"are die Mannschaft de{\s} Aeronef{\s} nicht gar so dringlich
besch\"aftigt gewesen, so h\"atte doch vielleicht Einer da{\s}
schwache Knistern wahrgenommen, welche{\s} jetzt dann und wann in dem
Ruff entstand; vielleicht h\"atte er auch den Geruch verbrannten
Pulver{\s} bemerkt. Da{\s} h\"atte ihn sicherlich so beunruhigt,
da{\ss} er dem Ingenieur davon Mittheilung machte. Bei genauer
Nachforschung konnte dann der Kasten, in dem der explodirende
K\"orper verborgen war, nicht unentdeckt bleiben. E{\s} w\"are also
noch Zeit gewesen, den wunderbaren {\glqq}Albatro{\s}{\grqq} und
Alle, die er trug, zu retten.

Die Leute arbeiteten aber auf dem Vorderdeck und wenigsten{\s}
zwanzig Meter entfernt von dem Ruff der Entflohenen. Noch rief sie
nicht{\s} nach diesem Theile de{\s} Deck{\s}, sowie nicht{\s} sie von
einer Besch\"aftigung ablenken konnte, die ihre volle Aufmerksamkeit
in Anspruch nahm.

Robur legte al{\s} geschickter Mechaniker pers\"onlich Hand mit an.
Er betrieb die Arbeit, ohne doch irgendwie zu vernachl\"assigen,
da{\ss} Alle{\s} mit gr\"o{\ss}ter Sorgfalt au{\s}gef\"uhrt wurde, da
e{\s} ihm ja darauf ankam, seine{\s} Apparate{\s} vollst\"andig Herr
zu werden. Gelang e{\s} ihm nicht, die Gefangenen bald wieder in
seine Gewalt zu bringen, so fanden diese vorau{\s}sichtlich
Gelegenheit, in ihr Vaterland zur\"uckzukehren. Dann wurden
jedenfall{\s} Nachforschungen angestellt und die Insel X konnte dabei
m\"oglicher Weise aufgefunden werden; damit aber w\"are da{\s} Ende
der Existenz gekommen, welche sich die Leute, die der
{\glqq}Albatro{\s}{\grqq} trug, geschaffen hatten -- da{\s} Ende
dieser \"ubermenschlichen, so zu sagen hocherhabenen Leben{\s}weise.

Eben jetzt trat Tom Turner an den Ingenieur heran. E{\s} war ein
Viertel nach ein Uhr.

{\glqq}Master Robur, begann er, mir scheint, die Brise hat Neigung
abzuflauen und mehr nach Westen umzuschlagen.

-- Und wa{\s} zeigt der Barometer? fragte Robur, nachdem er den
Himmel fl\"uchtig betrachtet.

-- Er h\"alt sich ziemlich genau auf demselben Punkte, antwortete der
Obersteuermann. Au{\ss}erdem kommt e{\s} mir vor, al{\s} ob die
Wolkenlagen unter dem {\glqq}Albatro{\s}{\grqq} sich senkten.

-- Ganz recht, Tom Turner, und in diesem Falle w\"are e{\s} nicht
unwahrscheinlich, da{\ss} \"uber dem Meere jetzt Regen fiele. Bleiben
wir jedoch \"uber der Regenzone schweben, so k\"ummert un{\s} da{\s}
ja nicht, und wir werden an der Vollendung unserer Arbeiten dadurch
nicht gest\"ort werden.

-- Wenn jetzt Regen f\"allt, meinte Tom Turner, so kann e{\s} nur
ein ganz feiner sein -- die Form der Wolken l\"a{\ss}t da{\s}
wenigsten{\s} muthma{\ss}en -- und h\"ochst wahrscheinlich legt sich
tiefer unten der Wind bald g\"anzlich.

-- Ohne Zweifel, Tom, antwortete Robur. Immerhin scheint e{\s} mir
zweckm\"a{\ss}iger, noch nicht hinunter zu gehen. Beeilen wir un{\s}
erst, alle erlittenen Besch\"adigungen au{\s}zubessern, dann k\"onnen
wir ja nach Belieben man\"ovriren, und da{\s} ist die
Hauptsache.{\grqq}

Wenige Minuten nach zwei Uhr war der erste Theil der Arbeit
vollendet. Nach Wiedereinsetzung der vorderen Triebschraube wurden
die jene treibenden Batterien in Th\"atigkeit gesetzt. Nach und nach
beschleunigte sich die Bewegung de{\s} {\glqq}Albatro{\s}{\grqq},
und, den Bug nach S\"udwest gerichtet, kehrte er mit mittlerer
Geschwindigkeit in der Richtung nach der Insel Chatam zur\"uck.

{\glqq}Tom, sagte Robur, e{\s} m\"ogen etwa zweieinhalb Stunden
verflossen sein, seit wir nach Nordost hin getrieben wurden. Die
Windrichtung hat sich, wie mir Compa{\ss}beobachtungen lehrten,
seitdem nicht ge\"andert. Ich sch\"atze also, da{\ss} wir binnen
h\"ochsten{\s} einer Stunde die Gestade der Insel wieder gefunden
haben k\"onnen.

-- Ich glaub' e{\s} auch, Master Robur, antwortete der
Obersteuermann, denn wir bewegen un{\s} jetzt mit einer Schnelligkeit
von zw\"olf Meter in der Secunde vorw\"art{\s}. Zwischen drei und
vier Uhr Morgen{\s} m\"u{\ss}te der {\glqq}Albatro{\s}{\grqq} seinen
Au{\s}gang{\s}punkt demnach wieder erreichen.

-- Da{\s} w\"are desto besser, Tom, erwiderte der Ingenieur. Wir
haben ein Interesse daran, noch w\"ahrend der Nacht einzutreffen und
ungesehen an'{\s} Land zu kommen. Die Fl\"uchtlinge halten un{\s}
f\"ur weit nach Norden verschlagen und sind jetzt gewi{\ss} nicht auf
ihrer Hut. Wenn der {\glqq}Albatro{\s}{\grqq} ganz nahe der Erde
hingleitet, werden wir versuchen, un{\s} hinter einigen hohen Felsen
der Insel zu verbergen. M\"u{\ss}ten wir dann selbst mehrere Tage bei
Chatam verweilen~...

-- So bleiben wir eben da, Master Robur, und h\"atten wir auch gegen
eine ganze Armee von Eingeborenen zu k\"ampfen~...

-- So k\"ampfen wir, Tom, k\"ampfen wir f\"ur unseren
{\glqq}Albatro{\s}{\grqq}!{\grqq}

Der Ingenieur wandte sich zu seinen neue Anordnungen erwartenden
Leuten zur\"uck.

{\glqq}Liebe Freunde, sagte er, noch ist die Stunde der Ruhe nicht
gekommen, wir m\"ussen bi{\s} zum Anbruch de{\s} Tage{\s} th\"atig
sein.{\grqq}

Alle erkl\"arten sich bereit.

Jetzt galt e{\s}, an der hinteren Triebschraube dieselben Reparaturen
vorzunehmen, welche an der vorderen schon au{\s}gef\"uhrt waren.
E{\s} handelte sich dabei um die gleichen Besch\"adigungen, die auch
die n\"amliche Ursache, jener Orkan bei der Fahrt \"uber den
antarktischen Continent, veranla{\ss}t hatte.

Um diese Schraube hereinzuholen, erschien e{\s} rathsam, die Fahrt
de{\s} Aeronef{\s} w\"ahrend einiger Minuten zu unterbrechen oder ihm
selbst eine R\"uckw\"art{\s}bewegung zu ertheilen. Auf ein Zeichen
Robur'{\s} legte der Hilf{\s}mechaniker die Maschine um, indem er die
vordere Schraube sich in entgegengesetztem Sinne drehen lie{\ss}, so
da{\ss} der Aeronef -- um den seetechnischen Au{\s}druck zu
gebrauchen -- {\glqq}\"uber Steuer zu gehen{\grqq} anfing.

Schon wollten sich Alle nach dem Hinterdeck begeben, al{\s} Tom
Turner ein eigenth\"umlicher Geruch auf\/fiel.

E{\s} waren die in dem Kasten jetzt angeh\"auften Gase der Lunte,
welche au{\s} der Cabine der Fl\"uchtlinge hervordrangen.

{\glqq}Hm~...? machte der Obersteuermann.

-- Wa{\s} giebt e{\s}? fragte Robur.

-- Riechen Sie nicht{\s}? Man k\"onnte sagen, e{\s} m\"usse Pulver
brennen.

-- Ihr habt Recht, Tom!

-- Und dieser Geruch dringt au{\s} dem letzten Ruff.

-- Ja ... sogar au{\s} derselben Cabine~...

-- Sollten die Elenden auch noch Feuer angelegt haben?

-- Und wenn e{\s} nicht nur Feuer w\"are? ... rief Robur. Sto{\ss}t
die Th\"ur ein, Tom, sto{\ss}t sie ein!{\grqq}

Der Obersteuermann ging aber kaum daran, diesem Befehle nachzukommen,
al{\s} eine furchtbare Explosion den {\glqq}Albatro{\s}{\grqq}
ersch\"utterte. Die Ruff{\s} flogen in St\"ucke. Die elektrischen
Lampen verl\"oschten, da ihnen der Strom pl\"otzlich fehlte, und
e{\s} ward vollst\"andig finster. Doch wenn auch gleichzeitig die
meisten Auftrieb{\s}schrauben verbogen oder theilweise zertr\"ummert
und dadurch wirkung{\s}lo{\s} geworden waren, so drehten sich
wenigsten{\s} noch mehrere nahe dem Bug ungest\"ort weiter.

P\"otzlich \"offnete sich auch der Aeronef ein wenig hinter dem
ersten Ruff, dessen Accumulatoren noch immer die vordere
Triebschraube in Th\"atigkeit erhielten, und der hintere Theil de{\s}
Deck{\s} senkte sich ebenso schnell nach abw\"art{\s}. Fast
gleichzeitig standen die hinteren Auftrieb{\s}schrauben still und der
{\glqq}Albatro{\s}{\grqq} st\"urzte in die Tiefe hinab.

Da{\s} bedeutete f\"ur die acht M\"anner, welche sich gleich
Schiffbr\"uchigen an diese{\s} Wrack klammerten, einen Sturz von
dreitausend Metern.

Derselbe mu{\ss}te obendrein noch um so schneller erfolgen, al{\s}
die vordere Triebschraube, deren Achse jetzt senkrecht stand, noch
immer arbeitete.

Da lie{\ss} sich Robur, der in dieser verzweifelten Lage eine ganz
au{\ss}erordentliche Kaltbl\"utigkeit an den Tag legte, bi{\s} zu dem
halb weggesprengten Ruff gleiten, ergriff den Steuerung{\s}hebel und
ver\"anderte sofort die Drehung{\s}richtung der Schraube, welche nun
statt vorw\"art{\s} nach aufw\"art{\s} trieb.

Der Absturz wurde dadurch zwar nicht aufgehalten, aber doch
wenigsten{\s} verlangsamt; da{\s} Wrack fiel nicht mehr mit der
zunehmenden Geschwindigkeit nieder, welche alle nur der Wirkung der
Schwerkraft unterworfenen K\"orper zeigen. Und wenn auch allen
lebenden Wesen auf dem {\glqq}Albatro{\s}{\grqq} noch immer der Tod
drohte, weil sie rettung{\s}lo{\s} in'{\s} Meer st\"urzen mu{\ss}ten,
so war e{\s} doch nicht mehr der Tod durch Erstickung inmitten der
wegen rasender Schnelligkeit de{\s} Falle{\s} unathembar werdenden
Luft.

Vierundzwanzig Secunden nach der Explosion war, wa{\s} vom
{\glqq}Albatro{\s}{\grqq} noch \"ubrig war, in den Fluthen versunken.



\newpage\begin{center}\label{kap17}
{\large \begin{antiqua}XVII.\end{antiqua}\\
Worin der Leser um zwei Monate r\"uckw\"art{\s} und auch um neun
Monate vorw\"art{\s}gef\"uhrt wird.\\\bigskip}
\end{center}



Einige Wochen fr\"uher, am 13. Juni, d.~h. am Tage nach der
denkw\"urdigen Sitzung, w\"ahrend der e{\s} im Weldon-Institut zu so
st\"urmischen Verhandlungen gekommen war, herrschte unter allen
Classen der Bewohner von Philadelphia, unter den Negern wie den
Wei{\ss}en, eine leichter zu constatirende, al{\s} zu beschreibende
Aufregung.

Schon in den ersten Morgenstunden unterhielt man sich \"uberall von
den unerwarteten, l\"armenden Zwischenf\"allen der Sitzung de{\s}
letzten Abend{\s}. Ein Eindringling, der Ingenieur zu sein angab, ein
Ingenieur, der den an sich unwahrscheinlichen Namen Robur -- Robur
der Sieger! -- f\"uhren wollte, eine Pers\"onlichkeit von unbekannter
Herkunft und namenloser Nationalit\"at hatte sich unerwartet im
Sitzung{\s}saale vorgestellt, die Ballonisten mit gr\"oblichen Reden
beleidigt, Diejenigen, welche der Lenkbarkeit von Aerostaten
huldigten, verspottet, und hatte dagegen die Vorz\"uge von specifisch
schwereren Apparaten gepriesen, ein ver\"achtliche{\s} Hohnlachen
unter dem wildesten Get\"ose au{\s}geschlagen und zu Drohungen
geradezu herau{\s}gefordert, um diese seinen Gegnern wieder al{\s}
Antwort in'{\s} Gesicht zu schleudern. Endlich war er, nachdem er den
Rednerstuhl unter dem Knattern von Revolvern ger\"aumt, verschwunden,
und trotz aller Nachforschungen hatte man keine weiteren Spuren von
ihm entdeckt.

Nat\"urlich waren solche Vorf\"alle dazu angetan, alle Zungen zu
wetzen und der Phantasie ein weite{\s} Feld zu er\"offnen. Daran
sollte e{\s} denn auch weder in Philadelphia, noch in den
sech{\s}unddrei{\ss}ig anderen Staaten der Union fehlen, ja
eigentlich wurde sogar die Alte Welt dadurch nicht minder erregt, wie
die Neue Welt.

Wie mu{\ss}te sich diese allgemeine Aufregung aber noch steigern,
al{\s} e{\s} am Abend de{\s} 13. Juni stadtkundig wurde, da{\ss}
weder der Vorsitzende, noch der Schriftf\"uhrer de{\s}
Weldon-Institut{\s} bi{\s} dahin in ihre Wohnungen zur\"uckgekehrt
waren, und hierbei handelte e{\s} sich um zwei geachtete, gelehrte
M\"anner von verh\"altni{\ss}m\"a{\ss}ig hoher Stellung. Am Vorabend
noch hatten sie den Sitzung{\s}saal verlassen al{\s} B\"urger, welche
ruhig nach ihrem Heim zur\"uckzukehren denken, al{\s} Hagestolze,
deren Nachhausekunft kein m\"urrisch gerunzelte{\s} Gesicht
verbittert h\"atte. Sollten Sie vielleicht gar absichtlich
verschwunden sein? Nein: mindesten{\s} hatten sie nicht{\s}
ge\"au{\ss}ert, wa{\s} zu diesem Glauben h\"atte verf\"uhren
k\"onnen; ja, e{\s} war sogar besprochen worden, da{\ss} sie schon am
n\"achsten Tage wieder nach dem Bureau de{\s} Club{\s} kommen und die
gewohnten Pl\"atze de{\s} Vorsitzenden und Schriftf\"uhrer{\s} bei
einer au{\ss}erordentlichen Sitzung einnehmen sollten. Welche man zur
weiteren Besprechung der Vorf\"alle de{\s} letzten Abend{\s} bestimmt
hatte.

Doch nicht nur jene beiden weitbekannten Pers\"onlichkeiten de{\s}
Staate{\s} Pennsylvanien waren spurlo{\s} verschwunden, auch von dem
Diener Frycollin kam keinerlei Nachricht, auch er war ebenso wenig zu
finden, wie sein Herr. Wahrlich, seit Toussaint Louverture, Faustin
Soulouque und Dessaline hatte noch kein Neger so viel von sich reden
gemacht. Er stand im Begriff, einen hervorragenden Platz sowohl unter
seinen dienenden Collegen in Philadelphia, wie unter allen jenen
Originalen einzunehmen, welche irgend eine Exentricit\"at in dem
sch\"onen Amerika schon in hellere{\s} Licht zu setzen hinreicht.

Auch am folgenden Tage nicht{\s} Neue{\s}. Weder die beiden Collegen,
noch Frycollin erschienen wieder. Ernste Beunruhigung. Beginnende
Aufregung. Vor den \begin{antiqua}Post and Telegraph
offices\end{antiqua} starke Ansammlung von Neugierigen, um etwaige
Nachricht noch ganz warm zu erhalten.

Vergebliche Liebe{\s}m\"uhe.

Und doch hatten so Viele deutlich genug gesehen, wie Beide in
lebhaftem Gespr\"ach au{\s} dem Weldon-Institut weggingen, den sie
erwartenden Frycollin mitnahmen und nachher die Walnut-Stra{\ss}e
hinabwanderten, um sich dem Fairmont-Park zuzuwenden.

Jem Cip, der Vegetarianer, hatte dem Pr\"asidenten noch die rechte
Hand gedr\"uckt und sich verabschiedet mit den Worten:

{\glqq}Auf morgen also!{\grqq}

Und William T. Forbe{\s}, der Fabrikant von Zucker au{\s} Leinwand,
r\"uhmte sich eine{\s} vertraulichen Handschlag{\s} von Phil
Evan{\s}, der ihm zweimal {\glqq}Auf Wiedersehen!{\grqq} zugerufen
hatte.

Mi{\ss} Doll und Mi{\ss} Mat Forbe{\s}, welche ein Band reinster
Freundschaft an Onkel Prudent fesselte, konnten sich \"uber sein
Verschwinden gar nicht beruhigen und schwatzten, nur um von ihm immer
etwa{\s} zu h\"oren, eher noch mehr, al{\s} gew\"ohnlich.

So verstrichen drei, vier, f\"unf, sech{\s} Tage, eine Woche, eine
zweite Woche ... Weder irgendwer, noch irgendwa{\s} leitete auf die
F\"ahrte der drei Verschwundenen.

Und doch hatte man die sorgsamsten Nachsuchungen im ganzen
Stadtviertel vorgenommen ... vergeblich; in allen nach dem Hafen zu
f\"uhrenden Stra{\ss}en ... nutzlo{\s}; weiter im Park selbst, unter
den Gruppen gr\"o{\ss}erer B\"aume und dichterer Geb\"usche ...
erfolglo{\s}! ... Ueberall nicht{\s}!

Auf der gro{\ss}en Waldbl\"o{\ss}e erkannte man jedoch, da{\ss} da
und dort da{\s} Gra{\s} ganz neuerding{\s} niedergedr\"uckt schien.
Diese Wahrnehmung erregte ihrer Unerkl\"arlichkeit wegen einigen
Verdacht. Ebenso wurden am Saume de{\s} dieselbe umschlie{\ss}enden
Walde{\s} Spuren eine{\s} stattgefundenen Kampfe{\s} entdeckt. Hatte
nun eine Bande von Landstreichern vielleicht die beiden Collegen zu
vorger\"uckter Nacht\/zeit hier in dem menschenleeren Parke getroffen
und \"uberfallen?

Da{\s} war ja m\"oglich. Die Polizei nahm auch eine
die{\s}bez\"ugliche regelrechte und mit gesetzlicher Langsamkeit
betriebene Untersuchung in die Hand. Man f\"uhrte Schleppnetze durch
den Schuylkill hin, schl\"ammte seinen Grund und befreite die Ufer
von dem Gewirr angeh\"auften Unkraute{\s}. Wenn auch da{\s}
vergeblich blieb, so war e{\s} doch nicht nutzlo{\s}, denn der
Schuylkill bedurfte einer gr\"undlichen S\"auberung gerade recht
n\"othig. O, e{\s} sind praktische Leute, die Aedilen von
Philadelphia!

Sp\"ater wandte man sich an die verbreitetsten Zeitungen. Anzeigen,
Reclamationen, wenn nicht gar Reclamen, wurden an alle demokratischen
und republikanischen Bl\"atter der Union -- ohne R\"ucksicht auf
deren Farbe -- versendet. Der \begin{antiqua}{\glqq}Daily
Negro{\grqq}\end{antiqua} da{\s} specielle Organ der schwarzen Rasse,
brachte Frycollin'{\s} Bildni{\ss} nach dessen letzter Photographie.
Man bot Belohnungen und versprach Preise Jedem, der von den drei
Abwesenden Nachricht geben k\"onnte, ja sogar allen Denen, die nur
irgend welche{\s} Anzeichen entdeckten, da{\s} auf deren F\"ahrte zu
leiten versprach.

{\glqq}F\"unftausend Dollar{\s}! F\"unftausend Dollar{\s} jedem
B\"urger, der~...{\grqq}

Vergeblich; die f\"unftausend Dollar{\s} verblieben in der Casse
de{\s} Weldon-Institut{\s}.

{\glqq}Nicht aufzufinden! Nicht aufzufinden! Nicht aufzufinden!!!
Onkel Prudent und Phil Evan{\s} au{\s} Philadelphia!{\grqq}

E{\s} versteht sich von selbst, da{\ss} der Club durch diese{\s}
unerkl\"arliche Verschwinden seine{\s} Vorsitzenden und seine{\s}
Schriftf\"uhrer{\s} in heillose Unordnung gerieth und von vornherein
sah derselbe sich durch diese Nothlage zu dem Beschlusse gezwungen,
die fr\"uher so eifrig betriebenen und schon ziemlich
fortgeschrittenen Arbeiten betreff{\s} Construction de{\s}
\begin{antiqua}Go a head\end{antiqua} auf unbestimmte Zeit
einzustellen. Wie h\"atten die anderen Mitglieder auch in Abwesenheit
der beiden Begr\"under und F\"orderer diese{\s} Unternehmen{\s}, dem
dieselben -- an Zeit und Geld -- einen Theil ihre{\s} Verm\"ogen{\s}
geopfert hatten, sich entschlie{\ss}en k\"onnen, ein Werk zu Ende zu
f\"uhren, wenn Jene fehlten, um e{\s} gleichsam zu kr\"onen?

Sie mu{\ss}ten sich also in Geduld fassen.

Gerade zu dieser Zeit ging auf'{\s} Neue die Rede von der
wunderbaren, merkw\"urdigen Erscheinung, welche mehrere Wochen
vorher alle Geister so lebhaft erregt hatte.

In der That war jener geheimni{\ss}volle Gegenstand wieder und
wiederholt wiedergesehen worden, wie er durch die h\"oheren Schichten
der Atmosph\"are schwebte. Freilich dachte kein Mensch an einen
Zusammenhang dieser auf\/fallenden Erscheinung mit dem nicht weniger
unerkl\"arlichen Verschwinden der beiden Mitglieder de{\s}
Weldon-Institut{\s}. E{\s} h\"atte auch einer au{\ss}ergew\"ohnlichen
Dosi{\s} von Einbildung{\s}kraft bedurft, diese beiden Thatsachen mit
einander in Verbindung zu bringen.

Auf jeden Fall war da{\s} Asteroid, die erkaltete Feuerkugel oder
da{\s} Luftungeheuer, wie man die Erscheinung nennen wollte, nun
unter Bedingungen gesehen worden, welche seine Gr\"o{\ss}e und
Gestalt besser abzusch\"atzen erlaubten. Zuerst in Canada \"uber den
Gebiet{\s}theilen, die sich von Ottawa bi{\s} Quebec erstrecken, und
zwar schon am n\"achsten Tage nach dem Verschwinden der beiden
Collegen; dann sp\"ater \"uber den Ebenen de{\s} Fernen Westen{\s},
al{\s} der {\glqq}Albatro{\s}{\grqq} sich an Schnelligkeit mit einem
Zuge der gro{\ss}en Pacific-Bahn ma{\ss}.

Von diesem Tage ab herrschte unter der gelehrten Welt keine
Ungewi{\ss}heit mehr; dieser K\"orper war kein Erzeugni{\ss} der
Natur, sondern ein Flieg-Apparat mit praktischer Anwendung der
Theorie de{\s} {\glqq}Schwerer, al{\s} die Luft{\grqq}. Und wenn der
Sch\"opfer und F\"uhrer diese{\s} Aeronef{\s} auch f\"ur seine Person
da{\s} bi{\s}herige Incognito noch aufrecht erhalten wollte,
jedenfall{\s} sah er davon, so weit e{\s} seine Maschine betraf,
jetzt ab, weil er dieselbe so dicht \"uber den Gebieten de{\s} Fernen
Westen{\s} sehen lie{\ss}. Die von ihm gew\"ahlte mechanische
Kraftquelle, wie die Natur der Maschinen, welche dem Apparate seine
Bewegung ertheilten, blieb vorl\"aufig freilich noch unbekannt.
Mindesten{\s} war jedoch au{\ss}er Zweifel gestellt, da{\ss} diesem
Aeronef eine ganz au{\ss}ergew\"ohnliche Fortbewegung{\s}f\"ahigkeit
innewohnte, denn nur wenige Tage sp\"ater meldete man schon sein
Erscheinen im Himmlischen Reiche, dann au{\s} den n\"ordlichen
Theilen von Hindostan und kurz darauf wieder au{\s} den Steppen
Ru{\ss}land{\s}.

Wer mochte nun jener k\"uhne Mechaniker sein, der \"uber so gro{\ss}e
bewegende Kr\"afte gebot, f\"ur den weder L\"ander, noch Meere eine
Grenze hatten, der in der Erdatmosph\"are wie in einem ihm allein
zugeh\"origen Gebiete schaltete und waltete? Sollte man glauben,
e{\s} k\"onne da{\s} jener Robur sein, der dem Weldon-Institut seine
Theorien so r\"ucksicht{\s}lo{\s} in'{\s} Gesicht geschleudert hatte,
al{\s} er an dem bewu{\ss}ten Abend erschien, um in die Utopien
betreff{\s} der lenkbaren Ballon{\s} eine klaffende Bresche zu legen?

Vielleicht kam einigen weiter blickenden K\"opfen dieser Gedanke. Und
-- wunderbarer Weise -- dennoch erhob sich Niemand zu der Annahme,
da{\ss} besagter Robur mit dem Verschwinden de{\s} Vorsitzenden und
de{\s} Schriftf\"uhrer{\s} vom Weldon-Institut in irgend welchem
Zusammenhange stehen k\"onnte.

Da{\s} blieb also noch weiter Geheimni{\ss}, bi{\s} eine Depesche von
Frankreich durch da{\s} tran{\s}atlantische Kabel am 6. Juli um elf
Uhr siebenundrei{\ss}ig Minuten in New-York eintraf.

Und wa{\s} meldete diese Depesche? Sie \"ubermittelte den Text
jene{\s} in Pari{\s} in einer Schnupftabak{\s}dose gefundenen
Document{\s} -- de{\s} Schriftst\"ucke{\s}, welche{\s} endlich
enth\"ullte, wa{\s} au{\s} den beiden M\"annern geworden war, um
welche die Union eben Trauer anlegen wollte.

Der Urheber der Entf\"uhrung war also doch Robur, der Ingenieur, der
au{\s}schlie{\ss}lich zu dem Zwecke nach Philadelphia kam, die
Theorie der Ballonisten gleichsam im Ei zu ersticken. Er war e{\s},
der auf dem Aeronef {\glqq}Albatro{\s}{\grqq} umherfuhr; er, der zur
Wiedervergeltung erfahrener Unbill Onkel Prudent nebst Phil Evan{\s}
und Frycollin obendrein in die L\"ufte entf\"uhrt hatte! Und diese
Personen konnte man al{\s} f\"ur immer verloren ansehen, wenn nicht
durch irgend welche Hilf{\s}mittel eine Maschine construirt wurde,
welche im Stande war, jenen m\"achtigen Apparat zu bek\"ampfen, und
wenn die irdischen Freunde Jener ihnen damit nicht zu Hilfe kamen.

Welche Erregung! Welche{\s} Staunen! Da{\s} Pariser Telegramm war an
da{\s} Bureau de{\s} Weldon-Institut{\s} adressirt gewesen. Die
Mitglieder de{\s} Club{\s} erhielten davon unverz\"uglich
Kenntni{\ss}. Nach zehn Minuten hatte ganz Philadelphia durch seine
Telephon{\s} die gro{\ss}e Neuigkeit erfahren, binnen einer Stunde
ganz Amerika, denn sie hatte sich elektrisch auf den zahllosen
Dr\"ahten der Neuen Welt verbreitet. Man wollte noch nicht recht
daran glauben und hielt e{\s} wohl f\"ur die Mystification eine{\s}
schlechten Witzbold{\s} -- sagten die Einen -- f\"ur ein
{\glqq}Einr\"auchern{\grqq} schlimmster Art -- meinten die Andern.
Wie w\"are e{\s} m\"oglich gewesen, diesen Raub in Philadelphia so im
Geheimen au{\s}zuf\"uhren? Wie h\"atte der {\glqq}Albatro{\s}{\grqq}
im Fairmont-Park zur Erde hernieder gehen k\"onnen, ohne am Horizont
de{\s} Staate{\s} Pennsylvanien bemerkt zu werden?

Recht sch\"on -- so lauteten die gew\"ohnlichen Argumente. -- Die
Ungl\"aubigen behielten zwar noch da{\s} Recht zu zweifeln, sollten
e{\s} aber sieben Tage nach dem Eintreffen de{\s} Telegramm{\s} schon
verlieren. Am 13. Juli ging da{\s} franz\"osische Packetboot
{\glqq}Normandie{\grqq} im Hudson vor Anker und -- brachte die
ber\"uhmte Schnupftabak{\s}dose mit. Die Eisenbahn bef\"orderte
dieselbe in gr\"o{\ss}ter Eile von New-York nach Philadelphia.

Ja, da{\s} war sie, die Dose de{\s} Vorsitzenden vom Weldon-Institut.
Jem Cip h\"atte an diesem Tage gut gethan, eine etwa{\s}
substantiellere Nahrung zu sich zu nehmen, denn er war, al{\s} er sie
erkannte, nahe daran, ohnm\"achtig umzusinken. Wie oft hatte er sich
darau{\s} ein Freundschaft{\s}prie{\s}chen zugelangt! Mi{\ss} Doll
und Mi{\ss} Mat erkannten sie ebenfall{\s}, diese Dose, welche sie so
oft mit dem heimlichen Wunsche betrachtet, eine{\s} Tage{\s} auch
ihre d\"urren Altjungfernfinger hinein zu senken. Und da waren ihr
Vater, William T. Forbe{\s}, Truk Milnor, Bat T. Fyn und viele Andere
au{\s} dem Weldon-Institut -- hundertmal hatten sie dieselbe in den
H\"anden ihre{\s} verehrten Vorsitzenden sich \"offnen und
schlie{\ss}en sehen. Endlich hatte sie da{\s} Zeugni{\ss} aller
Freunde f\"ur sich, die Onkel Prudent in der guten Stadt Philadelphia
besa{\ss}, deren Name -- wie man nicht oft genug wiederholen kann --
darauf hinweist, da{\ss} ihre Bewohner sich wie Br\"uder lieben.

Jetzt war also nach dieser Seite kein Schatten eine{\s} Zweifel{\s}
mehr aufrecht zu erhalten. Nicht allein die Dose de{\s} Vorsitzenden,
sondern besonder{\s} auch die von ihm herr\"uhrenden Schrift\/z\"uge
de{\s} Document{\s} erlaubten e{\s} auch den Ungl\"aubigsten nicht
mehr mit den Achseln zu zucken. Da begannen nun die Wehklagen und
verzweifelte H\"ande erhoben sich gen Himmel. Onkel Prudent und sein
College in einer Flugmaschine entf\"uhrt, ohne da{\ss} man ein Mittel
entdecken konnte, sie zu befreien!

Die Gesellschaft der Niagara-F\"alle, deren st\"arkster Action\"ar
Onkel Prudent war, h\"atte beinahe ihre Gesch\"afte eingestellt und
die Wasserf\"alle geschlossen. Die \begin{antiqua}Walton Watch
Company\end{antiqua} dachte schon daran, ihre Uhrenfabrik zu
liquidiren, da diese ihren Director Phil Evan{\s} eingeb\"u{\ss}t
hatte.

Ja, e{\s} herrschte allgemeine Trauer, und da{\s} Wort Trauer ist
hier gar nicht \"ubertrieben, denn manche hirnverbrannte K\"opfe, wie
man deren auch in den Vereinigten Staaten antrifft, bildeten sich
steif und fest ein, die beiden ehrenwerthen B\"urger niemal{\s}
wiederzusehen.

Nachdem er \"uber Pari{\s} hingefahren war, h\"orte man von dem
{\glqq}Albatro{\s}{\grqq} zun\"achst nicht weiter reden. Einige
Stunden sp\"ater war er \"uber Rom schwebend gesehen worden -- da{\s}
war Alle{\s}. Bei der bekannten Geschwindigkeit de{\s} Aeronef{\s},
mit der er \"uber Europa von Nord nach S\"ud und \"uber da{\s}
Mittelmeer von West nach Ost gefahren war, darf da{\s} ja nicht
Wunder nehmen. Und Dank eben dieser Schnelligkeit konnte ihn auch
kein Fernrohr an irgend einem Punkte seiner Fahrtlinie genauer
beobachten. Und h\"atten die Sternwarten ihr gesammte{\s} Personal
Tag und Nacht auf Vorposten gestellt, die Flugmaschine Robur de{\s}
Sieger{\s} h\"atte sich so weit und so hoch entfernt -- in
{\glqq}Ikarien{\grqq}, wie er zu sagen pflegte -- da{\ss} Alle
verzweifelt w\"aren, deren Spur je wieder aufzufinden.

Hier sei hinzugef\"ugt, da{\ss} wenn seine Geschwindigkeit \"uber dem
Ufer Afrika{\s} auch vermindert wurde, sich doch, weil jene{\s}
Document noch nicht bekannt war, Niemand versah, den Aeronef in den
H\"ohen de{\s} algerischen Himmel{\s} zu suchen. Auf jeden Fall wurde
er \"uber Timbuctu wahrgenommen; da{\s} Observatorium dieser
ber\"uhmten Stadt -- wenn sie \"uberhaupt ein solche{\s} besitzt --
hatte aber noch nicht Zeit gefunden, da{\s} Resultat seiner
Beobachtungen nach Europa mit\/zutheilen. Wa{\s} den K\"onig von
Dahomey betrifft, so h\"atte dieser gewi{\ss} eher zehntausend
Unterthanen, und seine Minister inbegriffen, um einen Kopf k\"urzer
machen lassen, ehe er zugestand, im Kampfe mit einer in der Luft
schwebenden Maschine unterlegen zu sein. Jeder fr\"ohnt eben seiner
kleinen Eigenliebe.

Weiterhin steuerte der Ingenieur Robur dann \"uber den Atlantischen
Ocean, wobei er zuerst nach dem Feuerlande und dann nach dem Cap Horn
kam. Ferner irrte er, etwa{\s} gegen seinen Willen, \"uber die
s\"udlichsten Landvesten und \"uber da{\s} au{\s}gedehnte Polargebiet
hinweg. Von diesen antarktischen Gegenden au{\s} war nat\"urlich
keine Nachricht zu erwarten.

Der Juli verrann, und kein menschliche{\s} Auge konnte sich r\"uhmen,
den Aeronef nur fl\"uchtig wieder erblickt zu haben.

Der August ging zu Ende, ohne da{\ss} sich an der Ungewi{\ss}heit
\"uber da{\s} Loo{\s} der beiden Gefangenen Robur'{\s} etwa{\s}
\"anderte. Man fing allm\"ahlich an, sich zu fragen, ob der
Ingenieur, nach dem Beispiele de{\s} Ikaru{\s}, diese{\s} \"altesten
Mechaniker{\s}, dessen die Sagengeschichte erw\"ahnt, nicht ein Opfer
seiner K\"uhnheit geworden sein m\"oge.

Endlich vergingen auch die ersten siebenundzwanzig Tage de{\s}
September{\s} ohne jede Aenderung der Sachlage.

Bekanntlich gew\"ohnt man sich ja in der Welt an Alle{\s}. E{\s}
liegt in der menschlichen Natur, mit der Zeit den Stachel de{\s}
Schmerze{\s} weniger zu empfinden; man vergi{\ss}t, weil e{\s}
nothwendig ist, einmal zu vergessen. In diesem Falle mu{\ss}te man
dagegen den Bewohnern diese{\s} Erdenthal{\s} zu ihrer Ehre
nachsagen, da{\ss} sie von der allgemeinen Regel abwichen; noch immer
ermattete nicht die warme Theilnahme an dem Loose zweier Wei{\ss}en
und eine{\s} Schwarzen, die wie durch den Propheten Elia{\s}
entf\"uhrt schienen, denen aber keine R\"uckkehr durch die Bibel
gewei{\s}sagt war.

In Philadelphia trat da{\s} nat\"urlich noch deutlicher zu Tage,
al{\s} an jedem anderen Orte; hier kamen dabei ja n\"ahere
pers\"onliche Beziehungen in'{\s} Spiel. Robur hatte den Onkel
Prudent und Phil Evan{\s} au{\s} Rache ihrer Heimat entfremdet,
hatte, wenn auch ohne jede{\s} Recht, eine grausame Wiedervergeltung
ge\"ubt. Doch war seine Rache damit gek\"uhlt? W\"urde er dieselbe
nicht auch noch anderen Collegen de{\s} Vorsitzenden und de{\s}
Schriftf\"uhrer{\s} vom Weldon-Institut f\"uhlen lassen? Und wer
konnte sich gesichert w\"ahnen gegen etwaige Angriffe jene{\s}
allm\"achtigen Beherrscher{\s} de{\s} Luftmeere{\s}?

Da durchlief am 28. September eine Neuigkeit die ganze Stadt: Onkel
Prudent und Phil Evan{\s} sollten danach am Nachmittage in der
Privatwohnung de{\s} Vorsitzenden vom Weldon-Institut wieder
aufgetaucht sein.

Da{\s} Merkw\"urdigste an dieser Botschaft war, da{\ss} sie sich
best\"atigte, obgleich die Meisten nicht daran glauben wollten.

Dennoch mu{\ss}te man sich der Thatsache f\"ugen. Da{\s} waren die
beiden Verschwundenen in Person -- nicht ihre Schatten -- und auch
Frycollin war mit ihnen zur\"uckgekehrt.

Die Mitglieder de{\s} Club{\s}, darauf deren Freunde und endlich eine
ungeheure Volk{\s}menge str\"omten vor Onkel Prudent'{\s} Hause
zusammen. Alle begr\"u{\ss}ten mit Jubelruf die beiden Collegen,
welche unter Hurrah{\s} und Hipp{\s} von Hand zu Hand getragen
wurden.

Hier befand sich Jem Cip, der sein Fr\"uhst\"uck -- ger\"ostete
Brotschnittchen mit gekochtem Lattig -- verlassen hatte, und auch
William T. Forbe{\s} nebst seinen beiden T\"ochtern Mi{\ss} Doll und
Mi{\ss} Mat. W\"are Onkel Prudent Mormone gewesen, heute h\"atte er
sie alle Beide zu Frauen bekommen; doch da{\s} war er nicht und hatte
auch nicht die geringste Absicht, e{\s} je zu werden. Hier waren
ferner Truk Milnor, Bat T. Fyn und endlich die \"ubrigen Mitglieder
de{\s} Club{\s}. E{\s} ist noch bi{\s} heutigen Tage{\s} ein
R\"athsel geblieben, wie Onkel Prudent und Phil Evan{\s} hatten
lebend au{\s} den Tausenden von Armen hervorgehen k\"onnen, welche
sie bei ihrem ersten Gange durch die Stadt ebenso viele Male zu
erdr\"ucken drohten.

An eben jenem Abende sollte da{\s} Weldon-Institut seine gewohnte
w\"ochentliche Sitzung abhalten. Man rechnete darauf, die beiden
Collegen ihre fr\"uheren Pl\"atze wieder einnehmen zu sehen. Da sie
\"ubrigen{\s} von ihren Abenteuern bi{\s}her noch nicht{\s} erz\"ahlt
hatten -- vielleicht hatte der Zudrang der Leute ihnen gar nicht die
n\"othige Zeit gew\"ahrt -- so hoffte man auch, da{\ss} sie nun von
den gehabten Eindr\"ucken w\"ahrend jener unfreiwilligen Reise
berichten w\"urden.

In der That hatten sich Beide au{\s} irgend welchem Grunde bi{\s}her
ganz stumm verhalten, und stumm blieb auch der Diener Frycollin, den
seine Stamme{\s}genossen vor toller Erregung fast geviertheilt
h\"atten.

Wa{\s} die beiden Collegen noch nicht gesagt und vielleicht hatten
sagen wollen, war Folgende{\s}:

Wir brauchen wohl kaum auf die dem Leser bekannten Vorg\"ange in der
Nacht vom 27. zum 28. Juli zur\"uck zu kommen; auf die k\"uhn
au{\s}gef\"uhrte Flucht de{\s} Vorsitzenden und de{\s}
Schriftf\"uhrer{\s} vom Weldon-Institut, auf ihre lebhafte Erregung
bei Durchwanderung der felsigen Insel Chatam, den auf Phil Evan{\s}
abgefeuerten Gewehrschu{\ss}, auf da{\s} durchschnittene Ankertau und
den {\glqq}Albatro{\s}{\grqq}, der damal{\s}, seiner Triebschrauben
entbehrend, durch den S\"udwestwind weit fortgetrieben und
gleichzeitig zu gro{\ss}er H\"ohe gewisserma{\ss}en emporgeschnellt
wurde. Darauf war derselbe bald au{\s} ihrem Gesicht{\s}krei{\s}
entschwunden.

Die Fl\"uchtlinge hatten nun nicht{\s} mehr zu f\"urchten. Wie
h\"atte Robur nach der Insel zur\"uckkehren k\"onnen, da seine
Schrauben noch drei bi{\s} vier Stunden au{\ss}er Stande waren, zu
functioniren?

Nach Ablauf dieser Zeit aber mu{\ss}te der durch die Explosion
zerst\"orte, {\glqq}Albatro{\s}{\grqq} zum elenden, auf dem Meere
treibenden Wrack geworden sein, und Diejenigen, welche er trug, waren
jedenfall{\s} nur noch in St\"ucke gerissene Leichen, die auch der
Ocean nicht wieder herau{\s}geben konnte.

Der entsetzliche Racheact mu{\ss}te dann vollkommen gelungen sein. Da
Onkel Prudent und Phil Evan{\s} sich al{\s} im Stande der Nothwehr
betrachteten, litten sie wegen dieser That an keinen
Gewissen{\s}bissen.

Phil Evan{\s} war durch die vom {\glqq}Albatro{\s}{\grqq} au{\s}
entsendete Kugel nur leicht verletzt worden. Alle Drei wanderten also
am Ufer hinauf, in der Hoffnung, Eingeborene anzutreffen.

Diese Hoffnung sollte nicht get\"auscht werden. Etwa f\"unfzig
halbwilde, vom Fischfange lebende Einwohner siedelten an der
Westk\"uste Chatam{\s}. Sie hatten den Aeronef nach der Insel
herabkommen sehen und bereiteten den Fl\"uchtlingen einen Empfang,
wie sie ihn al{\s} \"ubernat\"urliche Wesen verdienten. Man betete
sie an, mindesten{\s} fehlte daran nicht viel, und brachte sie in der
gr\"o{\ss}ten und sch\"onsten H\"utte unter. Frycollin fand gewi{\ss}
niemal{\s} wieder eine solche Gelegenheit, die Rolle al{\s} Gott der
Schwarzen spielen zu k\"onnen.

Wie sie vorau{\s}gesetzt, sahen Onkel Prudent und Phil Evan{\s} den
Aeronef nicht wieder zur\"uckkehren, und mu{\ss}ten darau{\s}
schlie{\ss}en, da{\ss} die schreckliche Katastrophe in gro{\ss}er
H\"ohe eingetreten sein werde. Nun w\"urde Niemand wieder von dem
Ingenieur Robur reden h\"oren, so wenig wie von seiner wunderbaren
Maschine, die seine Leute mit ihm dahingetragen hatte.

Jetzt galt e{\s} nur noch, eine Gelegenheit zur R\"uckkehr nach
Amerika abzuwarten, denn die Insel Chatam wird von Seefahrern wenig
besucht. So verstrich der ganze Monat August und die Fl\"uchtlinge
legten sich schon die Frage vor, ob sie am Ende nicht blo{\ss} ein
Gef\"angni{\ss} gegen ein andere{\s} eingetauscht h\"atten, mit dem
\"ubrigen{\s} Frycollin sich weit besser, al{\s} mit dem
{\glqq}Kerker in der Luft{\grqq}, abzufinden schien.

Endlich am 3. September erschien ein Schiff, um an der Insel Chatam
Wasser einzunehmen. Der Leser hat jedenfall{\s} nicht vergessen,
da{\ss} Onkel Prudent zur Zeit der Entf\"uhrung au{\s} Philadelphia
mehrere tausend Dollar{\s} Papiergeld bei sich f\"uhrte, d.~h. mehr
al{\s} nothwendig war, um nach Amerika zur\"uckkehren zu k\"onnen.
Nachdem sie ihren Verehrern, welche ihnen stet{\s} den
allergr\"o{\ss}ten Respect bewiesen hatten, herzlich gedankt,
schifften sich Onkel Prudent, Phil Evan{\s} und Frycollin nach
Aukland ein. Von ihren Schicksalen erz\"ahlten sie nicht{\s}, und
nach zwei Tagen schon langten sie in der Hauptstadt Neu-Seeland{\s}
an.

Hier nahm sie ein Packetboot de{\s} Stillen Ocean{\s} al{\s}
Passagiere auf, und am 20. September landeten die Ueberlebenden vom
{\glqq}Albatro{\s}{\grqq} nach h\"ochst gl\"ucklicher Ueberfahrt in
San Franci{\s}co. Sie hatten weder au{\s}gesprochen, wer sie waren,
noch woher sie kamen: doch da sie einen recht anst\"andigen Prei{\s}
f\"ur ihre Pl\"atze entrichteten, so w\"are e{\s} keinem
amerikanischen Capit\"an jemal{\s} eingefallen, weitere Fragen an die
Leute zu richten.

In San Franci{\s}co ben\"utzten Onkel Prudent, sein College und der
Diener Frycollin den ersten Zug der gro{\ss}en Pacific-Bahn und
trafen am 27. wohlbehalten in Philadelphia ein.

Da{\s} ist der gedr\"angte Bericht \"uber Alle{\s}, wa{\s} seit dem
Entweichen der Fl\"uchtlinge und ihrer Abfahrt von der Insel Chatam
vorgefallen war; und somit konnten an jenem Abende der Vorsitzende
und der Schriftf\"uhrer, inmitten eine{\s} ungeheuren Zudrang{\s},
ihre Pl\"atze im Weldon-Institut wieder einnehmen.

Niemal{\s} aber hatte weder der Eine, noch der Andere eine so
auf\/fallende Ruhe zur Schau getragen. Ihr Anblick allein h\"atte
niemal{\s} ahnen lassen, da{\ss} seit jener denkw\"urdigen Sitzung
vom 12. Juni irgend etwa{\s} Besondere{\s} vorgefallen sei. Diese
dreiundeinhalb Monate schienen in ihrem Leben gar nicht mit zu
z\"ahlen.

Nach den ersten Begr\"u{\ss}ung{\s}salven, welche Beide ohne da{\s}
Zucken nur eine{\s} Gesicht{\s}mu{\s}kel{\s} hinnahmen, bedeckte
Onkel Prudent da{\s} Haupt und ergriff er zuerst da{\s} Wort.

{\glqq}Ehrenwerthe B\"urger, sagte er, die Sitzung ist
er\"offnet.{\grqq}

Wahnsinniger und gewi{\ss} wohlberechtigter Beifall, denn wenn e{\s}
auch al{\s} etwa{\s} Au{\ss}ergew\"ohnliche{\s} nicht gelten konnte,
da{\ss} eine solche Wochenversammlung er\"offnet wurde, so erhielt
der Umstand doch ein au{\ss}ergew\"ohnliche{\s} Gewicht, da{\ss}
da{\s} durch Onkel Prudent unter Assistenz von Phil Evan{\s} geschah.

Der Vorsitzende lie{\ss} den in Zurufen und H\"andeklatschen
kundgegebenen Enthusia{\s}mu{\s} sich ruhig au{\s}toben. Dann fuhr er
fort:

{\glqq}In unserer letzten Sitzung, meine Herrn, kam e{\s} zu recht
lebhaftem Meinung{\s}au{\s}tausch (H\"ort! H\"ort!) zwischen den
Vertretern der Vorder- und der R\"uckenschraube f\"ur unseren Ballon,
den \begin{antiqua}Go a head\end{antiqua}. (Zeichen von
Verwunderung.) Wir haben inzwischen ein Au{\s}kunft{\s}mittel
erfunden, um die Vorder- und Hintersteuerer unter einen Hut zu
bringen, und da{\s} besteht einfach darin: Wir versehen eben beide
Enden de{\s} Nachen{\s} mit je einer Triebschraube.{\grqq}
(Stillschweigen vor allgemeinem Erstaunen.)

Da{\s} war Alle{\s}!

Ja, Alle{\s}, von der Entf\"uhrung de{\s} Vorsitzenden und de{\s}
Schriftf\"uhrer{\s} de{\s} Weldon-Institut{\s} fiel kein
Sterben{\s}w\"ortchen; kein Wort \"uber den Ingenieur Robur und den
{\glqq}Albatro{\s}{\grqq}; kein Wort \"uber die Art und Weise, wie
die Gefangenen hatten entkommen k\"onnen, und endlich kein Wort
\"uber da{\s} Schicksal de{\s} Aeronef{\s}, ob er noch durch da{\s}
Luftmeer schwebe und ob noch weitere Angriffe gegen Mitglieder de{\s}
Club{\s} zu bef\"urchten w\"aren.

Gewi{\ss} fehlte e{\s} den Ballonisten nicht an Lust, Onkel Prudent
und Phil Evan{\s} au{\s}zufragen; sie sahen dieselben aber so ernst,
so zugekn\"opft, da{\ss} e{\s} angezeigt schien, ihre Zur\"uckhaltung
zu respectiren. Wenn sie die Zeit zum Sprechen gekommen meinten,
w\"urden sie schon allein sprechen und Alle w\"urden sich geehrt
genug f\"uhlen, ihnen zuzuh\"oren.

Uebrigen{\s} konnte unter diesem Stillschweigen ja noch ein
Geheimni{\ss} verborgen liegen, da{\s} heute noch nicht enth\"ullt
werden durfte.

Da nahm Onkel Prudent unter einem, bi{\s}her bei den Sitzungen de{\s}
Weldon-Institut{\s} unerh\"orten Stillschweigen wieder da{\s} Wort.

{\glqq}Meine Herren, sagte er, e{\s} er\"ubrigt un{\s} nun blo{\ss}
noch, den Aerostaten \begin{antiqua}Go a head\end{antiqua}, der
bestimmt ist, sich da{\s} Luftmeer zu erobern, schleunigst der
Vollendung entgegen zu f\"uhren. -- Die Sitzung ist geschlossen.{\grqq}



\newpage\begin{center}\label{kap18}
{\large \begin{antiqua}XVIII.\end{antiqua}\\
Welche{\s} diese wahrhafte Geschichte zu Ende f\"uhrt, ohne sie zu
beendigen.\\\bigskip}
\end{center}



Am 29. April de{\s} folgenden Jahre{\s}, sieben Monate nach der so
unerwarteten R\"uckkehr de{\s} Onkel Prudent und Phil Evan{\s}, war
ganz Philadelphia in reger Bewegung. Um politische Fragen handelte e{\s}
sich dabei nicht, ebenso wenig um Wahlen oder Volk{\s}versammlungen.
Der auf Betreiben de{\s} Weldon-Institut{\s} nun vollendete Aerostat
\begin{antiqua}Go a head\end{antiqua} sollte endlich seinem
nat\"urlichen Element \"ubergeben werden.

Al{\s} Aeronaut f\"ur denselben war der ber\"uhmte Harry W. Tinder,
dessen wir schon zu Anfang dieser Erz\"ahlung erw\"ahnten, bestimmt
worden, und ihm hatte man noch einen erfahrenen Gehilfen beigegeben.

Die Passagiere bildeten der Vorsitzende und der Schriftf\"uhrer
de{\s} Weldon-Institut{\s}, denen diese Ehre gewi{\ss} vor allen
Anderen zukam, da e{\s} f\"ur sie so zu sagen eine Leben{\s}aufgabe
geworden war, pers\"onlich gegen jeden Apparat, der auf dem Principe
{\glqq}Schwerer, al{\s} die Luft{\grqq} beruhte, Einspruch zu
erheben.

Doch auch jetzt, nach sieben Monaten, sollten sie immer noch erst
anfangen, \"uber ihre Abenteuer zu berichten. Selbst Frycollin hatte,
wie sehr e{\s} ihn auch dazu dr\"angte, noch nicht vom Ingenieur
Robur und von dessen wunderbarer Maschine gesprochen. Offenbar
wollten Onkel Prudent und Phil Evan{\s} al{\s} eingefleischte und
unverbesserliche Ballonisten \"uberhaupt nicht, da{\ss} von dem
Aeronef oder einer anderen Flugmaschine jemal{\s} die Rede sei. Auch
wenn ihr Ballon, der \begin{antiqua}Go a head\end{antiqua}, noch
nicht die erste Stelle unter den zur Fortbewegung durch die Luft
bestimmten Apparaten einnehmen sollte, so wollten sie doch keine, von
irgendwelchem Anh\"anger der Aviation herr\"uhrende Erfindung dabei
anwenden lassen. Sie glaubten noch immer und wollten auch sp\"ater
nur glauben, da{\ss} da{\s} einzig wahre atmosph\"arische Vehikel der
Aerostat sei, und da{\ss} ihm allein die Zukunft geh\"ore.

Uebrigen{\s} existirte ja Derjenige, an dem sie eine so furchtbare,
ihrer Ansicht nach aber nur gerechte Rache genommen hatten, jetzt
schon l\"angst nicht mehr. Keiner von Denen, die er trug, hatte
seinen Untergang \"uberleben k\"onnen. Da{\s} Geheimni{\ss} de{\s}
{\glqq}Albatro{\s}{\grqq} lag jetzt in den unergr\"undlichen Tiefen
de{\s} Stillen Ocean{\s} begraben.

Die Annahme, da{\ss} der Ingenieur Robur einen Zuflucht{\s}ort, eine
rettende Insel im ungeheuren, verlassenen Ocean gefunden habe,
erschien nur al{\s} eine sehr gewagte Hypothese. Die beiden Collegen
behielten sich f\"ur sp\"ater die Entscheidung dar\"uber vor, ob
e{\s} angezeigt erscheine, nach dieser Richtung besondere
Nachforschungen zu veranlassen.

Man schritt also endlich zu dem gro{\ss}en Experimente, welche{\s}
da{\s} Weldon-Institut so lange Zeit und mit so gro{\ss}er Sorgfalt
vorbereitet hatte. Der \begin{antiqua}Go a head\end{antiqua} war der
vollendetste Typu{\s} dessen, wa{\s} im Bereiche der Aerostatik
bi{\s}her erfunden war -- da{\s}selbe wie der
\begin{antiqua}{\glqq}Inflexible{\grqq}\end{antiqua} und der
\begin{antiqua}{\glqq}Formidable{\grqq}\end{antiqua} (zwei neuere
franz\"osische Panzerschlachtschiffe) in der Schiff{\s}baukunst.

Der \begin{antiqua}Go a head\end{antiqua} besa{\ss} alle f\"ur einen
Aerostaten nur w\"unschen{\s}werthen Eigenschaften. Sein Volumen
gestattete ihm, bi{\s} zu den allergr\"o{\ss}ten H\"ohen, die ein
Ballon nur erreichen kann, aufzusteigen; seine Undurchl\"assigkeit
f\"ur Ga{\s}, sich unbegrenzt lange in der Luft zu erhalten; seine
Festigkeit, jeder Au{\s}dehnung der Gase ebenso zu widerstehen, wie
dem heftigsten Platzregen und st\"arksten Sturmwinde; sein
Fassung{\s}verm\"ogen, eine gen\"ugende Auftrieb{\s}kraft zu
entfalten, um au{\ss}er dem sonst n\"othigen Zubeh\"or eine
elektrische Maschine mit\/zunehmen, die seinen Propellern eine, jeder
bi{\s}her erreichten \"uberlegene Treibkraft verleihen konnte. Der
\begin{antiqua}Go a head\end{antiqua} hatte eine l\"angliche Gestalt,
um die horizontale Fortbewegung zu erleichtern. Seine Gondel, eine
derjenigen de{\s} Ballon{\s} der Capit\"ane Kreb{\s} und Renard
\"ahnliche Plattform, enthielt alle{\s} f\"ur Luftschiffer nothwendige
Ger\"ath und Werkzeug, physikalische Instrumente, Taue, Anker, Rollen
u.~s.~w., au{\ss}erdem die Apparate, Batterien und Accumulatoren,
welche seine mechanische Kraft lieferten. Diese Gondel trug vorne
eine Schraube und hinten neben einer gleichen Schraube ein
Steuerruder. Aller Wahrscheinlichkeit nach mu{\ss}te jedoch die
Arbeit{\s}leistung der Maschinen de{\s} \begin{antiqua}Go a
head\end{antiqua} weit hinter der der Apparate de{\s}
{\glqq}Albatro{\s}{\grqq} zur\"uckbleiben.

Der \begin{antiqua}Go a head\end{antiqua} war nach vollendeter
F\"ullung nach der Waldbl\"o{\ss}e im Fairmont-Park \"ubergef\"uhrt
worden, d.~h. genau nach derselben Stelle, an welcher fr\"uher der
Aeronef einige Stunden gelegen hatte.

Wir brauchen wohl nicht zu betonen, da{\ss} ihm die Auftrieb{\s}kraft
durch da{\s} leichteste aller Gase verliehen worden war. Da{\s}
gew\"ohnliche Leuchtga{\s} entwickelt per Cubikmeter nur eine solche
Hebekraft gleich 700 Gramm -- wa{\s} gegen die umgebende Luft nur
einen unbetr\"achtlichen Gewicht{\s}unterschied darstellt. Da{\s}
Wasserstoffga{\s} dagegen besitzt bei gleichem Volumen eine auf etwa
1100 Gramm zu sch\"atzende Steigekraft. Solche{\s}, nach dem
Verfahren und in den Special-Apparaten de{\s} ber\"uhmten Henry
Giffard dargestellte reine Wasserstoffga{\s} erf\"ullte den
ungeheuren Ballon. Da der \begin{antiqua}Go a head\end{antiqua} nun
einen Fassung{\s}raum von 40.000 Cubikmetern besa{\ss}, so entsprach
die Steigkraft seine{\s} Gase{\s} einem Gewichte von 40.000mal 1100
Gramm oder 44.000 Kilogramm.

Am Morgen de{\s} 20. April war Alle{\s} bereit. Um elf Uhr schon
schwankte der riesige Aerostat wenige Fu{\ss} \"uber dem Boden und
fertig, sich in die Luft zu erheben, majest\"atisch hin und her.

E{\s} herrschte ein pr\"achtige{\s} und wie f\"ur diesen Versuch
eigen{\s} gemachte{\s} Wetter. Vielleicht w\"are eine etwa{\s}
gr\"o{\ss}ere Windst\"arke w\"unschen{\s}werther gewesen, da sie die
Probe mehr beweisend gestaltet h\"atte. Man hat ja niemal{\s}
bezweifelt, da{\ss} ein Ballon in ganz ruhiger Luft nach Belieben
gelenkt werden k\"onne, in bewegter Atmosph\"are ist da{\s} aber ein
andere{\s} Ding und nur unter solchen Verh\"altnissen sollten
derartige Proben au{\s}gef\"uhrt werden.

Genug, jetzt war weder Wind zu versp\"uren, noch deutete etwa{\s}
darauf hin, da{\ss} solcher auftreten w\"urde. An jenem Tage sendete
Nordamerika au{\s} seinem unersch\"opflichen Vorrathe
au{\s}nahm{\s}weise keinen Sturm nach dem westlichen Europa, und
niemal{\s} h\"atte ein Tag g\"unstiger al{\s} dieser zur Vornahme
eine{\s} solchen aeronautischen Experimente{\s} gew\"ahlt werden
k\"onnen.

Kaum brauchen wir der ungeheuren, im Fairmont-Park aufgestauten
Menschenmenge, ebenso wenig der zahlreichen Bahnz\"uge zu erw\"ahnen,
welche Str\"ome von Neugierigen au{\s} allen Nachbarstaaten \"uber
Philadelphia ergossen hatten; auch nicht der Unterbrechung jeder
industriellen und commerciellen Th\"atigkeit, welche e{\s} Allen --
Chef{\s}, Beamten, Handwerkern, M\"annern und Frauen, Greisen und
Kindern, Congre{\ss}mitgliedern, Vertretern der bewaffneten Macht,
Magistrat{\s}personen, Reportern, wei{\ss}en und schwarzen
Eingeborenen, die auf der Waldbl\"o{\ss}e zusammengelaufen waren --
gestattete, diesem Schauspiele beizuwohnen. Oder sollten wir da{\s}
ger\"auschvolle Durcheinanderwogen dieser Volk{\s}mengen schildern,
die unerwarteten Bewegungen, da{\s} pl\"otzliche Dr\"angen und da{\s}
Jauchzen und Rufen de{\s} Mob{\s}? Sollen wir die Hipp! Hipp! Hipp!
nachz\"ahlen, welche von allen Seiten gleich dem Krachen von
Feuerwerk{\s}k\"orpern laut wurden, al{\s} Onkel Prudent und Phil
Evan{\s} auf der mit dem amerikanischen Sternenbanner geschm\"uckten
Plattform erschienen? Oder m\"u{\ss}ten wir e{\s} erst besonder{\s}
au{\s}sprechen, da{\ss} der gr\"o{\ss}te Theil dieser Neugierigen
vielleicht nicht gekommen war, um den \begin{antiqua}Go a
head\end{antiqua} zu sehen, sondern um sich die zwei
au{\ss}erordentlichen M\"anner zu betrachten, um welche die Alte Welt
die Neue beneidete?

Warum aber nur Zwei und nicht Drei? Warum nicht auch Frycollin? --
Da{\s} kam daher, da{\ss} Frycollin die Reise mit dem
{\glqq}Albatro{\s}{\grqq} f\"ur seine Ber\"uhmtheit al{\s} gen\"ugend
erachtete und er die Ehre, seinen Herrn zu begleiten, bescheiden
abgelehnt hatte. Er bekam also keinen Theil von den tollen
Jubelrufen, welche den Vorsitzenden und den Schriftf\"uhrer de{\s}
Weldon-Institut{\s} empfingen. E{\s} versteht sich von selbst,
da{\ss} von allen Mitgliedern der ber\"uhmten Gesellschaft keiner auf
dem f\"ur diese reservirten Platze innerhalb der Pf\"ahle und Leinen
fehlte, welche einen Theil der Lichtung abgrenzten. Hier waren Truk
Milnor, Bat T. Fyn, William T. Forbe{\s}, der seine beiden T\"ochter
Mi{\ss} Doll und Mi{\ss} Mat an den Armen f\"uhrte. Alle waren
erschienen, um durch ihre Anwesenheit zu bekr\"aftigen, da{\ss}
nicht{\s} jemal{\s} im Stande sei, die Anh\"anger de{\s}
{\glqq}Leichter, al{\s} die Luft{\grqq} zu trennen.

Gegen elf Uhr zwanzig Minuten verk\"undigte ein Kanonenschu{\ss} die
Beendigung der letzten Vorbereitungen.

Der \begin{antiqua}Go a head\end{antiqua} erwartete nur noch da{\s}
Signal zum Aufsteigen.

Ein zweiter Kanonenschu{\ss} donnerte um elf Uhr f\"unfundzwanzig.

Der nur noch durch seine Leitseile gehaltene \begin{antiqua}Go a
head\end{antiqua} erhob sich gegen f\"unfzehn Meter \"uber die
Lichtung. Am anderen Ende der Plattform stehend, legten Onkel Prudent
und Phil Evan{\s} die linke Hand auf die Brust, wa{\s} bedeuten
sollte, da{\ss} sie mit dem Zuschauerkreise eine{\s} Herzen{\s}
w\"aren. Dann streckten sie die rechte Hand nach dem Zenith au{\s},
um anzudeuten, da{\ss} der gr\"o{\ss}te, bi{\s} jetzt bekannte Ballon
endlich in Begriff stehe, von seinem \"uberirdischen Reiche Besitz zu
ergreifen.

Da legten sich hunderttausend H\"ande auf hunderttausend Br\"uste;
und hunderttausend andere erhoben sich zum Himmel.

Um elf Uhr drei{\ss}ig krachte ein dritter Kanonenschu{\ss}.

{\glqq}Alle{\s} lo{\s}!{\grqq} rief Onkel Prudent, die hergebrachte
Reden{\s}art ben\"utzend.

Und der \begin{antiqua}Go a head\end{antiqua} erhob sich
{\glqq}majest\"atisch{\grqq} -- da{\s} immer gebrauchte Beiwort in
der Beschreibung von beginnenden Luftfahrten.

In der That, e{\s} war ein pr\"achtige{\s} Schauspiel! Man h\"atte
ein Seeschiff zu sehen gemeint, da{\s} eben vom Stapel lief. Und war
da{\s} hier nicht auch ein Schiff, da{\s} in'{\s} Luftmeer abgelassen
wurde?

Der \begin{antiqua}Go a head\end{antiqua} stieg genau lothrecht in
die H\"ohe -- ein Bewei{\s} f\"ur die vollkommene Ruhe der
Atmosph\"are -- und hielt etwa zweihundertf\"unfzig Meter \"uber der
Erde still.

Hier begann nun die Vorf\"uhrung der Fahrt in wagrechter Richtung.

Der von seinen zwei Schrauben getriebene \begin{antiqua}Go a
head\end{antiqua} zog mit der Geschwindigkeit von zehn Metern in der
Secunde der Sonne entgegen. Da{\s} ist die Geschwindigkeit de{\s}
Walfische{\s} im freien Wasser. E{\s} ist auch gar nicht falsch,
jenen mit dem genannten Riesen der n\"ordlichen Meere zu vergleichen,
zumal da er auch die Gestalt jene{\s} Cetaceer{\s} hatte.

Eine neue Salve von Hurrah{\s} drang zu den geschickten Aeronauten
empor.

Hierauf f\"uhrte der \begin{antiqua}Go a head\end{antiqua} unter der
Wirkung seine{\s} Steuer{\s} allerlei krei{\s}f\"ormige, schiefe und
geradlinige Bewegungen au{\s}, welche ihm die Hand seine{\s}
Steuermanne{\s} aufn\"othigte. Er wendete in engem Kreise, ging nach
vorw\"art{\s}, nach r\"uckw\"art{\s}, um selbst die z\"ahesten
Widersacher der Lenkbarkeit von Ballon{\s} eine{\s} Besseren zu
belehren ... wenn e{\s} solche Widersacher hier gab! Und wenn e{\s}
dergleichen gegeben h\"atte, h\"atte man sie in die Pfanne gehauen!

Warum fehlte aber der Wind diesem herrlichen Experimente? Da{\s} war
bedauerlich. Unzweifelhaft h\"atte der \begin{antiqua}Go a
head\end{antiqua} alle Bewegungen ohne Z\"ogern au{\s}gef\"uhrt,
indem er entweder eine schr\"age Richtung einhielt, wie ein Schiff,
da{\s} dicht beim Winde segelte, oder der Luftstr\"omung gleich einem
Dampfer gerade entgegentrieb.

In diesem Augenblicke stieg der Aerostat um einige hundert Meter
h\"oher hinauf.

Man begreift wohl die Absicht. Onkel Prudent und seine Begleiter
suchten in den h\"oheren Luftschichten eine Str\"omung zu finden, um
die Probe zu vervollst\"andigen. Ein System von inneren Ballon{\s},
entsprechend der Schwimmblase der Fische, in welche man mittelst
Pumpen eine gewisse Menge Ga{\s} hineindr\"ucken konnte, gestattete
ihm n\"amlich, auf- und niederzusteigen. Ohne je Ballast
au{\s}zuwerfen, um h\"oher, oder Ga{\s} zu verlieren, um tiefer zu
gehen, war er im Stande, sich nach Belieben de{\s} Luftschiffer{\s}
in der Atmosph\"are zu heben oder zu senken. Au{\ss}erdem war er
jedoch am oberen Scheitel mit einem Ventil versehen, f\"ur den Fall,
da{\ss} er einmal sehr schnell herabzugehen gezwungen w\"are. Hier
waren demnach nur bereit{\s} bekannte Mittel vorgesehen, diese aber
bi{\s} zum h\"ochsten Grade der Vollkommenheit entwickelt.

Der \begin{antiqua}Go a head\end{antiqua} erhob sich also in
lothrechter Linie. Durch optische Wirkung verringerten sich seine
Dimensionen allm\"ahlich den Blicken. Gew\"ohnlich erscheint da{\s}
ziemlich merkw\"urdig f\"ur die Zuschauer, die sich, um gerade hinauf
zu sehen, fast die Hal{\s}wirbel brechen. Der ungeheure Walfisch
wurde so nach und nach zum Meerschwein, um endlich bi{\s} zur
Gr\"o{\ss}e de{\s} gew\"ohnlichen Gr\"undling{\s} herabzusinken.

Da die aufsteigende Bewegung nicht unterbrochen wurde, erreichte der
\begin{antiqua}Go a head\end{antiqua} eine H\"ohe von viertausend
Metern, blieb aber bei dem reinen, keine Spur von Dunst enthaltenden
Himmel vollkommen klar sichtbar.

Inde{\ss} hielt er sich fortw\"ahrend \"uber der Lichtung, al{\s}
w\"urde er daselbst von divergirenden Leinen festgehalten. Und wenn
eine riesenhafte Glocke \"uber die Umgegend gest\"urzt gewesen
w\"are, h\"atte die Luft darunter nicht ruhiger sein k\"onnen. Weder
in jener, noch in irgend einer anderen H\"ohe regte sich der leiseste
Hauch. Stark verkleinert durch die Entfernung, al{\s} ob man ihn
durch ein verkehrt gehaltene{\s} Fernrohr betrachtet h\"atte,
man\"ovrirte der Aerostat, ohne den geringsten Widerstand zu finden.

Pl\"otzlich drang ein Aufschrei au{\s} der Menge, ein Schrei, dem
sofort hunderttausend andere folgten. Alle Arme richteten sich nach
einem Punkte am Horizont, und zwar nach Nordwesten hin.

Dort im tiefen Azur ist ein sich bewegender K\"orper erschienen, der
n\"aher herankommt und gr\"o{\ss}er wird. Ist e{\s} ein Vogel, der
mit m\"achtigem Fl\"ugelschlage durch die h\"ochsten Luftschichten
schwebt? Ist'{\s} eine Feuerkugel, deren Bahn die Atmosph\"are in
schiefer Richtung durchschneidet? Jedenfall{\s} ist der r\"athselhaften
Erscheinung eine bedeutende Schnelligkeit eigen und sie mu{\ss} bald
\"uber die erstaunte Volk{\s}menge hinwegrauschen.

Ein Verdacht, der sich gleichsam elektrisch allen Gehirnen mittheilt,
verbreitet sich \"uber die ganze Lichtung.

E{\s} scheint jedoch, al{\s} ob auch der \begin{antiqua}Go a
head\end{antiqua} den fremdartigen Gegenstand bemerkt h\"atte.
Offenbar hat er da{\s} Gef\"uhl einer drohenden Gefahr empfunden,
denn pl\"otzlich steigert sich seine Geschwindigkeit und er flieht
nach Osten hin.

Ja, die Menge hat Alle{\s} begriffen. Ein von einem der Mitglieder
de{\s} Weldon-Institut{\s} au{\s}\-ge\-rufener Name wird von
zweihunderttausend Lippen wiederholt: {\glqq}Der 
{\glqq}Albatro{\s}{\grqq}! ... {\glqq}Albatro{\s}{\grqq}!{\grqq}

In der That, e{\s} ist der {\glqq}Albatro{\s}{\grqq}. Robur ist
e{\s}, der in den H\"ohen de{\s} Himmel{\s} wieder erscheint! Er
ist'{\s}, der gleich einem gigantischen Raubvogel auf den
\begin{antiqua}Go a head\end{antiqua} lo{\s}st\"urzt!

Und neun Monate vorher war der durch die Explosion zersprengte
Aeronef, die Schrauben zerbrochen und da{\s} Verdeck in zwei St\"ucke
zerrissen, doch vernichtet worden. Ohne die wunderbare Besonnenheit
de{\s} Ingenieur{\s}, der die Drehbewegung de{\s} vorderen
Propeller{\s} ver\"anderte, und diesen al{\s} Auftrieb{\s}schraube
wirken lie{\ss}, w\"are die ganze Besatzung de{\s}
{\glqq}Albatro{\s}{\grqq} schon durch die Schnelligkeit de{\s}
Sturze{\s} erstickt worden. Doch wenn sie auch dieser Gefahr
gl\"ucklich entronnen, wie kam e{\s}, da{\ss} sie nicht in den
Fluthen de{\s} Pacifischen Ocean{\s} ertrunken war?

Da{\s} kam daher, da{\ss} die Tr\"ummer de{\s} Verdeck{\s}, die
Fl\"ugel der Triebschrauben, die W\"ande der Ruff{\s} und wa{\s}
sonst noch vom {\glqq}Albatro{\s}{\grqq} \"ubrig war, ihn zur
schwimmenden Seetrift verwandelt hatten. Der verwundete Vogel war
in'{\s} Wasser gefallen, seine Fl\"ugel aber hielten ihn noch auf den
Wellen. Einige Stunden lang blieben Robur und seine Leute noch auf
diesem Wrack, dann fl\"uchteten sie in da{\s} auf dem Ocean wieder
gefundene Kautschukboot.

Die Vorsehung, f\"ur Diejenigen, welche an einen g\"ottlichen
Eingriff in irdische Dinge glauben -- der Zufall, f\"ur Diejenigen,
welche die Schw\"ache haben, an keine Vorsehung zu glauben -- kam den
Schiffbr\"uchigen zu Hilfe.

Wenige Stunden nach Sonnenaufgang wurden sie von einem Schiffe
bemerkt, da{\s} nicht allein Robur und seine Leute, sondern auch die
umher schwimmenden Tr\"ummer de{\s} Aeronef{\s} aufnahm. Der
Ingenieur begn\"ugte sich mit der Angabe, sein Fahrzeug sei durch
eine Collision zerst\"ort worden, und sein Incognito blieb auch bei
dieser Gelegenheit gewahrt.

Jene{\s} Schiff war ein englischer Dreimaster, der
{\glqq}\begin{antiqua}Two Friends\end{antiqua}{\grqq} von
Liverpool. E{\s} segelte nach Melbourne, wo e{\s} nach wenigen Tagen
eintraf.

Nun war man zwar in Australien, aber sehr fern von der Insel X, nach
der man doch baldigst zur\"uckkehren mu{\ss}te.

Unter den Tr\"ummern de{\s} hinteren Ruff{\s} hatte der Ingenieur
noch eine betr\"achtliche Geldsumme gefunden, die ihm, ohne einen
Anderen anzusprechen, alle Bed\"urfnisse seiner Leute zu bestreiten
gestattete. Kurz nach der Ankunft in Melbourne erwarb er eine kleine
Goelette von hundert Tonnen und auf dieser begab sich Robur, der auch
ein t\"uchtiger Seemann war, nach der Insel X zur\"uck.

Jetzt erf\"ullte ihn nur noch eine einzige fixe Idee -- sich zu
r\"achen. Doch um da{\s} zu k\"onnen, mu{\ss}te ein zweiter
{\glqq}Albatro{\s}{\grqq} gebaut werden, wa{\s} f\"ur den, der den
ersten construirt hatte, ja eine leichte Aufgabe war. Man verwendete
dabei, wa{\s} noch vom alten Aeronef brauchbar erschien, unter
anderen Maschinentheilen auch dessen Propeller, die mit allen
Tr\"ummern auf der Goelette verladen gewesen waren. Der
Mechani{\s}mu{\s} wurde mittelst neuer Batterien und Accumulatoren
wieder in Stand gesetzt. Kurz, binnen weniger al{\s} acht Monaten war
die ganze Arbeit beendigt und ein neuer {\glqq}Albatro{\s}{\grqq},
ganz gleich dem durch die Explosion zerst\"orten und eben so
m\"achtig wie dieser, stand bereit, durch die Luft abzusegeln.

Selbstverst\"andlich trug er auch dieselbe Mannschaft und ebenso
selbstverst\"andlich sch\"aumte diese Mannhaft vor Wuth auf Onkel
Prudent und Phil Evan{\s} im Besonderen, wie auf da{\s} ganze
Weldon-Institut im Allgemeinen.

Mit den ersten Tagen de{\s} April verlie{\ss} der
{\glqq}Albatro{\s}{\grqq} die Insel X. W\"ahrend dieser Luftfahrt
sollte sein Vor\"uberkommen von keinem Punkt der Erde au{\s} gemeldet
werden k\"onnen. So schwebte er also immer zwischen den Wolken hin.
Ueber Nordamerika an einer Ein\"ode de{\s} Far-West angelangt ging er
zur Erde. Da{\s} tiefste Incognito bewahrend, erfuhr der Ingenieur
hier, wa{\s} ihm da{\s} gr\"o{\ss}te Vergn\"ugen gew\"ahren
mu{\ss}te: da{\ss} da{\s} Weldon-Institut nun so weit sei, mit seinen
Probefahrten zu beginnen, und da{\ss} der \begin{antiqua}Go a
head\end{antiqua} mit Onkel Prudent und Phil Evan{\s} am 29. April
von Philadelphia au{\s} aufsteigen sollte.

Welch' herrliche Gelegenheit zur K\"uhlung jener Rache, die da{\s}
Herz Robur'{\s} und aller seiner Leute erf\"ullte! Eine schreckliche
Rache, welcher der \begin{antiqua}Go a head\end{antiqua} nicht
entrinnen sollte! Eine \"offentliche Rache, welche gleichzeitig die
Ueberlegenheit de{\s} Aeronef{\s} \"uber die Aerostaten und alle
Apparate dieser Art beweisen mu{\ss}te!

Au{\s} diesem Grunde also erschien an jenem Tage gleich dem Geier,
der au{\s} schwindelnder H\"ohe niederschie{\ss}t, der Aeronef \"uber
dem Fairmont-Park.

Ja, da{\s} war der {\glqq}Albatro{\s}{\grqq}, leicht erkannt selbst
von Denen, die ihn fr\"uher nie gesehen hatten.

Der \begin{antiqua}Go a head\end{antiqua} floh noch immer. Er begriff
jedoch, da{\ss} er durch eine Flucht in horizontaler Richtung
niemal{\s} zu entkommen verm\"oge. Er suchte sein Heil also in
verticaler Flucht, aber nicht durch Ann\"aherung an die Erde, denn da
h\"atte der Aeronef ihm den Weg verlegen k\"onnen, sondern indem er
sich in die Luft erhob, nach einer Zone, in der er vielleicht nicht
angegriffen werden konnte. Da{\s} war sehr k\"uhn, doch gleichzeitig
recht logisch gehandelt.

Inzwischen erhob sich aber auch der {\glqq}Albatro{\s}{\grqq} mit
ihm. Weit kleiner al{\s} der \begin{antiqua}Go a head\end{antiqua}
glich er dem Schwertfisch bei Verfolgung de{\s} Wal{\s}, den er mit
seinem Stachel durchbohrt, oder dem auf da{\s} Panzerschiff
zufliegenden Torpedo, der jene{\s} mit einem Schlage in die Luft zu
sprengen trachtet.

Die Zuschauer bemerkten da{\s} mit beklemmender Angst. Binnen wenigen
Augenblicken hatte der Aerostat eine H\"ohe von f\"unftausend Metern
erreicht. Der {\glqq}Albatro{\s}{\grqq} war ihm bei seiner
aufsteigenden Bewegung nachgefolgt. Er t\"anzelte jetzt gleichsam um
seine Seiten und umkreiste ihn in stetig vermindertem Umfange. Mit
einem Sprung konnte er ihn vernichten, indem er seine d\"unne H\"ulle
zerri{\ss}. Onkel Prudent und dessen Begleiter w\"aren durch einen
furchtbaren Absturz rein zerschmettert worden.

Die vor Schreck verstummten und nach Athem ringenden Zuschauer waren
von jener Art Entsetzen gepackt, da{\s} die Brust einschn\"urt und
die F\"u{\ss}e l\"ahmt, wenn man Einen au{\s} gro{\ss}er H\"ohe
herabst\"urzen sieht. Jetzt drohte ein Luftkampf, ein Kampf, der
nicht einmal die geringen Au{\s}sichten f\"ur Rettung wie ein
Wasserkampf darbot -- der erste dieser Art, aber gewi{\ss} nicht der
letzte, denn der Fortschritt geh\"ort zu den ehernen Gesetzen dieser
Welt. Und wenn der \begin{antiqua}Go a head\end{antiqua} an seiner
Seite da{\s} amerikanische Sternenbanner trug, so hatte der
{\glqq}Albatro{\s}{\grqq} auch seine Flagge, da{\s} schwarze
Fahnentuch mit der goldenen Sonne Robur de{\s} Sieger{\s}, entfaltet.

Der \begin{antiqua}Go a head\end{antiqua} wollte au{\s} dem Bereiche
seine{\s} Gegner{\s} zu kommen suchen, indem er sich noch weiter
erhob. Er warf den al{\s} Reserve mitgef\"uhrten Ballast au{\s}. Noch
einmal machte er einen Satz von tausend Metern und erschien jetzt nur
noch al{\s} ein Punkt im Luftraum. Der {\glqq}Albatro{\s}{\grqq}, der
ihm mit der gr\"o{\ss}ten Drehgeschwindigkeit seiner Schrauben
nacheilte, war schon v\"ollig unsichtbar geworden.

Pl\"otzlich erhob sich von der Erde ein Schrecken{\s}schrei.

Der \begin{antiqua}Go a head\end{antiqua} nahm sichtlich an
Gr\"o{\ss}e wieder zu, w\"ahrend auch der sich mit ihm senkende
Aeronef auf'{\s} Neue erschien. Jetzt war der Sturz da! Da{\s} in der
furchtbaren H\"ohe zu stark au{\s}gedehnte Ga{\s} hatte die H\"ulle
de{\s} Ballon{\s} gesprengt, und nur noch halb aufgeblasen fiel
dieser rasch herunter.

Der Aeronef dagegen, der nur die Bewegung seiner Auftrieb{\s}schrauben
gem\"a{\ss}igt hatte, sank mit abgemessener Geschwindigkeit herab. Er
fuhr an den \begin{antiqua}Go a head\end{antiqua} heran, al{\s}
dieser nur noch zw\"olfhundert Meter von der Erde entfernt war, und
n\"aherte sich ihm Bord an Bord.

Wollte Robur ihm den Gnadensto{\ss} geben? -- Nein, er wollte helfen,
wollte die Insassen retten!

Seine Man\"ovrirgeschicklichkeit war eine so erstaunliche, da{\ss}
der Aeronaut und sein Genosse auf da{\s} Verdeck de{\s} Aeronef{\s}
gelangen konnten.

Sollten Onkel Prudent und Phil Evan{\s} etwa die Unterst\"utzung
Robur'{\s} ablehnen, e{\s} verweigern, sich von ihm retten zu lassen?
Sie w\"aren e{\s} wahrlich im Stande gewesen! Die Leute de{\s}
Ingenieur{\s} bem\"achtigten sich jedoch derselben und schafften sie
mit Gewalt vom \begin{antiqua}Go a head\end{antiqua} nach dem
{\glqq}Albatro{\s}{\grqq}.

Da machte sich der Aeronef von jenem klar und blieb an derselben
Stelle, w\"ahrend der jetzt v\"ollig ga{\s}leere Ballon auf die
B\"aume neben der Lichtung niederfiel, wo er gleich einem riesigen
Fetzen h\"angen blieb.

Unten herrschte da{\s} Schweigen de{\s} Tode{\s}; e{\s} schien
wirklich, al{\s} wenn alle{\s} Leben au{\s} den Herzen der Menge
entflohen w\"are. Sehr Viele hatten gleich die Augen geschlossen, um
da{\s} Ende der Katastrophe nicht mit anzusehen.

Onkel Prudent und Phil Evan{\s} waren also wiederum die Gefangenen
de{\s} Ingenieur{\s} Robur geworden. Sollte er, nun er sie wieder
erlangt, mit ihnen noch einmal in'{\s} Luftmeer hinau{\s}fliegen,
wohin ihm Keiner folgen konnte?

Da{\s} war vielleicht zu vermuthen.

Indessen senkte sich der {\glqq}Albatro{\s}{\grqq}, statt h\"oher zu
steigen, langsam zur Erde nieder. Man glaubte, er wolle bi{\s}
auf'{\s} Land gehen, und die Menge dr\"angte sich, um ihm Platz zu
machen, au{\s}einander.

Die Erregung der Leute hatte jetzt den h\"ochsten Grad erreicht.

Zwei Meter \"uber der Erde hielt der {\glqq}Albatro{\s}{\grqq} an,
und unter tiefstem Stillschweigen lie{\ss} sich die Stimme de{\s}
Ingenieur{\s} vernehmen:

{\glqq}B\"urger der Vereinigten Staaten, sagte er, der Vorsitzende
und der Schriftf\"uhrer de{\s} Weldon-Institut{\s} sind wiederum in
meiner Gewalt. Hielte ich sie zur\"uck, so w\"urde ich nur von meinem
Rechte der Wiedervergeltung Gebrauch machen. Bei der in ihrer Seele
durch die Erfolge de{\s} {\glqq}Albatro{\s}{\grqq} entfachten
Leidenschaft aber sehe ich ein, da{\ss} ihr geistiger Zustand doch
nicht derart ist, um die Umw\"alzungen, welche die Beherrschung
de{\s} Luftmeere{\s} einst nach sich ziehen mu{\ss}, vollst\"andig zu
begreifen. Onkel Prudent und Phil Evan{\s}, Sie sind frei!{\grqq}

Der Vorsitzende, der Schriftf\"uhrer de{\s} Weldon-Institut{\s}, der
Aeronaut und sein Gehilfe hatten nur einen Sprung zu thun, um auf die
Erde zu gelangen.

Der {\glqq}Albatro{\s}{\grqq} erhob sich dann sofort um etwa zehn
Meter \"uber die Menge und Robur fuhr fort:

{\glqq}B\"urger der Vereinigten Staaten, mein Versuch ist gl\"ucklich
durchgef\"uhrt, doch meine Ansicht geht dahin, nicht{\s} zu
\"ubereilen, auch nicht einmal den Fortschritt. Die Wissenschaft darf
den Lande{\s}sitten und Gewohnheiten nicht zu sehr vorau{\s}eilen.
Die Menschheit soll nur schrittweise, nicht durch gewaltsame
Um\"anderungen vorw\"art{\s} kommen. Ich selbst w\"urde heute noch zu
zeitig auftreten, um alle widerstrebenden und getheilten Interessen
zu vereinigen. Die Nationen sind zum wirklichen Bunde noch nicht reif.

{\glqq}Ich ziehe also weiter und nehme mein Geheimni{\ss} mit mir.
F\"ur die Menschheit wird e{\s} de{\s}halb nicht verloren sein,
sondern ihr dereinst geh\"oren, wenn sie unterrichtet genug sein
wird, darau{\s} Vortheil zu ziehen, und weise genug, um e{\s} nicht
zu mi{\ss}brauchen. Heil Euch, B\"urger der Vereinigten Staaten, Heil
Euch, jetzt und immerdar!{\grqq}

Die Luft mit seinen vierundsiebenzig Schrauben peitschend und von den
beiden mit gr\"o{\ss}ter Kraft arbeitenden Propellern davongetragen,
verschwand der {\glqq}Albatro{\s}{\grqq} im Osten inmitten eine{\s}
Sturme{\s} von Hurrah{\s}, die jetzt der allgemeinen Bewunderung
Au{\s}druck gaben.

Die beiden, jetzt wie da{\s} ganze Weldon-Institut tief
gedem\"uthigten Collegen thaten da{\s} Einzige, wa{\s} sie thun
konnten -- sie schlichen nach ihren Behausungen zur\"uck, w\"ahrend
die Menge infolge einer pl\"otzlichen Sinne{\s}\"anderung nicht
\"ubel Lust zeigte, sie mit jetzt ganz angebrachtem bei{\ss}enden
Spotte zu begr\"u{\ss}en.

\begin{center}
\makebox[15em]{\hrulefill}\bigskip
\end{center}

Nun bleibt noch immer die Frage bestehen: {\glqq}Wer ist jener Robur?
Wird man da{\s} jemal{\s} erfahren?{\grqq}

Man wei{\ss} e{\s} schon heute. Robur ist da{\s} Wissen und K\"onnen
der Zukunft, vielleicht schon de{\s} n\"achsten Tage{\s} -- er ist
der sichere Schatz im Schoo{\ss}e kommender Zeiten.

Da{\ss} der {\glqq}Albatro{\s}{\grqq} noch immer durch die
Erdatmosph\"are hinschwebe, inmitten seine{\s} Reiche{\s}, da{\s} ihm
Niemand streitig machen kann, ist nicht zu bezweifeln; auch Robur der
Sieger wird seinem Versprechen gem\"a{\ss} eine{\s} Tage{\s}
wiederkehren und da{\s} Geheimni{\ss} einer Erfindung offenbaren,
welche die socialen und politischen Verh\"altni{\s} der Erde
g\"anzlich umgestalten d\"urfte.

Wa{\s} die Zukunft der Luftschifffahrt angeht, so geh\"ort diese den
Aeronef{\s}, nicht dem Aerostaten.

Den {\glqq}Albatrossen{\grqq} ist e{\s} noch vorbehalten, sich da{\s}
Reich der Luft endgiltig zu erobern.


\newpage\begin{center}
Inhalt
\end{center}

\parindent-1.5em

\begin{antiqua}I.\end{antiqua} Worin die gelehrte Welt sich ebenso
wenig Rath wei{\ss}, wie die ungelehrte.~\dotfill\pageref{kap01}

\begin{antiqua}II.\end{antiqua} In welchem die Mitglieder de{\s}
Weldon-Institut{\s} mit einander streiten, ohne zu einer
Uebereinstimmung zu gelangen.~\dotfill\pageref{kap02}

\begin{antiqua}III.\end{antiqua} In dem eine neue Pers\"onlichkeit
nicht besonder{\s} vorgestellt zu werden braucht, da sie da{\s}
selbst besorgt.~\dotfill\pageref{kap03}

\begin{antiqua}IV.\end{antiqua} In dem der Verfasser infolge einer
Bemerkung de{\s} Diener{\s} Frycollin den Mond wieder zu Ehren zu
bringen versucht.~\dotfill\pageref{kap04}

\begin{antiqua}V.\end{antiqua} In dem die Einstellung der
Feindseligkeiten zwischen dem Vorsitzenden und dem Schriftf\"uhrer
de{\s} Weldon-Institut{\s} beschlossen wird.~\dotfill\pageref{kap05}

\begin{antiqua}VI.\end{antiqua} Welche{\s} Ingenieure, Mechaniker und
andere Gelehrte vielleicht am besten
\"uberschlagen.~\dotfill\pageref{kap06}

\begin{antiqua}VII.\end{antiqua} In welchem Onkel Prudent und Phil
Evan{\s} sich noch nicht \"uberzeugen lassen
wollen.~\dotfill\pageref{kap07}

\begin{antiqua}VIII.\end{antiqua} Worin man sehen wird, da{\ss} Robur
sich entschlie{\ss}t, auf die ihm vorgelegte wichtige Frage zu
antworten.~\dotfill\pageref{kap08}

\begin{antiqua}IX.\end{antiqua} In dem der {\glqq}Albatro{\s}{\grqq}
fast zehntausend Kilometer zur\"ucklegt und da{\s} mit einem
merkw\"urdigen Sprunge endigt.~\dotfill\pageref{kap09}

\begin{antiqua}X.\end{antiqua} Worin man sehen wird, wie und warum
der Diener Frycollin in'{\s} Schlepptau genommen
wurde.~\dotfill\pageref{kap10}

\begin{antiqua}XI.\end{antiqua} In dem die Wuth de{\s} Onkel Prudent
mit dem Quadrat der Geschwindigkeit zunimmt.~\dotfill\pageref{kap11}

\begin{antiqua}XII.\end{antiqua} In dem der Ingenieur Robur handelt,
al{\s} ob er sich um einen der Monthyon-Preise bewerben
wollte.~\dotfill\pageref{kap12}

\begin{antiqua}XIII.\end{antiqua} In den Onkel Prudent und Phil
Evan{\s} einen ganzen Ocean durchfahren, ohne die Seekrankheit zu
bekommen.~\dotfill\pageref{kap13}

\begin{antiqua}XIV. \end{antiqua} In dem der {\glqq}Albatro{\s}{\grqq}
etwa{\s} au{\s}f\"uhrt, wa{\s} man vielleicht niemal{\s} h\"atte
au{\s}f\"uhren k\"onnen.~\dotfill\pageref{kap14}

\begin{antiqua}XV.\end{antiqua} Worin Dinge vorgehen, deren
Schilderung sich gewi{\ss} der M\"uhe lohnt.~\dotfill\pageref{kap15}

\begin{antiqua}XVI.\end{antiqua} Welche{\s} den Leser in einer
vielleicht beklagen{\s}werthen Ungewi{\ss}heit
l\"a{\ss}t.~\dotfill\pageref{kap16}

\begin{antiqua}XVII.\end{antiqua} Worin der Leser um zwei Monate
r\"uckw\"art{\s} und auch um neun Monate vorw\"art{\s} gef\"uhrt
wird.~\dotfill\pageref{kap17}

\begin{antiqua}XVIII.\end{antiqua} Welche{\s} diese wahrhafte
Geschichte zu Ende f\"uhrt, ohne sie zu
beendigen.~\dotfill\pageref{kap18}

\newpage

\small \pagenumbering{gobble}
\begin{verbatim}


End of the Project Gutenberg EBook Robur Der Sieger,
by Jules Verne

*** END OF THIS PROJECT GUTENBERG EBOOK ROBUR DER SIEGER ***

***** This file should be named 15559-pdf.pdf or 15559-pdf.zip *****
This and all associated files of various formats will be found in:
        http://www.gutenberg.org/1/5/5/5/15559/


Produced by K.F. Creiner and the Online Distributed Proofreading Team.

Updated editions will replace the previous one--the old editions
will be renamed.

Creating the works from public domain print editions means that no
one owns a United States copyright in these works, so the Foundation
(and you!) can copy and distribute it in the United States without
permission and without paying copyright royalties.  Special rules,
set forth in the General Terms of Use part of this license, apply to
copying and distributing Project Gutenberg-tm electronic works to
protect the PROJECT GUTENBERG-tm concept and trademark.  Project
Gutenberg is a registered trademark, and may not be used if you
charge for the eBooks, unless you receive specific permission.  If
you do not charge anything for copies of this eBook, complying with
the rules is very easy.  You may use this eBook for nearly any
purpose such as creation of derivative works, reports, performances
and research.  They may be modified and printed and given away--you
may do practically ANYTHING with public domain eBooks.
Redistribution is subject to the trademark license, especially
commercial redistribution.



*** START: FULL LICENSE ***

THE FULL PROJECT GUTENBERG LICENSE PLEASE READ THIS BEFORE YOU
DISTRIBUTE OR USE THIS WORK

To protect the Project Gutenberg-tm mission of promoting the free
distribution of electronic works, by using or distributing this work
(or any other work associated in any way with the phrase "Project
Gutenberg"), you agree to comply with all the terms of the Full
Project Gutenberg-tm License (available with this file or online at
http://gutenberg.net/license).


Section 1.  General Terms of Use and Redistributing Project
Gutenberg-tm electronic works

1.A.  By reading or using any part of this Project Gutenberg-tm
electronic work, you indicate that you have read, understand, agree
to and accept all the terms of this license and intellectual
property (trademark/copyright) agreement.  If you do not agree to
abide by all the terms of this agreement, you must cease using and
return or destroy all copies of Project Gutenberg-tm electronic
works in your possession. If you paid a fee for obtaining a copy of
or access to a Project Gutenberg-tm electronic work and you do not
agree to be bound by the terms of this agreement, you may obtain a
refund from the person or entity to whom you paid the fee as set
forth in paragraph 1.E.8.

1.B.  "Project Gutenberg" is a registered trademark.  It may only be
used on or associated in any way with an electronic work by people
who agree to be bound by the terms of this agreement.  There are a
few things that you can do with most Project Gutenberg-tm electronic
works even without complying with the full terms of this agreement.
See paragraph 1.C below.  There are a lot of things you can do with
Project Gutenberg-tm electronic works if you follow the terms of
this agreement and help preserve free future access to Project
Gutenberg-tm electronic works.  See paragraph 1.E below.

1.C.  The Project Gutenberg Literary Archive Foundation ("the
Foundation" or PGLAF), owns a compilation copyright in the
collection of Project Gutenberg-tm electronic works.  Nearly all the
individual works in the collection are in the public domain in the
United States.  If an individual work is in the public domain in the
United States and you are located in the United States, we do not
claim a right to prevent you from copying, distributing, performing,
displaying or creating derivative works based on the work as long as
all references to Project Gutenberg are removed.  Of course, we hope
that you will support the Project Gutenberg-tm mission of promoting
free access to electronic works by freely sharing Project
Gutenberg-tm works in compliance with the terms of this agreement
for keeping the Project Gutenberg-tm name associated with the work.
You can easily comply with the terms of this agreement by keeping
this work in the same format with its attached full Project
Gutenberg-tm License when you share it without charge with others.

1.D.  The copyright laws of the place where you are located also
govern what you can do with this work.  Copyright laws in most
countries are in a constant state of change.  If you are outside the
United States, check the laws of your country in addition to the
terms of this agreement before downloading, copying, displaying,
performing, distributing or creating derivative works based on this
work or any other Project Gutenberg-tm work.  The Foundation makes
no representations concerning the copyright status of any work in
any country outside the United States.

1.E.  Unless you have removed all references to Project Gutenberg:

1.E.1.  The following sentence, with active links to, or other
immediate access to, the full Project Gutenberg-tm License must
appear prominently whenever any copy of a Project Gutenberg-tm work
(any work on which the phrase "Project Gutenberg" appears, or with
which the phrase "Project Gutenberg" is associated) is accessed,
displayed, performed, viewed, copied or distributed:

This eBook is for the use of anyone anywhere at no cost and with
almost no restrictions whatsoever.  You may copy it, give it away or
re-use it under the terms of the Project Gutenberg License included
with this eBook or online at www.gutenberg.net

1.E.2.  If an individual Project Gutenberg-tm electronic work is
derived from the public domain (does not contain a notice indicating
that it is posted with permission of the copyright holder), the work
can be copied and distributed to anyone in the United States without
paying any fees or charges.  If you are redistributing or providing
access to a work with the phrase "Project Gutenberg" associated with
or appearing on the work, you must comply either with the
requirements of paragraphs 1.E.1 through 1.E.7 or obtain permission
for the use of the work and the Project Gutenberg-tm trademark as
set forth in paragraphs 1.E.8 or 1.E.9.

1.E.3.  If an individual Project Gutenberg-tm electronic work is
posted with the permission of the copyright holder, your use and
distribution must comply with both paragraphs 1.E.1 through 1.E.7
and any additional terms imposed by the copyright holder.
Additional terms will be linked to the Project Gutenberg-tm License
for all works posted with the permission of the copyright holder
found at the beginning of this work.

1.E.4.  Do not unlink or detach or remove the full Project
Gutenberg-tm License terms from this work, or any files containing a
part of this work or any other work associated with Project
Gutenberg-tm.

1.E.5.  Do not copy, display, perform, distribute or redistribute
this electronic work, or any part of this electronic work, without
prominently displaying the sentence set forth in paragraph 1.E.1
with active links or immediate access to the full terms of the
Project Gutenberg-tm License.

1.E.6.  You may convert to and distribute this work in any binary,
compressed, marked up, nonproprietary or proprietary form, including
any word processing or hypertext form.  However, if you provide
access to or distribute copies of a Project Gutenberg-tm work in a
format other than "Plain Vanilla ASCII" or other format used in the
official version posted on the official Project Gutenberg-tm web
site (www.gutenberg.net), you must, at no additional cost, fee or
expense to the user, provide a copy, a means of exporting a copy, or
a means of obtaining a copy upon request, of the work in its
original "Plain Vanilla ASCII" or other form.  Any alternate format
must include the full Project Gutenberg-tm License as specified in
paragraph 1.E.1.

1.E.7.  Do not charge a fee for access to, viewing, displaying,
performing, copying or distributing any Project Gutenberg-tm works
unless you comply with paragraph 1.E.8 or 1.E.9.

1.E.8.  You may charge a reasonable fee for copies of or providing
access to or distributing Project Gutenberg-tm electronic works
provided that

- You pay a royalty fee of 20% of the gross profits you derive from
     the use of Project Gutenberg-tm works calculated using the method
     you already use to calculate your applicable taxes.  The fee is
     owed to the owner of the Project Gutenberg-tm trademark, but he
     has agreed to donate royalties under this paragraph to the
     Project Gutenberg Literary Archive Foundation.  Royalty payments
     must be paid within 60 days following each date on which you
     prepare (or are legally required to prepare) your periodic tax
     returns.  Royalty payments should be clearly marked as such and
     sent to the Project Gutenberg Literary Archive Foundation at the
     address specified in Section 4, "Information about donations to
     the Project Gutenberg Literary Archive Foundation."

- You provide a full refund of any money paid by a user who notifies
     you in writing (or by e-mail) within 30 days of receipt that s/he
     does not agree to the terms of the full Project Gutenberg-tm
     License.  You must require such a user to return or
     destroy all copies of the works possessed in a physical medium
     and discontinue all use of and all access to other copies of
     Project Gutenberg-tm works.

- You provide, in accordance with paragraph 1.F.3, a full refund of
     any money paid for a work or a replacement copy, if a defect in
     the electronic work is discovered and reported to you within 90
     days of receipt of the work.

- You comply with all other terms of this agreement for free
     distribution of Project Gutenberg-tm works.

1.E.9.  If you wish to charge a fee or distribute a Project
Gutenberg-tm electronic work or group of works on different terms
than are set forth in this agreement, you must obtain permission in
writing from both the Project Gutenberg Literary Archive Foundation
and Michael Hart, the owner of the Project Gutenberg-tm trademark.
Contact the Foundation as set forth in Section 3 below.

1.F.

1.F.1.  Project Gutenberg volunteers and employees expend
considerable effort to identify, do copyright research on,
transcribe and proofread public domain works in creating the Project
Gutenberg-tm collection.  Despite these efforts, Project
Gutenberg-tm electronic works, and the medium on which they may be
stored, may contain "Defects," such as, but not limited to,
incomplete, inaccurate or corrupt data, transcription errors, a
copyright or other intellectual property infringement, a defective
or damaged disk or other medium, a computer virus, or computer codes
that damage or cannot be read by your equipment.

1.F.2.  LIMITED WARRANTY, DISCLAIMER OF DAMAGES - Except for the
"Right of Replacement or Refund" described in paragraph 1.F.3, the
Project Gutenberg Literary Archive Foundation, the owner of the
Project Gutenberg-tm trademark, and any other party distributing a
Project Gutenberg-tm electronic work under this agreement, disclaim
all liability to you for damages, costs and expenses, including
legal fees.  YOU AGREE THAT YOU HAVE NO REMEDIES FOR NEGLIGENCE,
STRICT LIABILITY, BREACH OF WARRANTY OR BREACH OF CONTRACT EXCEPT
THOSE PROVIDED IN PARAGRAPH F3.  YOU AGREE THAT THE FOUNDATION, THE
TRADEMARK OWNER, AND ANY DISTRIBUTOR UNDER THIS AGREEMENT WILL NOT
BE LIABLE TO YOU FOR ACTUAL, DIRECT, INDIRECT, CONSEQUENTIAL,
PUNITIVE OR INCIDENTAL DAMAGES EVEN IF YOU GIVE NOTICE OF THE
POSSIBILITY OF SUCH DAMAGE.

1.F.3.  LIMITED RIGHT OF REPLACEMENT OR REFUND - If you discover a
defect in this electronic work within 90 days of receiving it, you
can receive a refund of the money (if any) you paid for it by
sending a written explanation to the person you received the work
from.  If you received the work on a physical medium, you must
return the medium with your written explanation.  The person or
entity that provided you with the defective work may elect to
provide a replacement copy in lieu of a refund.  If you received the
work electronically, the person or entity providing it to you may
choose to give you a second opportunity to receive the work
electronically in lieu of a refund.  If the second copy is also
defective, you may demand a refund in writing without further
opportunities to fix the problem.

1.F.4.  Except for the limited right of replacement or refund set
forth in paragraph 1.F.3, this work is provided to you 'AS-IS', WITH
NO OTHER WARRANTIES OF ANY KIND, EXPRESS OR IMPLIED, INCLUDING BUT
NOT LIMITED TO WARRANTIES OF MERCHANTIBILITY OR FITNESS FOR ANY
PURPOSE.

1.F.5.  Some states do not allow disclaimers of certain implied
warranties or the exclusion or limitation of certain types of
damages. If any disclaimer or limitation set forth in this agreement
violates the law of the state applicable to this agreement, the
agreement shall be interpreted to make the maximum disclaimer or
limitation permitted by the applicable state law.  The invalidity or
unenforceability of any provision of this agreement shall not void
the remaining provisions.

1.F.6.  INDEMNITY - You agree to indemnify and hold the Foundation,
the trademark owner, any agent or employee of the Foundation, anyone
providing copies of Project Gutenberg-tm electronic works in
accordance with this agreement, and any volunteers associated with
the production, promotion and distribution of Project Gutenberg-tm
electronic works, harmless from all liability, costs and expenses,
including legal fees, that arise directly or indirectly from any of
the following which you do or cause to occur: (a) distribution of
this or any Project Gutenberg-tm work, (b) alteration, modification,
or additions or deletions to any Project Gutenberg-tm work, and (c)
any Defect you cause.


Section  2.  Information about the Mission of Project Gutenberg-tm

Project Gutenberg-tm is synonymous with the free distribution of
electronic works in formats readable by the widest variety of
computers including obsolete, old, middle-aged and new computers.
It exists because of the efforts of hundreds of volunteers and
donations from people in all walks of life.

Volunteers and financial support to provide volunteers with the
assistance they need, is critical to reaching Project Gutenberg-tm's
goals and ensuring that the Project Gutenberg-tm collection will
remain freely available for generations to come.  In 2001, the
Project Gutenberg Literary Archive Foundation was created to provide
a secure and permanent future for Project Gutenberg-tm and future
generations. To learn more about the Project Gutenberg Literary
Archive Foundation and how your efforts and donations can help, see
Sections 3 and 4 and the Foundation web page at
http://www.pglaf.org.


Section 3.  Information about the Project Gutenberg Literary Archive
Foundation

The Project Gutenberg Literary Archive Foundation is a non profit
501(c)(3) educational corporation organized under the laws of the
state of Mississippi and granted tax exempt status by the Internal
Revenue Service.  The Foundation's EIN or federal tax identification
number is 64-6221541.  Its 501(c)(3) letter is posted at
http://pglaf.org/fundraising.  Contributions to the Project
Gutenberg Literary Archive Foundation are tax deductible to the full
extent permitted by U.S. federal laws and your state's laws.

The Foundation's principal office is located at 4557 Melan Dr. S.
Fairbanks, AK, 99712., but its volunteers and employees are
scattered throughout numerous locations.  Its business office is
located at 809 North 1500 West, Salt Lake City, UT 84116, (801)
596-1887, email business@pglaf.org.  Email contact links and up to
date contact information can be found at the Foundation's web site
and official page at http://pglaf.org

For additional contact information:
     Dr. Gregory B. Newby
     Chief Executive and Director
     gbnewby@pglaf.org

Section 4.  Information about Donations to the Project Gutenberg
Literary Archive Foundation

Project Gutenberg-tm depends upon and cannot survive without wide
spread public support and donations to carry out its mission of
increasing the number of public domain and licensed works that can
be freely distributed in machine readable form accessible by the
widest array of equipment including outdated equipment.  Many small
donations ($1 to $5,000) are particularly important to maintaining
tax exempt status with the IRS.

The Foundation is committed to complying with the laws regulating
charities and charitable donations in all 50 states of the United
States.  Compliance requirements are not uniform and it takes a
considerable effort, much paperwork and many fees to meet and keep
up with these requirements.  We do not solicit donations in
locations where we have not received written confirmation of
compliance.  To SEND DONATIONS or determine the status of compliance
for any particular state visit http://pglaf.org

While we cannot and do not solicit contributions from states where
we have not met the solicitation requirements, we know of no
prohibition against accepting unsolicited donations from donors in
such states who approach us with offers to donate.

International donations are gratefully accepted, but we cannot make
any statements concerning tax treatment of donations received from
outside the United States.  U.S. laws alone swamp our small staff.

Please check the Project Gutenberg Web pages for current donation
methods and addresses.  Donations are accepted in a number of other
ways including including checks, online payments and credit card
donations.  To donate, please visit: http://pglaf.org/donate


Section 5.  General Information About Project Gutenberg-tm
electronic works.

Professor Michael S. Hart is the originator of the Project
Gutenberg-tm concept of a library of electronic works that could be
freely shared with anyone.  For thirty years, he produced and
distributed Project Gutenberg-tm eBooks with only a loose network of
volunteer support.

Project Gutenberg-tm eBooks are often created from several printed
editions, all of which are confirmed as Public Domain in the U.S.
unless a copyright notice is included.  Thus, we do not necessarily
keep eBooks in compliance with any particular paper edition.

Most people start at our Web site which has the main PG search
facility:

     http://www.gutenberg.net

This Web site includes information about Project Gutenberg-tm,
including how to make donations to the Project Gutenberg Literary
Archive Foundation, how to help produce our new eBooks, and how to
subscribe to our email newsletter to hear about new eBooks.
\end{verbatim}

\end{document}
